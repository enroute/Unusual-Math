
\chapter{算法}
\label{chap:algorithm}

\section{二进制}
\label{sec:binary-system}

\begin{example}
  有$1000$瓶水,其中有一瓶有剧毒(假设哪怕一个毒药分子在里面也能致命),其余的都是纯净水。蟑螂碰到剧毒药水在$2$小时内就会死亡。现在有$10$只蟑螂可以用来试验水是否有毒,有什么方法可以将有毒的一瓶找出来?哪种方法需要的时间最短?
\end{example}
\begin{proof}[提示]
  最直观的方法,一瓶一瓶的试,最坏的情况是试到最后一瓶才找出来有剧毒的水。需要的时间是$2\times1000=2000$小时。

  也可以用二分法,每次将范围缩小一半。第一次将1000瓶分两半,前500瓶各取一点混成一份,后500瓶各取一点混成一份,一只蟑螂喝其中一份,可分出剧毒水在哪500瓶里。以此类推,第二只蟑螂将500瓶缩小至250瓶。由于$2^{10}=1024>1000$,即使对最坏的情况,10只蟑螂也是够用的。在不考虑配制药水时间的情况下,这种方法需要的时间不超过最坏情况下的$2\times10=20$小时。

  还可以用二进制的方法。按以下方法配10瓶药水:
  \begin{quotation}
    将1000瓶药水编号,写出其编号的二进制形式,不足10位的在前面补零,使其编号都是十位的二进制,并排成一列,如下:
    \begin{center}
    \begin{tabular}{ccccccccccc}
      \toprule[2pt]
      \multirow{2}{*}{瓶编号} & \multicolumn{10}{c}{十位二进制编号}\\
      \cline{2-11} & 1 & 2 & 3 & 4 & 5 & 6 & 7 & 8 & 9 & 10\\
      \midrule
      1 & 0 & 0 & 0 & 0 & 0 & 0 & 0 & 0 & 0 & 1\\
      2 & 0 & 0 & 0 & 0 & 0 & 0 & 0 & 0 & 1 & 0\\
      3 & 0 & 0 & 0 & 0 & 0 & 0 & 0 & 0 & 1 & 1\\
      4 & 0 & 0 & 0 & 0 & 0 & 0 & 0 & 1 & 0 & 0\\
      5 & 0 & 0 & 0 & 0 & 0 & 0 & 0 & 1 & 0 & 1\\
      \multicolumn{11}{c}{$\cdots\cdots$}\\
      \hline
        & $1^*$ & 0 & $1^*$ & 0 & $1^*$ & 0 & 0 & 0 & 0 & 0\\
      \multicolumn{11}{c}{$^*$示例:喝了1、3和5号混合药水的蟑螂死了}\\  
      \bottomrule[2pt]
    \end{tabular}
    \end{center}

    用十位二进制编号第1位为1的药水配成第1瓶混合药水;

    用十位二进制编号第2位为1的药水配成第2瓶混合药水;

    用十位二进制编号第3位为1的药水配成第3瓶混合药水;
    
    $\cdots\cdots$

    如配制第10瓶药水,需要取第1、3、5、$\cdots$瓶药水混合。
  \end{quotation}

    配好药水后每个蟑螂分别喝一瓶混合药水。若喝了第$i$号混合药水的蟑螂死了,则说明剧毒药水的十位二进制编码在该位上为1,若蟑螂没死,则该位为0。两小时后观察十只蟑螂的存活状态,可得到剧毒药水的十位二进制编号。所需时间为配制药水所需的时间再加上两小时的观察时间。

    举例,如若第1、3、5号蟑螂死了,其余蟑螂没死,那么剧毒药水的十位二进制编号为$1010100000$,对应的十进制数为$2^9 + 2^7 + 2^5=672$。
\end{proof}