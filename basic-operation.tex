
\chapter{基本符号与运算}
\label{chap:basic-operations}

\section{阶乘}
\label{sec:factorial}

\begin{definition}[阶乘,Factorial]
  $\forall n\in\mathcal{Z}^+$,$1,2,3,\cdots,n$这$n$个正整数的乘积记为$n$的阶乘,用符号$n!$表示。即
  \begin{align}
    n!\equiv 1\times 2\times 3\times \cdots \times (n-1) \times n
  \end{align}
  $0$的阶乘定义为1,即
  \begin{align}
    0!\equiv 1
  \end{align}
\end{definition}

\begin{definition}[阶乘的等价定义]阶乘也可以通过递归定义。
  $\forall n\in\mathcal{Z}^+$,$n!$由下式定义
  \begin{align}
    n!\equiv
    \begin{cases}
      1               & n=0\\
      n\times (n-1)!  & n\ne 0
    \end{cases}
  \end{align}
\end{definition}


\begin{definition}[双阶乘,Double Factorial]双阶乘的定义分奇偶。
  \begin{enumerate}
  \item 当$n$是正奇数时,
    \begin{align}
      n!!\equiv 1\times 3\times 5\times \cdots \times (n-2) \times n
    \end{align}
  \item 当$n$是正偶数时,
    \begin{align}
      n!!\equiv 2\times 4\times 6\times \cdots \times (n-2) \times n
    \end{align}
  \item 0的双阶乘定义为1,即
    \begin{align}
      0!!\equiv 1
    \end{align}
  \end{enumerate}
\end{definition}

\begin{definition}
  同样,双阶乘也可以通过递归来定义。
  \begin{align}
    n!!\equiv
    \begin{cases}
      1                & n=0, 1\\
      n\times (n-2)!!  & n\ne 0, 1
    \end{cases}
  \end{align}
\end{definition}

\begin{example}
  $\forall n\in\mathcal{Z}^+$,
  \begin{align}
    n!! = \frac{n!}{(n-1)!!}
  \end{align}

  $\forall n=2k$, where $k$ is non-negative integer,
  \begin{align}
    n!! = 2^k\cdot k!
  \end{align}
\end{example}


\begin{definition}[Gamma函数]
  阶乘可以在实数或复数域上延拓\footnote{除去某些点外}。下面两个等价定义称为Gamma函数:
  \begin{align}
    \Gamma(x) \equiv \int_{t=0}^\infty e^{-t} \cdot t^{x-1} \dt
    = \int_{t=0}^1 (-\ln t)^x \dt
  \end{align}
  $\Gamma(x)$也记为$x!$。以下形式有时更容易找到积分与$\Gamma$之间的关系:
  \begin{align}
    \int_{t=0}^\infty e^{-t} \cdot t^{x} \dt = \Gamma(x + 1)
  \end{align}
\end{definition}

\begin{example}
  欧拉在研究阶乘$n!$时,先找到的函数是
  \begin{align}
    J(m, n)\equiv\int_0^1 x^m\cdot (1-x)^n \dx
  \end{align}
  显然$J(m,0)=1/(m+1)$。利用分部积分,容易得到以下性质
  \begin{align}
    J(m,n)=\frac{n}{m+1}J(m+1,n-1)
  \end{align}
  递推到$n=0$时,容易得到
  \begin{align}
    J(m,n)=\frac{n(n-1)(n-2)\cdots 1}{(m+1)(m+2)(m+3)\cdots (m+n)}\cdot J(m+n,0)
  \end{align}
  由此,欧拉得到了如下公式
  \begin{align}
    n!=(m+1)(m+2)(m+3)\cdots (m+n+1)\int_0^1 x^m\cdot (1-x)^n\dx
  \end{align}
  之后,再利用其他数学技巧,欧拉得到了$\Gamma$函数。
\end{example}

\begin{question}
  任意定义域内的$x$,有$\Gamma(x + 1) = x\cdot \Gamma(x)$。
\end{question}

\begin{question}
  对于非负整数$n$,$\Gamma(n + 1) = n! = n\cdot (n-1) \cdots 2\cdot 1$。
\end{question}

\begin{question}
  证明:
  \begin{align}
    \left(-\frac12\right)! = \sqrt\pi
  \end{align}
  实际上,
  \begin{align*}
    \left(-\frac12\right)! = \Gamma\left(-\frac12 + 1\right) = \Gamma\left(\frac12\right)
  \end{align*}
\end{question}

\begin{example}
  对于正奇数$2k+1$,有
  \begin{align*}
    \left(\frac{2k+1}{2}\right)! =&
    \frac{2k+1}2 \cdot\frac{2k-1}2 \cdot\frac{2k-3}2 \cdot \cdot\frac{1}2 \cdot \left(-\frac12\right)!\\
    =&\frac{(2k+1)!!}{2^{k+1}} \sqrt\pi
  \end{align*}
\end{example}

\begin{example}
  求定积分
  \begin{align}
    \int_0^{+\infty} x^2\cdot e^{-2x} \dx
  \end{align}
  令$2x\to t$,可得原式等于
  \begin{align}
    \frac18 \int_0^{+\infty} t^2\cdot e^{-t} \dt = \frac18 \cdot \Gamma(2 + 1) = \frac18 \cdot 2! = \frac14
  \end{align}
\end{example}

\begin{example}
  求定积分
  \begin{align}
    \int_0^{+\infty} x^2\cdot e^{-x^2} \dx
  \end{align}
  令$x^2\to t$再利用$\Gamma$函数,可得原式$=\Gamma(1/2 + 1) / 2 = \sqrt\pi / 4$。
\end{example}

\begin{question}
  证明:正态分布中的高斯积分等于$\Gamma(3/2)$,即
  \begin{align}
    \int_0^{+\infty} e^{-x^2}\dx = \Gamma\left(\frac12+1\right) = \frac{\sqrt\pi}2
  \end{align}
\end{question}


\begin{definition}[Beta函数]
  下面定义的函数称为Beta函数
  \begin{align}
    B(x,y)\equiv\int_{t=0}^1 t^{x-1} \cdot (1-t)^{1-y} \dt
  \end{align}
\end{definition}

\begin{question}
  任意定义域内的$x,y$,有
  \begin{align}
    B(x,y)=\frac{\Gamma(x)\Gamma(y)}{\Gamma(x+y)}
  \end{align}
\end{question}

\begin{question}
  证明:
  \begin{align}
    \frac21\cdot\frac23 \cdot\frac43 \cdot\frac45 \cdot\frac65 \cdot\frac67 \cdot\frac87
    \cdot\frac89\cdots=\frac\pi2
  \end{align}
\end{question}