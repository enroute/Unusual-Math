
\chapter{趣味集锦}
\label{chap:fun}

\section{数字}
\label{sec:fun-number}


\section{方程先放一边}
\label{sec:leave-equation-alone}

方程是一个非常强力的数学工具。然而有时不设求知数使用方程,直接寻找问题中的数理关系更能锻练、提高数学思维。

\begin{example}
  两个桶,一个装有红油漆,一个装有蓝油漆。一个装修工从蓝油漆桶里舀了一点到红油漆桶混合后,再从红油漆桶里舀了同等体积的混合油漆到蓝油漆桶里。问这时红油漆桶里的蓝油漆与蓝油漆桶里的红油漆这两者体积的数量关系。
\end{example}
\begin{proof}[提示]
  如果不考虑两种油漆混合后体积会变大或缩小的问题,那么这种问题是比较简单的。混合后总体积不变,那么就有红油漆里的蓝油漆,蓝油漆桶里的红油漆,这两者的体积是一样的。


  操作过后,红油漆桶里蓝油漆的体积,相当于此部分的红油漆与蓝油漆桶交换了。所以两者是一样多的。
\end{proof}

\begin{example}
  下图是$A,B,C$三个积木的两种摆放方式,问积木$A$的高度是多少?
  \begin{center}
    \begin{tikzpicture}[scale=.8]
      \begin{scope}
        \draw(0,0)rectangle(1,4) node[midway]{$A$};
        \filldraw[fill=red!20](1,0)rectangle(2,2.4) node[midway]{$B$};
        \filldraw[pattern=north west lines, pattern color=blue!20, draw=black](0,4)rectangle(1,5) node[midway]{$C$};
        \draw[|<->|](2.5,2.4)--(2.5,5)node[midway,fill=white]{$26$cm};
      \end{scope}
      \begin{scope}[shift={(4,0)}]
        \draw(0,0)rectangle(1,4) node[midway]{$A$};
        \filldraw[fill=red!20](0,4)rectangle(1,6.4) node[midway]{$B$};
        \filldraw[pattern=north west lines, pattern color=blue!20, draw=black](1,0)rectangle(2,1) node[midway]{$C$};
        \draw[|<->|](2.5,1)--(2.5,6.4)node[midway,fill=white]{$54$cm};
      \end{scope}
    \end{tikzpicture}
  \end{center}
\end{example}
\begin{proof}[提示]
  看起来$A$的高度像是26与54的平均值。从这个方向入手,则可以摆出下图来验证。为了省点篇幅,把两个图逆时针旋转$90^\circ$,即积木放倒,摆成横的,然后再两个图拼接起来。
  \begin{center}
    \begin{tikzpicture}[scale=.8]
      \begin{scope}
        \draw[dashed](5,2)rectangle(9,3)node[midway]{$A$};
        \filldraw[dashed,pattern=north west lines, pattern color=blue!20, draw=black](8,3)rectangle(9,4) node[midway]{$C$};
        \filldraw[dashed,fill=red!20](2.6,2)rectangle(5,3) node[midway]{$B$};
        \draw[|<->|](2.6,3.5)--(8,3.5)node[midway,fill=white]{$54$cm};

        \draw(1,0)rectangle(5,1) node[midway]{$A$};
        \filldraw[fill=red!20](2.6,1)rectangle(5,2) node[midway]{$B$};
        \filldraw[pattern=north west lines, pattern color=blue!20, draw=black](0,0)rectangle(1,1) node[midway]{$C$};
        \draw[|<->|](0,1.5)--(2.6,1.5)node[midway,fill=white]{$26$cm};
        \draw[|<->|](1,-.5)--(9,-.5)node[midway,fill=white]{$2A$};
      \end{scope}
    \end{tikzpicture}
  \end{center}
  由上图,$26+54=80$cm的长度往右平移一个$C$后即为两个$A$的长度(放倒之后是长度,竖直放是高度),从而原问题所求的$A$的高度(即放倒后的长度)为$80\div2=40$cm。

  上面的思路是把26和54两个长度首尾连接在一起。也可以把两个长度在一端对齐,看两者之间的差别,如下图。
  \begin{center}
    \begin{tikzpicture}[scale=.8]
      \def\A[#1,#2]{\begin{scope}[shift={(#1,#2)}]
          \draw(0,0)rectangle(4,1) node[midway]{$A$};
        \end{scope}
      }
      \def\B[#1,#2]{\begin{scope}[shift={(#1,#2)}]
          \filldraw[fill=red!20](0,0)rectangle(2.4,1) node[midway]{$B$};
        \end{scope}
      }
      \def\C[#1,#2]{\begin{scope}[shift={(#1,#2)}]
          \filldraw[pattern=north west lines, pattern color=blue!20, draw=black](0,0)rectangle(1,1) node[midway]{$C$};
        \end{scope}
      }
      \begin{scope}
        \A[1,0]
        \B[2.6,1]
        \C[0,0]
        \draw[|<->|](0,1.5)--(2.6,1.5)node[midway,fill=white]{$26$cm};
        \begin{scope}[dashed,shift={(0,2.2)}]
          \B[0,0]\A[2.4,0]\C[5.4,1]
          \draw[|<->|](0,1.5)--(5.4,1.5)node[midway,fill=white]{$54$cm};
        \end{scope}
        \draw[|<->|](5,.5)--(6.4,.5)node[midway,fill=white]{\small $\Delta$};
        \draw(0,-.5)--(0,-1.5);
        \draw[->](1,-.83)--(0,-.83);
        \draw[->](1,-1.13)--(0,-1.13);
        \node[right] at (1.5,-1){左对齐};

      \end{scope}
      \begin{scope}[shift={(10,0)}]
        \A[1,0]
        \B[2.6,1]
        \C[0,0]
        \begin{scope}[dashed,shift={(-1.4,2.2)}]
          \B[0,0]\A[2.4,0]\C[5.4,1]
        \end{scope}
        \draw[|<->|](-1.4,.5)--(0,.5)node[midway,fill=white]{\small $\Delta$};
        \draw(5,-.5)--(5,-1.5);
        \draw[->](4,-.83)--(5,-.83);
        \draw[->](4,-1.13)--(5,-1.13);
        \node[left] at (3.5,-1){右对齐};
      \end{scope}
    \end{tikzpicture}
  \end{center}
  由上面右图,可知$\Delta$处为$B-C$,从而左边$\Delta$处也为$B-C$,从而把$\Delta$代入
  \begin{align*}
    26 + B + \Delta = 54 + C,\quad\implies\quad
    26 + B + (B-C) = 54 + C
  \end{align*}
  有$B-C= (54-26)\div2=14$cm。如果早猜测到这个结论,也可以很容易地构造图形来解决。
  \begin{center}
    \begin{tikzpicture}[scale=.8]
      \def\A[#1,#2,#3]{\begin{scope}[shift={(#1,#2)}]
          \draw(0,0)rectangle(4,1) node[midway]{#3};
        \end{scope}
      }
      \def\B[#1,#2]{\begin{scope}[shift={(#1,#2)}]
          \filldraw[fill=red!20](0,0)rectangle(2.4,1) node[midway]{$B$};
        \end{scope}
      }
      \def\C[#1,#2]{\begin{scope}[shift={(#1,#2)}]
          \filldraw[pattern=north west lines, pattern color=blue!20, draw=black](0,0)rectangle(1,1) node[midway]{$C$};
        \end{scope}
      }
      \begin{scope}
        \A[1,0,]
        \B[2.6,1]
        \C[0,0]\C[2.6,0]
        \draw[|<->|](0,1.5)--(2.6,1.5)node[midway,fill=white]{$26$cm};
        \begin{scope}[dashed,shift={(0,2.2)}]
          \B[0,0]\A[2.4,0,$A$]\C[5.4,1]
          \draw[|<->|](0,1.5)--(5.4,1.5)node[midway,fill=white]{$54$cm};
        \end{scope}
        \draw[|<->|](5,.5)--(6.4,.5)node[midway,fill=white]{\small $\Delta$};
        \draw[|<->|](3.6,.5)--(5,.5)node[midway,fill=white]{\small $\Delta$};
        \draw[|<->|](2.6,-.5)--(5.4,-.5)node[midway,fill=white]{\small $2\Delta$};
        \draw[dashed,help lines](5.4,-1)--(5.4,4);

        % \draw(0,-.5)--(0,-1.5);
        % \draw[->](1,-.83)--(0,-.83);
        % \draw[->](1,-1.13)--(0,-1.13);
        % \node[right] at (1.5,-1){左对齐};

      \end{scope}
    \end{tikzpicture}
  \end{center}  
  将两个$\Delta$往左平移一个$C$的长度,则其对齐情况如上图,从而
  \begin{align*}
    2\Delta = 54 - 26 \quad\implies\quad \Delta = \frac{54-26}{2} = 14\text{cm}
  \end{align*}

  

  更通用一点,记图中$54,26$分别是$x,y$,则有
  \begin{align*}
    \begin{drcases}             % drcases in mathtools package
      B-C=\frac{x-y}{2}\\
      A+B-C=x
    \end{drcases}
    \implies A = x - \frac{x-y}{2} = \frac{x+y}{2}
  \end{align*}

  由上式亦可得到$A=(54+26)\div2=40$cm,即$A,B,C$三者的长度(单位:cm)分别为$(40,t+14,t)$,其中$t$是正数且由$B<A$有$t$满足$t+14<40\implies t<26$\footnote{由$C<26$也可以直接得到$t<26$。}。
\end{proof}

\begin{example}[列车齐头齐尾]
  两列车长度未知,只知道列车$A,B$速度(单位:米/秒)分别为$26,18$。若两车齐头出发,则30秒后$A$完全超过$B$;若两车齐尾出发,则25秒后$A$完全超过$B$。问两车的长度各多少?
\end{example}
\begin{proof}[提示]先画示意图。
  \begin{center}
    \begin{tikzpicture}[scale=1.0]
      \def\A[#1,#2]{\begin{scope}[shift={(#1,#2)}]
          \draw(0,0)rectangle(4,1)node[midway]{$A$};
        \end{scope}
      }
      \def\B[#1,#2]{\begin{scope}[shift={(#1,#2)}]
          \filldraw[fill=red!20](0,0)rectangle(3,1)node[midway]{$B$};
        \end{scope}
      }
      \begin{scope}
        \draw[->](1.5,2.5)--(2.5,2.5)node[midway,above]{行车方向};
        \A[0,1]\B[1,0]
        \node[below] at (2,0){齐头出发};
      \end{scope}
      \begin{scope}[shift={(6,0)}]
        \draw[->](1.5,2.5)--(2.5,2.5)node[midway,above]{行车方向};
        \A[0,1]\B[0,0]
        \node[below] at (2,0){齐尾出发};
      \end{scope}
      \begin{scope}[shift={(3,-4.5)}]
        \draw[->](3,2.5)--(4,2.5)node[midway,above]{行车方向};
        \A[3,1]\B[0,0]
        \node[below] at (3.5,0){结束状态};
      \end{scope}
    \end{tikzpicture}
  \end{center}
  以$B$车为参考(即保持$B$不动),则相当于$A$车以$26-18=8$米/秒的速度在超过$B$车,由上图,齐头出发时,在$B$上看,$A$车相当于以8米/秒的速度走了$A$车的长度;齐尾出发时,在$B$车上看时,$A$车相当于走了$B$车的长度。从而两车的长度(单位:米)分别为
  \begin{align*}
    A=8\times 30 = 240\\
    B=8\times 25 = 200&\qedhere
  \end{align*}
\end{proof}

\begin{example}[火车过桥]\label{ex:train-crossing-bridge}
  一火车正在匀速地通过一座长度为1000米的桥,其从开始上桥到完全下桥共花了120秒,整列火车完全在桥上的时间为80秒。问火车的长度与速度。
\end{example}
\begin{proof}[提示]
  先画示意图。
  \begin{center}
    \begin{tikzpicture}[scale=.8]
      \def\A[#1,#2]{\begin{scope}[shift={(#1,#2)}]
          \draw(0,.2)rectangle(2,1.2)node[midway]{火车};
        \end{scope}
      }
      \def\B{
        \filldraw[fill=red!20](0,0)rectangle(12,.2)node[midway]{桥};
      }
      \draw[->](5.5,2)--(6.5,2)node[midway,above]{行车方向};
      \begin{scope}
        \A[-2,0]\B
        \node at (6,.9){1. 开始上桥};
      \end{scope}
      \begin{scope}[shift={(0,-2)}]
        \A[0,0]\B
        \node at (6,.9){2. 完全上桥};
      \end{scope}
      \begin{scope}[shift={(0,-4)}]
        \A[10,0]\B
        \node at (6,.9){3. 开始下桥};
      \end{scope}
      \begin{scope}[shift={(0,-6)}]
        \A[12,0]\B
        \node at (6,.9){4. 完全下桥};
      \end{scope}
    \end{tikzpicture}
  \end{center}
  由上图,整个过桥过程的120秒时间,被4个时刻分为三段。

  第一段是开始上桥的时刻1到完全上了桥的时刻2之间的时间,其行驶距离是火车的车身长度。

  第二段是完全了桥的时刻2到开始准备下桥的时刻3之间的时间,其行驶时间是80秒。

  第三段是开始下桥的时刻3到完全下了桥的时刻4之间的时间,其行驶距离是火车的车身长度。

  三段的总时间是120秒,中间段是80秒,从而第一段与第三段的时间和为$120-80=40$秒。且这两段时间里行驶的路程都是火车的长度,其时间应该一样,即都为$40\div2=20$秒,从而前两段时间(即从时刻1到时刻3之间)共花了$120-20=100$秒,路程为$1000$米,火车的速度为$1000\div100=10$米/秒。

  最后一段(或者第一段)时间为$20$秒,路程为火车的长度,从而火车的长度为$10\times20=200$米。
\end{proof}

\begin{example}
  一辆匀速行驶的火车,完全通过一个观测点花了12秒。从开始进入一个200米长的月台到完全驶出月台花了22秒。问火车的行驶速度与长度。
\end{example}
\begin{proof}[提示]
  可以用例~\ref{ex:train-crossing-bridge}火车过桥类似的方法。

  还有一种方法。观测点是没有长度的,花了12秒,如果把观测点慢慢拉长到月台那么长,那么12秒就变为了22秒,也就是说多出来的$22-12=10$秒是由于观测点被拉长了所导致的,从而其速度为$200\div10=20$米/秒。
\end{proof}

\begin{example}[户口调查员]
  网格员在敲开一户人家的门,并询问主人有几个小孩,孩子们都有几岁。主人说他有三个女儿,并且她们年龄的乘积是$36$。

  网格员说这些信息还不足以算出他的女儿们的年龄。

  主人又说:“我就是告诉你她们年龄的总和,你还是不能算出他们的年龄。”

  “我希望你能告诉我更多的信息。”

  “好吧,我的大女儿蓝若她喜欢读书。”

  这时,网格员知道这家三个小孩的年龄了。你知道吗?
\end{example}
\begin{proof}[解答]
  三个年龄乘积为$36$的组合只有以下几种(其中第二行为三人年龄之和):

\begin{center}
\begin{tabular}{cccccccc}
  \hline
  (1,1,36) & (1,2,18) & (1,3,12) & (1,4,9) & (1,6,6) & (2,2,9) & (2,3,6) & (3,3,4)\\
  \hline
  38       & 21       & 16       & 14      & 13      & 13      & 11      & 10\\
  \hline
\end{tabular}
\end{center}

其中和相等的只有两种组合,即$(1,6,6)$和$(2,2,9)$,其和均为$13$。由于主人说就算知道年龄之和也算不出来,可知三人年龄组合只能是$(1,6,6)$和$(2,2,9)$两种这一。而由最后一句,主人有一个最大的女儿,而不是两个一样大的大女儿,从而可知三人年龄组合是$(2,2,9)$。
\end{proof}

\begin{question}
  一个经理有三个女儿,三个女儿的年龄加起来等于13,三个女儿的年龄乘起来等于经理自己的年龄。有一个下属已知道经理的年龄,但仍不能确定经理三个女儿的年龄,这时经理说只有一个女儿的头发是黑的,然后这个下属就知道了经理三个女儿的年龄。请问三个女儿的年龄分别是多少?
\end{question}
\begin{proof}[提示]\let\qed\relax%remove qed symbol
  这里说的是中国人。中国小孩特别小的时候头发是黄的,所以有黄毛丫头一说。慢慢长大之后头发才开始变黑。考虑三数为13的组合情况。
  \begin{align*}\setlength\arraycolsep{3pt}\renewcommand*{\arraystretch}{.9}
    \begin{array}{cccccccccccccc}
      1 & \times & 1 & \times & 11 & = & 11, &\quad
      1 & \times & 2 & \times & 10 & = & 20\\
      1 & \times & 3 & \times &  9 & = & 27, &\quad
      1 & \times & 4 & \times &  8 & = & 32\\
      1 & \times & 5 & \times &  7 & = & 35, &\quad
      1 & \times & 6 & \times &  6 & = & \underline{36}\\
      2 & \times & 2 & \times &  9 & = & \underline{36}, &\quad
      2 & \times & 3 & \times &  8 & = & 48\\
      2 & \times & 4 & \times &  7 & = & 56, &\quad
      2 & \times & 5 & \times &  6 & = & 60\\
      3 & \times & 3 & \times &  7 & = & 63, &\quad
      3 & \times & 4 & \times &  6 & = & 72\\
      3 & \times & 5 & \times &  5 & = & 75, &\quad
      4 & \times & 4 & \times &  5 & = & 80
    \end{array}
  \end{align*}
\end{proof}

\begin{example}
  扎西多吉是个藏人,喜欢随身带着一柄$1.5$米长的长刀。有一次他出远门需要坐火车,但列车安全员不允许将长刀作为手提行李带上车。如果托运,火车的托运规定行李箱又不能超过$1$米。扎西多吉怎么才能合法地将他的长刀带上火车呢?
\end{example}
\begin{proof}[解答]
若局限在二维平面内考虑,是无解的。考虑一个$1\times1\times1$的箱子,其对角线长度为$\sqrt3>1.5$米,可以放下扎西多吉的长刀。
\end{proof}

\begin{example}
\renewcommand{\thefootnote}{\fnsymbol{footnote}}%use symbol rather than number in footnote markers
  唐宋明清铸造钱币的机构叫钱监。有一次,钱监里储存了10箱黄金,每箱100块,每块一两。有一个官员,为了捞油水,把某一个箱子里的每块黄金都磨掉了一钱\footnote{一两等于十钱}。如果允许带一个精度及量程都足够用的称重工具回去,如何能称一次就找到被磨了的那一箱黄金?
\end{example}
\begin{proof}[提示]\let\qed\relax%
  给箱子编号。称法的关键在于称出的重量能与箱子编号一一对应。1号取一块,2号取2块,3号取3块,$\cdots\cdots$,10号取10块,放在一起称。若全都是足称的金块,则应该是$1+2+3+\cdots+10=55$两,但某箱有缺陷,故实际称出来会比55两少。如果是1号被磨,则缺1钱;如果是2号被磨,则缺2钱。缺多少钱就是哪号箱子被磨了。
\end{proof}

\begin{question}
  药厂仓库保存了某种药片共8罐。后来由于异常断电仓库温度失控,导致某一罐中药片与空气发生了化学反应从而变质了,其每片的质量也重了1克。问如何只称一次就判断出是哪个罐子里的药片变质了?
\end{question}

\begin{example}[空瓶换酒]\label{ex:beer}
  超市里某种啤酒在作促销,两元一瓶的啤酒,若集齐两个空瓶子则可以免费换一瓶。小胖手里有10元钱,那么在这个超市里他的钱最多可以让他喝到几瓶啤酒?
\end{example}
\begin{proof}[提示]\renewcommand{\thefootnote}{\fnsymbol{footnote}}%use symbol rather than number in footnote markers
  方法一,借空瓶。买一瓶会得一空瓶,再加一空瓶即可换一瓶。因此2元买一瓶,喝完剩一空瓶,再借一空瓶,换一瓶,喝完又剩一空瓶,把空瓶还了。此时2元喝了两瓶,手里无剩余,无外债。即2元最多可喝2瓶。也就是10元最多可喝10瓶。

  方法二,计算啤酒不含瓶的成本。价值上,两个空瓶等于一整瓶,从而空瓶的成本是$2/2=1$元,啤酒(不含瓶)的成本是$2-1=1$元,从而10元能喝到的啤酒(不含瓶)的数量最多为$10/1=10$瓶。\footnote{此方法只给出了上限,没有证明存在某种方法能达到上限}
\end{proof}

\begin{question}
  在例~\ref{ex:beer}中,若集齐3个空瓶才能换一瓶,则又如何?
\end{question}
\begin{proof}[提示]
  借一空瓶,买两瓶,则4元可喝3瓶。8元可喝6瓶。剩两元可喝1瓶,剩余一个空瓶子。一个空瓶子不足以换取一瓶啤酒里的酒水(不含瓶),即最多可喝7瓶。
\end{proof}


\begin{example}
  推推开关是一种按一次就转变接通与断开状态的开关。现有10个推推开关对应10盏灯。开始时所有灯都是灭的。现进行如下操作(顺序不定):
  \begin{enumerate}
  \item 编号为1的倍数的开关都按一次;
  \item 编号为2的倍数的开关都按一次;
  \item 编号为3的倍数的开关都按一次;
  \item $\cdots\cdots$;
  \item 编号为10的倍数的开关都按一次。
  \end{enumerate}
  这样操作之后,问10盏灯的亮灭状态。
\end{example}
\begin{proof}[提示]\let\qed\relax
  首先,操作顺序不影响最终状态,因为某盏灯的最终状态只与对应开关被按的次数(精确地说是奇偶性)相关。开关总共被按了奇数次的灯是亮的,被按了偶数次的灯是灭的。

  而一个开关被按的次数,相当于其因数(包含1与其本身)的个数。由定理~\ref{th:count-of-positive-divisors-iff-square-number}可知当且仅当完全平方数有奇数个正约数,从而编号为1,4,9,16,25,36,49,64,81,100的10盏灯是亮的,其余编号的灯是灭的。

  % \centering
  % \begin{tabular}{clcccccccc}
  %   \toprule
  %   开关编号 & \multicolumn{1}{c}{因数} & 因数个数 & 灯最终状态\\\midrule
  %   1        & 1    & 1        & 亮\\
  %   2        & 1,2  & 2        & 灭\\
  %   3        & 1,3  & 2        & 灭\\
  %   4        & 1,2,4& 3        & 亮\\
  %   5        & 1,5  & 2        & 灭\\
  %   6        & 1,2,3,6 & 4        & 灭\\
  %   7        & 1,7  & 2        & 灭\\
  %   8        & 1,2,4,8 & 4        & 灭\\
  %   9        & 1,3,9   & 3        & 亮\\
  %   10       & 1,2,5,10& 4        & 灭\\
  %   \bottomrule
  % \end{tabular}
\end{proof}


\begin{example}
  Farmer Daniel and Paul are selling eggs in the market. Professor Prince says it looks like Daniel has more eggs than Paul.

  Maybe Daniel has 70 eggs and Paul has 30? Prince says.

  You're right on the total number. We'll get the same money if all our eggs are sold. If we exchange the number of eggs, then I'll get 18 dollars. Says Paul.

  Yes. And I'll get only 8 dollars if I sold eggs with the number of Paul's. Says Daniel.

  How many eggs, respectively, do they bring to the market?
\end{example}
\begin{proof}[Hint]
  It's easy to solve with equations. Let $a,b$ denote the number of eggs Daniel and Paul have, $x$ is the unit price of eggs sold by Daniel. Then since they get the same money if all their eggs are sold, the unit price of eggs sold by Paul is $ax/b$. And obviously the following equations holds:
  \begin{align*}
    \begin{cases}
      ax/b\cdot a &=18\\
      x\cdot b    &= 8
    \end{cases} \overset{\text{Divide 2 equations}}{\implies}
    \left(\frac ab\right)^2=\frac{18}{8}=\left(\frac{3}{2}\right)^2
    \implies \frac ab=\frac32
  \end{align*}

  {\color{red}But, what if we're talking to primary students who're not aware of equations yet?}
\end{proof}