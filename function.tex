
\section{函数}
\label{sec:function}

\begin{definition}[对合,Involution]
  一个函数若存在反函数且其反函数为其本身,则称此函数为对合或对合函数。
\end{definition}

\begin{example}
  以下三个都是对合函数的例子:
  \begin{align*}
    f(x)=x,\quad g(x)=\frac1x,\quad h(x)=\frac{x}{x-1}
  \end{align*}
\end{example}

\begin{example}
  存在无限个对合函数。
\end{example}
\begin{proof}
  显然,一个函数是对合函数$\iff$其在笛卡尔坐标下的曲线关于直线$y=x$对称,这样的曲线是可以无限地构造出来。比如任意非零实数$a$,$f(x)=\frac{a}{x}$都是对合函数。或者,利用下面的定理构造。
\end{proof}

\begin{theorem}
  若$f$是对合函数,$g$有反函数$\bar g$,则$h\equiv g\circ f\circ\bar g$是对合函数。
\end{theorem}
\begin{proof}
  任意定义域内$x$,有
  \begin{align*}
    h(x)&= g(f(\bar g(x)))\\
    h(h(x))&=g(f(\underline{\bar g( g}(f(\bar g(x))) ))) = g(\underline{f(f}(\bar g(x)))) = g(\bar g(x)) = x \qedhere
  \end{align*}
\end{proof}
