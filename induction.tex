
\chapter{归纳}
\label{chap:induction}

\begin{definition}
  一个命题,若满足以下条件:
  \begin{enumerate}
  \item 存在非负整数$k_0$,命题在$n=k_0$时成立;
  \item 假设命题在$n\le k$时成立可以推出命题在$n=k+1$时成立。
  \end{enumerate}
  则该命题对任意整数$n\ge k_0$都成立。此证明方法称为数学归纳法。
\end{definition}

\begin{example}[求和]
  \begin{align*}
    \frac1{1\times2}+\frac1{2\times3}+\frac1{3\times4}+\cdots+\frac1{n\times (n+1)}
  \end{align*}
\end{example}

可以利用拆项,由$\dfrac1{k\times (k+1)}=\dfrac1k-\dfrac1{k+1}$,从而有
\begin{align*}
  &\frac1{1\times2}+\frac1{2\times3}+\frac1{3\times4}+\cdots+\frac1{(n-1)\times n}\\
  =&\left(\frac11-\frac13\right) + \left(\frac13-\frac14\right) + \cdots + \left(\frac1n-\frac1{n+1}\right)\\
  =&\frac11-\frac1{n+1}\\
  =&\frac{n}{n+1}
\end{align*}

若观察不到拆项的规律,可以考虑一下数学归纳法。记当$n=k$时的和为$S_k$,则
\begin{enumerate}
\item 当$n=1$时,有$S_1=\dfrac12$;
\item 当$n=2$时,有$S_2=\dfrac23$;
\item 当$n=3$时,有$S_3=\dfrac34$;
\item $\cdots$
\end{enumerate}
猜测,$\forall n$,有$S_n=\dfrac{n}{n+1}$。
\begin{proof}
  当$n=1$时,显然成立。

  设当$n\le k$时成立,考虑$n=k+1$的情况。
  \begin{align*}
    S_{k+1}&=S_k + \frac1{(k+1)\times(k+2)}\\
           &=\frac{k}{k+1}+\frac1{(k+1)\times(k+2)}\\
           &=\frac{k(k+2)+1}{(k+1)(k+2)}\\
           &=\frac{k^2+2k+1}{(k+1)(k+2)}\\
           &=\frac{k+1}{k+2}
  \end{align*}
  从而对任意整数$n\ge1$,猜测成立。
\end{proof}


