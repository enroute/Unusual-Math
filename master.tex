\documentclass[a4paper,12pt,book]{memoir}

% remove margin notes on book class of memoir
% \setmarginnotes not work
\setlrmarginsandblock{3.5cm}{3.5cm}{*}
\setulmarginsandblock{3.5cm}{*}{1}
\checkandfixthelayout 

%% To fix a bug in TeXLive 2018, ref.
%% https://tex.stackexchange.com/questions/427451/dot-between-chapter-number-and-figure-number-disapear-after-update
%% https://tex.stackexchange.com/questions/427995/memoir-on-tex-live-2018-breaks-figure-numbering-after-mainmatter
\makeatletter
\renewcommand{\counterwithin}{\@ifstar{\@csinstar}{\@csin}}
\makeatother

% comment this if don't want to show answer
\def\withanswer{1}

% Make different footmark reference the same footnote
\makeatletter
\newcommand\footnoteref[1]{\protected@xdef\@thefnmark{\ref{#1}}\@footnotemark}
\makeatother

% Turn on subsection numbering in memoir class
\setsecnumdepth{subsection}


\usepackage{metalogo}           %for \XeTeX, \XeLaTeX, \LuaTeX and \LuaLaTeX

%%% \DeclareUnicodeCharacter not compatible with xelatex
% \usepackage[utf8]{inputenc}
% \DeclareUnicodeCharacter{1F019}{\yibing}
% \DeclareUnicodeCharacter{1F01A}{\erbing}
% \DeclareUnicodeCharacter{1F01B}{\sanbing}
% \newcommand{\yibing}{\symbol{"1F019}}
% \newcommand{\yibing}{\fontfamily{Symbola}\selectfont\DeclareUnicodeCharacter{1F019}{kk}}

% Ref. Mahjong tiles in unicode: http://www.unicode.org/charts/PDF/U1F000.pdf
% Symbola font is one of those fonts that implements the mahjong range in unicode
\newcommand{\mahjongsymbol}[1]{\begingroup\Huge\setmainfont{Symbola}%
  \rule[-.3\baselineskip]{0pt}{\baselineskip}%strut to ensure the height
  \symbol{"#1}\endgroup}
\newcommand{\mjbing}[1]{
  \ifthenelse{#1=1}{\mahjongsymbol{1F019}}{% 1 bing
    \ifthenelse{#1=2}{\mahjongsymbol{1F01A}}{% 2 bing
      \ifthenelse{#1=3}{\mahjongsymbol{1F01B}}{% 3 bing
        \ifthenelse{#1=4}{\mahjongsymbol{1F01C}}{% 4 bing
          \ifthenelse{#1=5}{\mahjongsymbol{1F01D}}{% 5 bing
            \ifthenelse{#1=6}{\mahjongsymbol{1F01E}}{% 6 bing
              \ifthenelse{#1=7}{\mahjongsymbol{1F01F}}{% 7 bing
                \ifthenelse{#1=8}{\mahjongsymbol{1F020}}{% 8 bing
                  \ifthenelse{#1=9}{\mahjongsymbol{1F021}}{% 9 bing
                  }}}}}}}}}}
  
\newcommand{\yibing}{\mjbing1}
\newcommand{\erbing}{\mjbing2}
\newcommand{\sanbing}{\mjbing3}
\newcommand{\sibing}{\mjbing4}
\newcommand{\wubing}{\mjbing5}
% \newcommand{\yibing}{\mahjongsymbol{1F019}}
% \newcommand{\erbing}{\mahjongsymbol{1F01A}}
% \newcommand{\sanbing}{\mahjongsymbol{1F01B}}
% \newcommand{\sibing}{\mahjongsymbol{1F01C}}
% \newcommand{\wubing}{\mahjongsymbol{1F01D}}

% \newcommand{\yibing}{\begingroup\Huge\setmainfont{Symbola}\symbol{"1F019}\endgroup}
% \newcommand{\erbing}{\begingroup\Huge\setmainfont{Symbola}\symbol{"1F01A}\endgroup}
% \newcommand{\sanbing}{\begingroup\Huge\setmainfont{Symbola}\symbol{"1F01B}\endgroup}
% \newcommand{\sibing}{\begingroup\Huge\setmainfont{Symbola}\symbol{"1F01C}\endgroup}
% \newcommand{\wubing}{\begingroup\Huge\setmainfont{Symbola}\symbol{"1F01D}\endgroup}

%% subfigure is obsolete and conilict with tocloft package, use subcaption or subfig insted
% \usepackage{subfigure}
% \usepackage[subfigure]{tocloft}
% \PassOptionsToPackage{subfigure}{tocloft}
\usepackage{subcaption}

% toc style
\usepackage{titletoc}% http://ctan.org/pkg/titletoc
\setcounter{tocdepth}{2}% Display up to \subsection in ToC
% \titlecontents*{section}% <section>
\titlecontents*{subsection}% <section>
  [3.8em]% <left>
  {\small}% <above-code>
  {}% <numbered-entry-format>; you could also use {\thecontentslabel. } to show the numbers
  {}% <numberless-entry-format>
  {\ \thecontentspage}% <filler-page-format>
  [,\ \ \ \ ]% <separator>
  []% <end>

% For epigraphs
\usepackage{epigraph}

% Use varwidth to make epigraph length adaptive
\usepackage{varwidth}
\renewcommand{\epigraphsize}{\normalsize}
\setlength{\epigraphwidth}{0.8\textwidth}
\renewcommand{\textflush}{flushright}
\renewcommand{\sourceflush}{flushright}
% A useful addition
\newcommand{\epitextfont}{\kai}
\newcommand{\episourcefont}{\kai}

\makeatletter
\newsavebox{\epi@textbox}
\newsavebox{\epi@sourcebox}
\newlength\epi@finalwidth
\renewcommand{\epigraph}[2]{%
  \vspace{\beforeepigraphskip}
  {\epigraphsize\begin{\epigraphflush}
   \epi@finalwidth=\z@
   \sbox\epi@textbox{%
     \varwidth{\epigraphwidth}
     \begin{\textflush}\epitextfont#1\end{\textflush}
     \endvarwidth
   }%
   \epi@finalwidth=\wd\epi@textbox
   \sbox\epi@sourcebox{%
     \varwidth{\epigraphwidth}
     \begin{\sourceflush}\episourcefont#2\end{\sourceflush}%
     \endvarwidth
   }%
   \ifdim\wd\epi@sourcebox>\epi@finalwidth 
     \epi@finalwidth=\wd\epi@sourcebox
   \fi
   \leavevmode\vbox{
     \hb@xt@\epi@finalwidth{\hfil\box\epi@textbox}
     \vskip1.75ex
     \hrule height \epigraphrule
     \vskip.75ex
     \hb@xt@\epi@finalwidth{\hfil\box\epi@sourcebox}
   }%
   \end{\epigraphflush}
   \vspace{\afterepigraphskip}}}
\makeatother

%%% check mark and x mark
\usepackage{pifont}% http://ctan.org/pkg/pifont
\newcommand{\cmark}{\ding{51}}%
% \newcommand{\xmark}{\ding{55}}%
\newcommand{\xmark}{\times}
\newcommand{\redcross}{{\color{red}\ding{53}}}

\newcommand*\numcircled[1]{\raisebox{.5pt}{\textcircled{\raisebox{-.9pt} {#1}}}}


\usepackage[xetex,
            bookmarksnumbered=true,
            bookmarksopen=true,
            colorlinks=false,
            pdfborder={0 0 1},
            citecolor=blue,
            linkcolor=red,
            anchorcolor=green,
            urlcolor=blue,
            breaklinks=true,
            naturalnames  %与algorithm2e宏包协调
            ]{hyperref}

% For Chinese fonts
% \usepackage{fontspec}
% \setmainfont{SimSun}
% \usepackage[AutoFakeBold,SlantFont]{xeCJK}
% % \usepackage[SlantFont]{xeCJK}   %AutoFakeBold=true makes CJK characters in the generated PDF can't be copied

% \setCJKmainfont{SimSun}

%%% DON'T USE AutoFakeBold, or issues will arise:
%%% https://tex.stackexchange.com/questions/266894/titlesec-autofakebold-making-pdf-figures-bold
%%% https://stackoverflow.com/questions/4422054/why-is-latex-making-my-caption-bold
%%% Use the following font setting instead:
%%% https://www.jianshu.com/p/b1751078e28e
\usepackage{xeCJK}
\setCJKmainfont[ItalicFont={楷体}, BoldFont={黑体}]{宋体}%衬线字体 缺省中文字体为
\setCJKsansfont{黑体}
\setCJKmonofont{仿宋_GB2312}%中文等宽字体


\usepackage{multicol}
\usepackage{multirow}
\usepackage{ulem}               % For \sout, \uwave
\usepackage[usenames,dvipsnames]{pstricks}
\usepackage{pstricks-add}  % For \psrotate, \psbrace
\usepackage{epsfig}
\usepackage{pst-grad} % For gradients
\usepackage{pst-plot} % For axes
\usepackage{pst-node} % For nodes
\usepackage[space]{grffile} % For spaces in paths
\usepackage{etoolbox} % For spaces in paths
\AtBeginEnvironment{quote}{\kai\small} %for quote environment
\AtBeginEnvironment{quotation}{\kai\small} %for quotation environment
\patchcmd{\quotation}{1.5em}{2em}{}{}

\usepackage{pst-eucl}

\usepackage[english]{babel}     % For enquote
\usepackage{csquotes} 

\usepackage{tabularx}
\usepackage{colortbl}
% \usepackage[table]{xcolor}      %this will load colortbl package, for the \cellcolor

\usepackage{hhline}
% from color package?
\definecolor{LightCyan}{rgb}{0.88,1,1}

\usepackage{booktabs}       % 表格,横的粗线;\specialrule{1pt}{0pt}{0pt}, not compactible with vertical line
\usepackage{boldline}       % for \hlineB, bold horizontal lines compactable with vertical line

\usepackage{tikzpeople}
\usepackage{tikz,fp}
% \usetikzlibrary{angles}         % for perpendicular symbols
\usetikzlibrary{calc,intersections,patterns,angles,quotes,shapes}           %
\usepackage{tikz-3dplot}
\usetikzlibrary{positioning}    % right=of (SHAPE)
\usetikzlibrary{3d}
\usetikzlibrary{decorations.markings} %for decorate, arrow in the middle of a line
\usetikzlibrary{decorations.text}     %for text along a path
\usetikzlibrary{through}              %for [circle through] in \node
\usetikzlibrary{arrows.meta}          %for arrow options
\usetikzlibrary{shadings}             %for shadings

%% externalizing tikz graphs
%% https://tex.stackexchange.com/questions/1460/script-to-automate-externalizing-tikz-graphics
% \usetikzlibrary{external}
% \tikzexternalize[prefix=figures/]

\newcommand{\tikzmark}[2]{\tikz[overlay,remember picture,baseline] \node [anchor=base] (#1) {$#2$};}

\newcommand{\DrawVLine}[3][black, thick, opacity=0.5]{%
  \begin{tikzpicture}[overlay,remember picture]
    \draw[shorten <=0.3ex, #1] (#2.north) -- (#3.south);
  \end{tikzpicture}
}

\newcommand{\DrawHLine}[3][black, thick, opacity=0.5]{%
  \begin{tikzpicture}[overlay,remember picture]
    \draw[shorten <=0.2em, #1] (#2.west) -- (#3.east);
  \end{tikzpicture}
}

\usepackage{tkz-euclide} % for perpendicular symbols, \tkzMarkRightAngle, \tkzDefPoint
\usetkzobj{all}

% default style
\tikzset{every picture/.style={line join=round}}

\makeatletter
\def\calcLength(#1,#2)#3{%
  \pgfpointdiff{\pgfpointanchor{#1}{center}}%
               {\pgfpointanchor{#2}{center}}%
  \pgf@xa=\pgf@x%
  \pgf@ya=\pgf@y%
  \FPeval\@temp@a{\pgfmath@tonumber{\pgf@xa}}%
  \FPeval\@temp@b{\pgfmath@tonumber{\pgf@ya}}%
  \FPeval\@temp@sum{(\@temp@a*\@temp@a+\@temp@b*\@temp@b)}%
  \FProot{\FPMathLen}{\@temp@sum}{2}%
  \FPround\FPMathLen\FPMathLen5\relax
  \global\expandafter\edef\csname #3\endcsname{\FPMathLen}
}

\def\drawArc(#1,#2,#3){%
  \pgfmathanglebetweenpoints{\pgfpointanchor{#1}{center}}{\pgfpointanchor{#2}{center}}
  \let\StartAngle\pgfmathresult
  \pgfmathanglebetweenpoints{\pgfpointanchor{#1}{center}}{\pgfpointanchor{#3}{center}}
  \let\EndAngle\pgfmathresult
  \calcLength(#1,#2){myr@dius}
  \draw (#2) arc[start angle=\StartAngle, end angle=\EndAngle, radius=\myr@dius pt];
}

% tikz command to draw arc with center
\def\centerarc[#1](#2)(#3:#4:#5)% Syntax: [draw options] (center) (initial angle:final angle:radius)
{ \draw[#1] ($(#2)+({#5*cos(#3)},{#5*sin(#3)})$) arc (#3:#4:#5); }
\makeatother

\usepackage{pgfplots}           % for polor plot
\usepgfplotslibrary{polar}
% Change to 1.16, since the following output from xelate:
% Package pgfplots notification 'compat/show suggested version=true': you might b
% enefit from \pgfplotsset{compat=1.16} (current compat level: 1.8).
% \pgfplotsset{compat=1.14}


\makeatletter % For spaces in paths
    \patchcmd\Gread@eps{\@inputcheck#1 }{\@inputcheck"#1"\relax}{}{}

    % Font of theorem subtitle
    \def\th@plain{%
      \thm@notefont{}% same as heading font
      \kai % body font
    }
    \def\th@definition{%
      \thm@notefont{}% same as heading font
      % \bfseries \kai % body font
      \bfseries \hei % body font
    }
\makeatother

\def\answer#1{
  \ifx\withanswer\undefined\else 提示:\par\nopagebreak #1\fi
}

\usepackage{mathtools,amsthm,amsfonts,amssymb,bm}
% For \notdivides which has a negating line longer than \nmid in the
% `amssymb' package and shorter than the \centernot in the `centernot'
% package.
\usepackage{mathabx}

\usepackage{extarrows}          % for xlongequal
% use `fc-list|grep -i kai' to get installed font list
\setCJKfamilyfont{kai}{KaiTi}
\setCJKmonofont{NSimSun}
\newcommand{\kai}{\CJKfamily{kai}}
\newcommand{\hei}{\CJKfamily{hei}}

%% chapter style
% \usepackage{titlesec, blindtext, color}
% \definecolor{gray75}{gray}{0.75}
% \newcommand{\hsp}{\hspace{20pt}}
% \titleformat{\chapter}[hang]{\Huge\bfseries}{\thechapter\hsp\textcolor{gray75}{|}\hsp}{0pt}{\Huge\bfseries}
\makechapterstyle{box}{
  \renewcommand*{\printchaptername}{}
  \renewcommand*{\chapnumfont}{\normalfont\sffamily\huge\bfseries}
  \renewcommand*{\printchapternum}{
    \flushright
    \begin{tikzpicture}
      \draw[fill,color=black!30] (0,0) rectangle (2cm,2cm);
      \draw[color=white] (1cm,1cm) node { \chapnumfont\thechapter };
    \end{tikzpicture}
  }
  \renewcommand*{\chaptitlefont}{\normalfont\sffamily\Huge\bfseries}
  \renewcommand*{\printchaptertitle}[1]{\flushright\chaptitlefont##1}
}
\chapterstyle{box}
%% This won't work with the `babel' package
%% \renewcommand\chaptername{第~\thechapter~章}
%% We have to use the following fix
%% w/o using the babel package
%% \renewcommand{\figurename}{\kai 图}
%% w/ babel packge and English as language
\addto\captionsenglish{%
  % \renewcommand\chaptername{第~\thechapter~章}
  \renewcommand\chaptername{}
  \renewcommand\contentsname{目~~~~录}
  \renewcommand\tablename{表}
  \renewcommand\figurename{图}
  \renewcommand\figurename{\kai 图}  
}
% Or use the following
% \AtBeginDocument{%
%   \renewcommand\tablename{表}
% }

% % patch \newtheoremstyle to accept \newline<other tokens>
% \makeatletter
% \patchcmd{\newtheoremstyle}
%  {\def\@tempb{\newline}}
%  {\def\@tempb{\newline}\edef\@tempa{\unexpanded\expandafter{\@car#8\@nil}}}
%  {}{}
% \patchcmd{\newtheoremstyle}
%  {\def\thmheadnl{\newline}}
%  {\def\thmheadnl{#8}}
%  {}{}
% \makeatother

\newtheoremstyle{break}% name
  {15pt}%      Space above, empty = `usual value'
  {}%          Space below
  {\kai}%     Body font
  {}%         Indent amount (empty = no indent, \parindent = para indent)
  {\bfseries\hei}% Thm head font
  {.}%        Punctuation after thm head
  %{\newline\hspace*{\parindent}}% Space after thm head: \newline = linebreak
  {5pt plus 1pt minus 1pt}% Space after thm head: \newline = linebreak
  {}%         Thm head spec
\theoremstyle{break}
\newtheorem{theorem}{\hei 定理}[section]
\newtheorem{example}[theorem]{\hei 例} % Use the same counter as theorem
\newtheorem{property}[theorem]{\hei 性质} % Use the same counter as theorem
\newtheorem{question}[theorem]{\hei 题} % Use the same counter as theorem
\newtheorem{definition}[theorem]{\hei 定义} % Use the same counter as theorem
\newtheorem{corollary}{\hei 推论}[theorem]  % Reset after every theorem
\newtheorem{lemma}[theorem]{\hei 引理}  % Use the same counter as theorem

% \renewcommand*{\proofname}{证明}
\newcommand{\note}{\begingroup\bfseries\kai 注:\endgroup}
\newcommand{\think}{\begingroup\bfseries\kai 思考:\endgroup}
\newcommand{\hints}{\begingroup\bfseries\kai 提示:\endgroup}

\newcommand{\kurschak}{K\"ursch\'ak}
\newcommand{\term}[1]{\begingroup\bfseries\kai{#1}\endgroup}

% 填空的横线
%\newcommand{\blankline}{\underline{\hbox to 15mm{}}}
\newcommand{\blankline}{\rule{1.5cm}{0.15mm}}

\DeclareMathOperator{\atan}{atan}

% Change colon(:) in figure caption to period(.)
\usepackage{caption}
\captionsetup{labelsep=period}

\newcommand{\norm}[1]{\left\lVert{#1}\right\rVert}
\renewcommand*{\vec}[1]{\mathbf{#1}}

\newcommand*{\ihat}{\hat{\textbf\i}}
\newcommand*{\jhat}{\hat{\textbf\j}}
\newcommand*{\khat}{\hat{\textbf k}}

\renewcommand*{\le}{\leqslant}
\renewcommand*{\ge}{\geqslant}
\renewcommand*\labelenumi{(\theenumi)}

% Make proof environemtn skip the first line
\makeatletter
\renewenvironment{proof}[1][证明]{\par
  % \vspace{-\topsep}% remove the space after the theorem, HOWEVER A PROOF ENVIRONMENT NOT PRECEDED BY A THEOREM WILL BE TYPESET WRONGLY.
  \pushQED{\qed}%
  \normalfont %\topsep6\p@\@plus6\p@\relax
  \topsep0pt \partopsep0pt % no space before
  \trivlist
  \item[\hskip\labelsep
    \bfseries \kai%\itshape
    #1\@addpunct{.}]%\mbox{}\\*% new line after \proofname
}{%
  \popQED\endtrivlist\@endpefalse
}
\makeatother

% \newcommand*{\max}[1]{\begingroup\mathrm{max}\left(#1\right)\endgroup}
% \newcommand*{\min}[1]{\mathrm{min}\left(#1\right)}

\usepackage{enumitem}
% \setlist{nolistsep}
% \setitemize{labelindent=\parindent,leftmargin=*,align=left,itemindent=2em,noitemsep}%,leftmargin=*,topsep=2pt,partopsep=0pt}
\setitemize{labelindent=\parindent,leftmargin=*,itemsep=0em,partopsep=0em,parsep=0em,topsep=0em,itemindent=3.75em,align=left}
\setenumerate{labelindent=\parindent,leftmargin=*,itemsep=0em,partopsep=0em,parsep=0em,topsep=0em,itemindent=3.75em,align=left}

\setitemize[2]{labelindent=\parindent+2em,leftmargin=*,itemsep=0em,partopsep=0em,parsep=0em,topsep=0em,itemindent=5.75em,align=left}
\setenumerate[2]{labelindent=\parindent+2em,leftmargin=*,itemsep=0em,partopsep=0em,parsep=0em,topsep=0em,itemindent=5.75em,align=left}


\renewcommand{\baselinestretch}{1.1}
\renewcommand*{\arraystretch}{1.5}

% make \arraystrech in pmatrix/vmatrix/bmatrix in amsmath default to .9
\makeatletter
\renewcommand*\env@matrix[1][.9]{%
  \edef\arraystretch{#1}%
  \hskip -\arraycolsep
  \let\@ifnextchar\new@ifnextchar
  \array{*\c@MaxMatrixCols c}}
\makeatother

\usepackage{graphicx}%\resizebox
\graphicspath{{images/}} %Setting the graphicspath

\usepackage[makeroom]{cancel} % for xcancel/cancel/bcancel etc.

\usepackage{fancyvrb}           % for BVerbatim environment

\usepackage{verbatim}
% Center verbatim environment, with the help from verbatim package
\makeatletter
\def\verbatim@font{\normalfont\ttfamily\kai\Large
  \hyphenchar\font\m@ne
  \@noligs}
\makeatother
\newenvironment{pcontent}{%
  \par
  \centering
  \varwidth{\linewidth}%
  \verbatim
}{%
  \endverbatim
  \endvarwidth
  \par\vskip10pt
}

\newcommand{\ptitle}[1]{\par\vskip20pt{\centering\bfseries\kai\Large{#1}\par}\nopagebreak\vskip5pt\nopagebreak}
\newcommand{\pauthor}[1]{\par{\centering\kai【{#1}】\par}\nopagebreak\vskip10pt}
\newcommand{\premark}[1]{\par\kai\small【注释】{#1}\par\vskip5pt}
\newcommand{\ppreface}[1]{{\setlength{\parindent}{2em}\small\kai{#1}}\par\vskip5pt}

% \par indent
\usepackage{indentfirst}
% parindent to 2 characters
\setlength{\parindent}{2em}
\setlength{\parskip}{3pt}

\usepackage{ifthen}             % for \ifelsethen, \whiledo
% \ifthenelse{CONDITION}{POSITIVE}{NEGATIVE}
%     2 is an \ifthenelse{\isodd{2}}{ODD}{EVEN} number.
% \whiledo{CONDITION}{LOOP CONSTRUCT}
%     \newcounter{X}
%     \whiledo{\value{X}<10}{\value{X},\stepcounter{X}}

\usepackage{units}              % for \nicefrac command, a slashed fraction

\usepackage{harpoon}            % for \overrightharp

% 3 counters for loops
\newcounter{X}
\newcounter{Y}
\newcounter{Z}

% \include{myconfig}

% 选择题
\usepackage{multiplechoice}

\usepackage[chapter]{algorithm}          % the float environment
\usepackage{algorithmic} % algorithmic (NOT algorithmicx) package is suggested for IEEE journals
\floatname{algorithm}{算法}
\renewcommand{\algorithmicrequire}{\textbf{Input:}}
\renewcommand{\algorithmicensure}{\textbf{Output:}}

\newcommand{\dx}{\,\mathrm{d}x}
\newcommand{\dy}{\,\mathrm{d}y}
\newcommand{\dt}{\,\mathrm{d}t}

% \newcommand{\dz}{\mathrm{d}z}
% \newcommand{\gcd}{\mathrm{gcd}}
\newcommand{\lcm}{\mathrm{lcm}}

\newcommand{\trianglewithfournumbers}[4]{
  \coordinate(A) at (0,0);
  \coordinate(C) at (60:2);
  \coordinate(B) at (2,0);
  \coordinate(E) at (60:1);
  \coordinate(F) at (1,0);
  \coordinate(D) at ($(E)+(1,0)$);
  \draw(A)--(B)--(C)--cycle;
  \draw(D)--(E)--(F)--cycle;
  \node at ($1/3*(C)+1/3*(E)+1/3*(D)$) {#1};
  \node at ($1/3*(A)+1/3*(E)+1/3*(F)$) {#2};
  \node at ($1/3*(B)+1/3*(F)+1/3*(D)$) {#3};
  \node at ($1/3*(F)+1/3*(E)+1/3*(D)$) {#4};
}

%%%% 易经符号
% \newCJKfontfamily{\symbolazh}{Symbola} % this font has the required I Ching glyphs
\xeCJKsetcharclass{"4DC0}{"4DFF}{0} % set Yijing Hexagrams Symbols as non-CJK characters, otherwise one has to use the \symbolazh command instead
\newfontface{\symbola}{Symbola}

% 两仪
\newcommand{\yangyao}{{\symbola\char\numexpr"268A}}
\newcommand{\yinyao}{{\symbola\char\numexpr"268B}}
% 四象
\newcommand{\taiyang}{{\symbola\char\numexpr"268C}}
\newcommand{\shaoyin}{{\symbola\char\numexpr"268D}}
\newcommand{\shaoyang}{{\symbola\char\numexpr"268E}}
\newcommand{\taiyin}{{\symbola\char\numexpr"268F}}
% 八卦 0-7
\newcommand{\trigram}[1]{{\symbola\char\numexpr"2630+#1}}
% 六十四卦 0-63
\newcommand{\iching}[1]{{\symbola\char\numexpr"4DC0+#1}}



% comment this when compile the whole document
% \includeonly{basic-operation}
% \includeonly{basic-inequalities}
% \includeonly{proofs-without-words}
% \includeonly{golden-ratio}
% \includeonly{area}
% \includeonly{five-models}
% \includeonly{logic}
% \includeonly{geometry}
% \includeonly{graphs-of-functions}
% \includeonly{proofs-without-words}
% \includeonly{board-coverage}
% \includeonly{graphs}
% \includeonly{traps}
% \includeonly{intuition}
% \includeonly{introduction}
% \includeonly{monotonic-functions}
% \includeonly{algorithm}
% \includeonly{magic-square}
% \includeonly{geometric-inequality}
% \includeonly{number-theory}
% \includeonly{continued-fraction}
% \includeonly{calendar}
% \includeonly{diophantine-equation}
% \includeonly{methods}
% \includeonly{numerical-method}
% \includeonly{identities}
% \includeonly{nim-game}
% \includeonly{cover}
% \includeonly{infinite-series}
% \includeonly{tricks}
% \includeonly{game}
% \includeonly{pattern}
% \includeonly{insight}
% \includeonly{inequality}
% \includeonly{trigonometric-functions}
% \includeonly{set}
% \includeonly{area}
% \includeonly{discrete}
% \includeonly{fun}
% \includeonly{dp}
% \includeonly{matrix}
% \includeonly{color}
% \includeonly{fractal}
% \includeonly{genius-solution-from-the-book}
% \includeonly{genius-solution}
% \includeonly{math-by-physics}
% \includeonly{probability}
% \includeonly{circle}
% \includeonly{number-to-graph}
\includeonly{pengchengbei}
% \includeonly{charts}

% allow page break inside align environment
\allowdisplaybreaks
\begin{document}

%\hypersetup{CJKbookmarks=true}  %for \overline in section, NOT WORK!!

\frontmatter
\include{cover}

%% Remvoe the self reference of tableofcontents
\currentpdfbookmark{\contentsname}{tableofcontents}
\tableofcontents*
% \begin{KeepFromToc}
%   \tableofcontents
% \end{KeepFromToc}

\mainmatter
\include{introduction}
\include{pattern}
\include{number-theory}
\include{continued-fraction}
\include{diophantine-equation}

\chapter{历法}
\label{chap:calendar}

\epigraph{清明时节雨纷纷,路上行人欲断魂。\\借问酒家何处有,牧童遥指杏花村。}{唐·杜牧《清明》}

清明是中国从古流传至今的二十四节气之一,而公历是近代才传入中国的,为什么每年的清明不是公历4月4日就是4月5日呢?看完这节,希望你能想明白,一年中在2月28日之前的节气,每年都是固定日期的,而在2月28日之后的节气,每年都是差不多是固定日期的,最多相差一天,极少数相差两天。

\section{太阳历}
\label{sec:solar-calendars}

太阳历又称为阳历,是以地球绕太阳公转的运动周期为基础而制定的历法。

\begin{definition}[回归年,太阳年,Tropical Year,Solar Year]
  太阳年是指在地球上观察到太阳回到同一个地方所需要的时间,即地球绕太阳旋转一周所需要的时间,大约是$365.2422$天。
\end{definition}

日历年是日历中经过一年所包含的时间。历史上制定日历,除了记数之外,另外一个很重要的作用是指导农耕生产,因此自然是希望日历与太阳历越是一致越好,这样看日历就知道当天是处于什么季节。

为了记数方便,日历年都是取的整数,一般取为太阳年最接近的整数即365天。若日历年直接采用365天而不作其它修正的话,那么一年下来日历年与太阳年的误差为0.2422天,如下图。若不作修正的话,每个日历年上不同年份的同一个日期(比如1月1日),地球所在的位置就会不一样。这样的话,日历就只剩余记数的作用,而失去了指导生活生产(比如农业种植)的作用,因为在这样的日历下,有些年份的1月1日甚至会是在炎热的夏季。

置闰的方法可以修正这个误差,即通过在日历年中插入天数来调整日历年与太阳年的误差。由于$0.2422\times4\approx1$,儒略历(Julian Calendar)就采用每4年添加一日(闰日)的方式来修正,即儒略历的平年有365天,每隔4年就要添加一日(闰日),该年就是有366天的闰年。

\begin{center}
  \begin{tikzpicture}[scale=1.0]
    \begin{scope}
      % the sun
      \draw(0,0)node{\tiny Sun}circle(.3);
      \foreach \t in{0,30,60,90,120,150,180,210,240,270,300,330}{%
        \draw(\t:.5)--(\t:.7);
      }
      \draw(0,0)circle(2);
      \draw[fill=white](0:2)node[left]{\tiny Earth}circle(.1);
      \draw[-{>[scale=2.5,length=2,width=3]}](0:2.4)arc(0:45:2.4);
      \node[below]at(0,-2.5){开始位置};
    \end{scope}
    \begin{scope}[shift={(6,0)}]
      % the sun
      \draw(0,0)circle(.3);
      \foreach \t in{0,30,60,90,120,150,180,210,240,270,300,330}{%
        \draw(\t:.5)--(\t:.7);
      }
      \draw(0,0)circle(2);
      \draw[dashed,help lines,fill=white](0:2)circle(.1);
      \draw[fill=white](350:2)circle(.1);
      \draw[dashed,help lines,-{>[scale=2.5,length=2,width=3]}](0:2.4)arc(0:350:2.4);
      \draw[dashed,help lines,<->](350:2.8)arc(350:360:2.4);
      \draw[dashed,help lines](3,0)--(0,0)--(350:3);
      \path (355:2)--(355:3)node[sloped,right]{0.2422天};
      \node[below]at(0,-2.5){一个日历年之后};
    \end{scope}
  \end{tikzpicture}
\end{center}

按儒略历每4年一闰日,日历上过了4年为$1+365\times4$天,而实际的4个太阳年共有$365.2422\times4$天,4年下来其误差为$1+365\times4-365.2422\times4=1-0.2422\times4=0.0312$天,再扩大一下,日历上过了400年,则与实际的400个太阳年误差为3.12天,即比太阳年多跑了3.12天。为解决这个误差,格里历(Gregorian Calendar),即现在通用的公历,在每400年中去掉3个闰年。格里历的置闰规则如下:
\begin{quotation}
  四年一闰,百年不闰,四百年再闰。
\end{quotation}

剩余的每400年产生的$3.12-3=0.12$天的误差,就需要后人在用到时继续修正了。

\begin{definition}[闰年,Leap Year]
  2月份有28天的称为平年,2月份有29天的称为闰年。确定平年闰年的方法是:
  \begin{enumerate}
  \item 能被400整除的年份定为闰年;
  \item 否则能被100整除的年份定为平年;
  \item 否则能被4整除的年份定为闰年;
  \item 其余年份定为平年。
  \end{enumerate}
  即不能被4整除的都是平年;能被400整除的是闰年;不能被400整除但能被100整除的是平年;不能被100整除但能被4整除的是闰年。
\end{definition}

闰年置闰日的方法是在当年2月份增加一天,从原来的28天加到29天。

\begin{table}[htbp]
  \centering
  \caption{每月天数}
  \label{tab:days-of-months}
  \begin{minipage}{\textwidth}  %for footnote in tabular
  \centering
  \begin{tabular}{ccccccccccccc}
    \toprule
    月份 & 1  & 2          & 3  & 4  & 5  & 6  & 7  & 8  & 9  & 10 & 11 & 12\\\midrule
    天数 & 31 & 28\footnote{平年28天,闰年29天} & 31 & 30 & 31 & 30 & 31 & 31 & 30 & 31 & 30 & 31\\
    \bottomrule
  \end{tabular}
  \end{minipage}
\end{table}

\begin{example}
  公元1900年是平年,因为$400\notdivides 1900$,但$100\mid 1900$。公元2012年是闰年,因为$4\mid 2012$但$100\notdivides 2012$。
\end{example}

\begin{example}
  小明已经上小学4年级了,可是他却说:“我妈妈说我每个生日都没有落下,可是我到现在为止总共才给我过了3个生日。”小明说的有可能吗?
\end{example}
\begin{proof}[提示]
  比如闰年闰日过生日。
\end{proof}

\begin{example}[10年3650天]
  陈奕迅有首歌叫《十年》,吕珊有首歌叫《3650夜》。那么,十年到底有可能是几天?
\end{example}
\begin{proof}[提示]
  普通情况下,四年一闰。考虑到百年不闰的情况,十年有可能有一个闰年、两个闰年或三个闰年。
  {\small
  \begin{align*}
    \underline{2008},\ 2009,\ 2010,\ 2011,\ \underline{2012},\ 2013,\ 2014,\ 2015,\ \underline{2016},\ 2017,\ 2018,\ 2019,\ \underline{2020}
  \end{align*}
  }
  从2008--2017这十年包含了3个闰年(下划线部分),从2009--2018这十年包含了两个闰年。考虑百年不闰的1900年,以其为中心向两边扩散,则十年的范围有可能只包含一个闰年。
  {
  \begin{align*}
    1897,\ 1898,\ 1899,\ 1900,\ 1901,\ 1902,\ 1903,\ \underline{1904},\ 1905,\ 1906&\qedhere
  \end{align*}
  }
\end{proof}

\begin{example}
  小明是2013年入学的小学生,今年全家去旅行,过了三天后回家。到家后小明一连撕掉了3张日历。姨妈打电话过来问起小明什么时候去的旅行,小明说不记得了,只记得刚刚撕掉的3张日历的数字和是32。那么你知道小明旅行是在哪一年吗?
\end{example}
\begin{proof}[提示]
  撕掉3张日历,暗示了这3张正好对应小明旅行的3天日期。两种情况,一是3天都是同一月份,二是3天在不同的月份。

  如果3天是同一月份,则连续3天的日期之和应能被3整除,然而32并不能被3整除,排除这种情况。

  3天不在同一月份,则必须有一天是上月月末,一天是当月第一天即1号,另外一天可能是当月2号,也可能是上月倒数第二天。如果是上月月末一天外加当月1号2号两天,则上月月末日期数字为$32-1-2=29$。如果是当月1号外加上月最后两天,则上月最后两天的日期数字之和为$32-1=31$,不可能,因为每月最后两天日期数字之和最小的也有$27+28>31$。所以被撕掉的3天日期的唯一情况是$29,1,2$。

  从而当年是闰年。从2013年开始的闰年按顺序依次为:2016,2020,2024,$\cdots\cdots$。而小明是2013年上的小学,题目中说明全家旅行时小明仍然是小学生。如果是2020年的2月份去的旅行,那么小明在旅行时应该已经是初中二年级上学期了,所以题目中说的今年只能是2016年。
\end{proof}

\begin{example}
  柳如是在给岳怜花表演魔术。柳如是请岳怜花在一张纸上写上她的生日的月份,将这个数字乘以4,加13,然后再乘以25。之后减200,然后再加上她生日的日期(即生日年月日里的“日”),然后乘以2,再减40,再乘以50,然后再加上她出生年份的后两位数。柳如是请岳怜花将最后的结果说出来,岳怜花说是71512。柳如是想了一下,说:“那我知道你是什么时候出生的了,2012年6月10日。”岳怜花说:“太神奇了!可是你早就知道我的生日了呀。”
\end{example}
\begin{proof}[提示]
  这个问题其实与日历关系不大,只是个普通的数字游戏。记岳怜花的出生年(后两位)月日分别是$y$,$m$和$d$,最终结果为$s$,则
  \begin{align*}
    s ={}& \left\{ \left[ (4m + 13)\times 25 - 200 + d \right]\times 2 - 40 \right\} \times 50 + y \\
      ={}&       \left[ (100m + 325 - 200 + d) \times 2 - 40 \right] \times 50 + y\\
      ={}& (200m + 210 + 2d) \times 50 + y\\
      ={}& 10000m + 10500 + 100d + y
  \end{align*}
  所以$(s - 10500$这个数字,万位以上的数字就是$m$(月份),千位与百倍两数字组成的就是$d$(日),最后两位就是年。

  回到柳如是的魔术,$71512 - 1050 = 61012$,最后面两位$12$是年,紧接着两位$10$是日,剩余的$6$就是月。

  从后往前反过来推,也可以设计不同的数字魔法。比如若化简到最后的结果为
  \begin{align*}
    s = 10000y + 100m + d + 1234
  \end{align*}
  那么最后的结果减去1234之后,最后两位就是日,紧接着两位就是月,剩余最前面的就是年。
\end{proof}



\section{太阴历}
\label{sec:tai-yin-calendar}

太阴历(即常说的阴历,Lunar Calendar),是指按月亮的月相周期来安排的历法。太阴历的一年有12个朔望月,约354或355天。根据月亮绕地球运行一周定为一个月,称为朔望月(如表~\ref{fig:lunar-phases}),大约是29.53天,分为大月30日,小月29日。

\begin{figure}[htbp]
  \centering
  \scalebox{0.75}{\input{images/lunar-phases.tex}}
  \caption{月相}
  \label{fig:lunar-phases}
\end{figure}

农历是东亚地区传统广泛使用的阴阳历,据说是由黄帝所创,亦有说是夏朝时他创,所以也称为黄历、夏历。后来以西历格里历为公历,夏历就改称为了农历。西历通常又称为阳历,对应的农历就常被称为阴历。在农耕活动中,阳历更能反映春天播种、秋天收割的农业周期。然而由于古代历法是阴历用的是月相周期,所以古人就用了符合阳历规律的节气来指导农耕。即现在所说的阴历(农历)在天文学中实际上是一种阴阳合历。


\section{大明历}
\label{sec:da-ming-calendar}

南北朝时期的祖冲之在公元465(也有人说是公元462年,即刘宋大明六年,而非明朝)制定了大明历,亦称为甲子元历。祖冲之在此之前就测得了地球围绕太阳旋转一周的天数大约为$365\dfrac{9589}{39491}$,即约365.2428148天;一个朔望月为$29\dfrac{2090}{3939}$,即约29.5309天\footnote{与现代测得的朔望月长度相差不到1秒}。

中国古代历法中,冬至点是制订历法的起算点,然而在祖冲之之前,一直认为冬至点的位置是固定不变的。大明历首次采用了岁差的概念,这是中国历法上的第二次大变革。

\begin{definition}[岁差,Axial Precession\footnote{字面意思为自转轴进动。}]
  天文学中,一个天体由于其重力作用导致的自转轴指向在空间中缓慢且连续的变化称为岁差。
\end{definition}

%由于该数不是正整数,古人制定历法,很重要的一件事情就是对历法调整安排,使得制定的历法与季节的循环相匹配。

\section{节气}
\label{sec:jie-qi}

太阴国描述的是月相,与太阳历相差甚远。而古代中国是农耕社会,为了指导农耕,历算家们又制定了与太阳历吻合的节气。

\begin{quotation}
节气歌

春雨惊春清谷天,夏满芒夏暑相连,

秋处露秋寒霜降,冬雪雪冬小大寒。
\end{quotation}

古人把$360^\circ$黄经划分24等分,每隔$15^\circ$为一节气,其中$0^\circ$为春分,夏至是$90^\circ$,秋分是$180^\circ$,冬至是$270^\circ$。而太阳黄经在$15^\circ$时则定为清明,太阳直射北回归线之日即中国北方最热之时定为夏至,如图~\ref{fig:24-jie-qi}所示。

公历(即格里历)是根据太阳年来确定,其日期是与黄经基本是一一对应的,误差在一日之内(闰日),所以黄经每个角度对应的公历日期基本是固定的,所以节气在公历上的体现也基本是固定的,误差也在一日之内。

\begin{figure}[htbp]
  \centering
  \begin{tikzpicture}[scale=1.0]
    \draw(0,0)circle(5);
    \foreach \a/\v in {0/春分,15/清明,30/谷雨,45/立夏,60/小满,75/芒种,90/夏至,
      105/小暑,120/大暑,135/立秋,150/处暑,165/白露,180/秋分,
      195/寒露,210/霜降,225/立冬,240/小雪,255/大雪,270/冬至,
      285/小寒,300/大寒,315/立春,330/雨水,345/惊蛰
    }{
      \draw(0,0)--(90-\a:5.2)node[pos=1.1]{\v};
    }
    \foreach \a/\r/\v in {0/4.5/$0^\circ$, 90/4.3/$90^\circ$, 180/4.5/$180^\circ$, 270/4.2/$270^\circ$}{
      \draw[very thick](0,0)--(90-\a:5.2);
      \node[fill=white] at(90-\a:\r) {\v};
    }
    \foreach \a/\r/\v in {45/4.3/$45^\circ$, 135/4.3/$135^\circ$, 225/4.2/$225^\circ$, 315/4.2/$315^\circ$}{
      % \draw[very thick](0,0)--(90-\a:5.2);
      \node[fill=white] at(90-\a:\r) {\v};
    }
  \end{tikzpicture}
  \caption{二十四节气}
  \label{fig:24-jie-qi}
\end{figure}

节气是古人用于指导工作的,如春耕、播种等,与太阳与地球的相对位置密切相关的。节气实际上是一种阳历。即中国古代所使用的历法既包括阴历,也包括阳历(节气),所以实际上是一种阴阳历。



\section{古罗马日历}
\label{sec:Rome-calender}

学习英语的时候,通常都会有人提出疑问,按照英语的词根,sept是7,oct是8,
nov是9,dec是10,但为什么 September, October, November 和 December 不是
7、8、9、10月,而是9、10、11、12月?

这里面的说法五花八门,谁是谁非,这里不具体考究,此处记录只图一乐。其中
有一种说法,是当时的罗马日历,只有10个月,后来凯撒大帝(Julius Caesar)在
他生日的7月插入了以他名字Julius命名的月份,传到英语就成了现在的July。后
来,凯撒大帝的继任者,他的甥孙屋大维,为了和凯撒齐名,就选了8月,在
July之后插入了August这个月。为什么叫August呢,是因为罗马元老院在8月授予
屋大维Augustus的尊号。因为插入了这两个月,所以原来的 September,
October, November 和 December 就变成了现在的9、10、11和12月了。

还有一种说法,是古罗马日历只有10个月,后来在年尾加了2个月,然后再后来,
又把这新加的两个月移到了年初变成了新的1月和2月,原来的10个月都往后顺移
了两个月,所以原来的September, October, November 和 December 自然就变成
了9、10、11和12月了。凯撒和屋大维只是把7、8两月改了个名字。


% \begin{table}[htbp]
%   \centering
%   \caption{10 Months of Julius Calendar}
%   \label{tab:10-months-of-julius-calendar}
%   \begin{minipage}{\textwidth}  %for footnote in tabular
%   \centering
%   \begin{tabular}{llll}
%     \toprule
%     月份 & English           & Latin       & Meaning                                                                   \\\midrule
%     1    & January           & Ianuarius   & Month of Janus                                                            \\
%     2    & February          & Februarius  & Month of the Februa                                                       \\
%          & Intercalary Month & Mercedonius & Month of Wages                                                            \\           
%     3    & March             & Martius     & Month of Mars                                                             \\
%     4    & April             & Aprilis     & Month of Aphrodite – from which the Etruscan Apru might have been derived \\
%     5    & May               & Maius       & Month of Maia                                                             \\
%     6    & June              & Iunius      & Month of Juno                                                             \\
%     7    & July              & Quintilis   & Fifth Month (from the earlier calendar starting in March)                 \\
%     8    & August            & Sextilis    & Sixth Month                                                               \\
%     9    & September         & September   & Seventh Month                                                             \\
%     10   & October           & October     & Eighth Month                                                              \\
%     11   & November          & November    & Ninth Month                                                               \\
%     12   & December          & December    & Tenth Month                                                               \\
%     \bottomrule
%   \end{tabular}
%   \end{minipage}
% \end{table}
%% The first month in the Julian calendar was Martius for March, followed by Aprilis for April, Maius for May, Lunius for June, Quintilis or July, Sextilis or August, then September for the seventh month, October for the 8th, November for the 9th and December for the 10th month.

\subsection{First Roman Calendar}
\label{sec:first-roman-calendar}
第一个罗马历是太阴历,是基于月亮形状的周期变化而来的。由于新月出现的平均周期是29.5天,所以罗马太阴历中每个月是29天或者30天,一年总共是10个月共304天。这十个月的名字见表~\ref{tab:10-months-of-roman-calendar}。从December到来年的March之间的冬天日子,古罗马人是不算在日历当中的。

从表~\ref{tab:10-months-of-roman-calendar}也可以看出,前4个月是以神命名,后6个月则是以数字命名。

% The first Roman calendar was a lunar calendar, based on the Greek lunar calendars where months begin and end when new moons occur. Because the time between new moons averages 29.5 days, the Roman lunar calendar had either 29 or 30 days. It had 304 days subdivided into 10 months starting from March and ending with December (from the Latin word decem or ten in Latin), while no months were assigned to the winter days between December and March.

\begin{table}[htbp]
  \centering
  \caption{10个月的罗马历}
  \label{tab:10-months-of-roman-calendar}
  \begin{minipage}{\textwidth}  %for footnote in tabular
  \centering
  \begin{tabular}{lll}
    \toprule
    月份 & 拉丁语    & 含义               \\\midrule
    1    & Martius   & Month of Mars      \\
    2    & Aprilis   & Month of Aphrodite \\
    3    & Maius     & Month of Maia      \\
    4    & Iunius    & Month of Juno      \\
    5    & Quintilis & Fifth Month        \\
    6    & Sextilis  & Sixth Month        \\
    7    & September & Seventh Month      \\
    8    & October   & Eighth Month       \\
    9    & November  & Ninth Month        \\
    10   & December  & Tenth Month        \\
    \bottomrule
  \end{tabular}
  \end{minipage}
\end{table}

\subsection{十二个月的罗马历}
\label{sec:12-months-roman-calendar}
后来,罗马人发现一年304天实在与太阳年相差太多,于是就在December后面加了两个月。再后来,又把这新加的两个月放到年初,于是罗马历就成了一年十二个月共355天,如表~\ref{tab:12-months-of-roman-calendar}。
\begin{table}[htbp]
  \centering
  \caption{12个月的罗马历}
  \label{tab:12-months-of-roman-calendar}
  \begin{minipage}{\textwidth}  %for footnote in tabular
  \centering
  \begin{tabular}{lllc}
    \toprule
    月份  & 拉丁语    & 含义              & 天数 \\\midrule
    1     & Ianuarius & Month of Janus    & 29   \\
    2     & Februarius&                   & 28   \\
    3     & Martius   & Month of Mars     & 31   \\
    4     & Aprilis   & Month of Aphrodite& 29   \\
    5     & Maius     & Month of Maia     & 31   \\
    6     & Iunius    & Month of Juno     & 29   \\
    7     & Quintilis & Fifth Month       & 31   \\
    8     & Sextilis  & Sixth Month       & 29   \\
    9     & September & Seventh Month     & 29   \\
    10    & October   & Eighth Month      & 31   \\
    11    & November  & Ninth Month       & 29   \\
    12    & December  & Tenth Month       & 29   \\
    \bottomrule
  \end{tabular}
  \end{minipage}
\end{table}

然而,这355天与太阳历还是有不小的差距,于时罗马人再次次修订历法,在必要时(通常是每隔一年,即两年插一次)在2月和3月之间插入一个有27天的叫Intercalary(也叫Mercedonius)的月,以弥补历法与太阳历之间的差距。

再后来,凯撒大帝和屋大维分别将7、8两月按自己的名字和尊号改了名字,再传到英国,就有了现在的十二个月的叫法了。

\begin{example}
  当罗马官方机构 College of Pontiffs 宣布当年要增加Intercalary月份之后,当年的2月份就要改为只有23天,或者是24天(每四年一次,对应于闰年)。罗马历实行之后,到了公元前47年,结果发现罗马日历几乎比太阳相差了差不多两个半月。

  为什么呢?

  因为按照这个历法,将4年里日历总天数除以4,可以得到平均每年的天数为
  \begin{align*}
    & \left(4 \times 355 + 27 \times 2(\text{两次闰月}) - 5 (\text{一次23天的二月})\right.\\
    & \left. - 4 (\text{一次24天的二月})\right) \div 4\\
    = & 366\frac14
  \end{align*}
  与真实的太阳历每年天数有一天左右的差距,所以随着多年的积累,误差就越来越大了。
\end{example}

\begin{question}
  按照上题的描述,可以估算出来罗马大概是什么时候开始实行这个历法吗?
\end{question}


\begin{example}
  古罗马人将每周改为7天后,是按神给每一天命名的。传到英国后,Anglo--Saxon将其中4个改造成了自己的神,从而有了如今英语命名。
  \begin{table}[htbp]
    \centering
    \begin{tabular}{cllll}
      \toprule
      天     & 英文名    & 罗马的神         & Anglo--Saxon的神 & 其它含义 \\\midrule
      星期天 & Sunday    & 太阳神 Solis     &                  & 太阳     \\
      星期一 & Monday    & 月亮女神 Lunae   &                  & 月亮     \\
      星期二 & Tuesday   & 战神 Martis      & 战神 Tiu         & 火星     \\
      星期三 & Wednesday & 神使 Mercurii    & 主神 Woden       & 水星     \\
      星期四 & Thursday  & 天神 Iovis       & 雷神 Thor        & 木星     \\
      星期五 & Friday    & 爱神 Veneris     & 爱神 Frigg       & 金星     \\
      星期六 & Saturday  & 农业之神 Saturni &                  & 土星     \\
      \bottomrule
    \end{tabular}
    \caption{星期的命名}
    \label{tab:name-of-week-day}
  \end{table}
% Sunday (dies Solis) was the day of the god sun Sol. Monday (dies Lunae) was the day of the moon and celebrated the goddess Luna. Tuesday (dies Martis) was the day of Mars, the god of war. Wednesday (dies Mercurii) was the day of the god Mercury. Thursday (dies Iovis) was the day of the god Jupiter. Friday (dies Veneris) was the day of goddess Venus. Saturday (dies Saturni) was the day of god Saturn. 
\end{example}

\begin{example}
  古罗马人在每周的最后一天赶集,到城镇售卖他们的产品。在Julian历之前,古罗马日历中每年有355天,每周有8天,分别用$A$到$H$八个字母代替。
\end{example}


\section{天干地支}
\label{sec:gan-zhi}

天干地支是中国古代的记数方法。

\begin{definition}[天干]
  十天干是指:\nopagebreak

  \centering
  甲、乙、丙、丁、戊、己、庚、辛、壬、癸
\end{definition}

\begin{definition}[地支]
  十二地支是指:\nopagebreak

  \centering
  子、丑、寅、卯、辰、巳、午、未、申、酉、戌、亥
\end{definition}

天干和地支组合就是以「甲子」为首的六十干支循环,如表~\ref{tab:gan-zhi-loop}。这是一种循环配对,并不是天干任取一个,地支任取一个总数为$10\times12=120$种的配对。容易观察到,由于天干与地支的个数都是偶数,这种循环配对只有奇数配奇数,偶数配偶数的配法,如排第1的天干“甲”,便只能与排1、3、5、7、9、11的“子、寅、辰、午、申、戌”配对。

\begin{table}[htbp]
  \centering
  \caption{干支循环}
  \label{tab:gan-zhi-loop}
  \begin{tikzpicture}[scale=1.0]
    \setcounter{X}{0}
    \newcommand{\dizhi}{品}
    \foreach \y in{0,1,2,3,4,5}{%,6,7,8,9,10,11}{%
      \foreach \x/\tiangan in{0/甲,1/乙,2/丙,3/丁,4/戊,5/己,6/庚,7/辛,8/壬,9/癸}{%
        % \foreach \y in{0/子,1/丑,2/寅,3/卯,4/辰,5/巳,6/午,7/未,8/申,9/酉,10/戌,11/亥}{%
        % FIXME: \underline cause box raised!!
        \ifnum0=\theX      {\node at(\x,-\y*.8){\raisebox{-6pt}{\underline{\tiangan{}子}}};}
        \else\ifnum\theX=1 {\node at(\x,-\y*.8){\tiangan{}丑};}
        \else\ifnum\theX=2 {\node at(\x,-\y*.8){\tiangan{}寅};}
        \else\ifnum\theX=3 {\node at(\x,-\y*.8){\tiangan{}卯};}
        \else\ifnum\theX=4 {\node at(\x,-\y*.8){\tiangan{}辰};}
        \else\ifnum\theX=5 {\node at(\x,-\y*.8){\tiangan{}巳};}
        \else\ifnum\theX=6 {\node at(\x,-\y*.8){\tiangan{}午};}
        \else\ifnum\theX=7 {\node at(\x,-\y*.8){\tiangan{}未};}
        \else\ifnum\theX=8 {\node at(\x,-\y*.8){\tiangan{}申};}
        \else\ifnum\theX=9 {\node at(\x,-\y*.8){\tiangan{}酉};}
        \else\ifnum\theX=10{\node at(\x,-\y*.8){\tiangan{}戌};}
        \else\ifnum\theX=11{\node at(\x,-\y*.8){\tiangan{}亥};}
        \fi\fi\fi\fi\fi\fi\fi\fi\fi\fi\fi\fi
        \stepcounter{X}
        \ifnum\theX=12\setcounter{X}{0}\fi
      }
    }
  \end{tikzpicture}
\end{table}

\include{golden-ratio}

\include{basic-operation}

% 集合 -> 离散(含排列组合) -> 概率
\include{set}
\include{discrete}
\include{probability}

\include{tricks}
% \include{induction}

\chapter{三角形与三角函数}
\label{chap:triangle-and-trigonometric-functions}

\section{勾股定理}
\label{sec:pythagorean-theorem}

\begin{theorem}[勾股定理,毕达哥拉斯定理,Pythagorean Theorem]
  直角三角形中,斜边的平方等于两直角边的平方和。
\end{theorem}
\begin{proof}[提示]
  勾股定理作为平面几何上的一个非常重要的基础定理,经过多年的沉淀,人们发现了多种多样的证明方法,比如下图中的一种。
  \begin{center}
    \begin{tikzpicture}[scale=.4]
      \begin{scope}
        \coordinate(A)at(0,0);\coordinate(B)at(6,0);\coordinate(C)at(6,6);\coordinate(D)at(0,6);
        \coordinate(A')at(4,0);\coordinate(B')at(6,4);\coordinate(C')at(2,6);\coordinate(D')at(0,2);
        \fill[color=blue!10](A)--(A')--(D')--cycle
                            (B)--(B')--(A')--cycle
                            (C)--(C')--(B')--cycle
                            (D)--(D')--(C')--cycle;
        \draw(A) --(A')node[midway,below]{$a\vphantom{b}$}
                 --(B) node[midway,below]{$b$}
                 --(B')node[midway,right]{$a$}
                 --(C) node[midway,right]{$b$}
                 --(C')node[midway,above]{$a$}
                 --(D) node[midway,above]{$b$}
                 --(D')node[midway,left] {$a$}
                 --(A) node[midway,left] {$b$}
             (A')--(B')node[midway,sloped,above]{$c$}
                 --(C')node[midway,sloped,below]{$c$}
                 --(D')node[midway,sloped,below]{$c$}
                 --(A')node[midway,sloped,above]{$c$};             
      \end{scope}
      \begin{scope}[shift={(9,3.5)}]
        \node[right]at(0,0){
          \begin{minipage}{.5\linewidth}
            \begin{align*}
                        &&(a + b)^2  ={}& c^2 + 4\times \frac12ab\\
              &\implies{}& a^2 + b^2 ={}& c^2
            \end{align*}
          \end{minipage}};
      \end{scope}
    \end{tikzpicture}
  \end{center}
  将图中大正方形拆成4个小直角三角形和一个小正方形,求其面积可得。如果将上图沿边长为$c$的正方形的一条对角线切开,考虑得到的梯形的面积也能得出结论。这种切开的证明方法是第20任美国总统加菲尔德(James Abram Garfield)发现的,故也称之为“总统证法”。
\end{proof}

在西方,普遍认为这个定理是毕达哥拉斯(Pythagoras)先发现的。据说毕达哥拉斯在发现这个定理之后非常兴奋,宰了一百头牛来庆祝,所以这个定理也叫“百牛定理”。而在中国,有记载称周朝的商高发现了此定理,故在中国也称之为“商高定理”。

中国古代称直角三角形为勾股形,两直角边中较小者为勾,较长者为股,斜边为弦,所以中国人也称这个定理为勾股定理。商高曾说过“勾广三,股修四,经隅五”,此即后人所说的“勾三股四弦五”的来源。

\begin{question}
  猜猜赵爽在《九章算术》中下面描述的意思:“勾股各自乘,并之为玄实。开方除之,即玄\footnote{“玄”通“弦”。}。”
\end{question}

\begin{theorem}[倒数勾股定理,Inverse Pythagorean Theorem]
  直角三角形两直角边分别为$a$和$b$,斜边上的高为$h$,则
  \begin{center}
    \begin{tikzpicture}[scale=1.0]
      \coordinate(C)at(0,0);
      \coordinate(A)at(4,0);
      \coordinate(B)at(0,2);
      \coordinate(H)at($(A)!(C)!(B)$);
      \draw(C)--(A)node[midway,above]{$a$}
              --(B)node[midway,sloped,above]{$c$}
              --(C)node[midway,left]{$b$}
              --(H)node[midway,sloped,below]{$h$};
      \tkzMarkRightAngle(A,C,B)\tkzMarkRightAngle(C,H,A)
      \node[left]at(-2,1){$\dfrac 1{h^2} = \dfrac 1{a^2} + \dfrac 1{b^2}$};
    \end{tikzpicture}
  \end{center}
\end{theorem}
\begin{proof}
  记斜边为$c$,则
  \begin{align*}
    \frac 1{h^2} = \frac 1{a^2} + \frac 1{b^2} \iff a^2b^2 = (a^2+b^2)h^2 \iff ab = ch
  \end{align*}
  最后一个等号显然成立,因为等号两边都是直角三角形的面积的2倍。
\end{proof}

\section{三角函数}
\label{sec:trigometric-functions}

三角函数中最基本的是正弦与余弦,其余的都可是可以通过正弦与余弦给出。实际上正弦余弦在绝对值上也是可以互相给出的,只是符号(正负号)无法通过对方得出。

\begin{definition}[正弦,余弦]
  在直角坐标系中,单位圆上给定一条半径,记$x$轴到该半径的夹角为$\theta$,则
  \begin{enumerate}
  \item 半径在$x$轴方向上的有向投影\footnote{投影落在$x$正半轴为正,落在负半轴为负。}称为$\theta$的正弦值;
  \item 半径在$y$轴方向上的有向投影\footnote{投影落在$y$正半轴为正,落在负半轴为负。}称为$\theta$的余弦值。
  \end{enumerate}
\end{definition}

\begin{definition}[正切,余切,反正弦,反余弦]
  \begin{align*}
    \tan\theta\equiv{}&\frac{\sin\theta}{\cos\theta},& \cot\theta\equiv{}&\frac{\cos\theta}{\sin\theta}, &
    \csc\theta\equiv{}&\frac{1}{\sin\theta},         & \sec\theta\equiv{}&\frac{1}{\cos\theta}
  \end{align*}
\end{definition}

\subsection{基本性质}
\label{sec:basic-properties-of-trigonometric-functions}

\mbox{}\par\normalfont
% \mbox{}\par%prevent bold font if figure right after section command

\begin{figure}[htbp]
  \centering
  \begin{tikzpicture}[scale=1.0]
    \coordinate[label=below left:$O$](O)at(0,0);
    \coordinate[label=above right:$C$](C)at(40:3);\coordinate(C1)at($(C)!.5!90:(O)$);
    \coordinate(A1)at(1,0);\coordinate(B1)at(0,1);
    \tkzInterLL(C,C1)(O,A1)\tkzGetPoint{A}\node[below right] at(A){$A$};
    \tkzInterLL(C,C1)(O,B1)\tkzGetPoint{B}\node[above]at(B){$B$};
    \coordinate[label=below:$D$](D)at($(O)!(C)!(1,0)$);
    \coordinate(E)at($(O)!(C)!(B)$);
    \draw[help lines](O)circle(3);
    \draw[dashed](C)--(E)node[pos=.55,above]{$\cos\theta$};
    \draw(A)--(D)
            --(O)node[midway,below]{$\cos\theta$}
            --(B)node[midway,left]{$\csc\theta$}
            --(C)node[midway,sloped,above]{$\cot\theta$}
            --(A)node[midway,sloped,above]{$\tan\theta$}
         (O)--(C)node[midway,above,sloped]{1};
    \draw(D)--(C)node[pos=.4,sloped,above]{$\sin\theta$};
    \tkzMarkRightAngle(A,O,B)\tkzMarkRightAngle(C,D,A)\tkzMarkRightAngle(O,C,B)\tkzMarkRightAngle(C,E,O)
    \tkzDrawPoints(O,A,B,C,D)
    \draw pic["$\theta$",<->,draw=orange,angle eccentricity=1.6,angle radius=.6cm]{angle=A--O--C};
    \draw[|<->|]($(O)-(0,1)$)--($(A)-(0,1)$)node[midway,fill=white]{$\sec\theta$};
  \end{tikzpicture}
  \caption{三角函数在单位圆上的表示}
  \label{fig:trigonometric-funtion-on-unit-circle}
\end{figure}

图~\ref{fig:trigonometric-funtion-on-unit-circle}中的圆是单位圆,当$0<\theta<90^\circ$时可在图中找到各基本三角函数的值。由图,根据勾股定理容易得到到一些性质:
\begin{gather*}
  \sin^2\theta + \cos^2\theta = 1,\quad \csc^2\theta=\cot^2\theta+1,\quad \sec^2\theta=\tan^2\theta+1\\
  \sec^2\theta+\csc^2\theta = (\cot\theta+\tan\theta)^2\\
  \tan^2\theta=\sin^2\theta+(\sec\theta-\cos\theta)^2\\
  \cot^2\theta=\cos^2\theta+(\csc\theta-\sin\theta)^2
\end{gather*}

\section{余弦定理}
\label{sec:law-of-cosines}

\begin{theorem}[余弦定理,Law of Cosines]
  任意三角形,记三边边长分别是$a,b,c$,且$a$和$b$的夹角为$C$,则有
  \begin{align*}
    c^2=a^2+b^2-2ab\cos C
  \end{align*}
  \begin{center}
    \begin{tikzpicture}[scale=1.0]
      \coordinate (A) at (0,0);
      \coordinate (B) at (3,0);
      \coordinate (C) at (2, 1.5);
      \draw(A)--(B) node[midway, below]{$c$};
      \draw(B)--(C) node[midway,right]{$a$};
      \draw(C)--(A) node[midway,left]{$b$};
      \draw pic["$C$",<->,draw=orange,angle eccentricity=1.6,angle radius=.6cm]{angle=A--C--B};
    \end{tikzpicture}
  \end{center}
\end{theorem}
\begin{proof}
  余弦定理有证明方法有很多,这里有一种直观的构造方法。分$\angle C$是锐角及钝角两种情况(直角时可直接用勾股定理)。
  \begin{center}
    \begin{tikzpicture}[scale=1.0]
      \begin{scope}
        \coordinate(A) at (0,0);
        \coordinate(B) at (3,3);
        \coordinate(G) at (3,0);
        \coordinate(H) at (0,3);
        \coordinate(C) at (2.3,3.8);
        \coordinate(D) at ($(C)!1!90:(B)$);
        \coordinate(E) at ($(B)!1!-90:(C)$);
        \coordinate(F) at ($(G) + (E) - (B)$);
        % \fill[color=red!20](A)rectangle(B);
        % \fill[color=red!20](B)--(C)--(D)--(E)--cycle;
        \fill[color=red!20](B)--(C)--(H);
        \fill[pattern color=red!20,pattern=north east lines](B)--(E)--(F);
        \fill[pattern color=blue!20,pattern=north west lines](B)--(G)--(F);
        \draw(A)rectangle(B)--(E)--(D)--(C)--(B)--(F)--(E) (C)--(H) (G)--(F) ;
        \draw pic["$C$",<->,draw=orange,angle eccentricity=1.6,angle radius=.4cm]{angle=C--B--H};
        \node[below] at ($.5*(H)+.5*(B)$) {$a$};
        \node[above] at ($.5*(B)+.5*(C)$) {$b$};
        \node[above left] at ($.5*(H)+.5*(C)$) {$c$};
        \draw[dashed](B)--($(B)+(0,1)$) node(BB){};
        \draw[dashed](E)--($(BB)!(E)!(B)$);
      \end{scope}
      \begin{scope}[shift={(6,0)}]
        \coordinate(A) at (0,0);
        \coordinate(B) at (3,3);
        \coordinate(G) at (3,0);
        \coordinate(H) at (0,3);
        \coordinate(C) at (2.3,3.8);
        \coordinate(D) at ($(C)!1!90:(B)$);
        \coordinate(E) at ($(B)!1!-90:(C)$);
        \coordinate(F) at ($(G) + (E) - (B)$);

        \coordinate(U) at ($(H)!1!-90:(C)$);
        \coordinate(V) at ($(C)!1!90:(H)$);

        \fill[pattern=bricks, pattern color=orange](U)--(V)--(F)--cycle;
        \fill[pattern color=red!20,pattern=north east lines](C)--(D)--(V)--cycle;
        \fill[pattern color=blue!20,pattern=north west lines](H)--(A)--(U)--cycle;

        % \fill[color=red!20](A)rectangle(B);
        % \fill[color=red!20](B)--(C)--(D)--(E)--cycle;
        % \fill[pattern color=red!20,pattern=north west lines](B)--(C)--(H);
        % \fill[pattern color=blue!20,pattern=north east lines](B)--(G)--(F);
        % \fill[color=red!20](B)--(E)--(F);
        \draw(A)rectangle(B)--(E)--(D)--(C)--(B)--(F)--(E) (C)--(H) (G)--(F) ;
        % \draw pic["$C$",<->,draw=orange,angle eccentricity=1.6,angle radius=.4cm]{angle=C--B--H};
        \node[below] at ($.5*(H)+.5*(B)$) {$a$};
        \node[above] at ($.7*(B)+.3*(C)$) {$b$};
        \node[above left] at ($.5*(H)+.5*(C)$) {$c$};


        \draw[line width=1pt](H)--(U)--(V)--(C)--cycle;
        \draw[line width=1pt](A)--(H)--(U)--(A) (V)--(D)--(C)--(V);
        \draw[line width=1pt](A)--(G)--(F)--(U) (V)--(F)--(E)--(D);

        \draw pic["",draw=orange,angle eccentricity=1.6,angle radius=.5cm]{angle=A--H--U};
        \draw pic["",draw=orange,angle eccentricity=1.6,angle radius=.5cm]{angle=B--H--C};
        \draw pic["",draw=orange,angle eccentricity=1.6,angle radius=.55cm]{angle=A--H--U};
        \draw pic["",draw=orange,angle eccentricity=1.6,angle radius=.55cm]{angle=B--H--C};

        \draw pic["",draw=black,angle eccentricity=1.6,angle radius=.3cm]{angle=H--C--V};
        \draw pic["",draw=black,angle eccentricity=1.6,angle radius=.2cm]{angle=H--C--V};
        \draw pic["",draw=black,angle eccentricity=1.6,angle radius=.3cm]{angle=B--C--D};
        \draw pic["",draw=black,angle eccentricity=1.6,angle radius=.2cm]{angle=B--C--D};
        
        \draw pic["",<->, draw=black,angle eccentricity=1.6,angle radius=.4cm]{angle=G--A--U};
        \node[above right] at (.5,0) {$90^\circ - C$};
      \end{scope}

    \end{tikzpicture}
  \end{center}
  如左图,以三角形为基础,$a,b$两边向外做正方形,再以这两个正方形的两边作平行四边形。则带阴影的三个角形都是$a$为底,$b\cos C$为高,故其面积都为$\frac12ab\cos C$。

  对左图换一种如右图的切割方式,以$c$为边长向内作正方形,然后连接对应的顶点。则由“边角边”可知\tikz{\fill[draw,pattern=north west lines, pattern color=blue!20](0,0)--(1,0)--(.6,.4)--cycle}与\tikz{\fill[draw,pattern=north east lines, pattern color=red!20](0,0)--(1,0)--(.6,.4)--cycle}的两个三角形都与原三角形\tikz{\filldraw[draw=black,fill=red!20](0,0)--(1,0)--(.6,.4)--cycle}全等。从而白色的两个是全等的平行四边形,\tikz{\filldraw[pattern=bricks, pattern color=orange,draw=black](0,0)--(1,0)--(.6,.4)--cycle}也与原三角形全等。且白色平行四边形的一个内角为$90^\circ - C$,从而白色平行四边形的面积都是$ab\sin(90^\circ-C)=ab\cos C$。

  左右两图白色部分面积相等,从而有
  \begin{align*}
    c^2 + 2ab\cos C = a^2 + b^2
  \end{align*}

  钝角的情形则可按下图方式构造。
  \begin{center}
    \begin{tikzpicture}[scale=.9]
      \begin{scope}
        \coordinate(A) at (0,0);
        \coordinate(B) at (3,0);
        \coordinate(C) at (-2,1.8);
        \coordinate(D) at ($(B)!1!90:(A)$);
        \coordinate(E) at ($(A)!1!-90:(B)$);
        \coordinate(F) at ($(E) + (C) - (A)$);
        \coordinate(G) at ($(F)!1!-90:(E)$);
        \coordinate(H) at ($(E)!1!90:(F)$);
        \coordinate(I) at ($(H) + (B) - (A)$);
        \fill[color=blue!20](A)--(B)--(C)--cycle (C)--(F)--(G)--cycle;
        \draw(B)--(C)node[midway, above]{$c$};
        \draw(A)--(B)node[midway, below]{$b$};
        \draw(A)--(C)node[pos=.4,below left]{$a$};
        \draw pic["$C$",<->, draw=black,angle eccentricity=1.9,angle radius=.2cm]{angle=B--A--C};
        \draw(G)--(C)--(F)--(G)--(H)--(E)--(F) (A)--(E)--(D)--(B) (H)--(I)--(D);
        \node at ($.25*(A)+.25*(B)+.25*(D)+.25*(E)$) {$b^2$};
        \node at ($.25*(E)+.25*(F)+.25*(G)+.25*(H)$) {$a^2$};
        \node at ($.25*(E)+.25*(F)+.25*(A)+.25*(C)$) {$-ab\cos C$};
        \node at ($.25*(E)+.25*(D)+.25*(I)+.25*(H)$) {$-ab\cos C$};
      \end{scope}
      \begin{scope}[shift={(7.5,0)}]
        \coordinate(A) at (0,0);
        \coordinate(B) at (3,0);
        \coordinate(C) at (-2,1.8);
        \coordinate(D) at ($(B)!1!90:(A)$);
        \coordinate(E) at ($(A)!1!-90:(B)$);
        \coordinate(F) at ($(E) + (C) - (A)$);
        \coordinate(G) at ($(F)!1!-90:(E)$);
        \coordinate(H) at ($(E)!1!90:(F)$);
        \coordinate(I) at ($(H) + (B) - (A)$);
        \fill[color=blue!20](G)--(I)--(H)--cycle (I)--(D)--(B)--cycle;
        \draw[help lines](B)--(A)--(C) (G)--(F)--(C);
        \draw[help lines](H)--(E)--(F) (A)--(E)--(D)--(B);
        \draw(B)--(C)node[midway,above]{$c$} (C)--(G)--(H)--(I)--(D)--(B) (G)--(I)--(B);
        \node at ($.25*(B)+.25*(C)+.25*(G)+.25*(I)$) {$c^2$};
      \end{scope}
    \end{tikzpicture}
  \end{center}
  左图是先分别沿一边$a,b$向外作一正方形和平行四边形,然后再作另一正方形和平行四边形。右图是连接对应的顶点将图重新分割成边长为$c$的正方形与两个带阴影的三角形。左右两图空白面积相等,从而有
  \begin{align*}
    a^2+b^2-2ab\cos C=c^2&\qedhere
  \end{align*}
\end{proof}

\section{正弦定理}
\label{sec:the-law-of-sines}

\begin{theorem}[正弦定理,Law of Sines,Sine Rule]
  任意三角形,记三边边长为$a,b,$,对应的三个角为$A,B,C$,则
  \begin{align*}
    \frac{\sin A}{a} = \frac{\sin B}{b} = \frac{\sin C}{c}
  \end{align*}
\end{theorem}
\begin{proof}
  如下图,作高。
  \begin{center}
    \begin{tikzpicture}[scale=1.0]
      \begin{scope}
        \coordinate(A) at (0,0);
        \coordinate(B) at (3,0);
        \coordinate(C) at (2,1.5);
        \coordinate(D) at (2,0);
        \draw(A)--(B)node[midway,below]{$c$} --(C)node[pos=.6,right]{$a$}--(A)node[pos=.6,above left]{$b$};
        \draw[dashed](C)--(D)node[midway,left]{$h$};
        \tkzMarkRightAngle(A,D,C);
        \draw pic["$A$",<->,draw=orange,angle eccentricity=1.6,angle radius=.5cm]{angle=B--A--C};
        \draw pic["$B$",<->,draw=orange,angle eccentricity=1.6,angle radius=.3cm]{angle=C--B--A};
      \end{scope}
      \begin{scope}[shift={(6,0)}]
        \coordinate(A) at (0,0);
        \coordinate(B) at (3,0);
        \coordinate(C) at (-1,1.5);
        \coordinate(D) at (-1,0);
        \draw(A)--(B)node[midway,below]{$c$} --(C)node[pos=.4,above right]{$a$}--(A)node[pos=.6,below left]{$b$};
        \draw[dashed](A)--(D)--(C)node[midway,left]{$h$};
        \tkzMarkRightAngle(A,D,C);
        \draw pic["$A$",<->,draw=orange,angle eccentricity=1.6,angle radius=.3cm]{angle=B--A--C};
        \draw pic["$B$",<->,draw=orange,angle eccentricity=1.6,angle radius=.6cm]{angle=C--B--A};
      \end{scope}
    \end{tikzpicture}
  \end{center}
  则由正弦函数$\sin$的定义,有
  \begin{align*}
    h = b \sin A = a \sin B \quad\implies\quad \frac{\sin A}{a} = \frac{\sin B}{b}
  \end{align*}
  对于顶角是钝角的情况,由$\sin A = \sin (\pi - A)$,因此也有相同的结论。
\end{proof}


\section{角平分线}

\subsection{角平分线的长度}
\label{sec:length-of-angular-bisector}

三角形的角平分线的长度公式有好几种形式,下面定理给出的是其中比较常用的一种。

\begin{theorem}
  $AD$是$\triangle ABC$中$\angle A$的角平分线,则角平分线$AD$的长度$d$可由下面公式给出:
  \begin{align*}
    d=\frac{2 bc\cdot\cos\frac{A}{2}}{b+c}
  \end{align*}
\end{theorem}
\begin{proof}如图,考虑$\triangle ABC$的面积$S_{\triangle ABC}=S_{\triangle ABD}+S_{\triangle ACD}$,有
  \begin{center}
    \begin{tikzpicture}[scale=1.0]
      \coordinate[label=below left:$A$](A) at (0,0);
      \coordinate[label=below right:$B$](B) at (3,0);
      \coordinate[label=right:$D$](D) at (30:2.5);
      \coordinate(C') at (60:5);
      \tkzInterLL(A,C')(B,D)\tkzGetPoint{C}
      \tkzLabelPoint[above](C){$C$}
      \draw[line width=2pt](A)--(B) node[midway, below]{$c$};
      \draw[line width=2pt](A)--(C) node[midway, left]{$b$};
      \draw[line width=2pt](A)--(D) node[midway, above]{$d$};
      \draw(B)--(C);
      \draw pic["",<->,draw=orange,angle eccentricity=1.6,angle radius=.4cm]{angle=B--A--C};
    \end{tikzpicture}
  \end{center}
  \begin{align*}
    S_{\triangle ABC} = \frac12 bc\cdot \sin A = \frac12 bd\cdot \sin\frac{A}2 + \frac12 cd\cdot\sin\frac{A}2
  \end{align*}
  将$\sin\alpha=2\sin\frac{\alpha}2\cos\frac{\alpha}2$代入可得。
\end{proof}


\begin{example}
  任意正数$a,b,c$,有
  \begin{align*}
    \sqrt{a^2+ac+c^2}\le \sqrt{a^2-ab+b^2}+\sqrt{b^2-bc+c^2}
  \end{align*}
\end{example}
\begin{proof}[提示]利用余弦定理构造线段,如图:
  \begin{center}
    \begin{tikzpicture}[scale=1.0]
      \coordinate[label=below left:$D$] (D) at (0,0);
      \coordinate[label=below right:$C$] (C) at (3,0);
      \coordinate[label=above left:$A$] (A) at (120:2);
      \coordinate[label=above:$B$] (B) at (60:4);
      \draw[line width=2pt](D)--(A)node[midway,below left]{$a$};
      \draw[line width=2pt](D)--(C)node[midway,below left]{$c$};
      \draw[line width=2pt](D)--(B)node[midway,above left]{$b$};
      \draw(A)--(B)--(C)--cycle;
      \draw pic["$60^\circ$",<->,draw=orange,angle eccentricity=1.8,angle radius=.4cm]{angle=B--D--A};
      \draw pic["$60^\circ$",<->,draw=orange,angle eccentricity=1.6,angle radius=.6cm]{angle=C--D--B};
    \end{tikzpicture}
  \end{center}
  如上图,以点$D$为起点,用长度为$a,b,c$的线段及两个$60^\circ$的夹角构造图形,其中$AD=a$,$BD=b$,$CD=c$,$\angle ADB=BDC=60^\circ$。则由余弦定理有
  \begin{align*}
    AC=&\,\sqrt{a^2+ac+c^2}\\
    AB=&\,\sqrt{a^2-ab+b^2}\\
    BC=&\,\sqrt{b^2-bc+c^2}
  \end{align*}
  再由三角形$ABC$的两边和大于等于第三边(三角形退化为三顶点共线时等号成立),可知原不等式成立。当且仅当$B$落在线段$AC$上时等号成立,此时应用角平分线长度公式,有
  \begin{align*}
    b=\frac{2ac\cos60^\circ}{a+c}=\frac{ac}{a+c}&\qedhere
  \end{align*}
\end{proof}

\section{边长不等式}
\label{sec:triangle-inequality}

\begin{example}\label{ex:triangle-sides-inequality}
  三角形的三条边边长$a,b,c$满足
  \begin{align*}
    (a+b-c)(b+c-a)(c+a-b)\le abc
  \end{align*}
\end{example}
\begin{proof}
  令$s=\frac12(a+b+c)$是三角形的半周长,且
  \begin{align*}
    x\equiv s-a, \quad y\equiv s-b, \quad z\equiv s-c
  \end{align*}
  由三角形的性质,可知$x,y,z$都是正的,且
  \begin{gather*}
    a=y+z,\quad b=z+x,\quad c=x+y\\
    a+b-c = 2z, \quad b+c-a=2x, \quad c+a-b = 2y
  \end{gather*}
  代入并重新各项顺序排列,原不等式等价于$8xyz\le (x+y)(y+z)(z+x)$。对右边每项应用AM--GM不等式可得。
\end{proof}

\begin{example}\label{ex:triangle-sides-inequality-simplified}
  $a,b,c$是三角形的三边边长,则
  \begin{align*}
    (a-b)(b-c)(c-a)<abc
  \end{align*}
\end{example}
\begin{proof}
  由对称性,不妨设$a\le b\le c$,从而存在非负数$u,v$,使得
  \begin{align*}
    b=a+u,\quad c=a+u+v
  \end{align*}
  由$c<a+b$可知$a+u+v<a + (a+u)\implies a>v$。从而
  \begin{align*}
    (a-b)(b-c)(c-a)=(-u)(-v)(u+v) = uv(u+v)\\
    abc = a(a+u)(a+u+v) > v(v+u)(u+v) \ge uv(u+v)
  \end{align*}
  比较两式可得。
\end{proof}

\begin{example}
  若$a,b,c$是一个三角形的三边边长,则
  \begin{align*}
    \left| \frac{a-b}{a+b} + \frac{b-c}{b+c} + \frac{c-a}{c+a}
    \right|
    < \frac18
  \end{align*}
\end{example}
\begin{proof}[提示]
  应用例~\ref{ex:sum-is-negative-to-product},将和化为积的形式:
  \begin{align*}
    \left| \frac{a-b}{a+b} + \frac{b-c}{b+c} + \frac{c-a}{c+a} \right|
    =&\, \left| \frac{a-b}{a+b} \cdot \frac{b-c}{b+c} \cdot \frac{c-a}{c+a} \right|
    = \frac{|a-b|\cdot|b-c|\cdot|c-a|}{(a+b)(b+c)(c+a)}\\
    % <&\, \frac{2\sqrt{ab}\cdot 2\sqrt{bc}\cdot 2\sqrt{ca}}{(a+b)(b+c)(c+a)}\\
    % \le&\,\frac{}{(a+b)(b+c)(c+a)}\\
    \intertext{再由例~\ref{ex:triangle-sides-inequality-simplified}的结论,有}
    <&\, \frac{abc}{(a+b)(b+c)(c+a)}
    \le \frac{abc}{2\sqrt{ab} \cdot 2\sqrt{bc}\cdot 2\sqrt{ca}} = \frac18&&\qedhere
  \end{align*}
\end{proof}

\section{相似与全等}
\label{sec:similar-and-congruent}

\begin{definition}
  若两个三角形$\triangle ABC$与$\triangle DEF$三个角分别相等,即
  \begin{align*}
    \angle A = \angle D,\quad \angle B = \angle E,\quad \angle C = \angle F
  \end{align*}
  则称两个三角形相似。
\end{definition}

\begin{definition}[全等,Congruent]
  若两个相似三角形$\triangle ABC$与$\triangle DEF$对应的边长也相等,则称两三角形全等,通常记为$\triangle ABC \cong \triangle DEF$。
\end{definition}

\section{重要定理}
\label{sec:important-triangle-theorems}

在证明三点共线、三线共点时,经常需要用到梅涅劳斯定理及塞瓦定理。

\subsection{梅涅劳斯定理}
\label{sec:Menelaus's-theorem}

\begin{theorem}[梅涅劳斯定理,Menelaus's Theorem]\label{th:Menelaus's-theorem}
  给定三角形$\triangle ABC$,一条直线与三角形三边或其延长线分别相交于三个不同于三角形顶点的点$D$,$E$及$F$(如图所示),则
  \begin{align*}
    \frac{\vec{AF}}{\vec{FB}}\cdot \frac{\vec{BD}}{\vec{DC}} \cdot \frac{\vec{CE}}{\vec{EA}} = -1
  \end{align*}
  其中上式的比值$\frac{\vec{AF}}{\vec{FB}}$是指两向量$\vec{AF}$与$\vec{FB}$的比值,当其方向相同时比值为正实数,当其方向相反时比值为负实数,其绝对值为两向量的长度之比。

  反之,若$D$,$E$,$F$分别是三角形$\triangle ABC$三边或其延长线上一点,且
  \begin{align*}
    \frac{\vec{AF}}{\vec{FB}}\cdot \frac{\vec{BD}}{\vec{DC}} \cdot \frac{\vec{CE}}{\vec{EA}} = -1
  \end{align*}
  则$D$,$E$及$F$三点共线。
  \begin{center}
    \begin{tikzpicture}[scale=1.5]
        \coordinate[label=below left:$A$] (A) at (0,0);
        \coordinate[label=below:$B$] (B) at (3,0);
        \coordinate[label=above:$C$] (C) at (1,1.5);
        \coordinate (M) at (.5,1.5); \coordinate (N) at (5,-.5);
        \tkzInterLL(A,C)(M,N)\tkzGetPoint{E}
        \tkzInterLL(C,B)(M,N)\tkzGetPoint{D}
        \tkzInterLL(B,A)(M,N)\tkzGetPoint{F}
        \draw(A)--(B)--(C)--cycle;
        \draw[dashed,help lines](B)--(6,0);
        \draw[color=orange](M)--(N);
        \path (E)++(-.1,-.1)node[left]{$E$};
        \node[above right]at(D){$D$};
        \node[below]at(F){$F$};
    \end{tikzpicture}
  \end{center}
\end{theorem}
\begin{proof}[提示]
  首先,若$D$,$E$及$F$三点共线,则如图,$D$在$BC$内,$E$在$CA$内,$F$在$AB$外,从而
  \begin{align*}
   \frac{\vec{AF}}{\vec{FB}}<0,\quad \frac{\vec{BD}}{\vec{DC}}>0,\quad \frac{\vec{CE}}{\vec{EA}} >0
  \end{align*}
  从而其乘积是负数\footnote{若$D$,$E$,$F$均在三角形三边的延长线上,则三个比值都是负的,其乘积同样为负数。同时三个比值不会存在两负一正的情况,即两个点在三角形边的延长线上,另外一个点在三角形的边上。这是因为假设直线与三角形的一边(非延长线)有一交点,那么此直线在往另一边无限延伸的过程中必定穿过另外两条边中的一条,所以必有两个交点在三角形的边(非延长线)上。}。通过作$A$,$B$及$C$在线$DEF$上的高,记其高分别为$a,b,c$,如下图,
  \begin{center}
    \begin{tikzpicture}[scale=1.0]
        \coordinate[label=below left:$A$] (A) at (0,0);
        \coordinate[label=below:$B$] (B) at (4.5,0);
        \coordinate[label=above:$C$] (C) at (1.5,2.25);
        \coordinate (M) at (-.5,1.8); \coordinate (N) at (8.5,-.3);
        % \coordinate(F) at (6.5,0);
        \tkzInterLL(A,C)(M,N)\tkzGetPoint{E}
        \tkzInterLL(C,B)(M,N)\tkzGetPoint{D}
        \tkzInterLL(B,A)(M,N)\tkzGetPoint{F'}
        \coordinate(IA) at ($(E)!(A)!(F')$);
        \coordinate(IB) at ($(E)!(B)!(F')$);
        \coordinate(IC) at ($(E)!(C)!(F')$);
        \tkzMarkRightAngle(A,IA,D)\tkzMarkRightAngle(C,IC,E)\tkzMarkRightAngle(B,IB,D)
        \draw(A)--(B)--(C)--cycle;
        \draw[dashed](A)--(IA)node[midway,left]{$a$} (B)--(IB)node[midway,right]{$b$} (C)--(IC)node[pos=.6,right]{$c$};
        \draw[dashed,help lines](B)--(9,0);
        \draw[color=orange](M)--(N);
        \path (E)++(-.1,.1)node[above]{$E$};
        \node[above right]at(D){$D$};
        \node[below]at(F'){$F$};
        % \node[below]at(F){$F$};
        \tkzDrawPoints(D,E,F')
    \end{tikzpicture}
  \end{center}
  则由相似三角形容易知道
  \begin{align*}
    \left|\frac{\vec{AF}}{\vec{FB}}\right|=\frac{a}{b},\quad
    \left|\frac{\vec{BD}}{\vec{DC}}\right|=\frac{b}{c},\quad
    \left|\frac{\vec{CE}}{\vec{EA}}\right|=\frac{c}{a}
  \end{align*}
  三式相乘,可知其绝对值为1,从而有
  \begin{align*}
    \frac{\vec{AF}}{\vec{FB}}\cdot \frac{\vec{BD}}{\vec{DC}} \cdot \frac{\vec{CE}}{\vec{EA}} = -1
  \end{align*}
  其余情况(如$D$,$E$及$F$均在三角形外)也是类似的。

  至于其逆定理,设$DE$与$AB$相交于点$F'$,只需证明$F$与$F'$重合即可。
  \begin{center}
    \begin{tikzpicture}[scale=1.0]
        \coordinate[label=below left:$A$] (A) at (0,0);
        \coordinate[label=below:$B$] (B) at (4.5,0);
        \coordinate[label=above:$C$] (C) at (1.5,2.25);
        \coordinate (M) at (-.5,1.8); \coordinate (N) at (8.5,-.3);
        \coordinate(F) at (6.5,0);
        \tkzInterLL(A,C)(M,N)\tkzGetPoint{E}
        \tkzInterLL(C,B)(M,N)\tkzGetPoint{D}
        \tkzInterLL(B,A)(M,N)\tkzGetPoint{F'}
        % \coordinate(IA) at ($(E)!(A)!(F')$);
        % \coordinate(IB) at ($(E)!(B)!(F')$);
        % \coordinate(IC) at ($(E)!(C)!(F')$);
        % \tkzMarkRightAngle(A,IA,D)\tkzMarkRightAngle(C,IC,E)\tkzMarkRightAngle(B,IB,D)
        \draw(A)--(B)--(C)--cycle;
        % \draw[dashed](A)--(IA)node[midway,left]{$a$} (B)--(IB)node[midway,right]{$b$} (C)--(IC)node[pos=.6,right]{$c$};
        \draw[dashed,help lines](B)--(9,0);
        \draw[color=orange](M)--(N);
        \path (E)++(-.1,.1)node[above]{$E$};
        \node[above right]at(D){$D$};
        \node[below]at(F'){$F'$};
        \node[below]at(F){$F$};
        \tkzDrawPoints(D,E,F,F')
    \end{tikzpicture}
  \end{center}
  由前面的证明,有
  \begin{align*}
    \frac{\vec{AF'}}{\vec{F'B}}\cdot \frac{\vec{BD}}{\vec{DC}} \cdot \frac{\vec{CE}}{\vec{EA}} = -1
  \end{align*}
  又由条件
  \begin{align*}
    \frac{\vec{AF}}{\vec{FB}}\cdot \frac{\vec{BD}}{\vec{DC}} \cdot \frac{\vec{CE}}{\vec{EA}} = -1
  \end{align*}
  从而有
  \begin{align*}
    \frac{\vec{AF}}{\vec{FB}} = \frac{\vec{AF'}}{\vec{F'B}}
  \end{align*}
  两边加上$1=\frac{\vec{FB}}{\vec{FB}}=\frac{\vec{F'B}}{\vec{F'B}}$,则由向量的加法可知$\vec{AF}+\vec{FB}=\vec{AB}$及 $\vec{AF'}+\vec{F'B}=\vec{AB}$,从而有
  \begin{align*}
    \frac{\vec{AF} + \vec{FB}}{\vec{FB}} = \frac{\vec{AB}+\vec{F'B}}{\vec{F'B}} \,\ \implies\ \,
    \frac{\vec{AB}}{\vec{FB}} = \frac{\vec{AB}}{\vec{F'B}} \,\ \implies\ \,
    \vec{FB}=\vec{F'B}
  \end{align*}
  从而有$F'$与$F$重合。即$D$,$E$,$F$三点共线。
\end{proof}


\subsection{塞瓦定理}
\label{sec:ceva-theorem}

\begin{theorem}[塞瓦定理,Ceva's Theorem]
  给定三角形$\triangle ABC$,点$O$是任意一不在三角形边上的点,且$AO$,$BO$及$CO$分别与三角形三边相交于点$D$,$E$及$F$,则
  \begin{align*}
    \frac{AF}{FB}\cdot\frac{BD}{DC}\cdot\frac{CE}{EA}=1
  \end{align*}
  \begin{center}
    \begin{tikzpicture}[scale=1.5]
      \begin{scope}
        \coordinate[label=below left:$A$] (A) at (0,0);
        \coordinate[label=below right:$B$] (B) at (3,0);
        \coordinate[label=above:$C$] (C) at (1,1.5);
        \coordinate (O) at (1.5,.6);\path (O) ++(-.1,-.1)node[below]{$O$};
        \tkzInterLL(A,O)(B,C)\tkzGetPoint{D}
        \tkzInterLL(B,O)(C,A)\tkzGetPoint{E}
        \tkzInterLL(C,O)(A,B)\tkzGetPoint{F}
        \node[above right]at(D){$D$};
        \node[above left]at(E){$E$};
        \node[below]at(F){$F$};
        \draw(A)--(B)--(C)--cycle;
        \draw(A)--(D)--(C)--(F)--(A) (B)--(E)--(O);
      \end{scope}
      \begin{scope}[shift={(5,0)}]
        \coordinate[label=below left:$A$] (A) at (0,0);
        \coordinate[label=below right:$B$] (B) at (3,0);
        \coordinate[label=above:$C$] (C) at (1,1.5);
        \coordinate (O) at (1.5,.6);
        \tkzInterLL(A,O)(B,C)\tkzGetPoint{D}
        \tkzInterLL(B,O)(C,A)\tkzGetPoint{E}
        \tkzInterLL(C,O)(A,B)\tkzGetPoint{F}
        \fill[color=red!20](A)--(O)--(B)--cycle;
        \fill[pattern=bricks, pattern color=blue!20](B)--(O)--(C)--cycle;
        \fill[pattern=north west lines,pattern color=orange!20](A)--(O)--(C)--cycle;
        \node[above right]at(D){$D$};
        \node[above left]at(E){$E$};
        \node[below]at(F){$F$};
        \draw(A)--(B)--(C)--cycle;
        \draw(A)--(O)--(C) (B)--(O);
        \draw[help lines](D)--(O)--(E) (F)--(O);
        \path (O) ++(-.1,-.1)node[below]{$O$};
    \end{scope}
    \end{tikzpicture}
  \end{center}
\end{theorem}
\begin{proof}[提示]
  用面积法,考虑三个三角形$\triangle AOB$,$\triangle BOC$及$\triangle COA$的面积比。

  \begin{align*}
    & \frac{S_{\triangle ACF}}{S_{\triangle BCF}} = \frac{AF}{FB},\quad
      \frac{S_{\triangle AOF}}{S_{\triangle BOF}} = \frac{AF}{FB}\\
    \implies &
    \frac{S_{\triangle COA}}{S_{\triangle BOC}} = \frac{S_{\triangle ACF} - S_{\triangle AOF}}{S_{\triangle BCF}-S_{\triangle BOF}} = \frac{AF}{FB}
  \end{align*}
  以上是点$O$在三角形内部的情况。若点$O$在三角形外,则上式中的减号($-$)有可能要换为加号($+$)。同理可得
  \begin{align*}
    \frac{S_{\triangle BOC}}{S_{\triangle AOB}} = \frac{CE}{EA},\quad
    \frac{S_{\triangle AOB}}{S_{\triangle COA}} = \frac{BD}{DC}
  \end{align*}
  三式相乘,则有
  \begin{align*}
    \frac{AF}{FB}\cdot\frac{BD}{DC}\cdot\frac{CE}{EA}=
    \frac{S_{\triangle COA}}{S_{\triangle BOC}} \cdot
    \frac{S_{\triangle BOC}}{S_{\triangle AOB}} \cdot
    \frac{S_{\triangle AOB}}{S_{\triangle COA}} = 1
  \end{align*}

  除此之外,也可以用梅涅劳斯定理来证明。
\end{proof}

\begin{theorem}[塞瓦逆定理,Inverse of Ceva's Theorem]
  给定三角形$\triangle ABC$,$D,E,F$分别是三边或其延长线上一点,若
  \begin{align*}
    \frac{AF}{FB}\cdot\frac{BD}{DC}\cdot\frac{CE}{EA}=1
  \end{align*}
  则$AD$,$BE$及$CF$三线共点,或者三线互相平行。
  \begin{center}
    \begin{tikzpicture}[scale=1.5]
      \coordinate[label=below:$A$] (A) at (0,0);
      \coordinate[label=below:$B$] (B) at (3,0);
      \coordinate[label=above right:$C$] (C) at (1,1.5);
      \coordinate[label=above left:$O$] (O) at (.2,.6);
      \tkzInterLL(A,O)(B,C)\tkzGetPoint{D}
      \tkzInterLL(B,O)(C,A)\tkzGetPoint{E}
      \tkzInterLL(C,O)(A,B)\tkzGetPoint{F}
      \node[above]at(D){$D$};
      \node[below]at(E){$E$};
      \node[below left]at(F){$F$};
      \draw(A)--(B)--(C)--cycle;
      \draw(A)--(D)--(C)--(F)--(A) (B)--(E)--(O);
    \end{tikzpicture}
  \end{center}
\end{theorem}
\begin{proof}[提示]
  与梅涅劳斯定理的证明类似。考虑$D$,$E$,$F$均在三角形的三边内(不在延长线上)的情况,则$AD$与$BE$必有交点,记其交点为$O$,设$CO$与$AB$相交于点$F'$,只需证明$F'$与$F$重合即可。

  由前面的塞瓦定理,有
  \begin{align*}
    \frac{AF'}{F'B}\cdot\frac{BD}{DC}\cdot\frac{CE}{EA}=1
  \end{align*}
  再由条件
  \begin{align*}
    \frac{AF}{FB}\cdot\frac{BD}{DC}\cdot\frac{CE}{EA}=1
  \end{align*}
  比较两式,有
  \begin{align*}
    \frac{AF}{FB}=\frac{AF'}{F'B} \implies
    \frac{AF}{FB}+1=\frac{AF'}{F'B}+1 \implies
    % \frac{AF+FB}{FB}=\frac{AF'+F'B}{F'B}\,\implies\,
    \frac{AB}{FB}=\frac{AB}{F'B} \implies 
    FB=F'B
  \end{align*}
  从而$F$与$F'$重合,即$AD$,$BE$及$CF$三线共点。
\end{proof}

\subsection{西姆松定理}
\label{sec:simsons-theorem}

\begin{theorem}[西姆松定理,Simson's Theorem]
  给定三角形,从一点向三角形的三边所引垂线的垂足共线的充要条件是该点落在三角形的外接圆上。当共线时,此线称为西姆松线(Simson's Line)。

  \centering
  \begin{tikzpicture}[scale=1.0]
    \def\r{2}
    \begin{scope}
      \coordinate(O)at(0,0);
      \coordinate(A)at(210:\r);
      \coordinate(B)at(330:\r);
      \coordinate(C)at(70:\r);
      \coordinate(P)at(290:\r);
      \coordinate(D)at($(A)!(P)!(B)$);
      \coordinate(E)at($(B)!(P)!(C)$);
      \coordinate(F)at($(C)!(P)!(A)$);
      \tkzMarkRightAngle(P,D,A)\tkzMarkRightAngle(P,E,B)\tkzMarkRightAngle(P,F,C)
      \tkzDrawPoints(D,E,F)
      \draw(O)circle(\r) (A)--(B)--(C)--cycle (P)--(D) (P)--(E) (P)--(F)
      ($(F)+.3*(F)-.3*(E)$)--($(E)+.2*(E)-.2*(F)$);
      \draw[dashed](B)--(E);% (A)--(P)--(B);
    \end{scope}
    \begin{scope}[shift={(6,0)}]
      \coordinate(O)at(0,0);
      \coordinate[label=below left:$A$](A)at(210:\r);
      \coordinate[label=right:$B$](B)at(330:\r);
      \coordinate[label=above:$C$](C)at(70:\r);
      \coordinate[label=below:$P$](P)at(290:\r);
      \coordinate[label=above:$D$](D)at($(A)!(P)!(B)$);
      \coordinate[label=below:$E$](E)at($(B)!(P)!(C)$);
      \coordinate[label=above:$F$](F)at($(C)!(P)!(A)$);
      \draw pic["\tiny 1",fill=blue!20,draw=orange,angle eccentricity=1.4,angle radius=.5cm]{angle=F--D--A};
      \draw pic["\tiny 2",fill=red!20,draw=orange,angle eccentricity=1.4,angle radius=.5cm]{angle=E--D--B};
      \tkzMarkRightAngle(P,D,A)\tkzMarkRightAngle(P,E,B)\tkzMarkRightAngle(P,F,C)
      \tkzDrawPoints(D,E,F)
      \draw(O)circle(\r) (A)--(B)--(C)--cycle (P)--(D) (P)--(E) (P)--(F) (F)--(E);
      %($(F)+.3*(F)-.3*(E)$)--($(E)+.2*(E)-.2*(F)$);
      \draw[dashed](B)--(E) (A)--(P)--(B);
    \end{scope}
  \end{tikzpicture}
\end{theorem}
\begin{proof}[提示]
  三点共线,可用梅涅劳斯定理。考虑$\triangle APD$和$\triangle BPD$,容易知道
  \begin{align*}
    \frac{AD}{DB} = \frac{\cot\angle PBD}{\cot\angle PAD} = \frac{\cot\angle PBA}{\cot\angle PAB}
  \end{align*}
  同理可得另外两个比例,然后三式相乘,并注意到
  \begin{align*}
    \angle PAB=\angle PCB,\quad \angle PAC\angle PBC,\quad \angle PBC=\pi - \angle PAC
  \end{align*}
  就有
  \begin{align*}
    \frac{AD}{DB}\cdot \frac{BE}{EC}\cdot \frac{CF}{FA} = -1
  \end{align*}
  这是充分性。

  也可以利用四点共圆直接计算角度。连接$DE$和$DF$,则
  \begin{align*}
    D,E,F\text{三点共线}\iff{}& \angle 1=\angle 2 \iff \angle APF\angle BPE \\
    \iff{}& \angle APB + \angle C = \pi \quad(\text{因为}\angle FPE + \angle C=\pi)\\
    \iff{}& A,P,B,C\text{四点共圆}\qedhere
  \end{align*}
\end{proof}

\subsection{代沙格定理}
\label{sec:desargues-theorem}

\begin{theorem}[代沙格定理,Desargues' Theorem]
  若$\triangle ABC$和$\triangle A'B'C'$中,$AA'$、$BB'$和$CC'$所在的直线相交于一点$P$,则两个三角形每组对应边的交点(若存在时)共线,即$AB$与$A'B'$的交点$D$、$BC$与$B'C'$的交点$E$、$CA$与$C'A'$的交点$F$共线。
\end{theorem}
\begin{proof}[提示]
  严格的证明有不少需要考虑的地方,可以应用梅涅劳斯或者塞瓦定理证明。

  \begin{center}
    \begin{tikzpicture}[scale=1.0]
      \coordinate(A)at(0,0);
      \coordinate(B)at(2,-.5);
      \coordinate(C)at(3.5,1);
      \coordinate(P)at(2,4);
      \coordinate(A')at($.4*(A)+.6*(P)$);
      \coordinate(B')at($.7*(B)+.3*(P)$);
      \coordinate(C')at($.8*(C)+.2*(P)$);
      \tkzInterLL(A,B)(A',B')\tkzGetPoint{D}
      \tkzInterLL(B,C)(B',C')\tkzGetPoint{E}
      \tkzInterLL(C,A)(C',A')\tkzGetPoint{F}
      \tkzInterLL(B,E)(A',D)\tkzGetPoint{G}
      % \tkzInterLL(B,E)(A',F)\tkzGetPoint{H}
      \fill[color=red!20](A')--(B')--(C')--cycle (A)--(B)--(C)--cycle;
      \draw[dashed](A)--(F) (A')--(C') (C)--(C') (G)--(E);
      \draw(P)--(A) (P)--(B) (P)--(C')--(F) (B')--(E)
           (A)--(D)--(A')
           (B)--(G)
           (D)--(E);
      \foreach \v/\p in{P/above,A/below left,B/below,C/below,
        A'/left, B'/left, C'/above right,
        D/below right, E/above right, F/right%
      }{
        \tkzLabelPoints[\p](\v)
      }
      \tkzDrawPoints(P,A,B,C,A',B',C',D,E,F)
    \end{tikzpicture}
  \end{center}

  这里介绍一种直观的思路\footnote{这只有有助于理解的方法,并不是严格的数学证明。}:\underline{盯着图看,直到它变得显然成立为止}。怎么看呢,就是把图形立体化,把$\triangle ABC$和$\triangle A'B'C'$看作是三棱锥\mbox{$P$-$ABC$}的两个截面,那么所有的交点都落在平面$ABC$与平面$A'B'C'$的交线上,自然它们是共线的。
\end{proof}

\section{正三角形}
\label{sec:equilateral-triangle}

\begin{example}
  用一张A4纸折出一个正三角形。关键是第二步,对折找到中线上的点。
  \begin{center}
    \begin{tikzpicture}[scale=.75]
      \begin{scope}
        \coordinate(A)at(0,0);\coordinate(B)at(2,0);\coordinate(C)at(2,3);\coordinate(D)at(0,3);
        \draw(A)--(B)--(C)--(D)--(A);
        \draw[dashed]($.5*(A)+.5*(B)$)--($.5*(C)+.5*(D)$);
      \end{scope}
      \begin{scope}[shift={(3,0)}]
        \coordinate(A)at(0,0);\coordinate(B)at(2,0);\coordinate(C)at(2,3);\coordinate(D)at(0,3);
        \coordinate(E')at(30:1);\tkzInterLL(A,E')(B,C)\tkzGetPoint{E}
        \coordinate(F)at($(A)!1!60:(B)$);
        \coordinate(G)at($.5*(A)+.5*(B)$);\coordinate(H)at($.5*(C)+.5*(D)$);
        \coordinate(I)at($.5*(A)+.5*(E)$);\coordinate(J)at($.5*(A)+.5*(F)$);
        \draw(A)--(F)--(E)--(A)--(D)--(C)--(E);
        \draw[dashed](A)--(B)--(E) (H)--(G) (I)--(J);
        \tkzDrawPoint(F)
      \end{scope}
      \begin{scope}[shift={(6,0)}]
        \coordinate(A)at(0,0);\coordinate(B)at(2,0);\coordinate(C)at(2,3);\coordinate(D)at(0,3);
        \coordinate(E')at(30:1);\tkzInterLL(A,E')(B,C)\tkzGetPoint{E}
        \coordinate(F)at($(A)!1!60:(B)$);
        \coordinate(G)at($.5*(A)+.5*(B)$);\coordinate(H)at($.5*(C)+.5*(D)$);
        \coordinate(I)at($.5*(A)+.5*(E)$);\coordinate(J)at($.5*(A)+.5*(F)$);
        \tkzInterLL(A,F)(C,D)\tkzGetPoint{L}
        \tkzInterLL(E,F)(A,D)\tkzGetPoint{M}
        \draw(A)--(F)--(E)--(A)--(D)--(C)--(E);
        \draw[dashed](A)--(B)--(E) (H)--(G) (I)--(J) (L)--(F)--(M);
      \end{scope}
      \begin{scope}[shift={(9,0)}]
        \coordinate(A)at(0,0);\coordinate(B)at(2,0);\coordinate(C)at(2,3);\coordinate(D)at(0,3);
        \coordinate(E')at(30:1);\tkzInterLL(A,E')(B,C)\tkzGetPoint{E}
        \coordinate(F)at($(A)!1!60:(B)$);
        \coordinate(G)at($.5*(A)+.5*(B)$);\coordinate(H)at($.5*(C)+.5*(D)$);
        \coordinate(I)at($.5*(A)+.5*(E)$);\coordinate(J)at($.5*(A)+.5*(F)$);
        \tkzInterLL(A,F)(C,D)\tkzGetPoint{L}
        \tkzInterLL(E,F)(A,D)\tkzGetPoint{M}
        \fill[color=red!10](A)--(E)--(M)--cycle;
        \draw(A)--(B)--(C)--(D)--(A);
        \draw[dashed](E)--(A)--(L) (M)--(E) (H)--(G);
      \end{scope}
      \begin{scope}[shift={(12,0)}]
        \coordinate(A)at(0,0);\coordinate(B)at(2,0);\coordinate(C)at(2,3);\coordinate(D)at(0,3);
        \coordinate(E')at(30:1);\tkzInterLL(A,E')(B,C)\tkzGetPoint{E}
        \coordinate(F)at($(A)!1!60:(B)$);
        \coordinate(G)at($.5*(A)+.5*(B)$);\coordinate(H)at($.5*(C)+.5*(D)$);
        \coordinate(I)at($.5*(A)+.5*(E)$);\coordinate(J)at($.5*(A)+.5*(F)$);
        \tkzInterLL(A,F)(C,D)\tkzGetPoint{L}
        \tkzInterLL(E,F)(A,D)\tkzGetPoint{M}
        \fill[color=red!10](A)--(F)--(B)--cycle;
        \draw(A)--(B)--(C)--(D)--(A);
        \draw[dashed](E)--(A)--(L) (M)--(E) (H)--(G) (B)--(F);
      \end{scope}
    \end{tikzpicture}
  \end{center}
\end{example}

\subsection{正三角形的性质}
\label{sec:property-of-equilateral-triangle}
\begin{example}\label{ex:radius-of-equilateral-triangle}
  正三角形的边长为$a$,则其内切圆半径$r$与外接圆半径$R$分别是
  \begin{align*}
    r= \frac a{2\sqrt3},\quad R=\frac a{\sqrt3}
  \end{align*}
  \begin{center}
    \begin{tikzpicture}[scale=1.0]
      \coordinate (A) at (0,0); \coordinate (B) at (4,0); \coordinate (C) at (60:4);
      \coordinate (O) at ($1/3*(A) + 1/3*(B) + 1/3*(C)$);
      \coordinate (D) at (2,0);
      \draw(A)--(B)--(C) node[pos=.55,sloped,above right]{$a$} --cycle;
      \draw(A)--(O) node[midway,above]{$R$}--(D) node[midway,right]{$r$};
      \draw pic["$30^\circ$",<->,draw=orange,angle eccentricity=1.6,angle radius=.6cm]{angle=B--A--O};
      \tkzMarkRightAngle[color=blue](A,D,O)
      \tkzDrawPoint(O)
      \draw[|<->|](0,-.5)--(2,-.5) node[midway,below]{$a/2$};
      \draw[|<->|](2,-.5)--(4,-.5) node[midway,below]{$a/2$};
      % \node at (O) [circle through=(A)] {};
      % \node at (O) [circle through=({(1.5,0)})] {};
      \draw let \p1=($(O)-(D)$) in (O) circle({veclen(\x1,\y1)});
      % \draw let \p1=($(O)-(A)$) in (O) circle({veclen(\x1,\y1)});
      \draw let \p1=($(O)-(A)$) in (B) arc(-30:210:{veclen(\x1,\y1)});
    \end{tikzpicture}
  \end{center}
\end{example}
\begin{proof}[提示]
  如图,由三角函数有
  \begin{gather*}
    R\cos30^\circ = \frac a2\implies R = \frac a2 \div \frac{\sqrt3}{2} = \frac{a}{\sqrt3}\\
    r= R\sin30^\circ = \frac{a}{\sqrt3} \cdot \frac 12 =\frac{a}{2\sqrt3}\qedhere
  \end{gather*}
\end{proof}

\subsection{复数表示}
\label{sec:equilateral-triangle-with-complex-number}

\begin{theorem}
  复平面上三个不同的点的$z_1, z_2, z_3$是正三角形的顶点,当且仅当
  \begin{align*}
  z_1^2+z_2^2+z_3^2 = z_1z_2 + z_2z_3 + z_3z_1
  \end{align*}
\end{theorem}
\begin{proof}[提示]
  由于$z_1, z_2, z_3$两两不同,所以$z_1,z_3,z_3$三点组成正三角形,等价于两边长度相等且其夹角是$60^\circ$,即
  \begin{align*}
    \text{$z_1,z_2,z_3$组成正三角形}&\iff \frac{z_3-z_1}{z_2-z_1} = \cos(\pm60^\circ) + i\sin(\pm60^\circ) = \frac{1\pm\sqrt3i}{2}\\
                                    &\iff 2z_3 - z_1 - z_2=\pm\sqrt3i\left(z_2 - z_1\right)\\
    &\iff \left(2z_3 - z_1 - z_2\right)^2=-3\left(z_2 - z_1\right)^2\\
    &\iff z_1^2+z_2^2+z_3^2 = z_1z_2 + z_2z_3 + z_3z_1\qedhere
  \end{align*}
\end{proof}
\begin{corollary}
  复平面内两点$x,y$与原点组成正三角形,当且仅当$x,y$与原点互不相同且
  \begin{align*}
    x^2 + y^2 = xy
  \end{align*}
\end{corollary}

\begin{example}
记复平面上单位向量旋转$120^\circ$即$\frac{2\pi}{3}$后的向量为$w$,即
\begin{align*}
  w\equiv e^{\frac{2\pi}{3}i}=\cos\frac{2\pi}{3} + i \sin\frac{2\pi}{3} = -\frac12 + \frac{\sqrt3}{2}i
\end{align*}
则有
\begin{align*}
  1 + w + w^2 = 0
\end{align*}
\end{example}
\begin{proof}[提示]可以直接计算。也可以从向量图中直观地观察到。实际上,$w$也可以看作是一个旋转$120^\circ$的运算符。
  \begin{center}
    \begin{tikzpicture}[scale=1.0]
      \coordinate (A) at (2,0);
      \coordinate (B) at (120:2);
      \coordinate (C) at (240:2);
      \coordinate[label=below right:$O$] (O) at (0,0);
      \draw[help lines](0,0)circle(2);
      \tkzDrawPoint(O);
      \draw[->](O)--(A) node[pos=1,right]{$1$};
      \draw[->](O)--(B) node[pos=1,above left]{$w$};
      \draw[->](O)--(C) node[pos=1,below left]{$w^2$};
      \draw pic["$120^\circ$",<->,draw=orange,angle eccentricity=2.0,angle radius=.4cm]{angle=A--O--B};
      \draw pic["$120^\circ$",<->,draw=orange,angle eccentricity=2.0,angle radius=.4cm]{angle=B--O--C};
    \end{tikzpicture}
  \end{center}
  如上图,容易看出三个向量$1,w,w^2$的和是零向量。

  另一方面,由于$w^3=1$,从而对于任意整数$n$,$w^n, w^{n+1}, w^{n+2}$必是$1,w,w^2$这三者的某个排列,从而有
  \begin{align*}
    w^n + w^{n+1} + w^{n+2} = 0&\qedhere
  \end{align*}
\end{proof}

\begin{theorem}\label{th:equilateral-triangle-x+wy=0}
  复平面内两点$x,y$与原点组成正三角形,当且仅当
  \begin{align*}
    x + wy = 0,\quad \text{或者}\quad x + w^2y = 0 
  \end{align*}
\end{theorem}
\begin{proof}[提示]
  $x+wy=0$与$x+w^2y=0$的含义分别是将$y$绕原点$O$旋转$120^\circ$及旋转$-120^\circ$后与$x$成反向量。
  \begin{center}
    \begin{tikzpicture}[scale=2.0]
      \begin{scope}
        \coordinate[label=below:$O$] (O) at (0,0);
        \coordinate (A) at (20:1);
        \coordinate (B) at (80:1);
        \coordinate (C) at (200:1);
        \draw[help lines](O)circle(1) (A)--(B);
        \draw[->](O)--(A)node[right]{$x$};
        \draw[->](O)--(B)node[above]{$y$};
        \draw[->,dashed](O)--(C)node[left]{$wy$};
        \draw pic["$120^\circ$",->,draw=orange,angle eccentricity=1.6,angle radius=.6cm]{angle=B--O--C};
        \tkzDrawPoint(O);
      \end{scope}
      \begin{scope}[shift={(3.5,0)}]
        \coordinate[label=left:$O$] (O) at (0,0);
        \coordinate (A) at (20:1);
        \coordinate (B) at (80:1);
        \coordinate (C) at (260:1);
        \draw[help lines](O)circle(1) (A)--(B);
        \draw[->](O)--(A)node[right]{$y$};
        \draw[->](O)--(B)node[above]{$x$};
        \draw[->,dashed](O)--(C)node[below]{$\dfrac{y}{w} = w^2y$};
        \draw pic["$120^\circ$",<-,draw=orange,angle eccentricity=1.6,angle radius=.6cm]{angle=C--O--A};
        \tkzDrawPoint(O);
      \end{scope}
    \end{tikzpicture}
  \end{center}
  如上图,分两种情况,分别对应于$y$顺时针及逆时针旋转$120^\circ$后与$x$成反向量。当且仅当$x$与$y$长度相等且夹角为$60^\circ$时,即当且仅当$wy=-x$或者$w^2y=-x$时,$x,y$及原点$O$组成正三角形。
\end{proof}

\begin{theorem}\label{th:equilateral-triangle-z1+wz2+w2z3}
  复平面内三点$z_1,z_2,z_3$组成正三角形,当且仅当
  \begin{align*}
    z_1 + wz_2 + w^2z_3 = 0
  \end{align*}
\end{theorem}
\begin{proof}[提示]
  应用上一例中的记号$w$,有$1=-w-w^2$,从而
  \begin{align*}
    z_1 + wz_2 + w^2z_3 = 0 &\iff (-w-w^2)z_1 + wz_2 + w^2z_3 = 0 \\
                            & \iff w\left(z_2-z_1\right) + w^2\left(z_3-z_1\right) = 0\\
                            & \iff \left(z_2-z_1\right) + w\left(z_3-z_1\right) = 0
  \end{align*}
  于是将坐标原点移到$z_1$,在新坐标系下原$z_2,z_3$的坐标则变为了$z_2-z_1$及$z_3-z_1$,从而原命题变为
  \begin{quotation}
    复平面内两点$x,y$及原点组成正三角形,当且仅当
    \begin{align*}
      x + wy = 0
    \end{align*}
  \end{quotation}
  而这由定理~\ref{th:equilateral-triangle-x+wy=0}可得。
\end{proof}

\section{拿破仑三角形}
\label{sec:napoleon-triangle}

通常认为拿破仑定理是法国的著名军事家拿破仑·波拿巴(Napoleon Bonaparte)提出的。

\begin{theorem}[拿破仑定理,Napoleon' Theorem]\label{th:napoleon-theorem}
  任意三角形,以其三边分别向外做三个正三角形,则这三个正三角形的重心连线组成一个正三角形。此正三角形称为外拿破仑三角形。

  同样的,若以其三边分别向内做三个正三角形,则这三个正三角形的重心连线同样组成一个正三角形。此正三角形称为内拿破仑三角形。

  \begin{center}
    \begin{tikzpicture}[scale=1.0]
      \begin{scope}
        \coordinate[label=below left:$A$] (A) at (0,0);
        \coordinate[label=below right:$B$] (B) at (3,0);
        \coordinate[label=above:$C$] (C) at (2,2);
        \coordinate[label=below:$C'$] (C') at ($(B)!1!60:(A)$);
        \coordinate[label=above left:$B'$] (B') at ($(A)!1!60:(C)$);
        \coordinate[label=right:$A'$] (A') at ($(C)!1!60:(B)$);
        \coordinate[label=below:$D$] (D) at ($1/3*(A) + 1/3*(B) + 1/3*(C')$);
        \coordinate[label=right:$E$] (E) at ($1/3*(B) + 1/3*(C) + 1/3*(A')$);
        \coordinate[label=left:$F$] (F) at ($1/3*(C) + 1/3*(A) + 1/3*(B')$);
        \coordinate (D') at ($(A)!1!60:(D)$);
        \coordinate (E') at ($(B)!1!60:(E)$);
        \coordinate (F') at ($(C)!1!60:(F)$);
        \draw[help lines,dashed] (D)--(D') (E)--(E') (F)--(F');
        \draw[help lines,line width=2pt](A)--(C')--(B) (B)--(A')--(C) (C)--(B')--(A);
        \draw[line width=2pt](A)--(B)--(C)--cycle;
        \draw(D)--(E)--(F)--cycle (D')--(E')--(F')--cycle;
      \end{scope}
      \begin{scope}[shift={(5,0)}]
        \coordinate (A) at (0,0);
        \coordinate (B) at (3,0);
        \coordinate (C) at (2.2,.85);
        \coordinate (C') at ($(B)!1!-60:(A)$);
        \coordinate (B') at ($(A)!1!-60:(C)$);
        \coordinate (A') at ($(C)!1!-60:(B)$);
        \coordinate (D) at ($1/3*(A) + 1/3*(B) + 1/3*(C')$);
        \coordinate (E) at ($1/3*(B) + 1/3*(C) + 1/3*(A')$);
        \coordinate (F) at ($1/3*(C) + 1/3*(A) + 1/3*(B')$);
        \draw[help lines,line width=2pt](A)--(C')--(B) (B)--(A')--(C) (C)--(B')--(A);
        \draw[line width=2pt](A)--(B)--(C)--cycle;
        \draw(D)--(E)--(F)--cycle;
      \end{scope}
    \end{tikzpicture}
  \end{center}
\end{theorem}
\begin{proof}[提示]
  实际上,由于作一边向外与向内作的正三角形是关于此边对称的,从而由外拿破仑三角形的顶点分别以原三角形的三边作反射得到三个反射点,则此三点就是内拿破仑三角形的三个顶点。

  应用定理~\ref{th:equilateral-triangle-z1+wz2+w2z3},记图中各点在复平面内的坐标分别是$A$,$B$,$C$,$D$,$E$,$F$,$A'$,$B'$及$C'$,则由重心是三顶点的坐标平均值,有
  \begin{align*}
    D = \frac13\left(A+B+C'\right)\\
    E = \frac13\left(B+C+A'\right)\\
    F = \frac13\left(C+A+B'\right)
  \end{align*}
  从而(按$A,B,C$分组,带$'$或不带$'$,且每组要有系数$1,w,w^2$)
  \begin{align*}
    &D + wE + w^2F \\
    =\,& \frac13\left(A+B+C'\right)
    + \frac13\cdot w\left(B+C+A'\right)
       + \frac13\cdot w^2\left(C+A+B'\right)\\
    =\,&\frac13\left( A + wC + w^2B' \right) + \frac13\left( B + wA' + w^2C \right)
       + \frac13\left( C' + wB + w^2A \right)
  \end{align*}
  且对于三角正三角形$\triangle ABC'$,$\triangle BCA'$及$\triangle CAB'$,有
    \begin{align*}
    A + wC + w^2 B' = 0\\
    B + wA' + w^2 C = 0\\
    C' + wB + w^2 A = 0
  \end{align*}
  从而$D+wE+w^2F=0$,按定理~\ref{th:equilateral-triangle-z1+wz2+w2z3},$\triangle DEF$是正三角形。同样的,内拿破仑三角形也是正三角形。
\end{proof}

\begin{example}
  内外拿破仑三角形的重心与原三角形的重心重合。
\end{example}
\begin{proof}[提示]
  还是用复数。如定理~\ref{th:napoleon-theorem},有
  \begin{align*}
    D = \frac13\left(A+B+C'\right),\quad
    E = \frac13\left(B+C+A'\right),\quad
    F = \frac13\left(C+A+B'\right)
  \end{align*}
  同时,由于向量$BC'$是向量$BA$逆时针旋转$60^\circ$而得,从而有
  \begin{align*}
    C'-B=e^{\frac{\pi}6 i} (A-B) \quad\implies\quad C' = B + e^{\frac{\pi}6 i} (A-B)    
  \end{align*}
  同理可得
  \begin{align*}
    A' = C + e^{\frac{\pi}6 i} (B-C),\quad
    B' = A + e^{\frac{\pi}6 i} (C-A)
  \end{align*}
  从而有$A'+B'+C' = A+B+C$,代入,则有
  \begin{align*}
    D + E + F & = \frac13\left(A+B+C'\right) + \frac13\left(B+C+A'\right) + \frac13\left(C+A+B'\right) \\
              & = A + B + C
  \end{align*}
  从而外拿破仑三角形的重心$\frac13(D+E+F)$与原三角形的重心$\frac13(A+B+C)$重合。

  同理,内拿破仑三角形的重心与原三角形的重心重合。
\end{proof}


\begin{example}[美国,1956]
  在任意三角形的每边向外作顶角为$120^\circ$的等腰三角形,则这三个等腰三角形的三个顶点组成一个正三角形。
\end{example}
\begin{proof}[提示]
  此问题与拿破仑三角形类似,同样可以用复数解决。
  \begin{center}
    \begin{tikzpicture}[scale=1.0]
      \begin{scope}
        \coordinate[label=below left:$A$] (A) at (0,0);
        \coordinate[label=below right:$B$] (B) at (3,0);
        \coordinate[label=above:$C$] (C) at (2,2);
        \coordinate[label=below:$C'$] (C') at ($(B)!1/sqrt(3)!30:(A)$);
        \coordinate[label=above left:$B'$] (B') at ($(A)!1/sqrt(3)!30:(C)$);
        \coordinate[label=right:$A'$] (A') at ($(C)!1/sqrt(3)!30:(B)$);
        % \coordinate[label=below:$D$] (D) at ($1/3*(A) + 1/3*(B) + 1/3*(C')$);
        % \coordinate[label=right:$E$] (E) at ($1/3*(B) + 1/3*(C) + 1/3*(A')$);
        % \coordinate[label=left:$F$] (F) at ($1/3*(C) + 1/3*(A) + 1/3*(B')$);
        % \coordinate (D') at ($(A)!1!60:(D)$);
        % \coordinate (E') at ($(B)!1!60:(E)$);
        % \coordinate (F') at ($(C)!1!60:(F)$);
        % \draw[help lines,dashed] (D)--(D') (E)--(E') (F)--(F');
        \draw[help lines,line width=2pt](A)--(C')--(B) (B)--(A')--(C) (C)--(B')--(A);
        \draw[line width=2pt](A)--(B)--(C)--cycle;
        % \draw(D)--(E)--(F)--cycle (D')--(E')--(F')--cycle;
        \draw(A')--(B')--(C')--cycle;
      \end{scope}
    \end{tikzpicture}
  \end{center}
  如上图,记各点在复平面内的坐标分别为$A$,$B$,$C$,$A'$,$B'$及$C'$,则由三个顶角为$120^\circ$的等腰三角形$\triangle AC'B$,$\triangle BA'C$及$\triangle CB'A$,有
  \begin{align*}
    \begin{cases}
      A-C'=w(B-C')\\
      B-A'=w(C-A')\\
      C-B'=w(A-B')
    \end{cases}
    \implies
    \begin{cases}
      C' = \frac{A-wB}{1-w}\\
      A' = \frac{B-wC}{1-w}\\
      B' = \frac{C-wA}{1-w}
    \end{cases}
  \end{align*}
  上式中$A-C'=w(B-C')$是因为向量$\vec{C'A}$是向量$\vec{C'B}$逆时针旋转$120^\circ$而得,其余类似。从而
  \begin{align*}
    A' + wB' + w^2C' &= \frac{B-wC}{1-w} + w\cdot\frac{C-wA}{1-w} + w^2\cdot\frac{A-wB}{1-w}\\
                     &= \frac1{1-w}\left( (B-wC) + w(C-wA) + w^2(A-wB) \right)\\
                     &= \frac1{1-w}\left( B-wC + wC-w^2A + w^2A-w^3B \right)
  \end{align*}
  由$w^3=1$可知$A'+wB'+w^2C'=0$,从而$\triangle A'B'C'$是正三角形。
\end{proof}

\begin{example}[南京,1978]
  如图,$A_1,B_1,C_1$分别是三个正角形$\triangle ABC_2$,$\triangle BCA_2$及$\triangle CAB_2$的重心,则
  \begin{enumerate}
  \item $AA_2=BB_2=CC_2=\sqrt3 B_1C_1$;
  \item $\triangle A_1B_1C_1$是正三角形。
  \end{enumerate}
    \begin{center}
    \begin{tikzpicture}[scale=1.0]
      \begin{scope}[blend mode=multiply]
        \coordinate[label=below left:$A$] (A) at (0,0);
        \coordinate[label=below right:$B$] (B) at (3,0);
        \coordinate[label=above:$C$] (C) at (2,2);
        \coordinate[label=below:$C_2$] (C') at ($(B)!1!60:(A)$);
        \coordinate[label=above left:$B_2$] (B') at ($(A)!1!60:(C)$);
        \coordinate[label=right:$A_2$] (A') at ($(C)!1!60:(B)$);
        \coordinate[label=below left:$C_1$] (D) at ($1/3*(A) + 1/3*(B) + 1/3*(C')$);
        \coordinate[label=right:$A_1$] (E) at ($1/3*(B) + 1/3*(C) + 1/3*(A')$);
        \coordinate[label=left:$B_1$] (F) at ($1/3*(C) + 1/3*(A) + 1/3*(B')$);
        \coordinate (D') at ($(A)!1!60:(D)$);
        \coordinate (E') at ($(B)!1!60:(E)$);
        \coordinate (F') at ($(C)!1!60:(F)$);

        % \fill[color=red!20,opacity=.5](A)--(C)--(C')--cycle;
        % \fill[color=blue!20,opacity=.5](A)--(B)--(B')--cycle;
        
        % \draw[help lines,dashed] (D)--(D') (E)--(E') (F)--(F');
        \draw[help lines,line width=2pt](A)--(C')--(B) (B)--(A')--(C) (C)--(B')--(A);
        \draw[line width=2pt](A)--(B)--(C)--cycle;
        \draw(D)--(E)--(F)--cycle;% (D')--(E')--(F')--cycle;
        \draw(A)--(A') (B)--(B') (C)--(C');
      \end{scope}
      \begin{scope}[shift={(7,0)},blend mode=multiply]
        \coordinate[label=below left:$A$] (A) at (0,0);
        \coordinate[label=below right:$B$] (B) at (3,0);
        \coordinate[label=above:$C$] (C) at (2,2);
        \coordinate[label=below:$C_2$] (C') at ($(B)!1!60:(A)$);
        \coordinate[label=above left:$B_2$] (B') at ($(A)!1!60:(C)$);
        \coordinate[label=right:$A_2$] (A') at ($(C)!1!60:(B)$);
        \coordinate[label=below left:$C_1$] (D) at ($1/3*(A) + 1/3*(B) + 1/3*(C')$);
        \coordinate[label=right:$A_1$] (E) at ($1/3*(B) + 1/3*(C) + 1/3*(A')$);
        \coordinate[label=left:$B_1$] (F) at ($1/3*(C) + 1/3*(A) + 1/3*(B')$);
        \coordinate (D') at ($(A)!1!60:(D)$);
        \coordinate (E') at ($(B)!1!60:(E)$);
        \coordinate (F') at ($(C)!1!60:(F)$);

        \fill[color=red!20,opacity=.5](A)--(C)--(C')--cycle;
        \fill[color=blue!20,opacity=.5](A)--(B)--(B')--cycle;
        
        % \draw[help lines,dashed] (D)--(D') (E)--(E') (F)--(F');
        \draw[help lines,line width=2pt](A)--(C')--(B) (B)--(A')--(C) (C)--(B')--(A);
        \draw[line width=2pt](A)--(B)--(C)--cycle;
        \draw(D)--(E)--(F)--cycle;% (D')--(E')--(F')--cycle;
        \draw(A)--(A') (B)--(B') (C)--(C');
      \end{scope}
    \end{tikzpicture}
  \end{center}
\end{example}
\begin{proof}[提示]
  $A_1B_1C_1$是三角形$\triangle ABC$的外拿破仑三角形,从而是正三角形。

  将三角形$\triangle AC_2C$绕点$A$逆时针旋转$60^\circ$可得$\triangle ABB_2$,从而有$CC_2=BB_2$。同理可得$AA_2=BB_2=CC_2$。

  
  用余弦定理证明$AA_2=\sqrt3B_1C_1$。在$\triangle ACA_2$中,对角$\angle ACA_2$应用余弦定理,有
  \begin{align*}
    AA_2^2 &= AC^2 + A_2C^2 - 2AC\cdot A_2C\cdot\cos(C+\frac\pi6)\\
           &= AC^2 + BC^2 - 2AC\cdot BC\cdot\cos(C+\frac\pi6)
  \end{align*}
  其次,在$\triangle B_1AC_1$中,对角$\angle B_1AC_1$应用余弦定理,有
  \begin{align*}
    B_1C_1^2 & = B_1A^2 + C_1A^2 - 2B_1A\cdot C_1A\cdot \cos(A+30^\circ + 30^\circ) \\
             &= B_1A^2 + C_1A^2 - 2B_1A\cdot C_1A\cdot \cos(A+\frac\pi6)
  \end{align*}
  比较两式,一个有$\angle A$,一个有$\angle C$,不好比较。将$B_1C_1$换为$A_1B_1$,则有望将后式中的$\angle A$换为$\angle C$。对$\triangle A_1CB_1$中的$\angle C$应用余弦定理,有
  \begin{align*}
    A_1B_1^2 &= A_1C^2 + B_1C^2 - 2A_1C\cdot B_1C\cdot\cos(C + 30^\circ + 30^\circ)\\
             & = A_1C^2 + B_1C^2 - 2A_1C\cdot B_1C\cdot\cos(C + \frac\pi6)
  \end{align*}
  再由正三角形$\triangle BCA_2$及正三角形$\triangle ACB_2$,应用例~\ref{ex:radius-of-equilateral-triangle}的结论,有
  \begin{align*}
    A_1C = \frac{BC}{\sqrt3},\quad B_1C = \frac{AC}{\sqrt3}
  \end{align*}
  代入,则有
  \begin{align*}
    A_1B_1^2 = \frac13\left(BC^2 + AC^2 - 2 BC\cdot AC\cdot\cos(C + \frac\pi6)\right)
  \end{align*}
  比较$AA_2$的表达式,可知$AA_2^2 = 3A_1B_1^2$,从而有$AA_2=\sqrt3 A_1B_1$。由于外拿破仑三角形$\triangle A_1B_1C_1$是正三角形,从而有
  \begin{align*}
    AA_2=BB_2=CC_2=\sqrt3 A_1B_1 = \sqrt3 B_1C_1 = \sqrt3 C_1A_1 
  \end{align*}
  此证明中并没用到$AA_1$、$BB_1$及$CC_1$共点的结论。
\end{proof}

\begin{theorem}[拿破仑点]\label{th:napoleon's-point}
  如图,任意三角形$\triangle ABC$,以三边为底边分别向外(或向内)作相似的三个等腰三角形形成六边形,则三条对角线$AX$,$BY$,$CZ$共点。
  \begin{center}
    \begin{tikzpicture}[scale=1.0]
      \begin{scope}
        \coordinate[label=left:$A$](A) at (0,0);
        \coordinate[label=right:$B$](B) at (3,0);
        \coordinate[label=above:$C$](C) at (2.5,2.5);
        \coordinate[label=above right:$X$](X) at ($(C)!.5/cos(30)!30:(B)$);
        \coordinate[label=above left:$Y$](Y) at ($(A)!.5/cos(30)!30:(C)$);
        \coordinate[label=below:$Z$](Z) at ($(B)!.5/cos(30)!30:(A)$);
        % \fill[color=red!20](A)--(B)--(Z)--cycle;
        % \fill[color=blue!20](B)--(C)--(X)--cycle;
        % \fill[color=yellow!20](C)--(A)--(Y)--cycle;
        \draw(A)--(B)--(Z)--cycle (B)--(C)--(X)--cycle (C)--(A)--(Y)--cycle;
        \draw(A)--(X) (B)--(Y) (C)--(Z);
      \end{scope}
      \begin{scope}[shift={(5,0)}]
        \coordinate[label=left:$A$](A) at (0,0);
        \coordinate[label=right:$B$](B) at (3,0);
        \coordinate[label=above:$C$](C) at (2.5,2.5);
        \coordinate[label=above right:$X$](X) at ($(C)!.5/cos(30)!30:(B)$);
        \coordinate[label=above left:$Y$](Y) at ($(A)!.5/cos(30)!30:(C)$);
        \coordinate[label=below:$Z$](Z) at ($(B)!.5/cos(30)!30:(A)$);
        \fill[color=red!20](A)--(B)--(Z)--cycle;
        \fill[color=blue!20](B)--(C)--(X)--cycle;
        \fill[color=yellow!20](C)--(A)--(Y)--cycle;
        \draw(A)--(B)--(Z)--cycle (B)--(C)--(X)--cycle (C)--(A)--(Y)--cycle;
        \draw(A)--(X) (B)--(Y);% (C)--(Z);
        \draw pic["$\theta$",<->,draw=orange,angle eccentricity=1.6,angle radius=.3cm]{angle=B--C--X};
        \draw pic["$\theta$",<->,draw=orange,angle eccentricity=1.6,angle radius=.3cm]{angle=X--B--C};
        \draw pic["$\theta$",<->,draw=orange,angle eccentricity=1.6,angle radius=.3cm]{angle=Z--A--B};
        \draw pic["$\theta$",<->,draw=orange,angle eccentricity=1.6,angle radius=.3cm]{angle=A--B--Z};
        \draw pic["$\theta$",<->,draw=orange,angle eccentricity=1.6,angle radius=.3cm]{angle=C--A--Y};
        \draw pic["$\theta$",<->,draw=orange,angle eccentricity=1.6,angle radius=.3cm]{angle=Y--C--A};
      \end{scope}
    \end{tikzpicture}
  \end{center}
  当$\theta=\pi/6$时,三线的交点即为first Napoleon point;当$\theta=-\pi/6$时(即等腰三角形是向内的),三线的交点即为second Napoleon point。当$\theta=\pi/3$即向外作正三角形时三线的交点即为费马点(见\ref{sec:fermat-point})。
\end{theorem}
% \begin{proof}[提示]
%   此定理的证明并不平凡,若有兴趣,可以参考Eddy, R. H.; Fritsch, R. (June 1994). "The Conics of Ludwig Kiepert: A Comprehensive Lesson in the Geometry of the Triangle". Mathematics Magazine. 67 (3): 188--205.
% \end{proof}

拿破仑点可以更进一步地推广如下,上面定理是下面定理的一种特殊情形。
\begin{theorem}[推广]
  给定任意三角形,如图在其三边向外作三个三角形$\triangle XBC$,$\triangle AYC$及$\triangle ABZ$,使得相邻的两个角相等,即
  \begin{align*}
    \angle YAC = \angle ZAB = \alpha,\quad
    \angle ZBA = \angle XBC = \beta,\quad
    \angle XCB = \angle YCA = \gamma
  \end{align*}
  则$AX$,$BY$及$CZ$三线共点,该点称为雅可比点(Jacobi Point)。$\triangle XYZ$称为$\triangle ABC$的一个雅可比三角形(Jacobi Triangle)。
  \begin{center}
    \begin{tikzpicture}[scale=1.0]
      \coordinate[label=left:$B$] (B) at (0,0);
      \coordinate[label=right:$C$] (C) at (4,0);
      \coordinate[label=above:$A$] (A) at (3,2);
      % \coordinate[label=below:$X$] (X) at ($(C)!.8!30:(B)$);
      % \coordinate[label=below:$X$] (Y) at ($(C)!.8!-30:(A)$);
      % \tkzInterLL(A,X)(B,Y)\tkzGetPoint{N}
      % \tkzInterLL(C,N)(
      \coordinate(X1) at ($(B)!.5!-80:(C)$);\coordinate(X2) at ($(C)!.5!30:(B)$);\tkzInterLL(B,X1)(C,X2)\tkzGetPoint{X}
      \coordinate(Y1) at ($(A)!.5!60:(C)$);\coordinate(Y2) at ($(C)!.5!-30:(A)$);\tkzInterLL(A,Y1)(C,Y2)\tkzGetPoint{Y}
      \coordinate(Z1) at ($(A)!.5!-60:(B)$);\coordinate(Z2) at ($(B)!.5!80:(A)$);\tkzInterLL(A,Z1)(B,Z2)\tkzGetPoint{Z}
      \tkzInterLL(A,X)(B,C)\tkzGetPoint{X'}
      \tkzInterLL(B,Y)(C,A)\tkzGetPoint{Y'}
      \tkzInterLL(C,Z)(A,B)\tkzGetPoint{Z'}

      \draw pic["$\beta$",<->,draw=orange,angle eccentricity=1.8,angle radius=.3cm,fill=orange!20]{angle=X--B--C};
      \draw pic["$\beta$",<->,draw=orange,angle eccentricity=1.8,angle radius=.3cm,fill=orange!20]{angle=A--B--Z};

      \draw pic["$\gamma$",<->,draw=blue,angle eccentricity=1.6,angle radius=.5cm,fill=blue!20]{angle=B--C--X};
      \draw pic["$\gamma$",<->,draw=blue,angle eccentricity=1.6,angle radius=.5cm,fill=blue!20]{angle=Y--C--A};

      \draw pic["$\alpha$",<->,draw=red,angle eccentricity=1.6,angle radius=.3cm,fill=red!20]{angle=C--A--Y};
      \draw pic["$\alpha$",<->,draw=red,angle eccentricity=1.8,angle radius=.3cm,fill=red!20]{angle=Z--A--B};

      % \draw[help lines](X)--(Y)--(Z)--cycle;

      \draw(A)--(B)--(C)--cycle (B)--(X)--(C) (C)--(Y)--(A) (A)--(Z)--(B);
      \node[below] at (X) {$X$}; \node[above right] at (Y) {$Y$}; \node[above left] at (Z) {$Z$};
      \draw(A)--(X) (B)--(Y) (C)--(Z);
      \tkzInterLL(A,X)(B,Y)\tkzGetPoint{N}\tkzDrawPoint(N)
      % \node[label={[label distance=3pt]280:$N$}] at (N) {};
      \path (N) ++(.1,-.2) node[below]{$N$};
      \tkzDrawPoints(X',Y',Z')
      \path (X')++(.2,-.1)node[below]{$X'$};
      \path (Y')++(-.1,-.1)node[below]{$Y'$};
      \path (Z')++(-.1,0)node[left]{$Z'$};
    \end{tikzpicture}
  \end{center}
\end{theorem}
\begin{proof}[提示]利用正弦定理及塞瓦定理。记三角形的三个顶角分别是$A,B,C$,对应的三条边长分别是$a,b,c$。

  考虑$\triangle AYY'$及$\triangle CYY'$,利用正弦定理,有
  \begin{align*}
    &\frac{\sin\angle AYY'}{AY'} = \frac{\sin\alpha}{YY'}, \quad
      \frac{\sin\angle CYY'}{Y'C} = \frac{\sin\gamma}{YY'}\\[3pt]
    \implies& \frac{AY'}{Y'C} = \frac{\sin\alpha\cdot \sin\angle CYY'}{\sin\gamma\cdot\sin\angle AYY'}
              = \frac{\sin\alpha\cdot \sin\angle CYB}{\sin\gamma\cdot\sin\angle AYB}
  \end{align*}
  再对$\triangle AYB$和$\triangle CYB$应用正弦定理,有
  \begin{align*}
    &\frac{\sin\angle AYB}{c}=\frac{\sin(A+\alpha)}{BY},\quad
      \frac{\sin\angle CYB}{a}=\frac{\sin(C+\gamma)}{BY}\\
    \implies& \frac{\sin\angle AYB}{\sin\angle CYB}=\frac{a\sin(A+\alpha)}{c\sin(C+\gamma)}
  \end{align*}
  再由$\angle AYY' = \angle AYB$及$\angle CYY'=\angle CYB$,结合上面两式,有
  \begin{align*}
    \frac{AY'}{Y'C} = \frac{\sin\alpha\cdot c\cdot\sin(C+\gamma)}{\sin\gamma\cdot a\cdot\sin(A+\alpha)}
  \end{align*}
  同理可得出
  \begin{align*}
    \frac{CX'}{X'B}=\frac{\sin\gamma\cdot b\cdot\sin(B+\beta)}{\sin\beta\cdot c\cdot\sin(C+\gamma)},\quad
    \frac{BZ'}{Z'A}=\frac{\sin\beta\cdot a\cdot\sin(A+\alpha)}{\sin\alpha\cdot b\cdot\sin(B+\beta)}
  \end{align*}
  三式相乘,则有
  \begin{align*}
    \frac{AY'}{Y'C}\cdot\frac{CX'}{X'B}\cdot\frac{BZ'}{Z'A}=1
  \end{align*}
  由塞瓦定理,可知$AX'$,$BY'$,$CZ'$三线共点。
\end{proof}

\section{费马点}
\label{sec:fermat-point}

\begin{definition}[费马点,Fermat Point,Torricelli Point,Fermat-Torricelli Point]
  给定任意$\triangle ABC$,平面上到该三角形顶点距离之和取得最小值的点称为费马点。
\end{definition}
费马点也称为费马--托里拆利\footnote{埃万杰利斯塔·托里拆利是意大利的物理学家、数学家,也是气压计的发明者。}点。每个三角形都有唯一的一个费马点,且按定理~\ref{th:napoleon's-point}可由拿破仑三角形构造出来。

\begin{theorem}
  记$P$为$\triangle ABC$内一点,若$\angle APB=\angle BPC=\angle CPA$,则$P$是$\triangle ABC$的费马点。
\end{theorem}

\begin{example}[费马点的物理学解释]\label{ex:physics-of-fermat-point}
  将平面上所给的三个给定点$A,B,C$钻出洞来,再设有三条绳子系在一起,每条绳子各穿过一个洞口,而绳子的末端都绑有一个固定重量$m$的重物。假设摩擦力可以忽略,那么绳子会被拉紧,而绳结最后会停在平面一点的上方。可以证明,这个点$P$就是三个给定点所对应的费马点。首先,由于绳长是固定的,而绳子竖直下垂的部分越长,重物的位置也就越低,势能越低。在平衡态的时候,系统的势能达到最小值,也就是绳子竖直下垂的部分的长度达到最大值,因此水平的部分的长度达到最小值。而绳子的水平部分的长度就是$PA + PB + PC$,因此这时$PA + PB + PC$最小,也就是达到费马点。

  以平面为势能零点,记三根绳子总长为$l$,则平衡状态下系统的势能$E_p$为
  \begin{align*}
    E_p=-m(l-PA-PB-PC)=-ml+m(PA+PB+PC)
  \end{align*}
  即$PA+PB+PC$取得最小值等价于系统的势能最小,从而平衡时绳结所在的平面上的点$P$就是$\triangle ABC$的费马点。

  此处关键在于三条绳下的重物有相同的重量$m$。%若三个重物的重量不同,则平衡时的点$P$就是加权费马点?似乎不是的。
\end{example}

\begin{example}[Romanian National Olympiad]
  复数$u$、$v$和$w$满足
  \begin{align*}
    u|vw| + v|wu| + w|uv| = 0
  \end{align*}
  则有
  \begin{align*}
    |u-v|\cdot |v-w|\cdot |w-u| \ge 3\sqrt3 |uvw|
  \end{align*}
\end{example}
\begin{proof}[提示]
  引入记号如下:
  \begin{align*}
    u=|u|e^{\alpha i},\qquad
    v=|v|e^{\beta i},\qquad
    w=|w|e^{\gamma i}
  \end{align*}
  则由$u|vw| + v|wu| + w|uv| = 0$可知
  \begin{align*}
    e^{\alpha i} + e^{\beta i} + e^{\gamma i} = 0
  \end{align*}
  从而$u$、$v$和$w$之间两两夹角都是$120^\circ$\footnote{三个单位向量,其和为零向量等价于三个向量两两夹角为$120^\circ$。任取一个向量的方向及其正交方向,考虑三个向量在该正交方向上的投影,可知另两个向量关于取定向量对称。由此三个向量,两两关于另一个向量对称,从而两两夹角$120^\circ$。},即原点是$u$、$v$和$w$三点组成的三角形的费马点。再由余弦定理,有
  \begin{align*}
    |u-v|^2=|u|^2 + |v|^2 + |u|\cdot |v| \ge 3|u|\cdot |v| = 3|uv|
  \end{align*}
  同理求得$|v-w|$和$|w-u|$的下限估计,然后三式相乘可得。
\end{proof}

费马点是关于距离之和的最小值,那么使得距离平方和取得最小值的点是什么呢?

\begin{theorem}[重心]
  $\forall\triangle ABC$,以下两个关于重心的定义是等价的:
  \begin{enumerate}
  \item 平面上一点$G$若使得其到三角形三顶点的距离的平方和取到最小值,则称该点为三角形的重心;    
  \item 点$G\equiv (A+B+C)/3$称为三角形的重心。
  \end{enumerate}
\end{theorem}
\begin{proof}[提示]
  平面上任一点$P$,记其到三角形三顶点的距离平方和为$f(P)$,即
  \begin{align*}
    f(P)=PA^2 + PB^2 + PC^2
  \end{align*}
  若用直角坐标系并将$P\equiv(x,y)$代入则可得到关于$x,y$的二次多项式,且可得到当且仅当
  \begin{align*}
    x=\frac{x_a + x_b + x_c}{3},\qquad
    y=\frac{y_a + y_b + y_c}{3}
  \end{align*}
  时,$f(P)$取得最小值。
\end{proof}

\begin{theorem}[垂心]
  $\forall\triangle ABC$,以下两个关于垂心的定义是等价的:
  \begin{enumerate}
  \item 平面上一点$H$若使得其到三角形三边的距离的平方和取到最小值,则称该点为三角形的垂心;    
  \item 过三个顶点的三条高的交点称为垂心。
  \end{enumerate}
\end{theorem}

{\color{red}内心呢?外心呢?是否也有相应的规律?}

\begin{theorem}[费马点推广]
  给定平面上的凸$n$边形$A_1A_2\cdots A_n$,若平面内一点$O$满足
  \begin{align*}
    \angle A_1OA_2=\angle A_2OA_3=\cdots=\angle A_nOA_1
  \end{align*}
  则$O$是$n$边形的费马点,即平面内任意点$P$,有
  \begin{align*}
    \sum_{i-1}^n |PA_i|\ge \sum_{i}^n |OA_i|
  \end{align*}
\end{theorem}
\begin{proof}[提示]
  并不是所有的凸多边形都有等分$360^\circ$的费马点,如长方形的费马点(长方形中心)与四个顶点的连线组成的夹角并不是4个相等直角。此定理说明的是,若等分$360^\circ$的点存在的话,则此点必是费马点。

  直接用几何证明很难,可以考虑用复数。以$O$为原点,$\vec{OA_1}$方向为实轴建立复平面坐标系,且不妨设$A_1,A_2,\cdots,A_n$按顺时针方向。记各顶点对应的复数分别为$z_i$,$\epsilon\equiv e^{\frac{2\pi}n i}$,则$z_i\epsilon^i$的幅角都相同,从而有
  \begin{align*}
    \sum_{i=1}^n |OA_i| = \sum_{i=1}^n |z_i|
    = \sum_{i=1}^n |z_i \epsilon^{i}| = \left| \sum_{i=1}^n z_i \epsilon^{i} \right|
  \end{align*}
  上式第2个等号是由于$\epsilon$是单位复数,取模符号中再乘以一个$\epsilon$不会影响其值;第3个等号是由于取模符号内各个复数的幅角都相同,即是同一个方向的向量,取模符号可外移。

  同样,对于$P$,记其对应的复数为$p$,则
  \begin{align*}
    \sum_{i=1}^n |PA_i|={}&\sum_{i=1}^n |p-z_i|=\sum_{i=1}^n |(p-z_i)\epsilon^{i}|\\
    \ge{}& \left| \sum_{i=1}^n (p-z_i)\epsilon^i \right| = \left| p \sum_{i=1}^n \epsilon^i - \sum_{i=1}^n z_i \epsilon^i \right| = \left| \sum_{i=1}^n z_i \epsilon^i \right|
  \end{align*}
  最后一个等号是由于$\sum_{i=1}^n \epsilon^i=0$。从而原不等式成立。关键之处在于几个向量旋转(相当于乘以一个幅角为旋转角度的单位复数)后,取模符号与求和符号可交换位置。另外,此处证明并没有用到凸这个性质。
\end{proof}


\section{三角函数--续}
\label{sec:trigonometric-functions-cont.}

\subsection{和差公式}
\label{sec:sum-and-difference-formula}

\begin{theorem}[和差公式,Sum and Difference Formulas]$\forall \alpha, \beta$,有
  \begin{align*}
    \sin(\alpha+\beta)=\sin\alpha\cos\beta + \cos\alpha\sin\beta\\
    \cos(\alpha+\beta)=\cos\alpha\cos\beta - \sin\alpha\sin\beta
  \end{align*}
\end{theorem}
\begin{proof}[提示]\mbox{}\par
  \begin{center}
    \begin{tikzpicture}[scale=1.0]
      \coordinate[label=below left:$O$](O)at(0,0);
      \coordinate[label=below right:$C$](C)at(5,0);
      \coordinate[label=right:$B$](B)at(5,3);
      \coordinate[label=above:$A$](A)at($(B)!.8!-90:(O)$);
      % \coordinate[label=below:$D$](D)at($(O)!(A)!(C)$);
      \coordinate(D)at($(O)!(A)!(C)$);
      \coordinate[label=above right:$E$](E)at($(C)!(A)!(B)$);

      \coordinate[label=above left:$D$](F)at($(E)!(O)!(A)$);

      \fill[color=blue!20](A)--(O)--(B)--cycle;
      \fill[color=red!20](B)--(O)--(C)--cycle;
      \tkzMarkRightAngle(A,B,O)\tkzMarkRightAngle(B,C,O)%\tkzMarkRightAngle(A,D,O)
      \tkzMarkRightAngle(A,E,B)\tkzMarkRightAngle(O,F,A)

      \draw pic["$\beta$",<->,draw=orange,angle eccentricity=1.6,angle radius=.6cm]{angle=C--O--B};
      \draw pic["$\alpha$",<->,draw=orange,angle eccentricity=1.6,angle radius=.6cm]{angle=B--O--A};
      \draw pic["$\beta$",<->,draw=orange,angle eccentricity=1.6,angle radius=.6cm]{angle=E--B--A};
      \draw pic["$\alpha+\beta$",<->,draw=orange,angle eccentricity=2.0,angle radius=.6cm]{angle=F--A--O};

      \draw(O)--(D)%node[midway,below]{$\cos(\alpha+\beta)$}
              --(C)--(B)node[midway,right]{$\cos\alpha\sin\beta$}
              --(A)node[midway,sloped,above]{$\sin\alpha$}
           (O)--(B)node[midway,sloped,above]{$\cos\alpha$};
      \draw[line width=2pt](A)--(O)node[pos=.55,sloped,above right]{\bfseries 1};
      \draw[dashed](B)--(E)node[midway,right]{$\sin\alpha\cos\beta$}
                      --(A);%node[midway,above]{\small $\sin\alpha\sin\beta$}
                      %--(D);
      % \draw[help lines,|<->|,sloped,above]($(E)+(1,0)$)--($(C)+(1,0)$)node[midway,color=black]{$\sin(\alpha+\beta)$};
      \tkzDrawPoints(O,A,B,C,E,F)
      \draw[dashed,help lines](O)--(F)node[midway,left,color=black]{$\sin(\alpha + \beta)$}
                                 --(A);%node[midway,above,color=black]{\small $\cos(\alpha + \beta)$};
      \draw[help lines,|<->|]($(O)-(0,.8)$)--($(C)-(0,.8)$)node[midway,below,color=black]{$\cos\alpha\cos\beta$};
      \draw[help lines,|<->|]($(F)+(0,.8)$)--($(A)+(0,.8)$)node[midway,above,color=black]{$\cos(\alpha+\beta)$};
      \draw[help lines,|<->|]($(A)+(0,.8)$)--($(E)+(0,.8)$)node[midway,above,color=black]{$\sin\alpha\sin\beta$};
    \end{tikzpicture}
  \end{center}
  根据上面的图,由$OA=1$及两个角度$\alpha$和$\beta$出发,可以算出图中各线段的长度。从而有
  \begin{align*}
    \sin(\alpha+\beta) ={}& AD = CB + BE = \cos\alpha\sin\beta + \sin\alpha\cos\beta\\
    \cos(\alpha+\beta) ={}& OD = OC - AE = \cos\alpha\cos\beta - \sin\alpha\sin\beta &&\qedhere
  \end{align*}
\end{proof}

\begin{example}
  用欧拉公式$e^{i\theta} = \cos\theta + i\sin\theta$证明和差公式。
\end{example}
\begin{proof}[提示]由欧拉公式,有
  \begin{align*}
    e^{i(\alpha + \beta)} = \cos(\alpha + \beta) + i\sin(\alpha + \beta)
  \end{align*}
  另一方面,由$e^{i(x+y)} = e^{ix}\cdot e^{iy}$,有
  \begin{align*}
    e^{i(\alpha + \beta)} ={}& e^{i\alpha}\cdot e^{i\beta}
    = \left( \cos\alpha + i\sin\alpha \right) \cdot \left( \cos\beta + i\sin\beta \right)\\
    ={}& (\cos\alpha\cos\beta - \sin\alpha\sin\beta) + i(\sin\alpha\cos\beta + \cos\alpha\sin\beta)
  \end{align*}
  对比两式,由虚部及实部分别相等,则有
  \begin{align*}
    \sin(\alpha+\beta) = \sin\alpha\cos\beta + \cos\alpha\sin\beta\\
    \cos(\alpha+\beta) = \cos\alpha\cos\beta - \sin\alpha\sin\beta    &\qedhere
  \end{align*}
\end{proof}

\begin{example}
  将和差公式中的$\beta$用$-\beta$替换,并注意到$\sin$与$\cos$的奇偶性,则有
  \begin{align*}
    \sin(\alpha-\beta) = \sin\alpha\cos(-\beta) + \cos\alpha\sin(-\beta) = \sin\alpha\cos\beta - \cos\alpha\sin\beta\\
    \cos(\alpha-\beta) = \cos\alpha\cos(-\beta) - \sin\alpha\sin(-\beta) = \cos\alpha\cos\beta + \sin\alpha\sin\beta
  \end{align*}
\end{example}

\begin{theorem}
  $\forall \alpha,\beta$,有
  \begin{align*}
    \tan(\alpha+\beta) = \frac{\tan\alpha+\tan\beta}{1-\tan\alpha\tan\beta}
  \end{align*}
\end{theorem}
\begin{proof}[提示]
  利用和差公式,有
  \begin{align*}
    &\begin{cases}
      \sin(\alpha+\beta) = \sin\alpha\cos\beta + \cos\alpha\sin\beta\\
      \cos(\alpha+\beta) = \cos\alpha\cos\beta - \sin\alpha\sin\beta
    \end{cases}\\
    \implies& \tan(\alpha+\beta)=\frac{\sin(\alpha+\beta)}{\cos(\alpha+\beta)}
              =\frac{\sin\alpha\cos\beta + \cos\alpha\sin\beta}{\cos\alpha\cos\beta - \sin\alpha\sin\beta}
  \end{align*}
  分子分母同时除以$\cos\alpha\cos\beta$可得。此处没有考虑分母不能为零的情况。
\end{proof}

\begin{example}
  令$\tan(\alpha+\beta)=t$,则
  \begin{align*}
    t\cdot\tan\alpha\tan\beta + \tan\alpha + \tan\beta = t
  \end{align*}
  若$\alpha+\beta=30^\circ$,则$\tan(\alpha+\beta) = \frac{\sqrt3}{3}$,从而
  \begin{align*}
    \frac{\sqrt3}{3}\tan\alpha\tan\beta + \tan\alpha + \tan\beta = \frac{\sqrt3}{3}
  \end{align*}
  即
  \begin{align*}
    \tan\alpha\tan\beta + \sqrt3(\tan\alpha + \tan\beta) = 1
  \end{align*}
  如令$\alpha = 12^\circ$,$\beta = 30^\circ - \alpha = 18^\circ$,则有
  \begin{align*}
    \tan12^\circ\tan18^\circ + \sqrt3(\tan12^\circ + \tan18^\circ) = 1
  \end{align*}
\end{example}

\subsection{和差化积}
\label{sec:product-to-sum}

\begin{theorem}[和差化积,积化和差,Product-to-Sum Trigonometric Formulas]
  $\forall \alpha, \beta$,有
  \begin{align*}
    2\sin\alpha\sin\beta = \cos(\alpha-\beta) - \cos(\alpha+\beta)\\
    2\cos\alpha\cos\beta = \cos(\alpha-\beta) + \cos(\alpha+\beta)
  \end{align*}
\end{theorem}
\begin{proof}[提示]
  利用和差公式,有
  \begin{align*}
    \cos(\alpha-\beta) = \cos\alpha\cos\beta + \sin\alpha\sin\beta \\
    \cos(\alpha+\beta) = \cos\alpha\cos\beta - \sin\alpha\sin\beta
  \end{align*}
  两式相加及两式相减,则有
  \begin{align*}
    2\sin\alpha\sin\beta = \cos(\alpha-\beta) - \cos(\alpha+\beta) \\
    2\cos\alpha\cos\beta = \cos(\alpha-\beta) + \cos(\alpha+\beta)
  \end{align*}
  在不同场合,需要使用不同的形式,所以这些公式既叫和差化积,反过来也叫积化和差,是两个不同的方向。
\end{proof}

\begin{example}
  同样,利用$\sin$的和差公式,有
  \begin{align*}
    \sin(\alpha+\beta) = \sin\alpha\cos\beta + \cos\alpha\sin\beta \\
    \sin(\alpha-\beta) = \sin\alpha\cos\beta - \cos\alpha\sin\beta
  \end{align*}
  两式相加或相减,则有
  \begin{align*}
    2 \sin\alpha\cos\beta = \sin(\alpha + \beta) + \sin(\alpha - \beta)\\
    2 \cos\alpha\sin\beta = \sin(\alpha + \beta) - \sin(\alpha - \beta)&\qedhere
  \end{align*}
\end{example}

\begin{example}[和差化积公式互推]
  实际上,上述的和差化积公式都是可以互推的。比如,由$2\sin\alpha\sin\beta = \cos(\alpha-\beta) - \cos(\alpha+\beta)$,可得
  \begin{align*}
    \cos\alpha\cos\beta ={} & \sin(90^\circ - \alpha)\sin(90^\circ - \beta) \\
                        ={} & \cos\left( (90^\circ - \alpha) - (90^\circ - \beta) \right) - 
                              \cos\left( (90^\circ - \alpha) + (90^\circ - \beta) \right)\\
                        ={} & \cos(\beta - \alpha) - \cos\left( 180^\circ - (\alpha + \beta) \right) \\
                        ={} & \cos(\alpha - \beta) + \cos(\alpha + \beta)
  \end{align*}
  其余各式类似。可见,和差化积公式的本质还是和差公式,只要知道和差公式,和差化积公式很容易就能推导出来。
\end{example}

\begin{example}
  令$x \equiv \alpha - \beta$,$y\equiv \alpha + \beta$,则有
  \begin{align*}
    \alpha = \frac{y+x}{2},\quad \beta = \frac{y-x}{2}
  \end{align*}
  代入和差化积公式,则可以得到和差化积的另外一种形式:
  \begin{align*}
    \cos x - \cos y ={}& 2 \sin\frac{y+x}{2} \sin\frac{y-x}{2}\\[5pt]
    \cos x + \cos y ={}& 2 \cos\frac{y+x}{2} \cos\frac{y-x}{2}
  \end{align*}
  类似的,也有
  \begin{align*}
    \sin x + \sin y ={}& 2 \sin\frac{x+y}{2}\cos\frac{x-y}{2}\\[5pt]
    \sin x - \sin y ={}& 2 \cos\frac{x+y}{2}\sin\frac{x-y}{2}
  \end{align*}
\end{example}

\begin{example}
  $\forall \alpha, \beta$,有
  \begin{align*}
    \tan\left(\frac{\alpha + \beta}{2}\right) = \frac{\sin\alpha + \sin\beta}{\cos\alpha + \cos\beta}
  \end{align*}
\end{example}
\begin{proof}[提示]
  由和差公式,容易得到
  \begin{align*}
    \sin\alpha + \sin\beta ={}& 2 \sin\frac{\alpha+\beta}2 \cos\frac{\alpha-\beta}2 \\[3pt]
    \cos\alpha + \sin\beta ={}& 2 \cos\frac{\beta+\alpha}2 \cos\frac{\beta-\alpha}2
  \end{align*}
  两式相除并注意到$\cos$是偶函数可得结论。
\end{proof}

\begin{example}\label{ex:sin(alpha-beta)=0}
  若$\sin\gamma\ne0$,则
  \begin{align*}
    \sin(\gamma-\alpha)\sin\beta = \sin(\gamma-\beta)\sin\alpha \iff
    \sin(\alpha - \beta) = 0
  \end{align*}
  若额外有$\sin\alpha\ne 0$,$\sin\beta\ne 0$,则有
  \begin{align*}
    \frac{\sin(\gamma-\alpha)}{\sin\alpha} = \frac{\sin(\gamma-\beta)}{\sin\beta} \iff
    \sin(\alpha - \beta) = 0
  \end{align*}
\end{example}
\begin{proof}[提示]
  应用公式$\sin(a + b) = \sin a \cos b + \sin b \cos a$,将$a= \gamma$、$b=-\alpha$及$a=\gamma$、$b=-\beta$代入,有
  \begin{align*}
    &&\sin(\gamma-\alpha)\sin\beta ={}& \sin(\gamma-\beta)\sin\alpha \\
    \iff &&
           \left(\sin\gamma\cos\alpha - \sin\alpha\cos\gamma\right) \sin\beta ={}& 
           \left(\sin\gamma\cos\beta -  \sin\beta \cos\gamma\right) \sin\alpha\\
    \iff&&
          \cos\alpha\sin\beta\sin\gamma - \cancel{\sin\alpha\sin\beta\cos\gamma}={}&
          \sin\alpha\cos\beta\sin\gamma - \cancel{\sin\alpha\sin\beta\cos\gamma}\\
    \iff&& \sin\gamma(\sin\alpha\cos\beta - \cos\alpha\sin\beta) ={}& 0\\
    \iff&& \sin\gamma\sin(\alpha-\beta) ={}& 0\\
    \iff&& \sin(\alpha - \beta) ={}& 0 & \qedhere
  \end{align*}
\end{proof}

\begin{example}
  $\forall x$,有
  \begin{align*}
    \frac{\cos x-\cos 3x}{\cos x+\cos 3x} = \tan x \tan 2x
  \end{align*}
\end{example}
\begin{proof}[提示]
  利用和差化积公式,有
  \begin{align*}
    \cos x - \cos 3x ={}& 2\sin\frac{x + 3x}{2} \sin\frac{3x-x}{2} = 2\sin2x\sin x\\[3pt]
    \cos x + \cos 3x ={}& 2\cos\frac{x + 3x}{2} \cos\frac{3x-x}{2} = 2\cos2x\cos x
  \end{align*}
  两式相除可得。
\end{proof}

\subsection{倍角公式}
\label{sec:double-angle-formula}

\begin{theorem}[倍角公式]$\forall \alpha$,有
  \begin{align*}
    \sin 2\alpha ={}& 2\sin\alpha\cos\alpha\\
    \cos 2\alpha ={}& \cos^2\alpha - \sin^2\alpha = 2\cos^2\alpha - 1 = 1 - 2\sin^2\alpha\\[3pt]
    \tan 2\alpha ={}& \frac{2\tan\alpha}{1 - \tan^2\alpha}
  \end{align*}
\end{theorem}
\begin{proof}[提示]
  由和差公式可得。两式相除后分子分母同除以$\cos^2\alpha$,可得到$\tan2\alpha$关于$\tan\alpha$的表达式。
\end{proof}

\subsection{半角公式}
\label{sec:half-angle-formula}

\begin{theorem}[正弦余弦半角公式]$\forall \alpha$,有
  \begin{align*}
    \sin\frac\alpha2 ={} \pm\sqrt{\frac{1-\cos\alpha}2},\quad
    \cos\frac\alpha2 ={} \pm\sqrt{\frac{1+\cos\alpha}2}
  \end{align*}
\end{theorem}
\begin{proof}[提示]
  应用积化和差公式,有
  \begin{align*}
    2\sin\alpha\sin\alpha = \cos(\alpha - \alpha) - \cos(\alpha + \alpha) = 1 - \cos2\alpha
  \end{align*}
  化简可得$\sin$的半角公式。类似地可得$\cos$的半角公式。

  或者,由$\cos$的倍角公式,有
  \begin{align*}
    \cos\alpha = 1 - 2\sin^2\frac\alpha2, \quad \cos\alpha = 2\cos^2\frac\alpha2 - 1
  \end{align*}
  化简后可$\sin$与$\cos$的半角公式。
\end{proof}

\begin{theorem}[正切半角公式]$\forall \alpha$,有
  \begin{align*}
    \tan\frac{\theta}{2} ={}& \frac{\sin\theta}{1 + \cos\theta} = \frac{1 - \cos\theta}{\sin\theta} = \frac{\tan\theta}{\sec\theta + 1} = \frac{\sec\theta - 1}{\tan\theta} = \frac{1}{\csc\theta + \cot\theta}\\
                         ={}& \csc\theta - \cot\theta
  \end{align*}  
\end{theorem}
\begin{proof}[提示]
  考虑$\theta$是锐角的情况,作图,其中的圆是单位圆。
  \begin{center}
    \begin{tikzpicture}[scale=1.0]
      \coordinate[label=below:$O$](O)at(0,0);
      \coordinate[label=below:$A$](A)at(-4,0);
      \coordinate[label=below left:$B$](B)at(4,0);
      \coordinate[label=above right:$C$](C)at(40:4);
      \coordinate[label=below:$D$](D)at($(A)!(C)!(B)$);
      \coordinate(E1)at(0,4);
      \tkzInterLL(A,C)(O,E1)\tkzGetPoint{E}\node[above left]at(E){$E$};
      \coordinate(F1)at($(C)!1!90:(O)$);
      \tkzInterLL(C,F1)(O,E)\tkzGetPoint{F}\node[above]at(F){$F$};
      \tkzInterLL(C,F1)(O,B)\tkzGetPoint{G}\node[below]at(G){$G$};
      \coordinate(H1)at($(G)!.1!90:(C)$);\tkzInterLL(G,H1)(C,B)\tkzGetPoint{H}\node[below]at(H){$H$};

      \coordinate(P1)at($(G)!1!-90:(B)$);\tkzInterLL(G,P1)(A,C)\tkzGetPoint{P}\node[above right]at(P){$P$};

      \draw[help lines,dashed](B)arc(0:180:4);%(A)
      \draw pic["$\theta$",<->,draw=orange,angle eccentricity=1.6,angle radius=.6cm]{angle=D--O--C};
      \draw pic["$\theta$",<->,draw=orange,angle eccentricity=1.6,angle radius=.6cm]{angle=E--F--C};
      \draw pic["$\theta/2$",<->,draw=orange,angle eccentricity=1.8,angle radius=.8cm]{angle=D--A--C};
      \draw pic["\small$\frac\theta2$",<->,draw=orange,angle eccentricity=1.8,angle radius=.8cm]{angle=D--C--B};
      \draw pic["\small$\frac\theta2$",<->,draw=orange,angle eccentricity=1.8,angle radius=.8cm]{angle=B--C--G};
      \draw pic["$\frac{\pi-\theta}2$",draw=orange,angle eccentricity=1.6,angle radius=.5cm,fill=blue!20]{angle=C--E--F};
      \draw pic["$\frac{\pi-\theta}2$",draw=orange,angle eccentricity=1.6,angle radius=.5cm,fill=blue!20]{angle=F--C--E};
      \draw pic["",draw=orange,angle eccentricity=1.6,angle radius=.5cm,fill=blue!20]{angle=A--E--O};
      \draw pic["",draw=orange,angle eccentricity=1.6,angle radius=.25cm,fill=blue!20]{angle=H--B--G};
      \draw pic["",draw=orange,angle eccentricity=1.6,angle radius=.25cm,fill=blue!20]{angle=G--H--B};
      \draw pic["",draw=orange,angle eccentricity=1.6,angle radius=.25cm,fill=blue!20]{angle=C--B--D};
      \draw pic["",draw=orange,angle eccentricity=1.6,angle radius=.4cm,fill=blue!20]{angle=G--C--P};
      \draw pic["",draw=orange,angle eccentricity=1.6,angle radius=.4cm,fill=blue!20]{angle=C--P--G};

      \tkzMarkRightAngle(A,D,C)\tkzMarkRightAngle(B,C,A)\tkzMarkRightAngle(E,O,A)
      \tkzMarkRightAngle(O,C,F)\tkzMarkRightAngle(H,G,C)\tkzMarkRightAngle(P,G,A)
      \draw[dashed](C)--(P)--(G)node[midway,sloped]{\tiny |||};
      \draw[dashed](B)--(H)--(G)node[midway,sloped]{\tiny |};
      \draw[line width=1.5pt](E)--(O)node[pos=.57]{$\tan\frac\theta2$};
      \draw(A)--(O)node[midway,below]{$1$}
              --(D)node[midway,below]{$\cos\theta$}
              --(B)--(C)--(E)--(A)
           (O)--(C)node[midway,fill=white]{$1$}
              --(D)node[pos=.6,sloped,below]{$\sin\theta$};
      \draw(E)--(F)node[midway,sloped]{\tiny ||}
              --(C)node[midway,sloped]{\tiny ||}node[midway,sloped,above]{$\cot\theta$}
              --(G)node[midway,sloped,above]{$\tan\theta$} node[midway,sloped]{\tiny |||}
              --(B)node[midway,sloped]{\tiny |};

      \draw[dashed, help lines, |<->|]($(O) - (5,0)$)--($(F)-(5,0)$)node[midway,sloped,fill=white]{$\csc\theta$};
      \draw[dashed, help lines, |<->|]($(O) - (0,1.2)$)--($(B)-(0,1.2)$)node[midway,sloped,fill=white]{$1$};
      \draw[dashed, help lines, |<->|]($(O) - (0,2)$)--($(G)-(0,2)$)node[midway,sloped,fill=white]{$\sec\theta$};

      \tkzDrawPoints(O,A,B,C,D,E,F,G,H,P)
    \end{tikzpicture}
  \end{center}
  如图,有
  \begin{align*}
    EO = \tan\frac\theta2 = \frac{CD}{AD} = \frac{\sin\theta}{1 + \cos\theta}
  \end{align*}
  在上式中,分子分母同时除以$\cos\theta$,则有
  \begin{align*}
    \frac{\sin\theta}{1 + \cos\theta} = \frac{\tan\theta}{\sec\theta + 1}
  \end{align*}
  或者在$\triangle GCP$中,两底角相等均为$(\pi - \theta)/2$,从而$PG=CG=\tan\theta$,从而由$\triangle APG$可知
  \begin{align*}
    \tan\frac\theta2 = \frac{PG}{AG} = \frac{\tan\theta}{\sec\theta + 1}
  \end{align*}
  同时除以$\sin\theta$,则有
  \begin{align*}
    \frac{\sin\theta}{1 + \cos\theta} = \frac{1}{\csc\theta + \cot\theta}
  \end{align*}
  同时乘以$1-\cos\theta$并由$\sin^2\theta = (1+\cos\theta)(1-\cos\theta)$,则有
  \begin{align*}
    \frac{\sin\theta}{1 + \cos\theta} = \frac{1 - \cos\theta}{\sin\theta}
  \end{align*}
  或者由$\triangle CDB$,可知
  \begin{align*}
    \tan\frac\theta2 = \frac{DB}{DC} = \frac{1 - \cos\theta}{\sin\theta}
  \end{align*}

  容易知道$\triangle FEC$两底角相等均为$\frac\pi2 - \frac\theta2$,故$FE=FC=\cot\theta$,从而
  \begin{align*}
    \tan\frac\theta2 = EO = FO - FE = FO - FC = \csc\theta - \cot\theta
  \end{align*}
  同样,$\triangle GBH$中两底角相等均为$\frac\pi2 - \frac\theta2$,故$GH=GB=\sec\theta - 1$,从而
  \begin{align*}
    \tan\frac\theta2 = \frac{GH}{GC} = \frac{\sec\theta - 1}{\tan\theta}&\qedhere
  \end{align*}
\end{proof}

\begin{example}证明:
  \begin{align*}
    \left( \frac{1 - \tan\frac\theta2}{1 + \tan\frac\theta2} \right)^2 =
    \frac{1 - \sin\theta}{1+\sin\theta}, \quad
    \left( \tan\frac\theta2 \right)^2 = \frac{1-\cos\theta}{1+\cos\theta}
  \end{align*}
\end{example}
\begin{proof}[提示]
由
  \begin{align*}
    &\tan\frac\theta2 = \frac{1-\cos\theta}{\sin\theta}\\
    \implies&
    1 - \tan\frac\theta2 = \frac{\sin\theta - 1+\cos\theta}{\sin\theta},\quad
    1 + \tan\frac\theta2 = \frac{\sin\theta + 1-\cos\theta}{\sin\theta}
  \end{align*}
  两式相除,则有
  \begin{align*}
    \frac{1 - \tan\frac\theta2}{1 + \tan\frac\theta2} = 
    \frac{\sin\theta - 1+\cos\theta}{\sin\theta + 1-\cos\theta}
  \end{align*}
  两边平方,则有
  \begin{align*}
    \left( \frac{1 - \tan\frac\theta2}{1 + \tan\frac\theta2} \right)^2 ={}&
     \left( \frac{\sin\theta - 1+\cos\theta}{\sin\theta + 1-\cos\theta} \right)^2\\
    \left( \frac{1 - \tan\frac\theta2}{1 + \tan\frac\theta2} \right)^2 ={}&
             \frac{2(1 - \sin\theta - \cos\theta + \sin\theta\cos\theta)}{2(1+\sin\theta-\cos\theta-\sin\theta\cos\theta)}\\
    \left( \frac{1 - \tan\frac\theta2}{1 + \tan\frac\theta2} \right)^2 ={}&
             \frac{(1 - \sin\theta)(1 - \cos\theta)}{(1+\sin\theta)(1-\cos\theta)}\\
    \left( \frac{1 - \tan\frac\theta2}{1 + \tan\frac\theta2} \right)^2 ={}&
             \frac{1 - \sin\theta}{1+\sin\theta}
  \end{align*}

  而由
  \begin{align*}
    \tan\frac\theta2 = \frac{\sin\theta}{1+\cos\theta} = \frac{1-\cos\theta}{\sin\theta}
  \end{align*}
  两式相乘,可得
  \begin{align*}
    \left(\tan\frac\theta2\right)^2 = \frac{1-\cos\theta}{1+\cos\theta}&\qedhere
  \end{align*}
\end{proof}

\begin{theorem}[万能公式]$\forall \alpha$,有
  \begin{align*}
    \sin\alpha = \frac{2\tan\frac\alpha2}{1+\tan^2\frac\alpha2}, \quad
    \cos\alpha = \frac{1-\tan^2\frac\alpha2}{1+\tan^2\frac\alpha2},\quad
    \tan\alpha = \frac{2\tan\frac\alpha2}{1-\tan^2\frac\alpha2}
  \end{align*}  
\end{theorem}
\begin{proof}[提示]
  利用倍角公式(和差公式的特殊情形),有
  \begin{align*}
    \sin\alpha ={}& 2\sin\frac\alpha2\cos\frac\alpha2
                    = \frac{2\sin\frac\alpha2\cos\frac\alpha2}{\cos^2\frac\alpha2 + \sin^2\frac\alpha2}
                    = \frac{2\tan\frac\alpha2}{1 + \tan^2\frac\alpha2}\\
    \cos\alpha={}& \cos^2\frac\alpha2 - \sin^2\frac\alpha2
                   =\frac{\cos^2\frac\alpha2 - \sin^2\frac\alpha2}{\cos^2\frac\alpha2 + \sin^2\frac\alpha2}
                   =\frac{1-\tan^2\frac\alpha2}{1+\tan^2\frac\alpha2}    
  \end{align*}
  两式相除,可得$\tan\alpha$关于$\tan\frac\alpha2$的表达式。
\end{proof}

\section{解三角形}
\label{sec:solving-triangle}

\begin{example}如图,求$\angle\alpha$。
  \begin{center}
    \begin{tikzpicture}[scale=1.0]
      \begin{scope}
        \coordinate(B) at (0,0);
        \coordinate(C) at (4,0);
        \coordinate(D1) at ($(B)!1!20:(C)$); \coordinate(D2) at ($(C)!1!-110:(B)$);\tkzInterLL(B,D1)(C,D2)\tkzGetPoint{D}
        \coordinate(P1)at($(C)!1!-40:(B)$);\coordinate(P2)at($(D)!1!-80:(C)$);\tkzInterLL(C,P1)(D,P2)\tkzGetPoint{P}
        \coordinate(A1)at($(B)!1!30:(P)$);\tkzInterLL(B,A1)(D,P)\tkzGetPoint{A}
        \coordinate(E1)at($(D)!1!-50:(P)$);\tkzInterLL(B,P)(D,E1)\tkzGetPoint{E}
        \tkzInterLL(C,P)(B,D)\tkzGetPoint{Q}
        \draw(A)--(B)--(C)--(D)--(E)--cycle;
        \draw(A)--(D) (D)--(B)--(E) (C)--(P);
        \tkzDrawPoints(A,B,C,D,E,P,Q)
        \node[left]at(A){$A$};
        \node[below left]at(B){$B$};
        \node[below right]at(C){$C$};
        \node[right]at(D){$D$};
        \node[above]at(E){$E$};
        \path (P)++(-.1,-.2)node[left]{$P$};
        \node[above]at(Q){$Q$};
        \draw pic["$30^\circ$",<->,draw=orange,angle eccentricity=1.6,angle radius=.6cm]{angle=P--B--A};
        \draw pic["$\alpha$",<->,draw=orange,angle eccentricity=1.6,angle radius=.6cm]{angle=D--B--P};
        \draw pic["$20^\circ$",<->,draw=orange,angle eccentricity=1.6,angle radius=.8cm]{angle=C--B--D};
        \draw pic["$40^\circ$",<->,draw=orange,angle eccentricity=1.6,angle radius=.6cm]{angle=P--C--B};
        \draw pic["$70^\circ$",<->,draw=orange,angle eccentricity=1.7,angle radius=.3cm]{angle=D--C--P};
        \draw pic["$50^\circ$",<->,draw=orange,angle eccentricity=1.8,angle radius=.4cm]{angle=B--D--C};
        \draw pic["$30^\circ$",<->,draw=orange,angle eccentricity=1.6,angle radius=.6cm]{angle=P--D--B};
        \draw pic["$50^\circ$",<->,draw=orange,angle eccentricity=1.6,angle radius=.6cm]{angle=E--D--P};
        \draw pic["$30^\circ$",<->,draw=orange,angle eccentricity=1.6,angle radius=.6cm]{angle=C--P--D};
        \draw pic["$\theta$",<->,draw=orange,angle eccentricity=1.6,angle radius=.6cm]{angle=D--A--E};
      \end{scope}
      \begin{scope}[shift={(7.3,0)}]
        \coordinate(B) at (0,0);
        \coordinate(C) at (4,0);
        \coordinate(D1) at ($(B)!1!20:(C)$); \coordinate(D2) at ($(C)!1!-110:(B)$);\tkzInterLL(B,D1)(C,D2)\tkzGetPoint{D}
        \coordinate(P1)at($(C)!1!-40:(B)$);\coordinate(P2)at($(D)!1!-80:(C)$);\tkzInterLL(C,P1)(D,P2)\tkzGetPoint{P}
        \coordinate(A1)at($(B)!1!30:(P)$);\tkzInterLL(B,A1)(D,P)\tkzGetPoint{A}
        \coordinate(E1)at($(D)!1!-50:(P)$);\tkzInterLL(B,P)(D,E1)\tkzGetPoint{E}
        \tkzInterLL(C,P)(B,D)\tkzGetPoint{Q}

        \coordinate[label=below:$F$](F) at ($(D)!1!60:(P)$);

        \draw pic["$30^\circ$",<->,draw=orange,angle eccentricity=1.6,angle radius=.6cm]{angle=P--B--A};
        \draw pic["$\alpha$",<->,draw=orange,angle eccentricity=1.6,angle radius=.6cm]{angle=D--B--P};
        \draw pic["$20^\circ$",<->,draw=orange,angle eccentricity=1.6,angle radius=.8cm]{angle=C--B--D};
        \draw pic["$40^\circ$",<->,draw=orange,angle eccentricity=1.6,angle radius=.6cm]{angle=P--C--B};
        \draw pic["$70^\circ$",<->,draw=orange,angle eccentricity=1.7,angle radius=.3cm]{angle=D--C--P};
        \draw pic["$50^\circ$",<->,draw=orange,angle eccentricity=1.8,angle radius=.4cm]{angle=B--D--C};
        \draw pic["$30^\circ$",<->,draw=orange,angle eccentricity=1.6,angle radius=.6cm]{angle=P--D--B};
        \draw pic["$50^\circ$",<->,draw=orange,angle eccentricity=1.6,angle radius=.6cm]{angle=E--D--P};
        \draw pic["$30^\circ$",<->,draw=orange,angle eccentricity=1.6,angle radius=.6cm]{angle=C--P--D};
        \draw pic["$\theta$",<->,draw=orange,angle eccentricity=1.6,angle radius=.6cm]{angle=D--A--E};

        \draw pic["",draw=blue,fill=blue!20,angle eccentricity=1.6,angle radius=.3cm]{angle=F--D--C};
        \draw pic["",draw=blue,fill=blue!20,angle eccentricity=1.6,angle radius=.5cm]{angle=C--F--D};
        \draw pic["",draw=blue,fill=blue!20,angle eccentricity=1.6,angle radius=.4cm]{angle=F--B--C};
        \draw pic["",draw=blue,fill=blue!20,angle eccentricity=1.6,angle radius=.4cm]{angle=C--B--D};

        \draw(A)--(B)--(C)--(D)--(E)--cycle;
        \draw(A)--(D) (D)--(B)--(E) (C)--(P);
        \tkzDrawPoints(A,B,C,D,E,P,Q)
        \node[left]at(A){$A$};
        \node[below left]at(B){$B$};
        \node[below right]at(C){$C$};
        \node[right]at(D){$D$};
        \node[above]at(E){$E$};
        \path (P)++(-.1,-.2)node[left]{$P$};
        \node[above]at(Q){$Q$};

        \draw[help lines, dashed](P)--(F)--(D) (C)--(F)--(B);
        \tkzDrawPoints(F)
      \end{scope}
    \end{tikzpicture}
  \end{center}
\end{example}
\begin{proof}[提示]
  首先,三角形$\triangle BCD$和$\triangle CDP$是固定的,从而四边形$BCDP$是固定的,即$\angle\alpha$是固定的,从而$\triangle BDA$及$\triangle BDE$也是固定的,即图形若有解的话,其解是唯一的。

  可以应用三角函数。容易知道$\angle BQP = 60^\circ$,$\angle BPQ = 120^\circ - \alpha$,从而应用正弦定理,有
  \begin{align*}
    \frac{\sin (120^\circ - \alpha)}{\sin\alpha} ={} &\frac{BQ}{PQ} = \frac{BQ}{CQ}\cdot\frac{CQ}{DQ}\cdot\frac{DQ}{PQ}\\
    ={}& \frac{\sin40^\circ}{\sin20^\circ}\cdot \frac{\sin50^\circ}{\sin70^\circ}\cdot \frac{\sin30^\circ}{\sin30^\circ}\\
    ={}& \frac{\sin40^\circ \sin50^\circ}{\sin20^\circ\sin70^\circ} = \frac{2\sin40^\circ \cos40^\circ}{2\sin20^\circ\cos20^\circ} = \frac{\sin80^\circ }{\sin40^\circ} \\
  \end{align*}
  比较两边,分子与分母的角度和都是$120^\circ$,可猜测$\alpha = 40^\circ$。事实上,由例~\ref{ex:sin(alpha-beta)=0},有
  \begin{align*}
    \sin(\alpha - 40^\circ) = 0
  \end{align*}
  从而$\alpha = 40^\circ + k\pi, k\in\mathcal{Z}$。又由$0 < \alpha < \pi$,可知$\alpha = 40^\circ$。

  或者用三角形全等来做。以$PD$为边,向$BC$方向做等边三角形$PDF$。如题图中的右图,可以找到几个全等的三角形:$\triangle PCF \cong \triangle PCD$(SAS,Side-Angle-Side,边角边),从而
  \begin{align*}
    &FC=DC\\
    \implies & \angle CFD = \angle CDF = 30^\circ + 50^\circ - 60^\circ = 20^\circ = CBD\\
    \implies & \text{从而$CFBD$四点共圆}\\
    \implies & \angle FBC = \angle FDC = 20^\circ\\
    \implies & \angle FBD = 20^\circ + 20^\circ = 40^\circ    
  \end{align*}
  此外,$\triangle BDP \cong \triangle BDF$(以$D$为顶点的SAS),从而
  \begin{align*}
    \alpha = \angle PBD = \angle FBD = 40^\circ &\qedhere
  \end{align*}
\end{proof}

\begin{example}
  上题中,$B$,$C$,$D$,$F$其实是正18边形的其中四个顶点。如图~\ref{fig:bfcd-and-18-polygon},做$BFCD$的外接圆,不妨设此圆为单位圆。以$D$为第一个顶点$V_0$,逆时针找出外接圆上正18边形的其余17个顶点$V_i$,其中$(i=1,2,\cdots,17)$,则
  \begin{figure}[htbp]
    \centering
    \begin{tikzpicture}[scale=1.0]
      \coordinate[label=below left:$O$](O) at (0,0);
      \foreach \x in{0,1,2,3,4,5,6,7,8,9,10,11,12,13,14,15,16,17}{%
        \coordinate(V\x)at(360/18*\x:4.5);
        \tkzDrawPoint(V\x)
      }
      \foreach \x in{1,2,3,5,6,7,8,9,10,12,13,15,17}{%
        \node at(360/18*\x:5){$V_{\x}$};
      }
      \foreach \x in{0,11,14,16}{%
        \node at(360/18*\x:5.5){$(V_{\x})$};
      }
      \node[below right]at(V4){$V_4$};
      \draw[dashed, help lines](0,0)circle(4.5);
      \tkzDrawPoint(O)
      % \node[above right]at(V3){$V_3$};

      % \coordinate(B) at (0,0);
      % \coordinate(C) at (4,0);
      \coordinate(B) at (V11);\coordinate(C)at(V16);
      \coordinate(D1) at ($(B)!1!20:(C)$); \coordinate(D2) at ($(C)!1!-110:(B)$);\tkzInterLL(B,D1)(C,D2)\tkzGetPoint{D}
      \coordinate(P1)at($(C)!1!-40:(B)$);\coordinate(P2)at($(D)!1!-80:(C)$);\tkzInterLL(C,P1)(D,P2)\tkzGetPoint{P}
      \coordinate(A1)at($(B)!1!30:(P)$);\tkzInterLL(B,A1)(D,P)\tkzGetPoint{A}
      \coordinate(E1)at($(D)!1!-50:(P)$);\tkzInterLL(B,P)(D,E1)\tkzGetPoint{E}
      \tkzInterLL(C,P)(B,D)\tkzGetPoint{Q}

      \coordinate[label=below:$F$](F) at ($(D)!1!60:(P)$);

      \draw pic["$30^\circ$",<->,draw=orange,angle eccentricity=1.6,angle radius=.6cm]{angle=P--B--A};
      \draw pic["$\alpha$",<->,draw=orange,angle eccentricity=1.6,angle radius=.6cm]{angle=D--B--P};
      \draw pic["$20^\circ$",<->,draw=orange,angle eccentricity=1.6,angle radius=.8cm]{angle=C--B--D};
      \draw pic["$40^\circ$",<->,draw=orange,angle eccentricity=1.6,angle radius=.6cm]{angle=P--C--B};
      \draw pic["$70^\circ$",<->,draw=orange,angle eccentricity=1.7,angle radius=.3cm]{angle=D--C--P};
      \draw pic["$50^\circ$",<->,draw=orange,angle eccentricity=1.8,angle radius=.4cm]{angle=B--D--C};
      \draw pic["$30^\circ$",<->,draw=orange,angle eccentricity=1.6,angle radius=.6cm]{angle=P--D--B};
      \draw pic["$50^\circ$",<->,draw=orange,angle eccentricity=1.6,angle radius=.6cm]{angle=E--D--P};
      \draw pic["$30^\circ$",<->,draw=orange,angle eccentricity=1.6,angle radius=.6cm]{angle=C--P--D};
      \draw pic["$\theta$",<->,draw=orange,angle eccentricity=1.6,angle radius=.6cm]{angle=D--A--E};

      \draw pic["",draw=blue,fill=blue!20,angle eccentricity=1.6,angle radius=.3cm]{angle=F--D--C};
      \draw pic["",draw=blue,fill=blue!20,angle eccentricity=1.6,angle radius=.5cm]{angle=C--F--D};
      \draw pic["",draw=blue,fill=blue!20,angle eccentricity=1.6,angle radius=.4cm]{angle=F--B--C};
      \draw pic["",draw=blue,fill=blue!20,angle eccentricity=1.6,angle radius=.4cm]{angle=C--B--D};
      \draw pic["",draw=blue,fill=blue!20,angle eccentricity=1.6,angle radius=.4cm]{angle=O--D--B};

      \draw(A)--(B)--(C)--(D)--(E)--cycle;
      \draw(A)--(D) (D)--(B)--(E) (C)--(P);
      \tkzDrawPoints(A,B,C,D,E,P,Q)
      \node[below left]at(A){$A$};
      \node[below left]at(B){$B$};
      \node[below right]at(C){$C$};
      \node[right]at(D){$D$};
      \node[above]at(E){$E$};
      \path (P)++(-.1,-.2)node[left]{$P$};
      \node[above]at(Q){$Q$};

      \draw[help lines, dashed](P)--(F)--(D) (C)--(F)--(B);
      \tkzDrawPoints(F)

      \draw[dashed,help lines](V9)--(O)--(D) (V6)--(P)--(V7)--(A)--(V8);

      \tkzInterLL(C,P)(F,D)\tkzGetPoint{X}\tkzMarkRightAngle(Q,X,D)
    \end{tikzpicture}
    
    \caption{$BFCD$与正18边形}
    \label{fig:bfcd-and-18-polygon}
  \end{figure}

  \begin{enumerate}
  \item $\angle DCB = 70^\circ + 40^\circ = 110^\circ\implies B$与$V_{11}$重合。
  \item $\angle FCD = 180^\circ - 20^\circ - 20^\circ = 140^\circ \implies F$与$V_{14}$重合。
  \item $\angle FCD = 140^\circ$且$FC=DC\implies C$与$V_{16}$重合。
  \item $\angle DCP = 70^\circ\implies CP$与圆相交于$V_7$。
  \item $DV_9$与$CV_7$都过圆心$\implies DV_0$与$CP$相交于外接圆的圆心$O$。
  \item 根据$\angle BDE=80^\circ\implies DE$与圆相交于$V_3$。
  \item 根据$\angle CBE=60^\circ\implies BE$与圆相交于$V_4$。
  \item 根据$\angle BDP=30^\circ\implies DP$与圆相交于$V_8$。
  \end{enumerate}
  其余性质不一一列举,有兴趣的可自行再去挖掘。
\end{example}

\begin{example}
  求上题中的$\angle\theta$。
\end{example}
\begin{proof}[提示]
  $\theta=50^\circ$,类似求$\alpha$,用三角函数求解。

  在$\triangle APE$中,由于$\angle APE = \angle BPD = 180^\circ - \alpha - 30^\circ = 110^\circ$,从而$\angle AEP = 70^\circ - \theta$。由正弦定理,有
  \begin{align*}
    &\frac{\sin\theta}{\sin\angle AEP} = \frac{EP}{AP}\\[3pt]
    \implies&
        \frac{\sin\theta}{\sin(70^\circ -\theta)} =
        \frac{EP}{DP} \cdot \frac{DP}{BP} \cdot \frac{BP}{AP}\\
    \implies &
        \frac{\sin\theta}{\sin(70^\circ -\theta)} =
        \frac{\sin50^\circ}{\sin60^\circ} \cdot \frac{\sin40^\circ}{\sin30^\circ} \cdot \frac{\sin80^\circ}{\sin30^\circ}\\
    \implies &
        \sin60^\circ \sin30^\circ \sin30^\circ \sin\theta = \sin50^\circ\sin40^\circ\sin80^\circ\sin(70^\circ-\theta)\\
    \implies &
        \sqrt3\sin\theta = 8\cos40^\circ\sin40^\circ\sin80^\circ\sin(70^\circ-\theta)\\
    &\text{利用$\sin$的倍角公式}\\
    \implies & \sqrt3\sin\theta = 2\sin^280^\circ\sin(70^\circ-\theta)\\
    &\text{积化和差,} \sin\alpha\sin\alpha = \cos(\alpha-\alpha)-\cos(\alpha+\alpha) = 1-\cos2\alpha\\
    \implies & \sqrt3\sin\theta = 2(1-\cos160^\circ)\sin(70^\circ-\theta)\\
    \implies & \sqrt3\sin\theta = 2(1+\cos20^\circ)\sin(70^\circ-\theta)\\
    \implies & \sqrt3\sin\theta = 2\sin(70^\circ-\theta)+2\cos20^\circ\sin(70^\circ-\theta)\\
    \implies & \sqrt3\sin\theta = 2\sin(70^\circ-\theta)+\sin(20^\circ+70^\circ-\theta)-\sin(20^\circ-70^\circ+\theta)\\
    \implies & \sqrt3\sin\theta = 2\sin(70^\circ-\theta)+\cos\theta-\sin(\theta-50^\circ)\\
    \implies & \sqrt3\sin\theta -\cos\theta  = 2\sin(70^\circ-\theta)-\sin(\theta-50^\circ)\\
    &\text{和差公式}\\
    \implies & 2\sin(\theta-30^\circ) = 2\sin(70^\circ-\theta)-\sin(\theta-50^\circ)\\
    \implies & 2\left(\sin(\theta-30^\circ) -\sin(70^\circ-\theta)\right) + \sin(\theta-50^\circ) = 0\\
    &\text{和差化积}\\
    \implies & 4\cos20^\circ\sin(\theta-50^\circ) + \sin(\theta-50^\circ) = 0 \\
    \implies & \sin(\theta-50^\circ) = 0\\
    \implies & \theta = 50^\circ + k\pi,\quad k\in\mathcal{Z}
  \end{align*}
  又$0<\theta<\pi$,可知$\theta = 50^\circ$,即$AE=DE$。
\end{proof}


\begin{example}\label{ex:tan20-tan30-eq-tan10-tan50}
  Proof that $\tan 20\cdot\tan30=\tan10\cdot\tan50$.
\end{example}

\begin{example}
  Proof that $\tan20\cdot\tan30\cdot\tan40 = \tan10$.
\end{example}
\begin{proof}
  From example~\ref{ex:tan20-tan30-eq-tan10-tan50}, we have
  \begin{align*}
    \tan20\cdot\tan30\cdot\tan40 = & \tan10\cdot\tan50\cdot\tan40\\
    = & \tan10\cdot\tan50\cdot\cot50 = \tan10 \qed
  \end{align*}
\end{proof}


\begin{example}
  Proof that $\tan50 + \tan60 + \tan70= \tan50 \cdot \tan60 \cdot \tan70=\tan80$.
\end{example}
\include{five-models}           %section under triangle
\include{triangle-inequality}
% \include{fermat-point}


\chapter{面积}
\label{chap:area}

\section{基本公式}
\label{sec:basic-area-formula}

% \begin{table}[htbp]
%   \centering
%   \renewcommand{\arraystretch}{1.2}
%   % The > directive lets you basically inject the contained code
%   % before each entry in that column.
%   % 
%   % The point of \arraybackslash is to return \\ to its original
%   % meaning because the \centering command alters this and could
%   % possibly give you a noalign error during compilation.
%   % \newcolumntype{C}{ >{\centering\arraybackslash} m{1cm} }
%   % \newcolumntype{D}{ >{\centering\arraybackslash} m{2cm} }
%   % \newcolumntype{E}{ >{\centering\arraybackslash} m{4cm} }
%   % \newcolumntype{F}{ >{\centering\arraybackslash} m{6cm} }

%   % define "struts", as suggested by Claudio Beccari in
%   % a piece in TeX and TUG News, Vol. 2, 1993.
%   % \newcommand\Tstrut{\rule{0pt}{2.6ex}}         % = `top' strut
%   % \newcommand\Bstrut{\rule[-0.9ex]{0pt}{0pt}}   % = `bottom' strut
  
%   \begin{tabular}{cccl}
%     \hline
%     序号&图形 & 示例 & 公式\\
%     \hline\\[2pt]
%     1 & 三角形 & \tikz{\draw(2,2)--(0,0)--(3,0)--(2,2)--(2,0)
%                             (1.8,0)--(1.8,.2)--(2,.2);
%                  \draw[|<->|](0,-.3)--(3,-.3)node[midway,fill=white]{$a$};
%                  \draw[|<->|](3.5, 0)--(3.5, 2)node[midway,fill=white]{$h$};
%                  } & $S=\frac12 ah$\\
%     2 & 矩阵 & \tikz{
%                \draw(0,0)rectangle(3,2);
%                \draw[|<->|](0,-.3)--(3,-.3)node[midway,fill=white]{$a$};
%                \draw[|<->|](3.5,0)--(3.5,2)node[midway,fill=white]{$b$};
%                } & $S=ab$\\
%     \hline
%   \end{tabular}
%   \caption{基本面积公式}
%   \label{tab:basic-area-formula}
% \end{table}

一些基本图形的面积公式列表如下:
\begin{center}
\begin{tikzpicture}
  \begin{scope}[shift={(0,0)}]
    \draw(2,2)--(0,0)--(3,0)--(2,2)--(2,0)
    (1.8,0)--(1.8,.2)--(2,.2);
    \draw[|<->|](0,-.3)--(3,-.3)node[midway,fill=white]{$a$};
    \draw[|<->|](3.5, 0)--(3.5, 2)node[midway,fill=white]{$h$};
    \node[right]at (4,1){$S=\dfrac12 ah$};
  \end{scope}

  \begin{scope}[shift={(8,0)}]
    \draw(0,0)rectangle(3,2);
    \draw[|<->|](0,-.3)--(3,-.3)node[midway,fill=white]{$a$};
    \draw[|<->|](3.5,0)--(3.5,2)node[midway,fill=white]{$b$};
    \node[right]at (4,1){$S=ab$};
  \end{scope}

  \begin{scope}[shift={(0,-4)}]
    \draw[|<->|](1,2.3)--(2.5,2.3)node[midway,fill=white]{$b$};
    \draw[|<->|](0,-.3)--(3,-.3)node[midway,fill=white]{$a$};
    \draw[|<->|](3.5,0)--(3.5,2)node[midway,fill=white]{$h$};
    \draw(0,0)--(3,0)--(2.5,2)--(1,2)--cycle;
    \node[right]at (4,1){$S=\dfrac12 (a+b)h$};
  \end{scope}

  \begin{scope}[shift={(8,-4)}]
    \draw(1.5,1)circle(1.5);
    \draw[->](1.5,1)--(3,1)node[midway,above]{$r$};
    \node[right]at(4,1){$S=\pi r^2$};
  \end{scope}

  \begin{scope}[shift={(0,-8)}]
    \draw(0,1)node(O){}--(3,1)node(A){} arc(0:30:3) node(B){}--cycle;
    \draw[|<->|](0,.6)--(3,.6)node[midway,below]{$r$};
    \draw pic["$\theta$",<->,draw=orange,angle eccentricity=1.6,angle radius=.6cm]{angle=A--O--B};
    \node[right]at(4,1.5){$S=\dfrac12 \theta r^2$};
  \end{scope}
\end{tikzpicture}
\end{center}

\begin{example}
  已知正方形边长,求阴影面积。

  \centering
  \begin{tikzpicture}[scale=2.0]
    \draw[fill=blue!30](0,0)--(1,0)arc(-90:-180:.5)arc(-90:-180:.5)--cycle;
    \draw(0,0)--(1,0)--(1,1)--(0,1)--cycle;
    \draw(1,0)arc(-90:-270:.5) (1,1)arc(0:-180:.5);
  \end{tikzpicture}
\end{example}

\hints 作辅助线,分别求各部分面积。

\begin{center}
  \begin{tikzpicture}[scale=2.0]
    % \draw[fill=blue!30](0,0)--(1,0)arc(-90:-180:.5)arc(-90:-180:.5)--cycle;
    \draw(0,0)--(1,0)--(1,1)--(0,1)--cycle;
    \draw(1,0)arc(-90:-270:.5) (1,1)arc(0:-180:.5);
    \draw[dashed,pattern=north east lines,pattern color=blue!30](.5,.5)rectangle(1,1);
  \end{tikzpicture}
\end{center}

\begin{example}
  求阴影面积

  \centering
  \begin{tikzpicture}[scale=1.0]
    \begin{scope}[shift={(0,0)}]
      \draw(0,0)rectangle(2,2);
      \draw[fill=blue!30,even odd rule]
           ([shift=(0:2)]0,0)arc(0:90:2)
           ([shift=(90:2)]2,0)arc(90:180:2)
           ([shift=(180:2)]2,2)arc(180:270:2)
           ([shift=(270:2)]0,2)arc(270:360:2);
    \end{scope}

    \begin{scope}[shift={(3,0)}]
      \fill[color=blue!30] (0,0)--(2,0) arc(-90:-120:2) arc(-60:-90:2);
      \fill[color=blue!30] (2,0)--(2,2) arc(0:-30:2) arc(30:0:2);
      \fill[color=blue!30] (2,2)--(0,2) arc(90:60:2) arc(120:90:2);
      \fill[color=blue!30] (0,2)--(0,0) arc(180:150:2) arc(210:180:2);
      \draw(0,0)rectangle(2,2);
      \draw([shift=(0:2)]0,0)arc(0:90:2)
           ([shift=(90:2)]2,0)arc(90:180:2)
           ([shift=(180:2)]2,2)arc(180:270:2)
           ([shift=(270:2)]0,2)arc(270:360:2);
    \end{scope}

    \begin{scope}[shift={(6,0)}]
      \fill[color=blue!30]
           ([shift=(0:2)]0,0)arc(0:90:2)
           ([shift=(90:2)]2,0)arc(90:180:2)
           ([shift=(180:2)]2,2)arc(180:270:2)
           ([shift=(270:2)]0,2)arc(270:360:2);
      \fill[color=white,even odd rule]
           ([shift=(0:2)]0,0)arc(0:90:2)
           ([shift=(90:2)]2,0)arc(90:180:2)
           ([shift=(180:2)]2,2)arc(180:270:2)
           ([shift=(270:2)]0,2)arc(270:360:2);
      \draw(0,0)rectangle(2,2);
      \draw([shift=(0:2)]0,0)arc(0:90:2)
           ([shift=(90:2)]2,0)arc(90:180:2)
           ([shift=(180:2)]2,2)arc(180:270:2)
           ([shift=(270:2)]0,2)arc(270:360:2);
    \end{scope}
  \end{tikzpicture}
\end{example}
\begin{proof}[解]
  交点将圆弧平均分成三段,每段对应于$30^\circ$的圆弧。\hints 图中三角形是等边三角形。

  \begin{center}
  \begin{tikzpicture}[scale=1.0]
    \draw(0,0)rectangle(2,2);
    \draw([shift=(0:2)]0,0)arc(0:90:2)
         ([shift=(90:2)]2,0)arc(90:180:2)
         ([shift=(180:2)]2,2)arc(180:270:2)
         ([shift=(270:2)]0,2)arc(270:360:2);
    \coordinate (A) at (60:2);
    \draw[very thick](0,0)--(2,0)--(A)--cycle;
    \fill[pattern=north west lines,pattern color=blue!30](0,0)--(A)arc(60:90:2)--cycle;
    \fill[pattern=north east lines,pattern color=blue!30](2,0)--(2,2)arc(90:120:2)--cycle;
  \end{tikzpicture}
  \end{center}

  可按图中分割方式求解题目第二个图中阴影面积。
\end{proof}

\begin{example}
  由两个圆弧组成的图形称为半月形(lune)。求图中阴影部分半月形的面积。

  \centering
  \begin{tikzpicture}[scale=1.5]
    \coordinate (O) at (0,0);
    \coordinate (A) at (120:1); 
    \coordinate (B) at (60:1);
    \coordinate (C) at ($.5*(A) + .5*(B)$);
    \draw[fill=blue!30](B)arc(0:180:.5);
    \draw[fill=white](1,0)arc(0:180:1);
    \draw[dashed](A)--(B) node[midway,below]{$1$};
    \draw(-1,0)--(1,0) node[midway, below]{$2$};;
  \end{tikzpicture}
\end{example}
\begin{proof}[提示]
  可转换为求弓形面积。

  \begin{center}
    \begin{tikzpicture}[scale=2.0]
      \coordinate (O) at (0,0);
      \coordinate (A) at (120:1);
      \coordinate (B) at (60:1);
      \draw[pattern=north west lines, pattern color=blue!30](A)--(B) arc(60:120:1);
      \draw(A)--(O)--(B);
    \end{tikzpicture}
  \end{center}

  弓形面积可由上图中分割方式求出,即扇形面积减三角形面积。
\end{proof}

\begin{example}
  图中四边形为单位正方形,求阴影面积。

  \centering
  \begin{tikzpicture}[scale=1.0]
    \draw[fill=blue!30,even odd rule](0,0)arc(-90:0:2)--(2,0)arc(0:180:1);
    \draw(2,0)--(0,0)--(0,2)--(2,2);
  \end{tikzpicture}
\end{example}
\begin{proof}[解]
  除了积分,暂时未发现有其它好方法。(超过中小学知识?)
  \begin{align*}
    S&=\int_{x=0}^1 \left| \sqrt{0.5^2 - (x-0.5)^2} - \left(1-\sqrt{1-x^2}\right)\right| \dx\qedhere
  \end{align*}

  % 假设可以求出大圆弧将小圆弧切割成的两部分,小部分的弧角是$x$,则是可
  % 以求出面积与$x$之间的关系。连接圆弧交点与正方形右下角顶点,可以求得
  % 弓形面积
\end{proof}

\begin{example}
  已知三角形面积为$1$,求阴影面积。其中第一个图的线段长度的比例如图所示。后两图,若其中线段长度的比例都已知,则阴影面积又该如何求解?

  \centering
  \begin{tikzpicture}[scale=1.0]
    \begin{scope}[shift={(0,0)}]
      \coordinate (A) at (1.8,2);
      \coordinate (B) at (0,0);
      \coordinate (C) at (3,0);
      \coordinate (D) at ($.6*(C)$);
      \coordinate (E) at ($2/3*(A)+1/3*(C)$);
      \draw(A)--(B)--(C)--cycle (D)--(E);
      \fill[pattern=north west lines,pattern color=blue!30](C)--(D)--(E)--cycle;

      \coordinate (B') at ($(0,-.4) + (B)$);
      \coordinate (C') at ($(0,-.4) + (C)$);
      \coordinate (D') at ($(0,-.4) + (D)$);
      \draw[|<->|](B')--(D') node[midway,below]{$3x$};
      \draw[|<->|](D')--(C') node[midway,below]{$2x$};

      \coordinate (P)  at ($({.4*2/sqrt(5.44)}, {.4*1.2/sqrt(5.44)})$);
      \coordinate (C') at ($(P) + (C)$);
      \coordinate (E') at ($(P) + (E)$);
      \coordinate (A') at ($(P) + (A)$);
      \draw[|<->|](C')--(E') node[pos=.75,above right,sloped]{$2y$};
      \draw[|<->|](E')--(A') node[pos=.75,above right,sloped]{$y$};
    \end{scope}

    \begin{scope}[shift={(4.5,0)}]
      \coordinate (B) at (0,0);
      \coordinate (C) at (3,0);
      \coordinate (A) at (2,2);
      \coordinate (D) at ($.4*(B) + .6*(C)$);
      \coordinate (E) at ($.3*(A) + .7*(C)$);
      \coordinate (F) at ($.5*(A) + .5*(B)$);
      \draw(A)--(B)--(C)--cycle;
      \draw[pattern=north west lines,pattern color=blue!30](D)--(E)--(F)--cycle;
    \end{scope}

    \begin{scope}[shift={(9,0)}]
      \coordinate (B) at (0,0);
      \coordinate (C) at (3,0);
      \coordinate (A) at (2,2);
      \coordinate (D) at ($.4*(B) + .6*(C)$);
      \coordinate (E) at ($.3*(A) + .7*(C)$);
      \coordinate (F) at ($.5*(A) + .5*(B)$);
      \coordinate (G) at ($.8*(B) + .2*(C)$);
      \draw(A)--(B)--(C)--cycle;
      \draw[pattern=north west lines,pattern color=blue!30](D)--(E)--(F)--(G)--cycle;
    \end{scope}

  \end{tikzpicture}
\end{example}
\begin{proof}[解]\mbox{}\\
  \begin{center}
    \begin{tikzpicture}[scale=1.0]
    \begin{scope}[shift={(0,0)}]
      \coordinate (A) at (1.8,2);
      \coordinate (B) at (0,0);
      \coordinate (C) at (3,0);
      \coordinate (D) at ($.6*(C)$);
      \coordinate (E) at ($2/3*(A)+1/3*(C)$);
      \draw(A)--(B)--(C)--cycle (D)--(E);
      \fill[pattern=north west lines,pattern color=blue!30](C)--(D)--(E)--cycle;

      \coordinate (B') at ($(0,-.4) + (B)$);
      \coordinate (C') at ($(0,-.4) + (C)$);
      \coordinate (D') at ($(0,-.4) + (D)$);
      \draw[|<->|](B')--(D') node[midway,below]{$3x$};
      \draw[|<->|](D')--(C') node[midway,below]{$2x$};

      \coordinate (P)  at ($({.4*2/sqrt(5.44)}, {.4*1.2/sqrt(5.44)})$);
      \coordinate (C') at ($(P) + (C)$);
      \coordinate (E') at ($(P) + (E)$);
      \coordinate (A') at ($(P) + (A)$);
      \draw[|<->|](C')--(E') node[pos=.75,above right,sloped]{$2y$};
      \draw[|<->|](E')--(A') node[pos=.75,above right,sloped]{$y$};

      \draw[dashed](A)--(D);
    \end{scope}
  \end{tikzpicture}
\end{center}

  作辅助线,可以容易看出各个三角形之间的面积关系。
\end{proof}


\begin{example}
  已知正方形边长,求阴影面积。

  \begin{center}
    \begin{tikzpicture}[scale=1.0]
      \coordinate(A)at(0,0);
      \coordinate(B)at(2,0);
      \coordinate(C)at(2,2);
      \coordinate(D)at(0,2);
      \draw(A)--(B)--(C)--(D)--cycle;
      \draw[fill=blue!30, even odd rule] (A) arc(180:90:2) arc(90:270:1) arc(0:180:1);
    \end{tikzpicture}
  \end{center}
\end{example}
\begin{proof}[提示]
  变换,如图。
  \begin{center}
    \begin{tikzpicture}[scale=1.0]
      \coordinate(A)at(0,0);
      \coordinate(B)at(2,0);
      \coordinate(C)at(2,2);
      \coordinate(D)at(0,2);
      \draw(A)--(B)--(C)--(D)--cycle;
      \draw[fill=blue!30, even odd rule] (A) arc(180:90:2) arc(90:180:1) arc(90:180:1);
      \fill[pattern=crosshatch, pattern color=red!50](C) arc(90:180:1) arc(90:180:1) -- (C);
      \draw[dashed](A)--(C) (1,1)--(B);
      \draw(B) arc(0:90:1) arc(180:270:1);
    \end{tikzpicture}
  \end{center}

  原阴影面积等于$\frac14$圆减去一个等腰直角三角形。
\end{proof}

\section{面积法}
\label{sec:area-method}

有时解决问题可以升维或降维。比如在某些情况下求解线段问题时,可以升维成面积问题。反之,有时求解体积问题时可以降维成线段问题。

\begin{example}
  如图,$D$和$E$是$\triangle ABC$两边上一点,$O$是$AE$与$CD$的交点。若$\dfrac{CE}{BE}=\dfrac{u}{v}$,$\dfrac{AD}{BD}=\dfrac{x}{y}$,求$\dfrac{CO}{DO}$。
  \begin{center}
    \begin{tikzpicture}[scale=1.5]
      \coordinate[label=below left:$A$](A) at (0,0);
      \coordinate[label=below right:$B$](B) at (3,0);
      \coordinate[label=above:$C$](C) at (2,2);
      \coordinate[label=below:$D$](D) at ($.6*(A)+.4*(B)$);
      \coordinate[label=right:$E$](E) at ($.5*(B)+.5*(C)$);
      \coordinate[label=above left:$O$](O) at ($2/7*(C)+5/7*(D)$);
      \draw(A)--(D) node[midway,below]{$x$};
      \draw(D)--(B) node[midway,below]{$y$};
      \draw(B)--(E) node[midway,right]{$v$};
      \draw(E)--(C) node[midway,right]{$u$};
      \draw(C)--(A)--(O)--(E) (C)--(D);
      \tkzDrawPoints(A,B,C,D,E,O)
    \end{tikzpicture}
  \end{center}
\end{example}
\begin{proof}[提示]利用面积法。如图,设$\triangle AOD$与$\triangle COE$的面积分别为$M$和$N$。
  \begin{center}
    \begin{tikzpicture}[scale=1.5]
      \begin{scope}
        \coordinate[label=below left:$A$](A) at (0,0);
        \coordinate[label=below right:$B$](B) at (3,0);
        \coordinate[label=above:$C$](C) at (2,2);
        \coordinate[label=below:$D$](D) at ($.6*(A)+.4*(B)$);
        \coordinate[label=right:$E$](E) at ($.5*(B)+.5*(C)$);
        \coordinate[label=above left:$O$](O) at ($2/7*(C)+5/7*(D)$);
        \fill[color=red!20](A)--(D)--(O)--cycle;
        \fill[pattern=north west lines](C)--(E)--(O)--cycle;
        \draw(A)--(D) node[midway,below]{$x$};
        \draw(D)--(B) node[midway,below]{$y$};
        \draw(B)--(E) node[midway,right]{$v$};
        \draw(E)--(C) node[midway,right]{$u$};
        \draw(C)--(A)--(O)--(E) (C)--(D);
        \tkzDrawPoints(A,B,C,D,E,O)
        \node at ($1/3*(A)+1/3*(D)+1/3*(O)$) {$M$};
        \node[fill=white,circle] at ($1/3*(C)+1/3*(E)+1/3*(O)$) {$N$};
      \end{scope}
      \begin{scope}[shift={(0,-3)}]
        \coordinate[label=below left:$A$](A) at (0,0);
        \coordinate[label=below right:$B$](B) at (3,0);
        \coordinate[label=above:$C$](C) at (2,2);
        \coordinate[label=below:$D$](D) at ($.6*(A)+.4*(B)$);
        \coordinate[label=right:$E$](E) at ($.5*(B)+.5*(C)$);
        \coordinate[label=above left:$O$](O) at ($2/7*(C)+5/7*(D)$);
        \fill[color=red!20](A)--(D)--(O)--cycle;
        \fill[pattern=north west lines](C)--(E)--(O)--cycle;
        \draw(A)--(D) node[midway,below]{$x$};
        \draw(D)--(B) node[midway,below]{$y$};
        \draw(B)--(E) node[midway,right]{$v$};
        \draw(E)--(C) node[midway,right]{$u$};
        \draw(C)--(A)--(O)--(E) (C)--(D);
        \draw[dashed](B)--(O);
        \tkzDrawPoints(A,B,C,D,E,O)
        \node at ($1/3*(A)+1/3*(D)+1/3*(O)$) {$M$};
        \node[fill=white,circle] at ($1/3*(C)+1/3*(E)+1/3*(O)$) {$N$};
        \node(SBOE) at ($.5*(B)+.5*(E)+(2,0)$) {$S_{\triangle BOE} = \dfrac{v}{u}\cdot N$};
        \node(SBOD) at ($(B)+(-1,-1)$) {$S_{\triangle BOD}=\dfrac{y}{x}\cdot M$};
        \node(SAOC) at ($.5*(A)+.5*(C)+(-2,0)$) {$S_{\triangle AOC}=\dfrac{u}{v}\left(1+\dfrac{y}{x}\right)\cdot M$};
    \end{scope}
    \end{tikzpicture}
  \end{center}
  连接$BO$,可分别求得各部分的面积,即
  \begin{align*}
    &S_{\triangle BOD} = \frac{y}{x}\cdot M,\quad\quad
    S_{\triangle BOE} = \frac{v}{u}\cdot N\\
    \implies& S_{\triangle ABE} = \left(1 + \frac{y}{x}\right)\cdot M + \frac{v}{u}\cdot N \\
    \implies& S_{\triangle ACE} = \frac{u}{v}\cdot S_{\triangle ABE} =
              \frac{u}{v}\left(1+\frac{y}{x}\right)\cdot M + N\\
    \implies& S_{\triangle AOC} = \frac{u}{v}\left(1+\frac{y}{x}\right)\cdot M
  \end{align*}
  由此可得
  \begin{align*}
    \frac{CO}{DO}= & \frac{S_{\triangle AOC}}{S_{\triangle AOD}}= \frac{u}{v}\left(1+\frac{y}{x}\right)
  \end{align*}

  这里还可以得到关于$M$与$N$之间关系的结论。由类似的方法得到$S_{\triangle AOC}$关于$N$的表达式为
  \begin{align*}
    & S_{\triangle AOC} = \frac{x}{y}\left(1+\frac{v}{u}\right)\cdot N\\
    \implies & \frac{u}{v}\left(1+\frac{y}{x}\right)\cdot M = \frac{x}{y}\left(1+\frac{v}{u}\right)\cdot N\qedhere
  \end{align*}
\end{proof}

\begin{example}[正八边形]
  若正八边形最长的对角线为$a$,最短的对角线为$b$,则正八边形的面积是$ab$。
  \begin{center}
    \begin{tikzpicture}[scale=1.0]
      \begin{scope}
        \foreach \i in{0,1,2,3,4,5,6,7}{%
          \coordinate(N\i) at (360/8*\i:2);
        }
        \draw(N0)--(N1)--(N2)--(N3)--(N4)--(N5)--(N6)--(N7)--cycle;
        \draw[help lines](N2)--(N6)node[midway,right]{$a$};
        \draw[help lines](N3)--(N5)node[midway,right]{$b$};
      \end{scope}
      \begin{scope}[shift={(5,0)}]
        \foreach \i in{0,1,2,3,4,5,6,7}{%
          \coordinate(N\i) at (360/8*\i:2);
        }

        \coordinate(A) at ($(N1)!(N0)!(N3)$);
        \coordinate(B) at ($(N1)!(N4)!(N3)$);
        \coordinate(C) at ($(N5)!(N4)!(N7)$);
        \coordinate(D) at ($(N5)!(N0)!(N7)$);
        % \coordinate(E) at ($(N1)!(N2)!(N3)$);
        % \coordinate(F) at ($(N5)!(N6)!(N7)$);
        \fill[red!20](N1)--(N2)--(N3);
        \fill[red!20](N0)--(A)--(N1)--cycle (N3)--(B)--(N4)--cycle;
        \fill[pattern=north west lines,pattern color=blue!20](N5)--(N6)--(N7)--cycle;
        \fill[pattern=north west lines,pattern color=blue!20](N4)--(C)--(N5)--cycle (N7)--(D)--(N0)--cycle;

        \draw[dashed](A)--(B)--(C)--(D)--cycle;
        \draw(N0)--(N1)--(N2)--(N3)--(N4)--(N5)--(N6)--(N7)--cycle;
      \end{scope}
    \end{tikzpicture}
  \end{center}
\end{example}
\begin{proof}[提示]
  由上图,将正八边形切割后可重新拼装成一个$a\times b$的长方形。
\end{proof}

\begin{example}
  如图,$E$是正方形$ABCD$的$AB$边的中点,且阴影面积为$45$,求正方形的面积。
  \begin{center}
    \begin{tikzpicture}[scale=.8]
      \coordinate[label=below left:$A$](A)at(0,0);
      \coordinate[label=below right:$B$](B)at(3,0);
      \coordinate[label=above right:$C$](C)at(3,3);
      \coordinate[label=above left:$D$](D)at(0,3);
      \coordinate[label=below:$E$](E)at($.5*(A)+.5*(B)$);
      \tkzInterLL(B,D)(E,C)\tkzGetPoint{F}
      \fill[color=red!20](A)--(E)--(F)--(D)--cycle;
      \draw(A)--(B)--(C)--(D)--cycle (E)--(C) (D)--(B);
      \tkzLabelPoints[above](F)
      \begin{scope}[shift={(4,0)}]
        \coordinate(A)at(0,0);
        \coordinate(B)at(3,0);
        \coordinate(C)at(3,3);
        \coordinate(D)at(0,3);
        \coordinate(E)at($.5*(A)+.5*(B)$);
        \tkzInterLL(B,D)(E,C)\tkzGetPoint{F}
        % \fill[color=red!20](A)--(E)--(F)--(D)--cycle;
        \fill[pattern=crosshatch,pattern color=red](A)--(F)--(E)--cycle;
        \fill[color=blue!20](F)--(E)--(B)--cycle;
        \draw(A)--(B)--(C)--(D)--cycle (E)--(C) (D)--(B) (A)--(F);
      \end{scope}
      \begin{scope}[shift={(8,0)}]
        \coordinate(A)at(0,0);
        \coordinate(B)at(3,0);
        \coordinate(C)at(3,3);
        \coordinate(D)at(0,3);
        \coordinate(E)at($.5*(A)+.5*(B)$);
        \tkzInterLL(B,D)(E,C)\tkzGetPoint{F}
        % \fill[color=red!20](A)--(E)--(F)--(D)--cycle;
        \fill[pattern=crosshatch,pattern color=red](A)--(F)--(D)--cycle;
        \fill[color=blue!20](C)--(F)--(D)--cycle;
        \draw(A)--(B)--(C)--(D)--cycle (E)--(C) (D)--(B) (A)--(F);
        \draw[dashed](F)--(F |- D) (F)--(F -| D);
      \end{scope}
      \begin{scope}[shift={(12,0)}]
        \coordinate(A)at(0,0);
        \coordinate(B)at(3,0);
        \coordinate(C)at(3,3);
        \coordinate(D)at(0,3);
        \coordinate(E)at($.5*(A)+.5*(B)$);
        \tkzInterLL(B,D)(E,C)\tkzGetPoint{F}
        % \fill[color=red!20](A)--(E)--(F)--(D)--cycle;
        \fill[pattern=crosshatch,pattern color=red](A)--(F)--(B)--cycle;
        \fill[color=blue!20](C)--(F)--(B)--cycle;
        \draw(A)--(B)--(C)--(D)--cycle (E)--(C) (D)--(B) (A)--(F);
        % \draw[dashed](F)--(F |- D) (F)--(F -| D);
      \end{scope}
    \end{tikzpicture}
  \end{center}
\end{example}
\begin{proof}[提示]
  由后面几个图形,由两部分阴影面积相等可以知道只要得出$\triangle DFC$和$\triangle BFC$的面积比,即只要得出$DF:BF$就可以得到图中各种三角形的面积关系,从而由$S_{AEFD}=45$可以求出各个面积。

  \begin{center}
    \begin{tikzpicture}[scale=.8]
      \begin{scope}[shift={(0,0)}]
        \coordinate[label=below left:$A$](A)at(0,0);
        \coordinate[label=below right:$B$](B)at(3,0);
        \coordinate[label=above right:$C$](C)at(3,3);
        \coordinate[label=above left:$D$](D)at(0,3);
        \coordinate[label=below:$E$](E)at($.5*(A)+.5*(B)$);
        \tkzInterLL(B,D)(E,C)\tkzGetPoint{F}
        \coordinate[label=below:$G$](G)at(0,-3);
        % \fill[color=red!20](A)--(E)--(F)--(D)--cycle;
        \fill[pattern=crosshatch,pattern color=red](G)--(F)--(D)--cycle;
        \fill[color=blue!20](C)--(F)--(B)--cycle;
        \draw(A)--(B)--(C)--(D)--cycle (E)--(C) (D)--(B) (A)--(F);
        \draw[dashed](A)--(G)--(E);
        % \draw[dashed](F)--(F |- D) (F)--(F -| D);
        \tkzLabelPoints[above](F)
      \end{scope}      
      \begin{scope}[shift={(5,0)}]
        \coordinate(A)at(0,0);
        \coordinate(B)at(3,0);
        \coordinate(C)at(3,3);
        \coordinate(D)at(0,3);
        \coordinate(E)at($.5*(A)+.5*(B)$);
        \tkzInterLL(B,D)(E,C)\tkzGetPoint{F}
        \draw(A)--(B)--(C)--(D)--cycle (E)--(C) (D)--(B) (A)--(F);
        \node at($1/3*(A)+1/3*(E)+1/3*(F)$){1};
        \node at($1/3*(B)+1/3*(E)+1/3*(F)$){1};
        \node at($1/3*(B)+1/3*(C)+1/3*(F)$){2};
        \node at($1/3*(A)+1/3*(F)+1/3*(D)$){4};
        \node at($1/3*(C)+1/3*(F)+1/3*(D)$){4};
      \end{scope}
    \end{tikzpicture}
  \end{center}
  如上图,延长$CE$与$DA$相交于$G$,可知$AG=BC$,从而$DF:FB=DG:BC=2$。由此可得上面第二图中各面积关系比。
\end{proof}

\begin{example}
  由上面分析,容易得到正方形一个顶点与两条不相邻边的中点的连线将一条对角线三等分。
  \begin{center}
    \begin{tikzpicture}
      \begin{scope}
        \coordinate(A)at(0,0);
        \coordinate(B)at(3,0);
        \coordinate(C)at(3,3);
        \coordinate(D)at(0,3);
        \coordinate(E)at($.5*(A)+.5*(B)$);
        \coordinate(F)at($.5*(A)+.5*(D)$);
        \draw(A)--(B)--(C)--(D)--cycle (E)--(C)--(F) (D)--(B);
      \end{scope}      
    \end{tikzpicture}
  \end{center}
\end{example}


\section{Primary Skill}
\label{sec:primary-skill-in-calculating-area}

\begin{example}
  $AE=EF=FB$, $AG=2GC$, $S_{\triangle EFG}=8$. Find the area of the parallelogram.

  \centering
  \begin{tikzpicture}[scale=2.0]
    \coordinate[label=below left:$A$] (A) at(0,0);
    \coordinate[label=below right:$B$](B) at(3,0);
    \coordinate[label=above right:$C$](C) at(4,2);
    \coordinate[label=above left:$D$] (D) at(1,2);
    \coordinate[label=below:$E$]      (E) at(1,0);
    \coordinate[label=below:$F$]      (F) at(2,0);
    \coordinate[label=above left:$G$] (G) at($1/3*(A)+2/3*(C)$);
    \draw(A)--(B)--(C)--(D)--(A)--(C);
    \filldraw[pattern=north west lines, pattern color=black](E)--(F)--(G)--cycle;
  \end{tikzpicture}
\end{example}

\begin{proof}[Hint]
  Using propotion.

  \begin{center}
    \begin{tikzpicture}[scale=1.0]
      \begin{scope}
        \coordinate(A) at(0,0);
        \coordinate(B) at(3,0);
        \coordinate(C) at(4,2);
        \coordinate(D) at(1,2);
        \coordinate(E) at(1,0);
        \coordinate(F) at(2,0);
        \coordinate(G) at($1/3*(A)+2/3*(C)$);
        \draw(A)--(B)--(C)--(D)--(A)--(C);
        \filldraw[pattern=north west lines, pattern color=black](E)--(F)--(G)--cycle;
        \node at ($1/3*(E)+1/3*(F)+1/3*(G)$) {$8$};
      \end{scope}
      \begin{scope}[shift={(5,0)}]
        \coordinate(A) at(0,0);
        \coordinate(B) at(3,0);
        \coordinate(C) at(4,2);
        \coordinate(D) at(1,2);
        \coordinate(E) at(1,0);
        \coordinate(F) at(2,0);
        \coordinate(G) at($1/3*(A)+2/3*(C)$);
        \draw(A)--(B)--(C)--(D)--(A)--(C);
        \filldraw[pattern=north west lines, pattern color=black](A)--(B)--(G)--cycle;
        \node at ($1/3*(A)+1/3*(B)+1/3*(G)$) {$24$};
      \end{scope}
      \begin{scope}[shift={(10,0)}]
        \coordinate(A) at(0,0);
        \coordinate(B) at(3,0);
        \coordinate(C) at(4,2);
        \coordinate(D) at(1,2);
        \coordinate(E) at(1,0);
        \coordinate(F) at(2,0);
        \coordinate(G) at($1/3*(A)+2/3*(C)$);
        \draw(A)--(B)--(C)--(D)--(A)--(C);
        \filldraw[pattern=north west lines, pattern color=black](A)--(B)--(C)--cycle;
        \node at ($1/3*(A)+1/3*(B)+1/3*(C)$) {$36$};
      \end{scope}
    \end{tikzpicture}
  \end{center}
\end{proof}


\begin{example}
  Knowing $D,E$ are the middle point of $BC$ and $CA$ respectively, and $S_{\triangle ABC}=60$, find the area of the shadow $\triangle ABO$.
  \centering
  \begin{tikzpicture}[scale=2.0]
    \foreach \n/\x/\y/\p in{A/0/0/below left, B/3/0/below right, C/2/2/above}{
      \coordinate[label=\p:$\n$](\n)at(\x,\y);
    }
    \coordinate[label=above right:$D$](D)at($.5*(B)+.5*(C)$);
    \coordinate[label=above left:$E$](E)at($.5*(A)+.5*(C)$);
    \draw(A)--(B)--(C)--(A)--(D)
         (B)--(E);
    \tkzInterLL(A,D)(B,E)\tkzGetPoint{O}
    \tkzLabelPoints[above](O)
    \filldraw[pattern=north west lines, pattern color=black](A)--(O)--(B)--cycle;
  \end{tikzpicture}
\end{example}

\begin{proof}[Hint]
  The three parts have the same area. Or, the six parts have the same area.
  \begin{center}
    \begin{tikzpicture}[scale=1.5]
      \begin{scope}
        \foreach \n/\x/\y/\p in{A/0/0/below left, B/3/0/below right, C/2/2/above}{
          \coordinate[label=\p:$\n$](\n)at(\x,\y);
        }
        \coordinate(O)at($1/3*(A)+1/3*(B)+1/3*(C)$);
        \draw(A)--(B)--(C)--(A)--(O)--(B) (O)--(C);
        \fill[color=red!10](A)--(O)--(C);
        \fill[pattern=north west lines, pattern color=black](A)--(O)--(B);
        \fill[pattern=north east lines, pattern color=black](C)--(O)--(B);
      \end{scope}
      \begin{scope}[shift={(4,0)}]
        \foreach \n/\x/\y/\p in{A/0/0/below left, B/3/0/below right, C/2/2/above}{
          \coordinate[label=\p:$\n$](\n)at(\x,\y);
        }
        \coordinate(D)at($.5*(B)+.5*(C)$);
        \coordinate(E)at($.5*(C)+.5*(A)$);
        \coordinate(F)at($.5*(A)+.5*(B)$);
        \coordinate(O)at($1/3*(A)+1/3*(B)+1/3*(C)$);
        \draw(A)--(B)--(C)--(A)--(D) (B)--(E) (C)--(F);
        \fill[color=red!10](A)--(O)--(F)--cycle
                           (B)--(O)--(D)--cycle
                           (C)--(O)--(E)--cycle;
      \end{scope}
    \end{tikzpicture}
  \end{center}
\end{proof}


\begin{example}
  求图中阴影部分面积,其中的三角形是直角三角形。

  \centering
  \begin{tikzpicture}[scale=0.5]
    \draw[fill=red!30, even odd rule](0,0) arc(180:0:3) arc(270:90:2) --cycle;
    \draw(0,0)--(6,0)node[midway, below]{$6$} -- (6,4)node[midway, right]{$4$};
  \end{tikzpicture}
\end{example}
\begin{proof}[提示]
  大半圆 + 小半圆 = 三角形 + 阴影
\end{proof}


\begin{question}
  图中四边形是直角梯形,求其中两部分阴影面积的差。
  \begin{center}
    \begin{tikzpicture}[scale=1.0]
      \coordinate(A)at(0,0);
      \coordinate(B)at(8,0);
      \coordinate(C)at(2,6);
      \coordinate(D)at(0,6);
      \begin{scope}
        \clip (A)arc(180:135:8)--(B)--cycle;
        \fill[pattern=north west lines, pattern color=red!70](D)arc(180:315:2)--(C)--cycle;
      \end{scope}
      \begin{scope}
        \clip(A)--(D)arc(180:315:2)--(B)--(A);
        \fill[color=blue!20](A)arc(180:135:8)--(C)--(D)--cycle;
      \end{scope}
      \draw(A)--(B)node[midway, below]{8}--(C)--(D)node[midway,above]{2}--(A)node[midway, left]{6};
      \draw(A)arc(180:135:8) (D)arc(180:315:2);
      \node[draw,circle](none)at(1,5.3){\small $3$};
      \node[fill=white,draw,circle](none)at(0.5,3.9){\small $1$};
      \node[fill=white,draw,circle](none)at(2.2, 4.7){\small $2$};
      \tkzMarkRightAngle(B,A,D)
      \tkzMarkRightAngle(A,D,C)
    \end{tikzpicture}
  \end{center}
\end{question}
\begin{proof}[提示]
  记编号1、2、3的区域面积分别为$S_1, S_2, S_3$,则$S_1$与$S_2$的面积差与两者分别加上$S_3$之后的面积差相等,即
  \begin{align*}
    |S_1 - S_2| = |(\underline{S_1 + S_3}) - (\uwave{S_2 + S_3})|
  \end{align*}
  而$S_1+S_3$及$S_2+S_3$都容易求得。
\end{proof}


\begin{question}
  已知圆半径1,求阴影面积。
  \begin{center}
    \begin{tikzpicture}[scale=0.5]
      \coordinate(A)at(0,0);
      \coordinate(B)at(2,0);
      \coordinate(C)at($(A)!1!60:(B)$);
      \coordinate(D)at($(A)!1!60:(C)$);
      \fill[color=red!50](A)arc(240:180:2)arc(120:60:2)arc(120:180:2);
      \fill[color=red!50](A)arc(180:240:2)arc(300:360:2)arc(300:240:2);
      \fill[color=red!50](B)arc(300:360:2)arc(60:120:2)arc(60:0:2);
      \draw(A)circle(2);
      \draw(B)circle(2);
      \draw(C)circle(2);
    \end{tikzpicture}
  \end{center}
\end{question}
\begin{proof}[提示]
  作辅助线。
  \begin{center}
    \begin{tikzpicture}[scale=0.5]
      \coordinate(A)at(0,0);
      \coordinate(B)at(2,0);
      \coordinate(C)at($(A)!1!60:(B)$);
      \coordinate(D)at($(A)!1!60:(C)$);
      \fill[color=red!50](A)arc(240:180:2)--(C)--(A);
      \fill[color=red!50](A)arc(180:240:2)arc(300:360:2)arc(300:240:2);
      \fill[color=red!50](B)arc(300:360:2)arc(60:120:2)arc(60:0:2);
      \draw(A)circle(2);
      \draw(B)circle(2);
      \draw(C)circle(2);
      \draw[dashed](D)--(C)--(A);
    \end{tikzpicture}
  \end{center}
  可知,每个阴影的面积等于一个$60^\circ$的扇形面积,三个加起来等于半个圆。
\end{proof}


\begin{question}
  如图,一个单位圆和等腰直角三角形,求其中阴影面积。
  \begin{center}
    \begin{tikzpicture}[scale=1.0]
      \begin{scope}
        \draw[fill=blue!30,even odd rule](0,0) arc(0:360:1) -- (-2,0)--(0,2)--cycle;
      \end{scope}
      \begin{scope}[shift={(2,0)}]
        \coordinate(O) at (0,0);
        \coordinate(A) at (1,0);
        \coordinate(B) at (-1,0);
        \coordinate(C) at (1,2);
        \coordinate(D) at ($(C)!1!-45:(A)$);
        \coordinate(E) at (0,1);
        \fill[color=blue!30](O)circle(1);
        \fill[color=white](B)--(A) arc(270:225:2) -- cycle;
        \draw(A)arc(0:360:1)--(C)--(B)--cycle (A) arc(270:225:2);
      \end{scope}
    \end{tikzpicture}
  \end{center}
\end{question}
\begin{proof}[提示]
  先求圆内空白部分面积。
\end{proof}

\begin{example}
  如图,$\triangle ABC$与$\triangle ADE$都是边长为3的正三角形,$\angle DAC=30^\circ$。求图中阴影部分面积。

  \centering
  \begin{tikzpicture}[scale=1.0]
    \coordinate[label=above:$A$] (A) at (0,0);
    \coordinate[label=left:$B$] (B) at (225:3);
    \coordinate[label=left:$D$] (D) at (255:3);
    \coordinate[label=right:$C$] (C) at (285:3);
    \coordinate[label=right:$E$] (E) at (315:3);
    \coordinate[label=below:$F$] (F) at ($0.25*(B)+0.75*(C)$);
    \coordinate[label=above left:$O$] (O) at ($0.5*(B) + 0.5*(C)$);
    \fill[color=blue!10](A)--(D)--(F)--(C)--cycle;
    \draw(A)--(B)--(C)--cycle (A)--(D)--(E)--cycle;
  \end{tikzpicture}
\end{example}
\begin{proof}[提示]
  容易看出来,$AF$是整个图形的一个对称轴,从而$AF$是阴影角$\angle DAC$的角平分线,从而角平分线的性质,可以求得$DF$与$CF$的长度,从而可得阴影部分面积。

  也可以利用特殊直角三角形的性质。$\triangle ACF$与$\triangle ADF$都是边长为$3$的三角形,设其高$OF$为$x$,则由有一个$60^\circ$角的直角三角形$ODF$,有
  \begin{align*}
    DF = CF = \frac2{\sqrt3} x
  \end{align*}
  且由$BC = 2 OC$及$OC = OF +FC$,可知
  \begin{align*}
    x + \frac2{\sqrt3} x = \frac32
  \end{align*}
  从而可得$OF$。
\end{proof}


\section{数形结合}
\label{sec:combine-number-and-figure}

\begin{example}[希望杯]
  已知$a>0$,$b>0$,求以$\sqrt{a^2+b^2}$、$\sqrt{a^2+4b^2}$和$\sqrt{4a^2+b^2}$为边长的三角形的面积。
\end{example}
\begin{proof}[提示]
  已知三个边长的情况下,首先想到可用海伦公式求三角形面积
  \begin{align*}
    S=\sqrt{p(p-a)(p-b)(p-c)}
  \end{align*}
  其中$p=(a+b+c)/2$是半周长。然而这个计算量会非常大。

  根据边长的形式,可以考虑勾股定理
  \begin{center}
    \begin{tikzpicture}[scale=0.5]
      \begin{scope}
        \draw(0,0)--(3,0)node[midway,below]{$a$}--(0,2) node[midway,sloped,above]{$\sqrt{a^2+b^2}$}--cycle node[midway,left]{$b$};
      \end{scope}
      \begin{scope}[shift={(6,0)}]
        \draw(0,0)--(3,0)node[midway,below]{$a$}--(0,4) node[midway,sloped,above]{$\sqrt{a^2+4b^2}$}--(0,0) node[midway,left]{$2b$};
      \end{scope}
      \begin{scope}[shift={(12,0)}]
        \draw(0,0)--(6,0)node[midway,below]{$2a$}--(0,2) node[midway,sloped,above]{$\sqrt{4a^2+b^2}$}--(0,0) node[midway,left]{$b$};
      \end{scope}
    \end{tikzpicture}
  \end{center}
  画到网格上,可以得到下面的图形
  \begin{center}
    \begin{tikzpicture}[scale=0.5]
      \draw[fill=red!20](3,0)--(0,2)--(6,4)--cycle;
      \draw(0,0)--(3,0)node[midway,below]{$a$}--(6,0)node[midway,below]{$a$}--(6,4)--(0,4)--
           (0,2)node[midway,left]{$b$} --cycle node[midway,left]{$b$}
           (3,0)--(3,4) (0,2)--(6,2);
    \end{tikzpicture}
  \end{center}
  通过空白区域的面积,容易求得图中阴影三角形的面积。
\end{proof}

\begin{question}
  已知$a>0$,$b>0$,求以$\sqrt{a^2+b^2}$、$\sqrt{a^2+9b^2}$和$\sqrt{4a^2+4b^2}$为边长的三角形的面积。
\end{question}





\section{Misc}
\label{sec:misc}

\begin{example}
  Given a square with size $6\times6$ and a rectangle with size $3\times2$, find out the area difference of those not overlapped.

  \centering
  \begin{tikzpicture}[scale=1.0]
    \draw(0,0) rectangle(6,6);
  \end{tikzpicture}
\end{example}


\begin{example}
  % https://math.stackexchange.com/questions/1544719/point-in-the-interior-of-a-square
  A point in the interior of a square $ABCD$ is at distances 3, 4 and meters from the vertices $A$, $B$ and $C$, respectively. What's the area of $ABCD$?
\end{example}


\begin{example}[Same length of side, different height]
  Any point $P$ in the interior of a square $ABCD$, $S_{\triangle ABP} + S_{\triangle CDP}
  = S_{\triangle ACP} + S_{\triangle BDP}$.

  $\triangle ABP$ and $\triangle CDP$ both have a side the length of which is the that of the square, and the heights of them sum up to the length of square too, which makes the sum of the area of the two triangle the half of the square.
  

  Any point $P$ in the interior of square, connect it with the middle points each side. It split into two parts each of which sums to the same area.

  \begin{center}
    \begin{tikzpicture}[scale=1.0]
      \begin{scope}[shift={(0,0)}]
        \coordinate(A)at(0,0);
        \coordinate(B)at(2,0);
        \coordinate(C)at(2,2);
        \coordinate(D)at(0,2);
        \coordinate(P)at(0.7,1.1);
        \draw(A)--(B)--(C)--(D)--(A)--(P)--(B) (C)--(P)--(D);
        \fill[color=blue!20](A)--(P)--(D)--cycle;
        \fill[color=blue!20](B)--(P)--(C)--cycle;
      \end{scope}
      \begin{scope}[shift={(4,0)}]
        \coordinate(A)at(0,0);
        \coordinate(B)at(2,0);
        \coordinate(C)at(2,2);
        \coordinate(D)at(0,2);
        \coordinate(P)at(0.7,1.1);
        \coordinate(E)at($0.5*(A)+0.5*(B)$);
        \coordinate(F)at($0.5*(B)+0.5*(C)$);
        \coordinate(G)at($0.5*(C)+0.5*(D)$);
        \coordinate(H)at($0.5*(D)+0.5*(A)$);
        \draw(A)--(B)--(C)--(D)--cycle (E)--(P)--(F) (G)--(P)--(H);
        \fill[color=blue!20](A)--(E)--(P)--(H)--cycle;
        \fill[color=blue!20](C)--(G)--(P)--(F)--cycle;
      \end{scope}
    \end{tikzpicture}
  \end{center}
\end{example}

\begin{example}
  Any trapezoid $ABCE$ where $AB\parallel CD$, the two diaganl intersects in $P$, then $S_{\triangle ADP}=S_{\triangle BCP}$.
\end{example}


\begin{example}
  $\triangle ABC$, $AB$, $AC$ split into 5 parts equally. The area of the shadow region is 21. What's the area of the triangle?

  \begin{tikzpicture}
    \coordinate[label=left:$A$](A)at(0,0);
    \coordinate[label=below right:$B$](B)at(4,0);
    \coordinate[label=above right:$C$](C)at(3,2);
    \foreach \x/\y/\s in{4/1/5,3/2/5,2/3/5,1/4/5}{%
      \coordinate(B\y)at($\x/\s*(A)+\y/\s*(B)$);
      \coordinate(C\y)at($\x/\s*(A)+\y/\s*(C)$);
    }
    \foreach \a/\b/\c in{A/B1/C1, B1/B2/C2, B2/B3/C3, B3/B4/C4, B4/B/C}{%
      \fill[color=blue!20](\a)--(\b)--(\c)--cycle;
    }
    \draw(C)--(A)--(B)--(C)--(B4)--(C4)--(B3)--(C3)--(B2)--(C2)--(B1)--(C1);
  \end{tikzpicture}
\end{example}

\begin{example}
  Circle with radius 1, find the area of the shadow region.

  \begin{center}
    \begin{tikzpicture}[scale=1.0]
      \begin{scope}[shift={(0,0)}]
        \filldraw[black, fill=blue!20](-1,0) arc(-90:0:1) arc(180:270:1) arc(90:180:1) arc(0:90:1);
      \end{scope}
      \begin{scope}[shift={(3,0)}]
        \filldraw[black, fill=blue!20](0,0)circle(1);
        \filldraw[black, fill=white](-1,0) arc(-90:0:1) arc(180:270:1) arc(90:180:1) arc(0:90:1);
      \end{scope}
    \end{tikzpicture}
  \end{center}
\end{example}

Draw an auxilary square like the following
\begin{center}
  \begin{tikzpicture}
    \filldraw[black, fill=blue!20](-1,0) arc(-90:0:1) arc(180:270:1) arc(90:180:1) arc(0:90:1);
    \draw(-1,-1)rectangle(1,1);
  \end{tikzpicture}
\end{center}
Then the area of the shadow is that of the square subtracted by the 4 quater-of-circle, i.e.,
\begin{align*}
  S_{\text{shadow}} =& S_{\text{square}} - 4\times S_{\text{quarter of square}}\\
  =& 2\times2 - \pi \times 1\times 1 = 4-\pi
\end{align*}


\include{matrix}

\include{quadratic-equation}
\include{function}
\include{symetric}
\include{extreme}
\include{series}
\include{monotonic-functions}
\include{metric-space}
\include{fixed-point}
\include{identities}
% \include{perfect_squares}

\include{inequality}

% \chapter{不等式}
% \label{chap:inequality}


\section{基本不等式}
\label{sec:basic-inequalities}

\begin{definition}[算术平均数,Arithmetic Means, AM)]
  $\forall x_i$,下式称为其代数平均数
  \begin{align*}
    A_n\equiv\frac{\sum\limits_{i=1}^{n} x_i}{n}
    =\frac{x_1+x_2+x_3+\cdots+x_n}{n}
  \end{align*}
\end{definition}

\begin{definition}[几何平均数, Geometric Means, GM]
  $\forall x_i\ge0$,下式称为其几何平均数
  \begin{align*}
    G_n\equiv\sqrt[n]{\prod_{i=1}^{n}x_i}
    =\sqrt[n]{x_1 x_2 x_3\cdots x_n}
  \end{align*}
\end{definition}

\begin{definition}[调和平均数,Harmonic Means, HM]
  $\forall x_i\ge0$,下式称为其调和平均数
  \begin{align*}
    H_n\equiv\frac{n}{\sum\limits_{i=1}^{n}\dfrac1{x_i}}
    =\frac{n}{\dfrac1{x_1}+\dfrac1{x_2}+\cdots+\dfrac1{x_n}}
    =\frac{1}{\dfrac{\dfrac1{x_1}+\dfrac1{x_2}+\cdots+\dfrac1{x_n}}{n}}
  \end{align*}
\end{definition}

\begin{definition}[平方平均数,Quadratic Mean,QM]
  $\forall x_i$,下式称不其平方平均数
  \begin{align*}
    Q_n\equiv\sqrt{\dfrac{\sum\limits_{i=1}^n x_i^2}{n}}
    =\sqrt{\frac{x_1^2+x_2^2+\cdots+x_n^2}{n}}
  \end{align*}
  平均平方数也称为\term{均方根}(Root Mean Square)。
\end{definition}

若令
\begin{align*}
  \varphi(x_1,x_2,\cdots,x_n;p)\equiv\left(\frac{\sum_{i=1}^{n} x_i^p}{n} \right)^{\frac1p}
  =\left( \frac{x_1^p + x_2^p + \cdots + x_n^p}{n} \right)^{\frac1p}
\end{align*}
则显然有
\begin{align*}
  H_n \equiv \mathrm{HM}(x_1,x_2,\cdots,x_n)&=\varphi(x_1,x_2,\cdots,x_n;-1)\\
  A_n \equiv \mathrm{AM}(x_1,x_2,\cdots,x_n)&=\varphi(x_1,x_2,\cdots,x_n;1)\\
  Q_n \equiv \mathrm{QM}(x_1,x_2,\cdots,x_n)&=\varphi(x_1,x_2,\cdots,x_n;2)
\end{align*}

思考:是否能推出固定$x_1,x_2,\cdots,x_n$,则$\varphi(x_1,x_2,\cdots,x_n;p)$关于$p$是递增函数?若可以,则显然有$H_n\le A_n\le Q_n$。

\begin{theorem}[HM-GM-AM-QM]
  任意正数序列$x_i$,有
  \begin{align*}
    H_n\le G_n\le A_n\le Q_n
  \end{align*}
\end{theorem}
\begin{figure}[htbp]
  \centering
  \begin{tikzpicture}[scale=1.0]
    \draw[help lines](-6,0)--(6,0) arc (0:180:6);
    \draw[help lines,|<->|](-6,-0.6)--(3,-0.6) node[midway,fill=white]{$x_1$};
    \draw[help lines,|<->|](3,-0.6)--(6,-0.6) node[midway,fill=white]{$x_2$};
    \draw[help lines,|<->|](-6,-1.5)--(0,-1.5)node[midway,fill=white]{$\dfrac{x_1+x_2}2$};
    \draw[help lines,|<->|](0,-1.5)--(3,-1.5)node[midway,fill=white]{$\dfrac{x_1-x_2}2$};
    \coordinate (G) at (60:6);
    \coordinate (H) at (60:1.5);
    \tkzDefPoint(3,0){C}
    \tkzDefPoint(0,0){O}
    \tkzMarkRightAngle[draw,fill=white](O,H,C)
    \draw[very thick, blue](G)--(C) node[sloped,midway,fill=white]{GM};
    \draw[very thick, red](0,6)--(0,0) node[midway,sloped,fill=white]{AM};
    \draw[very thick](0,6)--(C) node[sloped,pos=0.3,fill=white]{QM};
    \draw[very thick, violet](H)--(G) node[sloped,pos=0.61,fill=white]{HM};
    \draw[very thick](-6,0)--(6,0);
    \fill (-6,0) circle(3pt);
    \fill (C) circle(3pt);
    \fill (6,0) circle(3pt);
    \fill (G) circle(3pt);
    \fill (H) circle(3pt);
    \fill (0,6) circle(3pt);
    \fill (O) circle(3pt);
    \draw[dashed](0,0)--(H);
    \draw[dashed](3,0)--(H);
  \end{tikzpicture}
  \caption{$n=2$时HM,GM,AM,QM的几何示意}
  \label{fig:HM-GM-AM-QM}
\end{figure}

\begin{lemma}\label{lemma:b-1-a}
  若$b\le1\le a$,则$ab\le a+b-1$。
\end{lemma}
\begin{proof}
  由$0\le (a-1)(1-b)=a+b-ab-1$可得。此引理比较重要,被应用到很多不等式的证明过程中。
\end{proof}

\begin{lemma}\label{lemma:product-ai-ge-n}
  $\forall a_i>0$,若满足$\prod_{i=1}^n a_i=1$,则以下不等式成立
  \begin{align}
    \sum_{i=1}^n a_i\ge n
  \end{align}
  当且仅当$a_1=a_2=\cdots=a_n=1$时上述等号成立。
\end{lemma}
\begin{proof}
  对$n$作数学归纳。当$n=1$时显然。当$n\ge2$时,不妨对$a_i$重新排列使得$a_1\equiv\max(a_1,a_2,\cdots,a_n)\ge1$,$a_2\equiv\min(a_1,a_2,\cdots,a_n)\le1$,那么以下$n-1$个数的序列
  \begin{align*}
    a_1a_2, a_3, a_4, \cdots, a_n
  \end{align*}
  满足$n-1$的情况,从而有
  \begin{align*}
    a_1a_2 + a_3+ a_4+ \cdots + a_n &\ge n-1
  \end{align*}
  当且仅当$a_1a_2=a_3=a_4=\cdots=a_n=1$时等号成立,又由引理\ref{lemma:b-1-a},$a_1+a_2 - 1\ge a_1a_2$,其中等号当且仅当$a_1=1$或者$a_2=1$时成立。代入后有
  \begin{align*}
    (a_1 + a_2 - 1) + a_3+ a_4+ \cdots + a_n &\ge n-1\\
    a_1 + a_2 + \cdots + a_n\ge n
  \end{align*}
  其中等号成立,当且仅当以下两个条件同时满足:
  \begin{enumerate}
  \item $a_1a_2=a_3=a_4=\cdots=a_n=1$
  \item $a_1=1$或者$a_2=1$
  \end{enumerate}
  由此条件可知当且仅当所有的$a_i$都等于1时等号成立。
\end{proof}

下面证明AM-GM不等式。
\begin{proof}
  令$g\equiv\sqrt[n]{\prod\limits_{i=1}^n a_i}$,对序列
  \begin{align*}
    \frac{a_1}{g}, \frac{a_2}{g}, \frac{a_3}{g}, \cdots, \frac{a_n}{g}
  \end{align*}
  应用引理\ref{lemma:product-ai-ge-n},有
  \begin{align*}
    \frac{a_1}{g} + \frac{a_2}{g} + \frac{a_3}{g} + \cdots + \frac{a_n}{g}&\ge n\\
    \frac{a_1 + a_2 + a_3 + \cdots + a_n}{n}&\ge g = \sqrt[n]{a_1a_2a_3\cdots a_n} \qedhere
  \end{align*}
\end{proof}

对于三角形,HM还有以下几何意义:
\begin{example}
  对任意三角形,其内切圆的半径是三角形三条高长度的调和平均值的三分之一,即
  \begin{align*}
    r=\frac13 HM(h_a,h_b,h_c)
  \end{align*}
\end{example}
\begin{proof}
  尝试用面积来解,如图~\ref{fig:incircle}所示设三边边长分别是$a,b,c$,
  其对应的高分别为 $h_a, h_b, h_c$,则三角形面积$S=\frac12
  ah_a = \frac12 bh_b = \frac12 ch_c$,同样有$S=\frac12 r(a+b+c)$,从而
  \begin{align*}
    & S = \frac12 r\left( \frac{2S}{h_c} +\frac{2S}{h_a} + \frac{2S}{h_b} \right)
    \quad \implies\quad  r=\frac{1}{\dfrac1{h_a} + \dfrac1{h_b} + \dfrac1{h_c}} \qedhere
  \end{align*}
\end{proof}

\begin{figure}[htbp]
  \centering
  \begin{tikzpicture}[scale=1]
    \tkzDefPoint[label=below left:$A$](0,0){A}
    \tkzDefPoint[label=below right:$B$](6,0){B}
    \tkzDefPoint[label=above:$C$](5,5){C}
    
    \tkzDefCircle[in](A,B,C)\tkzGetPoint{I}\tkzGetLength{rIN}
    \tkzDrawCircle[R](I,\rIN pt);
    \tkzDrawSegments(A,B B,C C,A)
    \tkzDrawSegments[dashed](I,A I,B I,C)
    \tkzDrawCircle[fillstyle=solid,R](I,2pt)

    \coordinate(IA) at ($(B)!(I)!(C)$);
    \coordinate(IB) at ($(C)!(I)!(A)$);
    \coordinate(IC) at ($(A)!(I)!(B)$);

    \tkzDrawSegments[dashed](I,IA I,IB I,IC)
    \tkzMarkRightAngle[color=blue](B,IC,I)
    \tkzMarkRightAngle[color=blue](C,IA,I)
    \tkzMarkRightAngle[color=blue](A,IB,I)
  \end{tikzpicture}
  \caption{三角形内切圆}
  \label{fig:incircle}
\end{figure}

\begin{example}
  若任意非负数$a,b,c$满足$(a+1)(b+1)(c+1)=8$,则$abc\le 1$。
\end{example}
\begin{proof}
  若$a,b,c$均非负,则由AM-GM不等式,有$a+1\ge 2\sqrt{a}$,从而
  \begin{align*}
    8=(a+1)(b+1)(c+1)\ge 2\sqrt{a}\times 2\sqrt{b} \times 2\sqrt{c}
  \end{align*}
  即$abc\le1$。当且仅当$a=b=c=1$时等号成立。
\end{proof}

上述条件不能扩展到任意实数,$a,b,c$一负两正或者三个都是负数,则显然$abc<0$;但对
于$a,b,c$两负一正,比如$a=-2,b=-2,c=7$,则$abc>1$。



\begin{question}\label{q:1/1+a+b}
  三个正数$a,b,c$的乘积是1,求证
  \begin{align*}
    \frac1{1+a+b} + \frac1{1+b+c} + \frac1{1+c+a} \le 1
  \end{align*}
  当且仅当$a=b=c=1$时等号成立。
\end{question}

先尝试猜测每一项的上界。
\begin{align*}
  1+a+b\ge 3\sqrt[3]{ab},\quad 1+b+c \ge 3\sqrt[3]{bc},\quad 1+c+a\ge 3\sqrt[3]{ca}
\end{align*}
从而
\begin{align*}
  \frac1{1+b+c} + \frac1{1+c+a} + \frac1{1+a+b} \le
  \frac13\left( \frac1{\sqrt[3]{ab}} + \frac1{\sqrt[3]{bc}} + \frac1{\sqrt[3]{ca}} \right)
\end{align*}
令$a\to0+, b\to 0+, c=\frac1{ab}$,则上式$\le$符号右边$\to+\infty$,从而没上界,此估计无用。

% 又任意正数$x,y$,有
% \begin{align*}
%   \sqrt{xy}\ge \frac2{\dfrac1x + \dfrac1y}%
%   % \implies  \frac1{xy}\le \frac2{\dfrac1{x^2} + \dfrac1{y^2}}
% \end{align*}

% 令$a_0=\sqrt[6]a, b_0=\sqrt[6]b, c_0=\sqrt[6]c$,代入,有
% \begin{align*}
%   \frac1{\sqrt[3]{ab}} = \frac1{\sqrt{a_0b_0}}\le \frac2{\dfrac1{a_0} + \dfrac1{b_0}}
%   =\frac{2a_0b_0}{a_0+b_0}
% \end{align*}
% 同理处理$\frac1{\sqrt[3]{bc}}$和$\frac1{\sqrt[3]{ca}}$,从而有
% \begin{align*}
%   \frac1{1+b+c} + \frac1{1+c+a} + \frac1{1+a+b} \le
%   \frac23 \left(
%   \frac{a_0b_0}{a_0+b_0} + \frac{b_0c_0}{b_0+c_0} + \frac{c_0a_0}{c_0+a_0} 
%   \right)
% \end{align*}
% 其中$a_0b_0c_0=1$,且上式当且仅当$a_0=b_0=c_0=1$时等号成立。{\color{red}往下似乎不好走了,尝试换个方法。}


\begin{lemma}
对任意$a,b,c\in\mathcal{R}$,有以下恒等式:
\begin{align*}
  (a+b+c)(ab+bc+ca)&=a^2b+a^2c+b^2c+b^2a+c^2a+c^2b + 3abc\\
                   &=(a+b)(b+c)(c+a)+abc
\end{align*}
\end{lemma}
\begin{proof}
  左右分别展开可得证。
\end{proof}

令$x=b+c, y=c+a, z=a+b$,代入有
\begin{align*}
  & \frac1{1+b+c}+\frac1{1+c+a}+\frac1{1+a+b}\le 1\\
  \iff & \underline{(1+x)(1+y)} + (1+y)(1+z) + (1+z)(1+x) \le \underline{(1+x)(1+y)(1+z)}\\
  \iff & \underline{z(1+x)(1+y)} - \underline{(1+y)(1+z)} - (1+z)(1+x) \ge 0\\
  \iff & \underline{zx(1+y)} - (1+y) - 1-x-z-\underline{zx}\ge 0\\
  \iff & zxy - 2 - (x+y+z)\ge 0\\
  \iff & (a+b)(b+c)(c+a) - 2 - 2(a+b+c)\ge 0\\
  \iff & (a+b+c)(ab+bc+ca) - abc -2(a+b+c)\ge 2 \quad(\text{把}abc=1\text{代入})\\
  \iff & (a+b+c)(ab+bc+ca-2)\ge 3
  % \iff & (1+c+a)(1+a+b) + (1+b+c)(1+a+b) + (1+b+c)(1+c+a) \le (1+b+c)(1+c+a)(1+a+b)\\
  % \iff & (b+c)(1+c+a)(1+a+b) - (1+b+c)(1+a+b) - (1+b+c)(1+c+a) \ge 0\\
  % \iff & (b+c)(c+a)(1+a+b) - (1+a+b) - (1+b+c)(1+c+a) \ge 0\\
  % \iff & (b+c)(c+a)(a+b) - (1+a+b) - (1+c+a) - (b+c) \ge 0\\
  % \iff & (b+c)(c+a)(a+b) - 2(1+a+b+c)\ge 0\\
  % \iff & cab+b^2c+a^2b+ab^2 +c^2a+bc^2+a^2c+abc - 2(1+a+b+c)\ge 0\\
  % \iff & b^2c+a^2b+ab^2 +c^2a+bc^2+a^2c - 2(a+b+c)\ge 0\\
  % \iff & (a+b+c)(ab+bc+ca)-3abc -2(a+b+c)\ge 0\\
  % \iff & (a+b+c)(ab+bc+ca-2)\ge 3
\end{align*}
而由AM-GM不等式,有$a+b+c\ge 3\sqrt[3]{abc}=3, ab+bc+ca\ge 3\sqrt[3]{ab\cdot bc\cdot ca}=3$,可得。


\begin{question}
  上题中能否推广到任意正整数$n$,若正数$a_1,a_2,\cdots,a_n$的乘积是$1$,且令$S$表示其和,即
  \begin{align*}
    S\equiv\sum_{i=1}^n a_i
  \end{align*}
  则有
  \begin{align*}
    \sum_{i=1}^n \frac{1}{1 - a_i + S}\le 1
  \end{align*}
  当且仅当$a_1=a_2=\cdots=a_n=1$时等号成立?
\end{question}

当$n=1$时,$a_1=1$,有
\begin{align*}
  \sum_{i=1}^n \frac{1}{1 - a_i + S} = \frac{1}{1} = 1
\end{align*}

当$n=2$时,由$a_1a_2=1$,代入后恒有
\begin{align*}
  \sum_{i=1}^n \frac{1}{1 - a_i + S} &= \frac{1}{1+a_2} + \frac1{1+a_1}\\
                                     &= \frac{1+a_1 + 1 + a_2}{(1+a_1)(1+a_2)}\\
                                     &=1
\end{align*}

当$n=3$时,由题\ref{q:1/1+a+b}已得证。若用引理\ref{lemma:product-ai-ge-n},有$S\ge 3$,从而
\begin{align*}
  \sum_{i=1}^n \frac{1}{1 - a_i + S} &\le \sum_{i=1}^n \frac{1}{1 - a_i + 3}\\
                                     &=   \sum_{i=1}^n \frac{1}{4 - a_i}\\
\end{align*}
{\color{red}上式可不一定,比如$a_i>4$呢?这方法不一定可行。}

用数学归纳法。设$n\le k$时成立,考虑$n=k+1$的情况。考虑数列$\{a_1,a_2,\cdots, a_{k-1}, a_ka_{k+1}\}$,共$k$个数,且其乘积为$1$,从而有
\begin{align*}
  \sum_{i=1}^{k} \frac{1}{1 - a_i' + S_{k+1}'}
\end{align*}
其中
\begin{align*}
  a_i'=
  \begin{cases}
    a_i &i=1,2,\cdots,k-1\\
    a_ka_{k+1} &i=k
  \end{cases} \quad
  S_{k+1}'= a_1+a_2+\cdots +a_{k-1} + a_ka_{k+1}
\end{align*}

%%%%%%%%%%%%%%%%%%%%%%%%%%%%%%%%%%%%%%%%
%%% Basic inequality examples
%%%%%%%%%%%%%%%%%%%%%%%%%%%%%%%%%%%%%%%%
\begin{example}
  证明对$\forall a,b,c\in\mathcal{R}$,有$a^2+b^2+c^2\ge ab+bc+ca$。
\end{example}
\begin{proof}
  由AM--GM不等式,有
\begin{align*}
  a^2+b^2\ge 2ab,\quad b^2+c^2\ge 2bc,\quad c^2+a^2\ge 2ca
\end{align*}
三式相加并除以2可得。该不等式可以推广:$\forall n>1, x_i\in\mathcal{R}(i=1,2,\cdots,n)$,有
\begin{align*}
  \sum_{i=1}^n x_i^2\ge \sum_{i=1}^n x_ix_{i+1}
\end{align*}
其中$x_{n+1}=x_1$。
\end{proof}

\begin{example}
  对任意非负数$a,b$及正整数$n\ge2$,有 $(n-1)a^n + b^n\ge na^{n-1}b$。
\end{example}
对序列$a^n,a^n,\cdots,a^n,b^n$(其中有$n-1$项$a^n$应用AM-GM不等式,有)
\begin{align*}
  \underbrace{a^n + a^n + \cdots + a^n}_{n-1\text{个}} + b^n \ge
  n\times \sqrt[n]{\left(a^n\right)^{n-1} b^n}
  = na^{n-1}b
\end{align*}
当$n=3,4,5$时,有
\begin{align*}
  2a^3 + b^3&\ge 3a^2b\\
  3a^4 + b^4&\ge 4a^3b\\
  4a^5 + b^5&\ge 5a^4b
\end{align*}

\begin{example}
  若正数$a,b,c$的乘积为$1$,求$ab+bc+ca$的极值。
\end{example}
\begin{proof}[提示]
\begin{align*}
  ab+bc+ca\ge 3\times\sqrt[3]{ab\cdot bc\cdot ca}=3\times\sqrt[3]{(abc)^2}=3
\end{align*}
当且仅当$a=b=c=1$时等号成立。另一方面,令$a=b=n,c=1/n^2$,则
\begin{align*}
  ab+bc+ca>ab=n^2\to+\infty (n\to+\infty)
\end{align*}
即$ab+bc+ca$无上界。
\end{proof}


\begin{example}
  若正数$a,b,c$的乘积为$1$,求$a+b+c$的极值。
\end{example}
\begin{align*}
  a+b+c\ge 3\times\sqrt[3]{abc}=3
\end{align*}
当且仅当$a=b=c=1$时等号成立。另一方面,令$a=b=n,c=1/n^2$,则
\begin{align*}
  a+b+c>a=n\to+\infty(n\to+\infty)
\end{align*}
即$a+b+c$无上界。

\begin{example}
  找出所有满足下列等式的实数$a,b,c,d$
  \begin{align*}
    a^2+b^2+c^2+d^2=a(b+c+d)
  \end{align*}
\end{example}

$\forall a,b,c,d\in\mathcal{R}$,有
\begin{align*}
  \left(\frac a2\right)^2\ge 0,\quad \left(\frac a2\right)^2 + b^2\ge ab,
  \quad \left(\frac a2\right)^2 + c^2\ge ac, \quad \left(\frac a2\right)^2 + d^2\ge ad
\end{align*}
四式相加,有
\begin{align*}
  a^2+b^2+c^2+d^2\ge a(b+c+d)
\end{align*}
当且仅当$a=0, \frac a2=b=c=d$时即$a=b=c=d=0$时成立。即原题只有$a=b=c=d=0$这一个解。

\begin{example}
  若正数$a,b,c$的平方和为1,求下式的最小值
  \begin{align*}
    S=\frac{a^2b^2}{c^2} + \frac{b^2c^2}{a^2} + \frac{c^2a^2}{b^2}
  \end{align*}
\end{example}

由
\begin{align*}
  \frac{a^2b^2}{c^2} + \frac{b^2c^2}{a^2}\ge 2b^2,\quad
  \frac{b^2c^2}{a^2} + \frac{c^2a^2}{b^2}\ge 2c^2,\quad
  \frac{c^2a^2}{b^2} + \frac{a^2b^2}{c^2}\ge 2a^2
\end{align*}
三式相加并除以2,有
\begin{align*}
  S\ge a^2+b^2+c^2=1
\end{align*}
当且仅当$\dfrac{a^2b^2}{c^2} = \dfrac{b^2c^2}{a^2} = \dfrac{c^2a^2}{b^2}$即$a=b=c=\frac1{\sqrt3}$时等号成立。


\begin{example}
  $x,y$是小于1的正数,则
  \begin{align*}
    \frac1{1-x^2} + \frac1{1-y^2} \ge \frac2{1-xy}
  \end{align*}
\end{example}
由$x+y\ge2\sqrt{xy}$,有
\begin{align*}
  \frac1{1-x^2} + \frac1{1-y^2} &\ge \frac2{\sqrt{(1-x^2)(1-y^2)}} &&\text{等号成立}\iff x=\pm y\\
  &=\frac2{\sqrt{1+x^2y^2-x^2-y^2}} \\
  &\ge \frac2{\sqrt{1+x^2y^2-2xy}} &&\text{等号成立}\iff x=y\\
  &=\frac2{\sqrt{(1-xy)^2}} \\
  &=\frac2{1-xy}
\end{align*}
当且仅当$x=y$时等号成立。

\begin{example}
  对任意非负数$a,b$,有
  \begin{align*}
    a^3+b^3\ge a^2b+ab^2
  \end{align*}
  推广:对任意非负数$a,b,c$,有
  \begin{align*}
    a^3+b^3+c^3\ge a^2b+b^2c+c^2a
  \end{align*}
  上述结论对任意$n>1$个非负数$a_1,a_2,\cdots,a_n$是否成立,即
  \begin{align*}
    \sum_{i=1}^n a_i^3\ge a_1^2a_2 + a_2^2a_3 + \cdots + a_{n-1}^2a_n + a_n^2a_1
  \end{align*}
\end{example}
由$a^3+b^3-a^2b-ab^2=a^2(a-b)+b^2(b-a)=(a-b)^2(a+b)\ge0$可得,当且仅当$a=b$时等号成立。

由$2a^3+b^3\ge3a^2b$,轮换$a,b,c$,有
\begin{align*}
  2a^3+b^3\ge3a^2b,\quad 2b^3+c^3\ge3b^2c,\quad 2c^3+a^3\ge3c^2a
\end{align*}
三式相加并除以3可得。

\begin{theorem}[Shapiro不等式,Shapiro's Cyclic Inequalities]
  $n$是正整数,序列$\{x_1,x_2,\cdots,x_n\}$是正数序列,则
  \begin{enumerate}
  \item 若$n$是正奇数且$3\le n\le 23$,则
    \begin{align*}
      \sum_{i=1}^{n} \frac{x_i}{x_{i+1} + x_{i+2}} \ge \frac{n}{2}  
    \end{align*}
    其中$x_{n+1} = x_1, x_{n+2} = x_2$。当且仅当$x_1=x_2=\cdots=x_n$时等号成立;

  \item 若$n$且正偶数且$4\le n\le 12$,则
    \begin{align*}
      \sum_{i=1}^{n} \frac{x_i}{x_{i+1} + x_{i+2}} \ge \frac{n}{2}  
    \end{align*}
    当且仅当$x_1=x_3=x_5=\cdots=x_{n-1}$且$x_2=x_4=x_6=\cdots=x_n$时等号成立;

  \item 若$n$是大于12的偶数或者是大于23的奇数,则存在正数序列$\{x_1,x_2,\cdots,x_n\}$,使得
    \begin{align*}
      \sum_{i=1}^{n} \frac{x_i}{x_{i+1} + x_{i+2}} < \frac{n}{2}  
    \end{align*}
  \end{enumerate}
\end{theorem}
\begin{proof}[说明]
  这里仅说明一下其证明历史。B.~A.~Troesch在1989年证明了(1),P.~J.~Bushell \& J.~B.~McLeod在2002年证明了(2),而(3)则是早在1979年就被 J.~L.~Searcy \& B.~A.~Troesch所证明。
\end{proof}

\begin{example}[Nesbitt不等式]
  对任意正数$a,b,c$,有
  \begin{align*}
    \frac{a}{b+c}+\frac{b}{c+a}+\frac{c}{a+b}\ge\frac32
  \end{align*}
\end{example}
\begin{proof}[提示]
  这是Shapiro不等式在$n=3$时的情形。有多种巧妙的证明方法。下面是其中三种。
  \begin{enumerate}
  \item 利用凸函数的Jensen不等式。记$S=a+b+c$,则
    \begin{align*}
      f(x)=\frac{x}{S-x}
    \end{align*}
    在$x\in[0,S)$上是凸的,应用Jensen不等式,则有
    \begin{align*}
      \frac{f(a) + f(b) + f(c)}{3}\ge f\left(\frac{a+b+c}{3}\right)=f\left(\frac{S}{3}\right)=\frac12
    \end{align*}
  \item 用换元法,消除难处理的分母。令$x=a+b,y=b+c,z=c+a$,则$x,y,z$是正数,且有
    \begin{align*}
      a=\frac{x-y+z}2,\quad b=\frac{x+y-z}2,\quad c=\frac{-x+y+z}2
    \end{align*}
    代入,有
    \begin{align*}
      \frac{a}{b+c}+\frac{b}{c+a}+\frac{c}{a+b}
      &= \frac{x-y+z}{2y} + \frac{x+y-z}{2z} + \frac{-x+y+z}{2x}\\
      &= \frac12\left(\frac{x+z}{y} + \frac{x+y}{z} + \frac{y+z}{x}\right)-\frac32\\
      &= \frac12\left( \underbrace{\left(\frac xy + \frac yx\right)}_{\text{应用AM--GM不等式}}
        +\left(\frac zy + \frac yz\right)
        +\left(\frac xz + \frac zx\right)
        \right)-\frac32\\
      &\ge \frac12\times ( 2 + 2 + 2) - \frac32=\frac32
      % &=\frac32
    \end{align*}
    当且仅当$\dfrac xy=\dfrac yx$, $\dfrac zy=\dfrac yz$, $\dfrac xz=\dfrac zx$即$x=y=z$时等号成立,亦即$a=b=c$时等号成立。

  \item 直接利用AM--HM不等式。由
    \begin{align*}
      & \frac{(a+b)+(b+c)+(c+a)}{3}\ge \frac{3}{\dfrac1{a+b}+\dfrac1{b+c}+\dfrac1{c+a}}\\
      \iff & \big[(a+b)+(b+c)+(c+a)\big]\left(\dfrac1{a+b}+\dfrac1{b+c}+\dfrac1{c+a}\right)\ge 9\\
      \iff & 2(a+b+c)\left(\dfrac1{a+b}+\dfrac1{b+c}+\dfrac1{c+a}\right)\ge 9\\
      \iff & 2\left(1+\dfrac{c}{a+b}+1+\dfrac{a}{b+c}+1+\dfrac{b}{c+a}\right)\ge 9
    \end{align*}
    展开可得。其等号成立的充要条件是$a+b=b+c=c+a$,即$a=b=c$。$\qedhere$
  \end{enumerate}
\end{proof}

\begin{question}
  证明Shapiro不等式在$n=4$时的情况,即对任意正数$a,b,c,d$,有
  \begin{align*}
    \frac{a}{b+c}+\frac{b}{c+d}+\frac{c}{d+a}+\frac{d}{a+b}\ge 2
  \end{align*}
\end{question}
% \begin{proof}[提示]
%   应用\ref{lemma:titu}的T2引理,有
%   \begin{align*}
%          &\frac{a}{b+c}+\frac{b}{c+d}+\frac{c}{d+a}+\frac{d}{a+b}\\
%     =\   &\frac{a^2}{a(b+c)}+\frac{b^2}{b(c+d)}+\frac{c^2}{c(d+a)}+\frac{d^2}{d(a+b)}\\
%     \ge\ &\dfrac{(a+b+c+d)^2}{a(b+c) + b(c+d) + c(d+a) + d(a+b)}
%   \end{align*}

%   \begin{align*}
%        & \frac{a}{b+c}+\frac{b}{c+d}+\frac{c}{d+a}+\frac{d}{a+b}\\
%     =\ & \left(\frac{a}{b+c}+\frac{c}{d+a}\right) + \left(\frac{b}{c+d}+\frac{d}{a+b}\right)\\
%     \ge\ & \dfrac{(a+c)^2}{a(b+c)+c(d+a)} + \dfrac{(b+d)^2}{b(c+d)+d(a+b)}\\
%     =\ & \dfrac{a^2+2ac+c^2}{ab+ac+cd+da}
%   \end{align*}
% \end{proof}

% 同样应用换元法简化分母,令$w=a+b$,$x=b+c$,$y=c+d$,$z=d+a$。先反求用$w$,$x$,$y$和$z$表示$a$:
% \begin{enumerate}
% \item 先看$w$,比$a$多了个$b$;
% \item $x$有$b$,$w-x=a-c$,又多减了个$c$;
% \item $y$里有$c$,$w-x+y=a+d$,则又多加了个$d$;
% \item $z$里有$d$,$w-x+y-z=2a$,多加了个$a$。
% \end{enumerate}
% 从而有$a=(w-x+y-z)/2$。类似地,有
% \begin{align*}
%   a&=\frac{w-x+y-z}2 & b&=\frac{x-y+z-w}2\\
%   c&=\frac{y-z+w-x}2 & d&=\frac{z-w+x-y}2
% \end{align*}
% 代入,有
% \begin{align*}
%      &\frac{a}{b+c}+\frac{b}{c+d}+\frac{c}{d+a}+\frac{d}{a+b}\\
%   =\ &\frac{w-x+y-z}{2x} + \frac{x-y+z-w}{2y} + \frac{y-z+w-x}{2z} + \frac{z-w+x-y}{2w}\\
%   =\ &\frac12\left( \frac{w+y-z}{x} + \frac{x+z-w}{y} + \frac{y+w-x}{z} + \frac{z+x-y}{w} - 4\right)
% \end{align*}
% 有负号,不能直接应用AM--GM不等式。代入继续化简,要证明的不等式等价于
% \begin{align*}
%        & \frac12\left( \frac{w+y-z}{x} + \frac{x+z-w}{y} + \frac{y+w-x}{z} + \frac{z+x-y}{w} - 4\right) \ge 2\\
%   \iff & \frac{w+y-z}{x} + \frac{x+z-w}{y} + \frac{y+w-x}{z} + \frac{z+x-y}{w} \ge 8\\
%   \iff & wyz(w+y-z) + xzw(x+z-w) + ywx(y+w-x) + zxy(z+x-y) \ge 8wxyz\\
%   \iff & w^2(yz-xz+xy) + x^2(wz-wy+yz) + y^2(
% \end{align*}


% {\color{red}似乎没用?}


\begin{question}
  证明:对任意正数$a,b,c$,有
  \begin{align*}
    \frac{a^3}{a^2+ab+b^2}+\frac{b^3}{b^2+bc+c^2}+\frac{c^3}{c^2+ca+a^2}
    \ge
    \frac{a+b+c}{3}
  \end{align*}
\end{question}

\begin{question}
  对任意正数$a_1,a_2,\cdots,a_n,b_1,b_2,\cdots,b_n$,有
  \begin{align*}
    \sum_{i=1}^n\frac{a_ib_i}{a_i+b_i}\le
    \frac{\sum\limits_{i=1}^n a_i \cdot \sum\limits_{i=1}^n b_i}
         {\sum\limits_{i=1}^n a_i + \sum\limits_{i=1}^n b_i}
  \end{align*}
\end{question}
对$n$用数学归纳法。

\begin{question}
  若$a_1,a_2,\cdots,a_n,b_1,b_2,\cdots,b_n$是正数,则
  \begin{align*}
    \sum_{i=1}^n \sqrt{a_ib_i}
    \le
    \sqrt{\sum_{i=1}^n a_i \cdot \sum_{i=1}^n b_i}
  \end{align*}
\end{question}
对$n$用数学归纳法。

\begin{question}
  对正数$a,b,c$,证明
  \begin{align*}
    \frac{5a^3-ab^2}{a+b} & \ge 3a^2-b^2\\
    \frac{5a^3-ab^2}{a+b} + \frac{5b^3-bc^2}{b+c} + \frac{5c^3-ca^2}{c+a} & \ge 2(a^2+b^2+c^2)
  \end{align*}
\end{question}
只需证明第一条即可,而第一条又等价于$2a^3+b^3\ge 3a^2b$。

\begin{question}
  对大于$2$的整数$n$,及非负数$x_1,x_2,\cdots,x_n$,若$x_1=0, x_n=1$,则存在整数$j\in[1,2,\cdots,n-1]$,使得
  \begin{align*}
    \left| x_{j-1} - 2x_j + x_{j+1} \right| \ge \frac{4}{n^2}
  \end{align*}
\end{question}
令$x_0=0,x_{n+1}=1$,且对$k=0,1,2,\cdots,n$,令$y_k=x_{k+1}-x_k$,则$y_0=y_n=0$,且$y_0+y_1+y_2+\cdots+y_n=1$。用反证法。假设对所有的$j=1,2,\cdots,n$,不等式不成立,从而有
\begin{align*}
  y_j - y_{j-1} \le \left| y_j-y_{j-1} \right| = \left| x_{j-1} - 2x_j + x_{j+1} \right| < \frac{4}{n^2}
\end{align*}
对$j=1,2,\cdots,k$加起来,则有
\begin{align*}
  y_k&=y_k-y_0=(y_1-y_0) + (y_2-y_1) + \cdots + (y_k-y_{k-1})\\
  &<\underbrace{\frac{4}{n^2}+\frac{4}{n^2}+\cdots+\frac{4}{n^2}}_{k\text{个}}\\
  &=\frac{4k}{n^2}
\end{align*}
同样$y_{j-1}-y_j<\frac{4}{n^2}$,加起来,有
\begin{align*}
  y_k&=y_{k}-y_n=(y_k-y_{k+1})+(y_{k+1}-y_{k+2})+\cdots+(y_{n-1}-y_{n})\\
  &<\underbrace{ \frac{4}{n^2}+\frac{4}{n^2}+\cdots+\frac{4}{n^2}}_{n-k\text{个}}\\
  &=\frac{4(n-k)}{n^2}
\end{align*}

若$n$是奇数,则
\begin{align*}
  1&=y_0+y_1+y_2+\cdots+y_n\\
  &=\left(y_1+y_1+\cdots+y_{\frac{n-1}2}\right)
  + \left(y_{\frac{n+1}2}+y_{\frac{n+1}2+1}+\cdots+y_n\right)\\
  & <\frac4{n^2}\left(
  \left(1+2+\cdots+\frac{n-1}2\right) +
  \left( \left(n-\frac{n+1}2\right) + \left(n-\frac{n+1}2-1\right) + \cdots + 1\right)
  \right)\\
  &=2\cdot\frac4{n^2}\left(1+2+\cdots+\frac{n-1}2\right)\\
  &=\frac8{n^2}\left( \left(1+\frac{n-1}2\right)\cdot\frac{n-1}2\cdot\frac12 \right)\\
  &=\frac{n^2-1}{n^2}<1
\end{align*}
矛盾。

若$n$是偶数,则
\begin{align*}
  1&=y_0+y_1+y_2+\cdots+y_n\\
  &=\left(y_1+y_1+\cdots+y_{\frac{n}2-1}\right)
  + \left(y_{\frac{n}2+1}+y_{\frac{n}2+2}+\cdots+y_n\right)
\end{align*}
剩余略。



\begin{example}[Proof GM $le$ AM]
  For non negative sequence $(a_1, a_2, \cdots, a_n)$, if there's one $a_i=0$, then GM=0, which is obvious that GM$le$AM, iff $a_1=a_2=\cdots=a_n=0$ the equal sign holds.

  So we just consider positive sequence $(a_1, a_2, \cdots, a_n)$. Let $\alpha$ be the arithmetic mean. If the numbers in the sequence is not all equal, then there must exist two elements $x$, $y$ s.t. $x < \alpha < y$.
  Let $X\equiv\alpha$, $Y\equiv  x+y-\alpha$, then $X+Y=x+y$ and also $XY\ge xy$, since
  \begin{align*}
    xy=&(\alpha-u)(\alpha+v)=\alpha^2+(v-u)\alpha-uv\\
    XY=&\alpha(x+y-\alpha)=\alpha(\alpha-u+\alpha+v-\alpha)=\alpha(\alpha-u+v)=\alpha^2+(v-u)\alpha
  \end{align*}
  which means, if we change the sequence of $(a_1,a_2,\cdots,a_n)$ in to another sequence simply by replace the two elements $x, y$ into $\alpha, Y$, then the sum and the AM of the sequence is not changed, but the GM get bigger, if not equal. Repeating this stratedgy, we get a sequnce consists of $n$ $\alpha$, i.e.
  \begin{align*}
    (\alpha,\quad \alpha,\quad \alpha,\quad \cdots,\quad \alpha)
  \end{align*}
  the GM of which is no less than that of the original one.
\end{example}

\begin{question}
  Proof QM--AM inequality.
\end{question}

\begin{question}
  Proof GM--HM inequality.
\end{question}


\include{cauchy-schwarz}
\include{minkowski}
\include{geometric-inequality}

\include{pigeonhole}
\include{buffalo-way}

\include{pythagorean_triples}
\include{circle}
\include{geometry}
\include{convex}
\include{graphs-of-functions}

\include{proofs-without-words}
\include{board-coverage}
\include{graphs}

\include{insight}
\include{fun}

\chapter{方法与策略}
\label{chap:methods}

\section{归纳}
\label{sec:induction}

\begin{definition}
  一个命题,若满足以下条件:
  \begin{enumerate}
  \item 存在非负整数$k_0$,命题在$n=k_0$时成立;
  \item 假设命题在$n\le k$时成立可以推出命题在$n=k+1$时成立。
  \end{enumerate}
  则该命题对任意整数$n\ge k_0$都成立。此证明方法称为数学归纳法。
\end{definition}

\begin{example}[求和]
  \begin{align*}
    \frac1{1\times2}+\frac1{2\times3}+\frac1{3\times4}+\cdots+\frac1{n\times (n+1)}
  \end{align*}
\end{example}

可以利用拆项,由$\dfrac1{k\times (k+1)}=\dfrac1k-\dfrac1{k+1}$,从而有
\begin{align*}
  &\frac1{1\times2}+\frac1{2\times3}+\frac1{3\times4}+\cdots+\frac1{(n-1)\times n}\\
  =&\left(\frac11-\frac13\right) + \left(\frac13-\frac14\right) + \cdots + \left(\frac1n-\frac1{n+1}\right)\\
  =&\frac11-\frac1{n+1}\\
  =&\frac{n}{n+1}
\end{align*}

若观察不到拆项的规律,可以考虑一下数学归纳法。记当$n=k$时的和为$S_k$,则
\begin{enumerate}
\item 当$n=1$时,有$S_1=\dfrac12$;
\item 当$n=2$时,有$S_2=\dfrac23$;
\item 当$n=3$时,有$S_3=\dfrac34$;
\item $\cdots$
\end{enumerate}
猜测,$\forall n$,有$S_n=\dfrac{n}{n+1}$。
\begin{proof}
  当$n=1$时,显然成立。

  设当$n\le k$时成立,考虑$n=k+1$的情况。
  \begin{align*}
    S_{k+1}&=S_k + \frac1{(k+1)\times(k+2)}
             =\frac{k}{k+1}+\frac1{(k+1)\times(k+2)}\\
           &=\frac{k(k+2)+1}{(k+1)(k+2)}
           =\frac{k^2+2k+1}{(k+1)(k+2)}
           =\frac{k+1}{k+2}
  \end{align*}
  从而对任意整数$n\ge1$,猜测成立。
\end{proof}



\section{特殊点方法}
\label{sec:special-point-method}

\begin{example}
  求解方程$x\cdot \lfloor x\rfloor=80$,其中$\lfloor x\rfloor$表示不大于$x$的最大整数(向下取整)。
\end{example}
\begin{proof}[解]
  可用猜测法。$x\cdot \lfloor x\rfloor$与$\lfloor x\rfloor ^2$接近,而与$80$最近的完全平方数是$81$,可以考虑$\lfloor x\rfloor=\pm 9$的情况。若$\lfloor x\rfloor=9$,则$x=80/9$,从而$\lfloor x\rfloor=8\ne 9$,矛盾。若$\lfloor x\rfloor=-9$,则$x=-80/9$,这个是吻合的。这种猜测法只是给出了某个解,并没有证明这个解就是所有的解。

  下面是完整解法。由定义,显然有$\lfloor x\rfloor\le x \le \lfloor x\rfloor + 1$。由原题,首先有$x>1$或者$x<-1$。

  \begin{enumerate}
  \item 设$x>1$,则
    \begin{align*}
      \begin{cases}
        \lfloor x\rfloor \cdot \lfloor x\rfloor  \le 80 &\implies \lfloor x\rfloor < 9\\
        \left(\lfloor x\rfloor + 1\right)\cdot \lfloor x\rfloor \ge 80 &\implies \lfloor x\rfloor \ge 9
      \end{cases}
    \end{align*}
    此时无解。
  \item 设$x<-1$,则同样有
    \begin{align*}
      \begin{cases}
        \lfloor x\rfloor \cdot \lfloor x\rfloor  \ge 80 &\implies \lfloor x\rfloor \le -9\\
        \left(\lfloor x\rfloor + 1\right)\cdot \lfloor x\rfloor \le 80 &\implies \lfloor x\rfloor \ge -9
      \end{cases}
    \end{align*}
    此时$\lfloor x\rfloor=-9$,从而$x=-\dfrac{80}{9}$。\qedhere
  \end{enumerate}
\end{proof}


\begin{example}
  已知单位圆上两相互垂直的弦$AB$与$CD$相交于点$M$,则$AB^2+(CM-DM)^2$是常数。

  \centering
  \begin{tikzpicture}[scale=1.0]
    \draw(0,0)circle(2);
    \coordinate[label=right:$B$] (B) at (30:2);
    \coordinate[label=left:$A$] (A) at (150:2);
    \coordinate[label=above:$C$] (C) at (110:2);
    \coordinate[label=below:$D$] (D) at (250:2);
    \tkzInterLL(A,B)(C,D)\tkzGetPoint{M}\tkzLabelPoints[above right](M)
    \tkzMarkRightAngle[color=blue](C,M,A)
    \draw(A)--(B) (C)--(D);
  \end{tikzpicture}
\end{example}
\begin{proof}
  考虑当$AB$与$CD$都是直径的特殊情况,此时$AB=2,CM=DM$,从而有$AB^2+(CM-DM)^2=2^2+0=4$。

  尝试证明对任意两垂直的弦,有$AB^2+(CM-DM)^2=4$。联想勾股定理,需要以$AB$为直角边,直径为斜边作直角三角形,作过点$A$的直径$AE$,则由勾股定理有$AB^2+BE^2=2^2$,从而只再需要证明$BE=|CM-DM|$即可。

  \begin{center}
    \begin{tikzpicture}[scale=1.0]
      \draw(0,0)circle(2);
      \coordinate[label=right:$B$] (B) at (30:2);
      \coordinate[label=left:$A$] (A) at (150:2);
      \coordinate[label=above:$C$] (C) at (110:2);
      \coordinate[label=below:$D$] (D) at (250:2);
      \coordinate[label=below right:$E$] (E) at (330:2);
      \coordinate[label=above right:$O$] (O) at (0,0);
      \tkzDrawPoint(O)
      \tkzInterLL(A,B)(C,D)\tkzGetPoint{M}\tkzLabelPoints[above right](M)
      \tkzMarkRightAngle[color=blue](C,M,A)
      \tkzMarkRightAngle[color=blue](A,B,E)

      \coordinate[label=left:$M'$] (M') at ($(C)!(E)!(D)$);
      \draw(A)--(B)--(E)--cycle (C)--(D);
      \draw[dashed](E)--(M');
      \tkzMarkRightAngle[color=blue](E,M',D)
      \draw[dashed](-3,0)--(3,0)node[pos=0,below]{$L$} node[pos=1,below]{$N$};
    \end{tikzpicture}
  \end{center}

  由上图,只需证明$DM'=CM$即可得到$BE=|DM-CM|$。而这可由对称性,即图形中的相关点关于直线$LN$对称可得。
\end{proof}


\section{构造法}
\label{sec:construction-method}

\begin{example}
  已知$a,b,c$满足方程组
  \begin{align*}
    a+b&=8\\
    ab-c^2+8\sqrt2c&=48
  \end{align*}
  求方程$bx^2+cx-a=0$的根。
\end{example}
\begin{proof}[解]
  变换条件中的方程组的形式,有
  \begin{align*}
    a+b&=8\\
    ab&=c^2-8\sqrt2c+48
  \end{align*}
  以$a,b$为根,构造一元二次方程$(y-a)(y-b)\equiv y^2 - (a+b)y + ab = 0$,即
  \begin{align*}
    y^2 - 8y + (c^2-8\sqrt2c+48) = 0
  \end{align*}
  由于此方程存在两根$a,b$,从而
  \begin{align*}
    & \Delta = (-8)^2 -4(c^2-8\sqrt2c + 48)\ge 0\\
    \implies& c^2-8\sqrt2c + 32\le 0\\
    \implies& \left(c-4\sqrt2\right)^2\le 0\\
    \implies& c=4\sqrt2
  \end{align*}
  从而可解得$a,b$,代入原方程可解得$x$。
\end{proof}


\begin{example}
  若实数$x,y$满足方程组
  \begin{align*}
    \frac{x}{3^3+4^3} + \frac{y}{3^3+6^3} &= 1\\[3pt]
    \frac{x}{5^3+4^3} + \frac{y}{5^3+6^3} &= 1
  \end{align*}
  求$x+y$。
\end{example}
\begin{proof}[解]
  观察原方程组,区别在于$3^3$和$5^3$,把$3^3$和$5^3$看作是以下方程的两个根
  \begin{align*}
    \frac{x}{t+4^3} + \frac{y}{t+6^3} = 1
  \end{align*}
  化简可得
  \begin{align*}
    t^2 - (x+y-4^3-6^3)t - (6^3x + 4^3y - 4^3\cdot 6^3)=0
  \end{align*}
  由韦达定理,有$3^3+5^3=x+y-4^3-6^3$,从而可得$x+y$。
\end{proof}


\section{奇偶分析法}
\label{sec:even-odd-method}

\begin{example}
  能否用下面5种图形各一个拼成一个$4\times5$的长方形?

  \centering
  \begin{tikzpicture}[scale=.5]
    \draw(0,0)grid(1,3)grid(2,2);
    \draw(3,0)grid(5,2);
    \draw(6,0)grid(7,4);
    \draw(8,0)grid(10,1) (9,1)grid(11,2);
    \draw(12,0)grid(15,1) (13,1)grid(14,2);

    \begin{scope}[shift={(5,-5)}]
      \draw(0,0)grid(5,4);
    \end{scope}
  \end{tikzpicture}
\end{example}
\begin{proof}[解]
  将$4\times5$的图形如棋盘般染色,如下图。
  \begin{center}
    \begin{tikzpicture}[scale=.5]
      \draw(0,0)grid(5,4);
      \fill(0,0)rectangle(1,1)
           (0,2)rectangle(1,3)
           (1,1)rectangle(2,2)
           (1,3)rectangle(2,4)
           (2,0)rectangle(3,1)
           (2,2)rectangle(3,3)
           (3,1)rectangle(4,2)
           (3,3)rectangle(4,4)
           (4,0)rectangle(5,1)
           (4,2)rectangle(5,3);
    \end{tikzpicture}
  \end{center}
  则无论如何摆放,前4种图形都会覆盖两个白格子两个黑格子,最后一种图形\tikz[scale=.25]{\draw(0,0)grid(3,1)(1,1)grid(2,2);}都会覆盖奇数个白格子奇数个黑格子(一个白格子加三个黑格子,或者三个白格子加一个黑格子)。按奇偶性,5种图形各一个,无法覆盖偶数个白格子和偶数个黑格子。
\end{proof}

\begin{example}
  若$a_1,a_2,a_3,\cdots,a_7$是$1,2,3,\cdots,7$这$7$个数的一个排列,则
  \begin{align*}
    P\equiv (a_1 - 1)(a_2-2)\cdots(a_7-7)
  \end{align*}
  是偶数。
\end{example}
\begin{proof}
  反证。若$P$是奇数,则$a_1-1, a_2-2, a_3-3,\cdots, a_7-7$都是奇数,从而其和也是奇数。然而$a_1,a_2,\cdots,a_7$是$1,2,\cdots,7$的排列,从而其和$(a_1-1) + (a_2-2) + \cdots + (a_7-7)=(a_1+a_2+\cdots + a_7)-(1+2+\cdots+7)=0$应为偶数。矛盾。
\end{proof}

\begin{example}
  在区间$(1,\sqrt2)$内任取$n$个点,按大小顺序记为$x_1,x_2,\cdots,x_n$,再加上$x_0=1$,$x_{n+1}=\sqrt2$

  {\color{red}后面的呢?}
\end{example}

\begin{example}
  某班有49位学生,坐成7行7列。每个座位的前、后、左、右的座叫做它的“邻座”。有没有一种调换位置的方案,使得这49位同学都换到他的邻座上去?\footnote{出自罗增儒的《中学数学竞赛的内容与方法》。}
\end{example}
\begin{proof}[解]用奇偶性。
    \begin{center}
    \begin{tikzpicture}[scale=.5]
      \draw(0,0)grid(7,7);
      \foreach \x/\y in{%
        1/0, 3/0, 5/0,
        0/1, 2/1, 4/1, 6/1,
        1/2, 3/2, 5/2,
        0/3, 2/3, 4/3, 6/3,
        1/4, 3/4, 5/4,
        0/5, 2/5, 4/5, 6/5,
        1/6, 3/6, 5/6}{
        \fill(\x,\y)rectangle(\x+1,\y+1);
      }
    \end{tikzpicture}
  \end{center}
  每个人换位置后都会改变奇偶性,即白变黑,黑变白,从而若有一种能实现的换法则必须黑白同样多,而这显然是不成立的。
\end{proof}

\section{假设}
\label{sec:assumption}

\begin{example}[鸡兔同笼]
  一个笼子里着好多鸡和兔子。上面看去,可以数到有10个头,从下面看去,可以数到有28只脚。问笼子里有几只鸡几只兔子?
\end{example}
\begin{proof}[提示]
  这个用方程来说显然是非常简单的。若用假设法的话,则更直观。

  假设笼子里的全是鸡,那10个头就是10个鸡,总共应该有$10\times2=20$只脚。现在多了$28-20=8$只脚,只能是假设的10个鸡里有几个应该要换成兔子。而每1个鸡换成1个兔子,脚的总数就会多$4-2=2$只,所以应该是$8\div 2=4$只鸡要换成兔子。即兔子有$4$只,鸡有$10-4=6$只。

  若假设全是兔子,也可以用类似的分析方法得到相同的答案。
\end{proof}

\begin{example}
  27 apples in $3\times3\times3$ box. A worm, in the apple which
  resides at the center of the box, wants to eat all of the apples
  with minimum move, i.e., it can move up, down, forward, backward,
  left or right. Can it walk through all of the apples without
  touching an apple second time?
\end{example}
\begin{proof}[hint]
  Even odd coloring all of the apples, then there's 13 apples the same
  color(make it red) as the center one, 14 apples the other color(make
  it green). Then after every move, the target apple changes color,
  i.e, From the starting red(R) apple, the color sequence should be

  R, G, R, G, R, G, $\cdots$, R, G, R

  Since there're 27 apples in total, the first and last must be R,
  which means there should be 14 Red apples and 13 Green
  apple. Conflicts.

  \begin{center}
    \begin{tikzpicture}[scale=.8]
      % horizontal lines
      \foreach \y in{0,1,2}{
        \filldraw[fill=gray!20](0,\y,0)--(0,\y,2)--(2,\y,2)--(2,\y,0)--cycle;
        \draw(0,\y,1)--(2,\y,1);
        \draw(1,\y,0)--(1,\y,2);
      }
      % vertical lines
      \foreach \x in{0,1,2}{
        \foreach \z in{0,1,2}{
          \ifnum2=\x
          \draw(\x,0,\z)--(\x,2,\z);
          \else\ifnum2=\z
          \draw(\x,0,\z)--(\x,2,\z);
          \else
          \draw[dashed](\x,0,\z)--(\x,2,\z);
          \fi\fi
        }}
      \foreach \x in{0,1,2}{
        \foreach \y in{0,1,2}{
          \foreach \z in{0,1,2}{
            \pgfmathsetmacro\result{\x + \y + \z}
            \ifodd\result
              \fill[ball color=red](\x,\y,\z)circle(.1);
            \else
              \fill[ball color=green](\x,\y,\z)circle(.1);
            \fi
          }}}
      \fill[ball color=red](1,1,1)circle(.2);
      \begin{scope}[shift={(4,0)}]
        \node(0,-3){Top layer};
        \foreach \y in{2}{
          \filldraw[fill=gray!20](0,\y,0)--(0,\y,2)--(2,\y,2)--(2,\y,0)--cycle;
          \draw(0,\y,1)--(2,\y,1);
          \draw(1,\y,0)--(1,\y,2);
          \foreach \x in{0,1,2}{
            \foreach \z in{0,1,2}{
              \pgfmathsetmacro\result{\x + \y + \z}
              \ifodd\result
              \fill[ball color=red](\x,\y,\z)circle(.1);
              \else
              \fill[ball color=green](\x,\y,\z)circle(.1);
              \fi
            }}}
      \end{scope}
      \begin{scope}[shift={(8,0)}]
        \node(0,-3){Middle layer};
        \foreach \y in{1}{
          \filldraw[fill=gray!20](0,\y,0)--(0,\y,2)--(2,\y,2)--(2,\y,0)--cycle;
          \draw(0,\y,1)--(2,\y,1);
          \draw(1,\y,0)--(1,\y,2);
          \foreach \x in{0,1,2}{
            \foreach \z in{0,1,2}{
              \pgfmathsetmacro\result{\x + \y + \z}
              \ifodd\result
              \fill[ball color=red](\x,\y,\z)circle(.1);
              \else
              \fill[ball color=green](\x,\y,\z)circle(.1);
              \fi
            }}}
        \fill[ball color=red](1,1,1)circle(.2);
      \end{scope}
      \begin{scope}[shift={(12,0)}]
        \node(0,-3){Bottom layer};
        \foreach \y in{0}{
          \filldraw[fill=gray!20](0,\y,0)--(0,\y,2)--(2,\y,2)--(2,\y,0)--cycle;
          \draw(0,\y,1)--(2,\y,1);
          \draw(1,\y,0)--(1,\y,2);
          \foreach \x in{0,1,2}{
            \foreach \z in{0,1,2}{
              \pgfmathsetmacro\result{\x + \y + \z}
              \ifodd\result
              \fill[ball color=red](\x,\y,\z)circle(.1);
              \else
              \fill[ball color=green](\x,\y,\z)circle(.1);
              \fi
            }}}
      \end{scope}
    \end{tikzpicture}
  \end{center}
\end{proof}


\section{齐次与换元}
\label{sec:homogeneous-and-substitution}

\begin{example}[美国竞赛题]
  求解方程
  \begin{align*}
    4^x + 6^x = 9^x
  \end{align*}
\end{example}
\begin{proof}[提示]
  两边除以$4^x$或者$6^x$或者$9^x$,再换元。以除$4^x$为例,可得
  \begin{align*}
                               4^x + 6^x &= 9^x\\
    \iff\quad 1 + \left(\frac32\right)^x &= \left[\left(\frac32\right)^x\right]^2
  \end{align*}
  令$t\equiv (3/2)^x$,可得一个一元二次方程。
\end{proof}


\begin{example}[安徽竞赛题]
  求解方程
  \begin{align*}
    4^x + 9^x + 25^x = 6^x + 10^x + 15^x
  \end{align*}
\end{example}
\begin{proof}[提示]
  观察方程中数字
  \begin{align*}
    4 &= 2^2,       & 9 &= 3^2,        & 25 &= 5^2\\
    6 &= 2\times 3, & 10&= 2\times  5, & 15 &= 3\times 5
  \end{align*}
  可以联想$a^2+b^2+c^2$与$ab+bc+ca$的关系。令
  \begin{align*}
    a\equiv 2^x, \quad b\equiv 3^x, \quad c\equiv 5^x
  \end{align*}
  代入则有
  \begin{align*}
    a^2 + b^2 + c^2 = ab + bc + ca
  \end{align*}
  两边乘2可得
  \begin{align*}
    (a-b)^2 + (b-c)^2 + (c-a)^2 = 0
  \end{align*}
  即当且仅当$a=b=c$时等号成立。
\end{proof}



\begin{example}
  若 $x\in\mathcal{R}$,求解方程:
  \begin{align*}
    \frac1{x^2+11x-8} + \frac1{x^2+2x-8} + \frac1{x^2-13x-8} = 0
  \end{align*}
\end{example}
\begin{proof}
  观察分母共同点。显然,$x\ne 0$。从而 $\exists a\in\mathcal{R}$,使得 $ax = x^2-8$。
  \begin{align*}
    \frac1{(a+11)x} + \frac1{(a+2)x} + \frac1{(a-13)x} = 0\\
    (a^2-11-26)+(a^2-2-143)+(a^2+13+22)=0\\
    3a^2=147\implies a=\pm7
  \end{align*}
  再将$a$代入$ax=x^2-8$,可得$x$。
\end{proof}



% \begin{example}
%   Given $a+b+c$, $\any x\in\{\mathcal{R}-\{a,b,c\}\}$,
% \end{example}



\section{其它}
\label{sec:misc-method}

\begin{example}[德国竞赛题]
  求解方程组
  \begin{align*}
    \begin{cases}
      x^3 + 1 - xy^2 - y^2 = 0\\
      y^3 + 1 - x^2y - x^2 = 0
    \end{cases}
  \end{align*}
\end{example}
\begin{proof}[提示]
  联想立方差公式。两式相减可得
  \begin{align*}
    \underline{x^3 - y^3} - \uwave{xy^2} - y^2 + \uwave{x^2y} + x^2 = 0
  \end{align*}
  两两组合,因式分解可得
  \begin{align*}
    (x-y)(x+y)(x+y+1)=0
  \end{align*}
  分三种情况$x-y=0$、$x+y=0$及$x+y+1=0$分别考虑可得。
\end{proof}


\begin{example}
  Find the range of $a$, s.t. hyperbolic
  \begin{align*}
    f(x)\equiv ax^2 + 4ax
  \end{align*}
  has common point with line segment between point $P(2,2)$ and $Q(2+2a, 5a)$.
\end{example}
\begin{proof}[Hint]
  The curve $f$ and point $Q$ are both dynamic, make one of them static.

  Let $t=y/a$, then it's equivalent to the curve of
  \begin{align*}
    g(x) = x^2 + 4x
  \end{align*}
  has common point with line segment between points $(2, 2/a)$ and $(2+2a, 5)$, which equivalent to
  \begin{align*}
    \left(g(2) - 2/a\right) \cdot \left( g(2+2a) - 5 \right) \le 0
  \end{align*}
  i.e.
  \begin{align*}
    (2a+3)(2a+1)(2a-3)/a \ge 0
  \end{align*}
  Consider the intervals $(-\infty, -2/3)$, $(-2/3, -1/2)$, $(-1/2, 0)$, $(0, 3/2)$ and $(3/2, +\infty)$ and the ending points respectively, the result can be easily deduced.
\end{proof}



\chapter{数形转换}
\label{chap:number-to-graph}

\section{方程与图形}
\label{sec:equation-and-graph}

\begin{example}
  已知实数$x,y$满足$2x+3y=6$,求$xy$的最大值。
\end{example}
\begin{proof}[提示]
  此类问题的常见的方法是换元求二次方程最大最小值。如果结合图形考虑,如在笛卡尔坐标下,则
  \begin{align*}
    2x+3y=6
  \end{align*}
  是一条直线,而$xy$则是面积(不考虑符号),如图。
  
  \begin{center}
    \begin{tikzpicture}[scale=1.0]
      \coordinate[label=above right:${(x,y)}$] (A) at(1.5,1);
      \coordinate[label=below left:2](B) at (0,2);
      \coordinate[label=below:3](C) at (3,0);

      \fill[color=red!20](0,0) rectangle(A) node[midway]{$xy$};
      \node at(0.75,0.5) {$xy$};
      \draw[->](-1,0)--(4,0) node[below right]{$x$};
      \draw[->](0,-1)--(0,3) node[left]{$y$};
      \draw(-1.5,3)--(4.5,-1);
      \tkzDrawPoints(A,B,C)
    \end{tikzpicture}
  \end{center}  

  显然,$xy$只能在直线$2x+3y=6$落于第一象限的部分内取得最大值,因为其它象限直线上的点与原点组成的矩形面积$xy$是“负”的。原问题与下面的几何问题等价:

  直角边长分别为2和3的直角三角形,在斜边上任取一点向两直角边作垂线,求两垂线与直角边构成的矩形面积最大时该点的位置。
\end{proof}

\begin{example}
  先特殊化,等腰直角三角形斜边上任取一点,求对应矩形面积取得最大值时该点的位置。
\end{example}
\begin{proof}[提示]
  直觉上应该是斜边的中点\footnote{数学上直觉很重要。}。这里猜不出来也不要紧,可以想象一下,在顶点的时候,面积是零,所以从顶点出发往中间走的时候,面积是增加的。又由于两个顶点其实是对称的,想象两个动点以相同的速度分别从两个顶点往中间出发,面积都是增加的,当两个动点在斜边的中点相遇的时候,面积应该就是最大值。

  \begin{center}
    \begin{tikzpicture}[scale=1.0]
      \begin{scope}[shift={(0,0)}]
        \coordinate(O)at(0,0);
        \coordinate(A)at(4,0);
        \coordinate(B)at(0,4);
        \coordinate(C)at(2,2);
        \coordinate(Cx)at(2,0);
        \coordinate(Cy)at(0,2);
        
        \coordinate(D)at(2.5,1.5);
        \coordinate(E)at(1.5,2.5);
        \fill[color=red!10](O)rectangle(C);
        \draw(O)--(A)--(B)--cycle;
        \draw(Cx)--(C)--(Cy);
        \tkzDrawPoints(C)
      \end{scope}
      \begin{scope}[shift={(4.5,0)}]
        \coordinate(O)at(0,0);
        \coordinate(A)at(4,0);
        \coordinate(B)at(0,4);
        \coordinate(C)at(2,2);
        \coordinate(Cx)at(2,0);
        \coordinate(Cy)at(0,2);
        \coordinate(D)at(1.5,2.5);
        \coordinate(Dx)at(1.5,0);
        \coordinate(Dy)at(0,2.5);
        %\fill[color=red!10](O)rectangle(C);
        \fill[color=red!10](C)--(D)--(Dx)--(Cx)--cycle;
        \fill[pattern=north east lines](C)--(D)--(Dy)--(Cy)--cycle;
        \draw[dashed](Cx)--(C)--(Cy);
        \draw(Dx)--(D)--(Dy);
        \draw(O)--(A)--(B)--cycle;
        \tkzDrawPoints(C)
      \end{scope}
      \begin{scope}[shift={(9,0)}]
        \coordinate(O)at(0,0);
        \coordinate(A)at(4,0);
        \coordinate(B)at(0,4);
        \coordinate(C)at(2,2);
        \coordinate(Cx)at(2,0);
        \coordinate(Cy)at(0,2);
        \coordinate(D)at(1.5,2.5);
        \coordinate(Dx)at(1.5,0);
        \coordinate(Dy)at(0,2.5);
        \coordinate(E)at(2.5,1.5);
        \coordinate(Ey)at(0,1.5);
        % \fill[color=red!10](O)rectangle(C);
        \fill[color=red!10](C)--(E)--(Ey)--(Cy)--cycle;
        \fill[pattern=north east lines](C)--(D)--(Dy)--(Cy)--cycle;
        \draw[dashed](Cx)--(C)--(Cy) (E)--(Ey);
        \draw(Dx)--(D)--(Dy);
        \draw(O)--(A)--(B)--cycle;
        \tkzDrawPoints(C)
      \end{scope}
    \end{tikzpicture}
  \end{center}

  由图,可以看出,当动点离开中点时,面积是减小的(因为增加的面积小于减小的面积,从而面积净增量是负的,把第二图中的增加区域等价变换到第三图中的同等染色区域就很明显了)。也就是动点在斜边中点时,矩形面积达到最大,为等腰直角三角形的一半。

  \begin{center}
    \begin{tikzpicture}[scale=1.0]
      \coordinate(O)at(0,0);
      \coordinate(A)at(4,0);
      \coordinate(B)at(4,4);
      \coordinate(C)at(0,4);
      \coordinate(D)at(2,2);
      \coordinate(Dx1)at(2,0);
      \coordinate(Dx2)at(2,4);
      \coordinate(Dy1)at(0,2);
      \coordinate(Dy2)at(4,2);
      \coordinate(E)at(1.5,2.5);
      \fill[color=red!10](Dx1)--(D)--(Dy2)--++(0,0.5)--(E)--(1.5,0)--cycle;
      \fill[pattern=north east lines](D)--(Dx2)--++(-0.5,0)--(E)--(0,2.5)--(Dy1)--cycle;
      \draw(O)--(A)--(B)--(C)--cycle (A)--(C) (0,2)--(4,2) (2,0)--(2,4);
      \draw[dashed](1.5,0)--(1.5,4) (0,2.5)--(4,2.5);
      \tkzDrawPoints(D, E)
    \end{tikzpicture}
  \end{center}

  把等腰直角三角形补充为一个正方形,则更明显。当动点从中点移动时,减小的一圈是往外扩张(与$\frac14$个小正方形比)的,而增加的一圈则是往内收缩的,其面积大小是很明显的\footnote{实际上其面积差值是中间重叠部分的两倍。}。
\end{proof}

\begin{example}
  利用上面等腰直角三角形的结论,求满足$2x+3y=6$是$xy$的最大值。
\end{example}
\begin{proof}[提示]
  换元,使直线的斜率的绝对值为1,从而变换为等腰直角三角形。令$u\equiv2x, v\equiv3y$,则原问题变换为求满足$u+v=6$下的$uv/6$的最大值。$uv/6$取得最大值时$(u,v)$点与$uv$取得取大值时的$(u,v)$点是一致的,从而当$u=v=3$时,$uv/6=3\times3/6=1.5$为最大值。
\end{proof}

\section{图形变代数}
\label{sec:graph-to-algebra}

\begin{example}[2014 AMC12B 21]
  单位正方形$ABCD$中,矩形$JKHG$与$EBCF$全等。问$AE$的长度是多少?
  \begin{center}
    \begin{tikzpicture}[scale=1.0]
      \coordinate[label=below left:$A$](A)at(0,0);
      \coordinate[label=below right:$B$](B)at(4,0);
      \coordinate[label=above right:$C$](C)at(4,4);
      \coordinate[label=above left:$D$](D)at(0,4);

      \coordinate[label=below:$E$](E)at(1.0718, 0);
      \coordinate[label=above:$F$](F)at(1.0718, 4);
      \fill[color=red!10](A)--(E)--(F)--(D)--cycle;

      \coordinate[label=below:$J$](J)at(2, 0);
      \coordinate[label=left:$K$](K)at(1.0718, 0.536);
      \coordinate[label=above:$H$](H)at(3.0718, 4);
      \coordinate[label=right:$G$](G)at(4, 3.464);

      \filldraw[fill=blue!10](G)--(H)--(K)--(J)--cycle;

      \draw(A)--(B)--(C)--(D)--cycle;
      \draw(E)--(F);
    \end{tikzpicture}
  \end{center}
  \begin{align*}
    (A)\quad \frac12 (\sqrt6 - 2)\qquad
    (B)\quad \frac14             \qquad
    (C)\quad 2 - \sqrt3          \qquad
    (D)\quad \frac{\sqrt3}6      \qquad
    (E)\quad 1 - \frac{\sqrt2}2  \qquad
  \end{align*}
\end{example}
\begin{proof}[提示]
  第一种,使用几何方法,计算简单,但需要深刻的图形洞察力。$JG=AD=1$,猜测$J$是中点,则$BJ=0.5$,从而$\angle JGB=30^\circ$,剩余简单。

  第二种,设变量,导方程组,变换为含三角函数的代数运算。这种对图形洞察力的要求没那么高,但涉及的计算量有可能会非常复杂。
\end{proof}

\include{numerical-method}
\include{math-by-physics}


\chapter{逻辑}
\label{chap:logic}
\def\BKC{\cellcolor{gray!50}}   % cell background color

\section{排除法}
\begin{example}
  警察在审问四个嫌疑犯:甲,乙,丙,丁。他们的回话如下:
  \fbox{\parbox{\textwidth}{
      \begin{itemize}
      \item 甲说:我不是罪犯
      \item 乙说:丁是罪犯
      \item 丙说:乙是罪犯
      \item 丁说:我不是罪犯
      \end{itemize}
    }}
  研究完其它材料后警察发现以上四人只有一个人说假话。警察已经知道说假话的是罪犯,说真话的不是罪犯。那么,谁是罪犯?
\end{example}
\begin{proof}[提示]
  如果注意到乙和丁的话是互斥的,就会知道乙和丁中必有一人说的假话。从而甲和丙说的是真话,从而由丙的话可知乙说的是假话是罪犯。
  
  如果注意不到乙和丁的话是互斥的这一点,那么可以用通用的方法来找出说假话的人,即甲乙丙丁四人逐一排除,若剩余一人无法排除,则就是此人说假话。
  \begin{enumerate}
  \item 假设甲说假话,则其余三人说的都是真话,即以下四句为真:
    \begin{align*}
      \text{\numcircled1 甲是罪犯;\numcircled2 丁是罪犯;\numcircled3 乙是罪犯;\numcircled4 丁不是罪犯}
    \end{align*}
    从而有以下表格:

    \begin{center}
      \begin{tabular}{c|c|c|c|c}
        \hline
             & 甲 & 乙 & 丙 & 丁\\
        \hline
        罪犯 & $\checkmark^1$  & $\checkmark^3$  &    & $\checkmark^2\ \xmark^4$\\
        \hline
      \end{tabular}
    \end{center}

    其中\checkmark 代表此人是罪犯,$\xmark$代表此人不是罪犯,数字代表该符号对应于第几句话。由表格,可以看出两个矛盾的地方:\numcircled1 共有3个罪犯,但实际应只有一个罪犯;\numcircled2 丁既是罪犯又不是罪犯(既有\checkmark 符号,又有$\xmark$符号)。
    
    从而排除甲说假话。

  \item 假设乙说的假话,则其余三人说的是真话,即以下四句为真:
    \begin{align*}
      \text{\numcircled1 甲不是罪犯;\numcircled2 丁不是罪犯;\numcircled3 乙是罪犯;\numcircled4 丁不是罪犯}
    \end{align*}
    从而有以下表格:
    \begin{center}
      \begin{tabular}{c|c|c|c|c}
        \hline
             & 甲 & 乙 & 丙 & 丁\\
        \hline
        罪犯 & $\xmark^1$  & $\checkmark^3$  &    & $\xmark^2\ \xmark^4$\\
        \hline
      \end{tabular}
    \end{center}

    找不到矛盾。

  \item 假设丙说的是假话,则欺其余三人说的是真话,即以下四句为真:
    \begin{align*}
      \text{\numcircled1 甲不是罪犯;\numcircled2 丁是罪犯;\numcircled3 乙不是罪犯;\numcircled4 丁不是罪犯}
    \end{align*}
    从而有以下表格:

    \begin{center}
      \begin{tabular}{c|c|c|c|c}
        \hline
             & 甲 & 乙 & 丙 & 丁\\
        \hline
        罪犯 & $\xmark^1$  & $\xmark^3$  &    & $\checkmark^2\ \xmark^4$\\
        \hline
      \end{tabular}
    \end{center}
    从表格可以看出丁既是罪犯又不是罪犯,矛盾。排除丙说假话。
    
  \item 假设丁说的是假话,则欺其余三人说的是真话,即以下四句为真:
    \begin{align*}
      \text{\numcircled1 甲不是罪犯;\numcircled2 丁是罪犯;\numcircled3 乙是罪犯;\numcircled4 丁是罪犯}
    \end{align*}
    从而有以下表格:

    \begin{center}
      \begin{tabular}{c|c|c|c|c}
        \hline
             & 甲 & 乙 & 丙 & 丁\\
        \hline
        罪犯 & $\xmark^1$  & $\checkmark^3$  &    & $\checkmark^2\ \checkmark^4$\\
        \hline
      \end{tabular}
    \end{center}

    此处隐藏的条件是,说假话的是罪犯,说真话的不是罪犯。由于只有一个人说假话,从而只有一个罪犯,而由上面的表格容易看出乙和丁都是罪犯,这与只有一个罪犯矛盾。排除丁说假话。
  \end{enumerate}

  这样排除完甲丙丁三人,剩余一人,乙就是说假话的人,即罪犯就是乙。
\end{proof}


\begin{example}[歌唱家的年龄]
  甲乙丙丁四人正在讨论某位歌唱家的年龄:
  \begin{itemize}
  \item 甲说:她不会超过25岁。
  \item 乙说:她不会超过30岁。
  \item 丙说:她绝对在35岁以上。
  \item 丁说:她在40岁以下。
  \end{itemize}
  事实是这四人中只有一人说对了。那么请猜测下歌唱家的年龄范围。
\end{example}
\begin{proof}[提示]
  记歌唱家的真实年龄是$x$,那么四人的话等价于
  \begin{align*}
    \text{\numcircled1} x\le25,\quad \text{\numcircled2} x\le30,\quad \text{\numcircled3} x>35,\quad \text{\numcircled4} x<40
  \end{align*}
  如果用数轴表示,则会非常直观。

  \begin{center}
    \begin{tikzpicture}[scale=.2]
      \draw[<->](-5,0)--(55,0);
      \foreach \x in {0,10,20,30,40,50}{
        \draw(\x,0) node[below]{$\x$}--(\x,1);
      }
      \draw[->](25,0)--(25,3)--(-5,3)node[left]{$x\le25$};
      \draw[->](30,0)--(30,6)--(-5,6)node[left]{$x\le30$};
      \draw[->](35,0)--(35,3)--(55,3)node[right]{$x>35$};
      \draw[->](40,0)--(40,9)--(-5,9)node[left]{$x<40$};
      \draw[fill=black](25,0)circle(8pt) (30,0)circle(8pt);
      \draw[fill=white](40,0)circle(8pt) (35,0)circle(8pt);
    \end{tikzpicture}
  \end{center}
  或者下面这种表示方式:
  \begin{center}
    \begin{tikzpicture}[scale=.2]
      \draw[<->](-5,0)--(55,0);
      \foreach \x in {0,10,20,30,40,50}{
        \draw(\x,0) node[below]{$\x$}--(\x,1);
      }
      \draw[->,line width=2pt](25,-3)--(-5,-3)node[left]{$x\le25$};
      \draw[->,line width=2pt](30,-6)--(-5,-6)node[left]{$x\le30$};
      \draw[->,line width=2pt](35,-3)--(55,-3)node[right]{$x>35$};
      \draw[->,line width=2pt](40,-9)--(-5,-9)node[left]{$x<40$};
      \draw[fill=black](25,-3)circle(10pt) (30,-6)circle(10pt);
      \draw[fill=white](40,-9)circle(10pt) (35,-3)circle(10pt);
    \end{tikzpicture}
  \end{center}

  其中实心圈表示包含该点,空心圈表示不含此点。由上两图可以看出,只有$(30,35]$及$[40,+\infty)$这两个区间才是只被一个条件所覆盖的。
\end{proof}


\begin{example}
  有个遭遇到海难的人最终在一个小岛上顽强地生存了下来。有一天他被毒蛇咬伤了,需要用$3$升的纯净水来配药解毒,一点不能多一点不能少。但由于物资匮乏,他手里只有一个$5$升和$6$升的两个容器,以及之前他用蒸馏的方法获取到的足够多的备用纯净水。由于时间紧迫,请帮他用最少的步骤量出精确的$3$升纯净水。
\end{example}
\begin{proof}[提示]
  其中一种方法如表格~\ref{tab:water-ops}所示。经过若干步骤后6升容器里有纯净水3升。
  \begin{table}[htbp]
    \centering
    \caption{倒水操作}
    \label{tab:water-ops}
    \renewcommand*{\arraystretch}{1.0}
    \begin{tabular}{clcc}
      \toprule[1.5pt]
      % Use multicolumn to make the multirow center horizontally
      \multirow{2}{*}{步骤} & \multicolumn{1}{c}{\multirow{2}{*}{操作}} & \multicolumn{2}{c}{操作后容器含水量(升)} \\
      \cline{3-4}
           &      & $5$升容器 & $6$升容器 \\
      \midrule[1pt]
      0 & (开始状态)                       & 0 & 0\\
      \midrule[1pt]
      1 & 用备用纯净水灌满6升容器          & 0 & 6\\
      2 & 用6升容器里的水灌满5升容器       & 5 & 1\\
      3 & 清空5升容器                      & 0 & 1\\
      4 & 6升容器里的剩余的水全倒入5升容器 & 1 & 0\\
      5 & 用备用纯净水灌满6升容器          & 1 & 6\\
      6 & 用6升容器里的水灌满5升容器       & 5 & 2\\
      7 & 清空5升容器                      & 0 & 2\\
      8 & 6升容器里的剩余的水全倒入5升容器 & 2 & 0\\
      9 & 用备用纯净水灌满6升容器          & 2 & 6\\
     10 & 用6升容器里的水灌满5升容器       & 5 & 3\\
      \bottomrule[1.5pt]
    \end{tabular}
  \end{table}

  也可以用有向图的方法。记$(x,y)$分别为5升容器和6升容器中装了多少升水,则开始状态为$(0,0)$,目标状态为$(x,3)$或者$(3,y)$。
  \begin{center}\tiny
    \begin{tikzpicture}[scale=1.3]
      \node(N00)[draw,circle,pattern=north west lines,pattern color=blue!50]at(2,0){$0,0$};
      \node(N56)[draw,circle]at(1,0){$5,6$};

      \node(N50)[draw,circle]at(1,-1){$5,0$};
      \node(N05)[draw,circle]at(2,-1){$0,5$};
      \node(N55)[draw,circle]at(3,-1){$5,5$};
      \node(N46)[draw,circle]at(4,-1){$4,6$};
      \node(N40)[draw,circle]at(5,-1){$4,0$};
      \node(N04)[draw,circle]at(6,-1){$0,4$};
      \node(N54)[draw,circle]at(7,-1){$5,4$};
      \node(N36)[draw,circle,fill=red!50]at(8,-1){$3,6$};

      \node(N06)[draw,circle]at(1,1){$0,6$};
      \node(N51)[draw,circle]at(2,1){$5,1$};
      \node(N01)[draw,circle]at(3,1){$0,1$};
      \node(N10)[draw,circle]at(4,1){$1,0$};
      \node(N16)[draw,circle]at(5,1){$1,6$};
      \node(N52)[draw,circle]at(6,1){$5,2$};
      \node(N02)[draw,circle]at(7,1){$0,2$};
      \node(N20)[draw,circle]at(8,1){$2,0$};
      \node(N26)[draw,circle]at(9,1){$2,6$};
      \node(N53)[draw,circle,fill=red!50]at(10,1){$5,3$};

      \foreach \x/\y in{N00/N50,%
        N00/N06,N06/N51,N51/N01,N01/N10,N10/N16,N16/N52,N52/N02,N02/N20,N20/N26,N26/N53,%
        N05/N00%
      }{
        \draw[->](\x)--(\y);
      }
      \foreach \x/\y in{N06/N56,N50/N56,N50/N05,N05/N55,N55/N46,N46/N40,N40/N04,N04/N54,N54/N36}{
        \draw[->](\x)edge[bend left=20](\y);
        \draw[->](\y)edge[bend left=20](\x);
      }
      \foreach \x in{N40,N04}{
        \draw[->](\x)edge[bend right=20](N00);
      }
      \draw[->](N54)edge[bend left=30](N55);
      \draw[->](N54)edge[bend left=40](N50);
    \end{tikzpicture}
  \end{center}

  上图中箭头方向表示可由一个状态经过一个操作后到达另一个状态。当然,图中的结点及有向箭头是不全的\footnote{比如结点$(5,1)$经过一个操作后,可以达到的结点还有$(5,0)$、$(6,1)$,及$(5,6)$。},但也足以给出两种可以量出3升水的操作路径。
\end{proof}

\begin{question}
  古时候有家酒馆的老板特别节省,他只购买了一个7两和一个11两的勺子给客人打酒。有个刁钻的客人想要为难老板,于是一连10天跑去酒馆打酒,第一天只打一两,第二天打二两,第三天打三两,$\cdots\cdots$,第十天打十两。如果你是酒馆的老板,你如何应对这个客人的要求?
\end{question}
\begin{proof}[提示]
  类似地,可不停试验。

  其实可以归结为$7x+11y=1$的整数解问题,其解法参考第~\ref{chap:diophantine-equation}章~\nameref{chap:diophantine-equation}。其一个特解是$x=-3$,$y=2$。所以2个11两的倒满7两3次,可剩1两。$7x+11y=k$问题类似,其中$k$是任意正整数。
\end{proof}


\begin{example}
  小毛是学校出名的机灵鬼,也是家里让人头疼的调皮蛋。有一天他急急忙忙写完作业就要出动玩,妈妈便说还有一道题,要做完了才能出去玩。妈妈的题目是这样的:有6个杯子排成一排,前面3个装满了水,后面3个是空的。若只能移动一个杯子,能否将装满水的水杯与空杯间隔开?
\end{example}
\begin{proof}[提示]
  若“移动”还包括倒水的动作,那么是有可能的。只需要将前三杯中间的杯子里的水倒到后三杯中间的空杯子里,然后再放回原位即可。
  \begin{center}
    \begin{tikzpicture}[scale=1.0]
      \foreach \x in {0,1,2,3,4,5}{
      \begin{scope}[shift={(2 * \x,0)}]
        \coordinate(A)at(0,1.2);
        \coordinate(B)at(0.2,0);
        \coordinate(C)at(0.6,0);
        \coordinate(D)at(0.8,1.2);
        \coordinate(W1)at(0.05,1);
        \coordinate(W2)at(0.1,1.1);
        \coordinate(W3)at(0.2,1);
        \coordinate(W4)at(0.3,1.1);
        \coordinate(W5)at(0.4,1);
        \coordinate(W6)at(0.5,1.1);
        \coordinate(W7)at(0.6,1);
        \coordinate(W8)at(0.7,1.1);
        \coordinate(W9)at(0.75,1);
        \ifthenelse{\x<3}{
          \fill[pattern=dots,pattern color=blue!30](W1)--(W2)--(W3)--(W4)--(W5)--(W6)--(W7)--(W8)--(W9)--(C)--(B)--cycle;
          \draw[color=blue!30](W1)--(W2)--(W3)--(W4)--(W5)--(W6)--(W7)--(W8)--(W9);
        }{}
        \draw(A)--(B)--(C)--(D);
      \end{scope}
      }
      \node(X)at(2.4,-.2){};
      \node(Y)at(8.4,-.2){};
      \draw[->](X) to[out=330,in=210](Y);
      \node at(5.4,-.7){倒水};
    \end{tikzpicture}
  \end{center}
  % don't show qed symbol for this example only
  \let\qed\relax
\end{proof}

\begin{example}[爱因斯坦难题]\label{ex:einstein's-problem}
  在一条街上,有5座房子,喷了5种颜色。每个房子了住着不同国籍的人,每个人
  喝着不同的饮料,抽不同品牌的香烟,养着不同的宠物,这有一些他们的信
  息:
  \begin{enumerate}
  \item 英国人住在红房子里,
  \item 瑞典人养了一条狗,
  \item 丹麦人喝茶,
  \item 绿房子在白房子左边(注:是指紧挨着的左边),
  \item 绿房子主人喝咖啡,
  \item 抽PALL MALL烟的人养了一只鸟,
  \item 黄房子主人抽DUNHILL烟,
  \item 住在中间那间房子的人喝牛奶,
  \item 挪威人住第一间房子,
  \item 抽BLENDS烟的人住在养猫人的旁边,
  \item 养马的人住在DUNHILL烟的人旁边,
  \item 抽BLUE MASTER烟的人喝啤酒,
  \item 德国人抽PRINCE烟,
  \item 挪威人住在蓝房子旁边,
  \item 抽BLENDS烟的人的邻居喝矿泉水。
  \end{enumerate}
  那么谁在养鱼?
\end{example}
\begin{proof}[提示]
  做表格,一个个排除,一个个填,与数独类似。

  首先由(8)和(9),可以先填入下表:
    \begin{center}
      \renewcommand*{\arraystretch}{1.0}
      \begin{tabular}{l|l|l|l|l|l}
        \hline
        房号     & 1      & 2 & 3    & 4 & 5\\\hline
        颜色     &        &   &      &   &  \\\hline
        国籍     & 挪威人 &   &      &   &  \\\hline
        饮料     &        &   & 牛奶 &   &  \\\hline
        烟       &        &   &      &   &  \\
        宠物     &        &   &      &   &  \\
        \hline
      \end{tabular}
    \end{center}

    再由(14),填入蓝房子
    \begin{center}
      \renewcommand*{\arraystretch}{1.0}
      \begin{tabular}{l|l|l|l|l|l}
        \hline
        房号     & 1      & 2  & 3    & 4 & 5\\\hline
        颜色     &        & 蓝 &      &   &  \\\hline
        国籍     & 挪威人 &    &      &   &  \\\hline
        饮料     &        &    & 牛奶 &   &  \\\hline
        烟       &        &    &      &   &  \\\hline
        宠物     &        &    &      &   &  \\
        \hline
      \end{tabular}
    \end{center}

    再由(1)和(4),红、绿、白房子只能都在3、4 、5号房子里,从而1号是黄房子。由(7)可填入1号房子主人抽什么。再由(11)可填入马。
    \begin{center}
      \renewcommand*{\arraystretch}{1.0}
      \begin{tabular}{l|l|l|l|l|l}
        \hline
        房号     & 1      & 2  & 3    & 4 & 5\\\hline
        颜色     & 黄     & 蓝 &      &   &  \\\hline
        国籍     & 挪威人 &    &      &   &  \\\hline
        饮料     &        &    & 牛奶 &   &  \\\hline
        烟       & DUNHILL&    &      &   &  \\\hline
        宠物     &        & 马 &      &   &  \\
        \hline
      \end{tabular}
    \end{center}

    再由(4),“绿白”要紧挨着填入,只能填“34”或“45”号房子。由(5),“绿”不能是3号房子,从而“绿白”要填入“45”号房子,继而3号房子是红房子,再由(1),填入英国人。由(5)填入咖啡。
    \begin{center}
      \renewcommand*{\arraystretch}{1.0}
      \begin{tabular}{l|l|l|l|l|l}
        \hline
        房号     & 1      & 2  & 3     & 4     & 5\\\hline
        颜色     & 黄     & 蓝 & 红    & 绿    &白\\\hline
        国籍     & 挪威人 &    & 英国人&       &  \\\hline
        饮料     &        &    & 牛奶  & 咖啡  &  \\\hline
        烟       & DUNHILL&    &       &       &  \\\hline
        宠物     &        & 马 &       &       &  \\
        \hline
      \end{tabular}
    \end{center}

    此时,划掉已经无用的条件,剩余如下:
    \begin{enumerate}
    \item \sout{英国人住在红房子里,}
    \item 瑞典人养了一条狗,
    \item 丹麦人喝茶,
    \item \sout{绿房子在白房子左边(注:是指紧挨着的左边),}
    \item \sout{绿房子主人喝咖啡,}
    \item 抽PALL MALL烟的人养了一只鸟,
    \item \sout{黄房子主人抽DUNHILL烟,}
    \item \sout{住在中间那间房子的人喝牛奶,}
    \item \sout{挪威人住第一间房子,}
    \item 抽BLENDS烟的人住在养猫人的旁边,
    \item \sout{养马的人住在DUNHILL烟的人旁边,}
    \item 抽BLUE MASTER烟的人喝啤酒,
    \item 德国人抽PRINCE烟,
    \item \sout{挪威人住在蓝房子旁边,}
    \item 抽BLENDS烟的人的邻居喝矿泉水。
    \end{enumerate}

    由(3),“丹麦人”和“喝茶”两个一起填入的只剩下2号和5号房。由(12),“抽BLUE MASTER”和“喝啤酒”两个一起填入的也只剩下2号和5号房。再由(15),还有人“喝矿泉水”。此时可知“喝矿泉水”的是1号房,并且2号房抽BLENDS烟。
    \begin{center}
      \renewcommand*{\arraystretch}{1.0}
      \begin{tabular}{l|l|l|l|l|l}
        \hline
        房号     & 1      & 2      & 3     & 4     & 5\\\hline
        颜色     & 黄     & 蓝     & 红    & 绿    &白\\\hline
        国籍     & 挪威人 &        & 英国人&       &  \\\hline
        饮料     & 矿泉水 &        & 牛奶  & 咖啡  &  \\\hline
        烟       & DUNHILL& BLENDS &       &       &  \\\hline
        宠物     &        & 马     &       &       &  \\
        \hline
      \end{tabular}
    \end{center}

    此时再由(12),能够一起填入抽BLUE MASTER和喝啤酒的就只剩下5号房了,从而可在5号房填入BLUE MASTER和啤酒。这样之后由(3),丹麦人和苶就只能填在2号房了。再划掉(3)、(12),表格及剩余条件如下。
    \begin{center}
      \renewcommand*{\arraystretch}{1.0}
      \begin{tabular}{l|l|l|l|l|l}
        \hline
        房号     & 1      & 2      & 3     & 4     & 5           \\\hline
        颜色     & 黄     & 蓝     & 红    & 绿    &白           \\\hline
        国籍     & 挪威人 & 丹麦人 & 英国人&       &             \\\hline
        饮料     & 矿泉水 & 茶     & 牛奶  & 咖啡  &啤酒         \\\hline
        烟       & DUNHILL& BLENDS &       &       &BLUE MASTER  \\\hline
        宠物     &        & 马     &       &       &  \\
        \hline
      \end{tabular}
    \end{center}

    \begin{enumerate}
    \item \sout{英国人住在红房子里,}
    \item 瑞典人养了一条狗,
    \item \sout{丹麦人喝茶,}
    \item \sout{绿房子在白房子左边(注:是指紧挨着的左边),}
    \item \sout{绿房子主人喝咖啡,}
    \item 抽PALL MALL烟的人养了一只鸟,
    \item \sout{黄房子主人抽DUNHILL烟,}
    \item \sout{住在中间那间房子的人喝牛奶,}
    \item \sout{挪威人住第一间房子,}
    \item 抽BLENDS烟的人住在养猫人的旁边,
    \item \sout{养马的人住在DUNHILL烟的人旁边,}
    \item \sout{抽BLUE MASTER烟的人喝啤酒,}
    \item 德国人抽PRINCE烟,
    \item \sout{挪威人住在蓝房子旁边,}
    \item 抽BLENDS烟的人的邻居喝矿泉水。
    \end{enumerate}

    此时由(13),德国人抽PRINCE烟只能在4号房。由(2),剩下的瑞典人养狗只能在5号房。由(6),PALL MALL烟和养鸟在3号房。
    \begin{center}
      \renewcommand*{\arraystretch}{1.0}
      \begin{tabular}{l|l|l|l|l|l}
        \hline
        房号     & 1      & 2      & 3         & 4     & 5           \\\hline
        颜色     & 黄     & 蓝     & 红        & 绿    &白           \\\hline
        国籍     & 挪威人 & 丹麦人 & 英国人    & 德国人&瑞典人       \\\hline
        饮料     & 矿泉水 & 茶     & 牛奶      & 咖啡  &啤酒         \\\hline
        烟       & DUNHILL& BLENDS & PALL MALL & PRINCE&BLUE MASTER  \\\hline
        宠物     &        & 马     & 鸟        &       &狗           \\
        \hline
      \end{tabular}
    \end{center}

    此时由(10),养猫只能在1号房。从而表格中只剩余4号房养什么宠物未知。而由题目是问谁养鱼,故养鱼只能填在4号房。
\end{proof}

\begin{example}
  六所房子排成一排,每所房子颜色不同,每位屋主来自不同的地方,吃着不同的水果,喝着不同的饮料,喜欢不同的运动,养不同的宠物。下面是关于他们的一些信息:\
  \begin{enumerate}%\vspace{-2cm}
  \begin{multicols}{2}
  \item 前3位无养鸡和吃葡萄者。
  \item 吃桔子的人也养马。
  \item 北京人在养鸟者旁边。
  \item 湖南人住黄房子。
  \item 四川人旁边的人吃桔子。
  \item 绿房子的主人吃西瓜。
  \item 上海人在白房子旁边。
  \item 江苏人喜欢打排球。
  \item 红色屋主喝牛奶。
  \item 养猫人在第一所房子。
  \item 打篮球的人养狗。
  \item 喝咖啡的人吃香蕉。
  \item 打棒球的人养鸡。
  \end{multicols}%\vspace{-2cm}
  \item 喝茶者在喝矿泉水者的右边
  \item 爱吃香蕉者和打棒球者住在中间(第3、4位置)
  \item 后3位中没有黑房子,有吃苹果者。
  \item 篮球、垒球、足球,这3项运动的爱好者互不相邻。
  \item 上海人和黑房子之间隔有3个位置。
  \item 白色房子既不在中间,也不在最后,与蓝房子相邻。
  \item 喝可乐者和山东人之间仅隔喝啤酒者。
  \item 爱吃苹果的人也爱打桌球。
  \item 吃桃子者与打桌球者之间仅隔扔垒球者。
  \end{enumerate}
那么谁养鱼?
\end{example}

\newcolumntype{Y}{>{\centering\arraybackslash}X}
\begin{center}
  %https://tex.stackexchange.com/questions/60601/evenly-distributing-column-widths
  
  \begin{tabularx}{\textwidth}{|c *{7}{|Y}}
    \hline
    房间 & 1 & 2 & 3 & 4 & 5 & 6\\\hline
    颜色 &   &   &   &   &   &  \\\hline
    地方 &   &   &   &   &   &  \\\hline
    水果 &   &   &   &   &   &  \\\hline
    饮料 &   &   &   &   &   &  \\\hline
    运动 &   &   &   &   &   &  \\\hline
    宠物 &   &   &   &   &   &  \\\hline
  \end{tabularx}
\end{center}

\noindent\begin{minipage}{\textwidth}\setlength{\parindent}{2em}
先根据(10),填猫。根据(1)、(13)和(15),填棒球、鸡和香蕉。再由(12)填入咖啡。
\begin{center}
  \begin{tabularx}{\textwidth}{|c *{7}{|Y}}
    \hline
    房间 & 1 & 2 & 3     & 4 & 5 & 6\\\hline
    颜色 &   &   &       &       &   &  \\\hline
    地方 &   &   &       &       &   &  \\\hline
    水果 &   &   & 香蕉  &       &   &  \\\hline
    饮料 &   &   & 咖啡  &       &   &  \\\hline
    运动 &   &   &       & 棒球  &   &  \\\hline
    宠物 & 猫&   &       & 鸡    &   &  \\\hline
  \end{tabularx}\vspace{.5cm}
\end{center}
\end{minipage}

\noindent\begin{minipage}{\textwidth}\setlength{\parindent}{2em}
由(21)和(22),苹果、桌球、桃子和垒球只能是以下两种排列方式:\nopagebreak
\begin{center}
  \begin{tabular}{|c|c|c|}
    \hline
    \BKC 苹果 & & \BKC 桃子\\\hline
    & & \\\hline
    \BKC 桌球 & \BKC 垒球 &\\\hline
  \end{tabular}
  \hspace{2cm}
  \begin{tabular}{|c|c|c|}
    \hline
    \BKC 桃子 & & \BKC 苹果\\\hline
    & & \\\hline
     & \BKC 垒球 & \BKC 桌球\\\hline
  \end{tabular}\vspace{.5cm}
\end{center}
\end{minipage}

\noindent\begin{minipage}{\textwidth}\setlength{\parindent}{2em}
再由(16)苹果在4/5/6号房,从而只有上面第二种排列能放入到表格中:\nopagebreak
\begin{center}
  \begin{tabularx}{\textwidth}{|c *{7}{|Y}}
    \hline
    房间 & 1 & 2 & 3     & 4    & 5    & 6   \\\hline
    颜色 &   &   &       &      &      &     \\\hline
    地方 &   &   &       &      &      &     \\\hline
    水果 &   &   & 香蕉  & \BKC 桃子  &      & \BKC 苹果 \\\hline
    饮料 &   &   & 咖啡  &      &      &     \\\hline
    运动 &   &   &       & 棒球  & \BKC 垒球 & \BKC 桌球 \\\hline
    宠物 & 猫&   &       & 鸡    &      &     \\\hline
  \end{tabularx}\vspace{.5cm}
\end{center}
\end{minipage}

\noindent\begin{minipage}{\textwidth}\setlength{\parindent}{2em}
由(17),足球、篮球占据了房间1和3。由(11),打篮球养狗的人只能在(3),从而足球在(1)。由(8),剩下打排球的江苏人在(2)。由(1),葡萄在(5)。\nopagebreak
\begin{center}
  \begin{tabularx}{\textwidth}{|c *{7}{|Y}}
    \hline
    房间 & 1   & 2 & 3     & 4    & 5    & 6   \\\hline
    颜色 &     &   &       &      &      &     \\\hline
    地方 &     & \BKC 江苏  &       &      &      &     \\\hline
    水果 &     &   & 香蕉  & 桃子  & \BKC 葡萄 & 苹果 \\\hline
    饮料 &     &   & 咖啡  &      &      &     \\\hline
    运动 & \BKC 足球 & \BKC 排球 & \BKC 篮球   & 棒球  & 垒球 & 桌球 \\\hline
    宠物 & 猫  &   & \BKC 狗    & 鸡    &      &     \\\hline
  \end{tabularx}\vspace{.5cm}
\end{center}
\end{minipage}

\noindent\begin{minipage}{\textwidth}\setlength{\parindent}{2em}
此时剩余1号房和2号房两种水果。由(5),桔子不能在1号房,从而桔子在2号房。由(6),填入绿色和西瓜。由(2),填入马。\nopagebreak
\begin{center}
  \begin{tabularx}{\textwidth}{|c *{7}{|Y}}
    \hline
    房间 & 1   & 2     & 3     & 4    & 5    & 6   \\\hline
    颜色 & \BKC 绿色 &      &       &      &      &     \\\hline
    地方 &     & 江苏  &       &      &      &     \\\hline
    水果 & \BKC 西瓜 & \BKC 桔子 & 香蕉  & 桃子  & 葡萄 & 苹果 \\\hline
    饮料 &     &     & 咖啡   &      &      &     \\\hline
    运动 & 足球 & 排球 & 篮球   & 棒球  & 垒球 & 桌球 \\\hline
    宠物 & 猫  & \BKC 马  & 狗    & 鸡    &      &     \\\hline
  \end{tabularx}\vspace{.5cm}
\end{center}
\end{minipage}

\noindent\begin{minipage}{\textwidth}\setlength{\parindent}{2em}
由(16),黑色在2号或3号,从而由(18),上海在6号或7号(不存在),从而黑色在2号,上海在6号。由(19),白色在5号。
\begin{center}
  \begin{tabularx}{\textwidth}{|c *{7}{|Y}}
    \hline
    房间 & 1   & 2     & 3    & 4    & 5    & 6   \\\hline
    颜色 & 绿色 & \BKC 黑色 &  & % \BKC 黄
    & \BKC 白色  &     \\\hline
    地方 &     & 江苏  &      & % \BKC 湖南
    &      & \BKC 上海 \\\hline
    水果 & 西瓜 & 桔子 & 香蕉  & 桃子  & 葡萄 & 苹果 \\\hline
    饮料 &     &      & 咖啡  &      &      &     \\\hline
    运动 & 足球 & 排球 & 篮球  & 棒球  & 垒球 & 桌球 \\\hline
    宠物 & 猫  &  马  & 狗    & 鸡    &      &     \\\hline
  \end{tabularx}\vspace{.5cm}
\end{center}
\end{minipage}

\noindent\begin{minipage}{\textwidth}\setlength{\parindent}{2em}
由(19),蓝色在4、6。由(9),红色在4、6。从而由(4),剩余的一个黄色在3,填入湖南与黄色。由(5),四川人在1号。
\begin{center}
  \begin{tabularx}{\textwidth}{|c *{7}{|Y}}
    \hline
    房间 & 1   & 2     & 3   & 4    & 5    & 6   \\\hline
    颜色 & 绿色 & 黑色 & \BKC 黄色 &    & 白色  &     \\\hline
    地方 & \BKC 四川 & 江苏  & \BKC 湖南 &     &      & 上海 \\\hline
    水果 & 西瓜 & 桔子 & 香蕉  & 桃子  & 葡萄 & 苹果 \\\hline
    饮料 &     &      & 咖啡  &      &      &     \\\hline
    运动 & 足球 & 排球 & 篮球  & 棒球  & 垒球 & 桌球 \\\hline
    宠物 & 猫  &  马  & 狗    & 鸡    &      &     \\\hline
  \end{tabularx}\vspace{.5cm}
\end{center}
\end{minipage}

\noindent\begin{minipage}{\textwidth}\setlength{\parindent}{2em}
由(20),山东、可乐、啤酒的排列形式只有以下两种:
\begin{center}
  \begin{tabular}{|c|c|c|}
    \hline
    \BKC 山东 &     &\\\hline
         &     &\\\hline
         & \BKC 啤酒 & \BKC 可乐\\\hline
  \end{tabular}\hspace{2cm}
  \begin{tabular}{|c|c|c|}
    \hline
         &     & \BKC 山东\\\hline
         &     &\\\hline
     \BKC 可乐& \BKC 啤酒 & \\\hline
  \end{tabular}\vspace{.5cm}
\end{center}
\end{minipage}
由此容易将表格剩余内容填满。

% \makeatletter
% \newcommand{\thickhline}{%
%     \noalign {\ifnum 0=`}\fi \hrule height 1pt
%     \futurelet \reserved@a \@xhline
% }
% \newcolumntype{"}{@{\vrule width 1pt\hskip\tabcolsep}}
% \newcolumntype{!}{@{\hskip\tabcolsep\vrule width 1pt}}
% \makeatother

\noindent\begin{minipage}{\textwidth}\setlength{\parindent}{2em}
\begin{center}
  \begin{tabularx}{\textwidth}{|c *{7}{|Y}}
    \hline
    房间 & 1   & 2     & 3   & 4    & 5    & 6   \\\hline
    颜色 & 绿色 & 黑色 & 黄色 &    & 白色  &     \\\hline
    地方 & 四川 & 江苏  & 湖南 & \BKC 山东 &      & 上海 \\\hline
    水果 & 西瓜 & 桔子 & 香蕉  & 桃子  & 葡萄 & 苹果 \\\hline
    饮料 &     &      & 咖啡  &      & \BKC 啤酒 & \BKC 可乐 \\\hline
    运动 & 足球 & 排球 & 篮球  & 棒球  & 垒球 & 桌球 \\\hline
    宠物 & 猫  &  马  & 狗    & 鸡    &      &     \\\hline
  \end{tabularx}\vspace{.5cm}
\end{center}
\end{minipage}

由(3)和(19),填入北京、鸟、红色、牛奶。所以是\underline{北京人养鱼}。

\noindent\begin{minipage}{\textwidth}\setlength{\parindent}{2em}
\begin{center}
  \begin{tabularx}{\textwidth}{|c *{7}{|Y}}
    \hline
    房间 & 1   & 2     & 3   & 4    & 5    & 6   \\\hline
    颜色 & 绿色 & 黑色 & 黄色 & \BKC 红色 & 白色  &     \\\hline
    地方 & 四川 & 江苏  & 湖南 & 山东 & \BKC 北京 & 上海 \\\hline
    水果 & 西瓜 & 桔子 & 香蕉  & 桃子  & 葡萄 & 苹果 \\\hline
    饮料 &     &      & 咖啡  & \BKC 牛奶 & 啤酒 & 可乐 \\\hline
    运动 & 足球 & 排球 & 篮球  & 棒球  & 垒球 & 桌球 \\\hline
    宠物 & 猫  &  马  & 狗    & 鸡    &      & \BKC 鸟 \\\hline
  \end{tabularx}\vspace{.5cm}
\end{center}
\end{minipage}


\begin{question}
  小赵、小张、小王、小李、小白,这五人喜欢不同的运动,喜欢吃不同的食物,
  每个人都有一个老婆,这天他们同时到希尔顿去吃饭,分别坐了不同的包间,
  五个包间同在一侧,排列在一起。在这一侧有且只有五个包间。请根据下面提
  示回答,杨玉环是谁的老婆。

  \begin{enumerate}
  \item 苏州包间在伦敦包间的左边(隔壁)
  \item 在大理包间的主人喜欢打乒乓球
  \item 喜欢打篮球的人有一个喜欢吃猪肉的邻居
  \item 小李的包间在最左边
  \item 黄蓉的老公在喜欢打乒乓球的人旁边
  \item 喜欢打篮球的人的包间在赵飞燕的老公的包间隔壁
  \item 小赵的包间名字叫东京
  \item 喜欢打网球的人老婆名叫苏蓉蓉
  \item 小张喜欢跑步
  \item 小白的老婆是任盈盈
  \item 苏州包间主人喜欢吃梨
  \item 小王喜欢吃苹果
  \item 在中间那个包间的人喜欢吃牛肉
  \item 小李的包间在纽约包间的旁边
  \item 喜欢游泳的人爱吃土豆
  \end{enumerate}
\end{question}

\begin{proof}[提示]
  根据线索,依次填入表格。
  \begin{itemize}
  \item 首先可以填完比较明显的线索:小李、纽约、牛肉。
  \item 再根据苏州、伦敦、梨的关系,苏州、梨只能是在第4列,伦敦在第5列。
  \item 再填入小赵、东京。
  \item 再填大理、乒乓球。
\end{itemize}

\noindent\begin{minipage}{\textwidth}\setlength{\parindent}{2em}
\begin{center}
  \begin{tabularx}{\textwidth}{|c *{6}{|Y}}
    \hline
         & 1    & 2    & 3     & 4    & 5    \\\hline
    包间 & 大理 & 纽约 & 东京  & 苏州 & 伦敦 \\\hline
    主人 & 小李 &      & 小赵  &      &      \\\hline
    运动 & 乒乓 &      &       &      &      \\\hline 
    食物 &      &      & 牛肉  & 梨   &      \\\hline
    老婆 &      &      &       &      &      \\\hline
  \end{tabularx}\vspace{.5cm}
\end{center}
\end{minipage}

此时,根据游泳、土豆、小王、苹果,可知2、5列的食物被土豆、苹果占满,剩余一个食物猪肉只能在第1列。其余依次填入即可。
\end{proof}



\begin{example}
  黛西参加学校举行的小学生数学比赛,比赛结束后,艾玛问黛西得了第几名,黛西故意卖关子,说:“我考的分数、名次和我的年龄的乘积是1958,你猜猜看。”请算算黛西的分数、名次和年龄。
\end{example}
\begin{proof}[提示]
  将由于$1958=2\times11\times 89$,1958有如下的三数连乘分解方式,并注意到黛西,有
  \begin{align*}
    \setlength\arraycolsep{2pt}
    \begin{array}{cccccccl}
    1958&=&1 & \times & 1  &\times & 1958 &\quad\implies\text{年龄是1或1958,不可能}\hfill\\
    1958&=&1 & \times & 2  &\times & 979  &\quad\implies\text{年龄是1、2或者979,不可能}\hfill\\
    1958&=&1 & \times & 11 &\times & 178  &\quad\implies\text{年龄可能是11,分数可能是1或者179}\\
    1958&=&1 & \times & 22 &\times & 89   &\quad\implies\text{年龄不可能}\\
    1958&=&2 & \times & 11 &\times & 89   &\quad\implies\text{年龄可能是11,分数可能是2或者89}
    \end{array}
  \end{align*}
  由此,出现的数字有1、2、11、22、89、178、979、1958共8个。
  
  第一个信息:小学生。由此年龄只能是11岁。
  
  第二个信息:参加比赛。能参加比赛的,成绩一般不会太差,排除分数不在$[0,100]$区间且分数特别低的比如1分2分的情况,从而只剩下11岁89分第2名这种情况。
\end{proof}

\begin{example}[失窃的艺术品]
  一件艺术失窃案发生后,警察询问了六名嫌疑人是谁干的。图~\ref{fig:statements-of-suspects}是他们的陈述。

  % \begin{multicols}{3}
  % Alex说:
  % \begin{enumerate}
  % \item 不是Blake;
  % \item 不是Drew;
  % \item 不是Emery。
  % \end{enumerate}

  % Blake说:
  % \begin{enumerate}
  % \item 不是Alex;
  % \item 不是Chris;
  % \item 不是Emery。
  % \end{enumerate}

  % Chris说:
  % \begin{enumerate}    
  % \item 不是Blake;
  % \item 不是Frankie;
  % \item 不是Emery。
  % \end{enumerate}

  % Drew说:
  % \begin{enumerate}
  % \item 不是Alex;
  % \item 不是Frankie;
  % \item 不是Chris。
  % \end{enumerate}

  % Emery说:
  % \begin{enumerate}
  % \item 不是Chris;
  % \item 不是Drew;
  % \item 不是Frankie。
  % \end{enumerate}

  % Frankie说:
  % \begin{enumerate}
  % \item 不是Chris;
  % \item 不是Drew;
  % \item 不是Alex。
  % \end{enumerate}

  % \end{multicols}

  \begin{figure}[htbp]
    \centering
    \begin{tikzpicture}[scale=1.0]
      \begin{scope}[shift={(0,0)}]
        \node[text width=3.5cm,rotate=30] at (0,0) {\fbox{\vbox{
            Alex说:
            \begin{enumerate}
            \item 不是Blake;
            \item 不是Drew;
            \item 不是Emery。
            \end{enumerate}
          }}};
      \end{scope}
      \begin{scope}[shift={(4.5,0)}]
        \node[text width=3.5cm,rotate=0] at (0,0) {\fbox{\vbox{
              Blake说:
              \begin{enumerate}
              \item 不是Alex;
              \item 不是Chris;
              \item 不是Emery。
              \end{enumerate}
          }}};
      \end{scope}
      \begin{scope}[shift={(9,0)}]
        \node[text width=3.5cm,rotate=-30] at (0,0) {\fbox{\vbox{
              Chris说:
              \begin{enumerate}    
              \item 不是Blake;
              \item 不是Frankie;
              \item 不是Emery。
              \end{enumerate}
          }}};
      \end{scope}


      \begin{scope}[shift={(0,-4.5)}]
        \node[text width=3.5cm,rotate=-30] at (0,0) {\fbox{\vbox{
              Drew说:
              \begin{enumerate}
              \item 不是Alex;
              \item 不是Frankie;
              \item 不是Chris。
              \end{enumerate}
          }}};
      \end{scope}
      \begin{scope}[shift={(4.5,-4.5)}]
        \node[text width=3.5cm,rotate=0] at (0,0) {\fbox{\vbox{
              Emery说:
              \begin{enumerate}
              \item 不是Chris;
              \item 不是Drew;
              \item 不是Frankie。
              \end{enumerate}
          }}};
      \end{scope}
      \begin{scope}[shift={(9,-4.5)}]
        \node[text width=3.5cm,rotate=30] at (0,0) {\fbox{\vbox{
              Frankie说:
              \begin{enumerate}
              \item 不是Chris;
              \item 不是Drew;
              \item 不是Alex。
              \end{enumerate}
            }}};
      \end{scope}
    \end{tikzpicture}
    \caption{嫌疑人供词}
    \label{fig:statements-of-suspects}
  \end{figure}
  
  经过调查,警察发现他们当中恰好有四个人每人说了一句假话,其余所有话都是真的。那么,到底罪犯是谁?  
\end{example}
\begin{proof}[提示]
  还是用假设--排除的方法。
  \begin{enumerate}
  \item 假设Alex是罪犯,那么可以得到下面的表格。
    \begin{center}
    \begin{tabular}{cccccc}
      \toprule
      Alex & Blake & Chris & Drew & Emery & Frankie\\\midrule
      $\checkmark\checkmark\checkmark$ & $\xmark\checkmark\checkmark$ & $\checkmark\checkmark\checkmark$ & $\xmark\checkmark\checkmark$ & $\checkmark\checkmark\checkmark$ & $\checkmark\checkmark\xmark$\\                                         
      \bottomrule
    \end{tabular}
    \end{center}
    其中$\xmark$表示一句假话,$\checkmark$表示一句真话。由上表可以看到,只有3个人每人说了一句假话,与恰好有4人每人说了一句假话矛盾。
    
  \item 假设Blake是罪犯,那么有
    \begin{center}
      \begin{tabular}{cccccc}
        \toprule
        Alex & Blake & Chris & Drew & Emery & Frankie\\\midrule
        $\xmark\checkmark\checkmark$ & $\checkmark\checkmark\checkmark$ & $\xmark\checkmark\checkmark$ & $\checkmark\checkmark\checkmark$ & $\checkmark\checkmark\checkmark$ & $\checkmark\checkmark\checkmark$\\
        \bottomrule
      \end{tabular}
    \end{center}
  \item 依此类推,若能排除五人,剩余一人不能排除,则剩余的人就是罪犯了。\hfill $\qedhere$
  \end{enumerate}
\end{proof}

\begin{example}
  在一场数学竞赛里,Alvin、Brian和Carlos都获得了一枚奖牌,其中一人得了金牌,一人得了银牌,一人得了铜牌。David猜:“Alvin得了金牌,Brian没得金牌,Carlos没得铜牌。”事实上David只猜对了一个。那么三人分别得了什么奖牌?
\end{example}
\begin{proof}[提示]
  三人三种奖牌,即三奖牌的全排列,共有$P_3^3=3!=3\times2\times1=6$种,如表~\ref{tab:permutation-of-medals}所示:
  \begin{table}[htbp]
    \centering
    \caption{三种奖牌的全排列}
    \label{tab:permutation-of-medals}
    \begin{tabular}{ccccc}
      \toprule
      序号 & Alvin & Brian & Carlos & David的猜测\\\midrule
      1    & 金    & 银    & 铜     & $\checkmark\checkmark\xmark$\\
      2    & 金    & 铜    & 银     &\\
      3    & 银    & 金    & 铜     &\\
      4    & 银    & 铜    & 金     &\\
      5    & 铜    & 金    & 银     &\\
      6    & 铜    & 银    & 金     &\\\bottomrule
    \end{tabular}
  \end{table}

  分析每一种情况。比如,假设是第一种,那么David的三句猜测的对错如表~\ref{tab:permutation-of-medals}所示。可见David猜对了两人,矛盾,可排除这种情况。其余类推。若最后只剩余一种情况无法排除,那就是它了。
\end{proof}


\begin{example}[找火车]
  在一个洞察力大赛中,5名参赛者在桌子边上一字排开,他们从桌上的图案中观察到的飞机数量有$(26,86,123,174,250)$,观察到的火车数则有$(5,42,45,98,105)$,而他们的年龄则分别为$(21,23,31,36,40)$。根据下面的信息,请判断各人的位置,年龄,观察到的飞机和火车数量以及衣服的颜色。
  \begin{enumerate}
  \item Simon找到的火车比他找到的飞机少44;
  \item Keith的年龄是36;
  \item 最右边的人比Simon年轻8岁,并且他找到了174架飞机;
  \item James穿着米黄色的衣服,并且跟Simon比起来他找到的火车数少了37;
  \item 穿着绿色衣服的人比他左手边的人年轻19岁;
  \item Steven找到了105辆火车和250架飞机;
  \item 站中间的人是31岁,穿着蓝色的衣服,并且他找到了42辆火车;
  \item Alan站在最左边,找到了26架飞机,并且他找到的火车数量比他找到的飞机数量多了72;
  \item 穿红衣服的人比Keith大4岁,并且不在穿蓝衣服的人的旁边;
  \item 在31岁的人旁边但不在找到26架飞机的人旁边的那个人穿着橙色的衣服,找到了45辆火车。
  \end{enumerate}
\end{example}
\begin{proof}[提示]
  首先填入可以直接使用的条件,可得表~\ref{tab:contester-info-of-train-spotting}。然后再作一个年龄差值表,如表~\ref{tab:contester-info-of-age-diff}所示。
  \begin{table}[htbp]
    \centering
    \caption{洞察力比赛参赛者信息}
    \label{tab:contester-info-of-train-spotting}
    \begin{tabularx}{.8\textwidth}{|>{\columncolor{LightCyan}}c|*{5}{>{\centering\arraybackslash}X|}}
      \hline
      \rowcolor{LightCyan}
      编号 & 1    & 2 & 3  & 4 & 5\\\hline
      名字 & Alan &   &    &   &  \\\hline
      年龄 &      &   & 31 &   &  \\\hline
      颜色 &      &   & 蓝 &   &  \\\hline
      飞机 & 26   &   &    &   & 174 \\\hline
      火车 & 98   &   & 42 &   &  \\\hline
    \end{tabularx}
  \end{table}

  \begin{table}[htbp]
    \centering
    \caption{年龄差值}
    \label{tab:contester-info-of-age-diff}
    \begin{tabularx}{.6\textwidth}{|>{\columncolor{LightCyan}}c|*{5}{>{\centering\arraybackslash}X|}}
      % \cline{2-6} % \cline not compactible with colortblb
      \hhline{~-----}
      \cline{2-6}
      \rowcolor{LightCyan}
      \multicolumn{1}{c|}{\cellcolor{white}} 
         & 21 & 23 & 31 & 36 & 40\\\hline
      21 &    & 2  & 10 & 15 & 19\\\hline
      23 &    &    &  8 & 13 & 17\\\hline
      31 &    &    &    &  5 & 9 \\\hline
      36 &    &    &    &    & 4 \\\hline
      40 &    &    &    &    &   \\\hline
    \end{tabularx}
  \end{table}
  
  根据(10),在31岁的人的旁边,但不在找到26架飞机的人的旁边,只能是编号为4的人,从而4号穿橙色衣服且找到了45辆火车
  
  同时观察年龄差值表,每个差值只有一种组合。观察条件(3),年龄差值为8的年龄组合为23和31,其中Simon年龄较大为31,最右边的年龄较小为23,填入Simon与23。再由(1),填入Simon找到的飞机数。如下表。
  \begin{center}
    \begin{tabularx}{.8\textwidth}{|>{\columncolor{LightCyan}}c|*{5}{>{\centering\arraybackslash}X|}}
      \hline
      \rowcolor{LightCyan}
      编号 & 1    & 2 & 3     & 4 & 5  \\\hline
      名字 & Alan &   & \cellcolor{blue!25}Simon &   &   \\\hline
      年龄 &      &   & 31    &    & \cellcolor{blue!25}23  \\\hline
      颜色 &      &   & 蓝    & \cellcolor{red!25}橙 &   \\\hline
      飞机 & 26   &   & \cellcolor{blue!25}86    &    & 174 \\\hline
      火车 & 98   &   & 42    & \cellcolor{red!25}45 &     \\\hline
    \end{tabularx}
  \end{center}

  考虑(5),年龄差为19的组合为21和40。从而21、40可以并且只能填在2、1号,同时填入绿衣服。而由(9),年龄差为4的组合是36和40,从而40岁的人穿红衣服。
  \begin{center}
    \begin{tabularx}{.8\textwidth}{|>{\columncolor{LightCyan}}c|*{5}{>{\centering\arraybackslash}X|}}
      \hline
      \rowcolor{LightCyan}
      编号 & 1    & 2 & 3     & 4  & 5  \\\hline
      名字 & Alan &   & Simon &    &    \\\hline
      年龄 & \cellcolor{blue!25}40   & \cellcolor{blue!25}21& 31    &    & 23  \\\hline
      颜色 & \cellcolor{blue!25}红   & \cellcolor{blue!25}绿& 蓝    & 橙 &     \\\hline
      飞机 & 26   &   & 86    &    & 174 \\\hline
      火车 & 98   &   & 42    & 45 &     \\\hline
    \end{tabularx}
  \end{center}

  划掉已经用掉的条件,剩余如下:
  \begin{enumerate}
  \item \sout{Simon找到的火车比他找到的飞机少44;}
  \item Keith的年龄是36;
  \item \sout{最右边的人比Simon年轻8岁,并且他找到了174架飞机;}
  \item James穿着米黄色的衣服,并且跟Simon比起来他找到的火车数少了37;
  \item \sout{穿着绿色衣服的人比他左手边的人年轻19岁;}
  \item Steven找到了105辆火车和250架飞机;
  \item \sout{站中间的人是31岁,穿着蓝色的衣服,并且他找到了42辆火车;}
  \item \sout{Alan站在最左边,找到了26架飞机,并且他找到的火车数量比他找到的飞机数量多了72;}
  \item \sout{穿红衣服的人比Keith大4岁,并且不在穿蓝衣服的人的旁边;}
  \item \sout{在31岁的人旁边但不在找到26架飞机的人旁边的那个人穿着橙色的衣服,找到了45辆火车。}
  \end{enumerate}  

  由(4),填入4号。由(6),填入2号。
  \begin{center}
    \begin{tabularx}{.8\textwidth}{|>{\columncolor{LightCyan}}c|*{5}{>{\centering\arraybackslash}X|}}
      \hline
      \rowcolor{LightCyan}
      编号 & 1    & 2      & 3     & 4  & 5    \\\hline
      名字 & Alan & \cellcolor{blue!25}Steve  & Simon &    & \cellcolor{red!25}James\\\hline
      年龄 & 40   & 21     & 31    &    & 23   \\\hline
      颜色 & 红   & 绿     & 蓝    & 橙 & \cellcolor{red!25}米黄 \\\hline
      飞机 & 26   & \cellcolor{blue!25}250    & 86    &    & 174  \\\hline
      火车 & 98   & \cellcolor{blue!25}105    & 42    & 45 &      \\\hline
    \end{tabularx}
  \end{center}

  剩余的一一填入即可。
\end{proof}



\section{极端值}
\label{sec:extream-point}

\begin{example}
  有一个考试试卷只有5道题,答对3道或以上就及格。在100个参加考试的人中,有81人答对第一题,91人答对第二题,85人答对第三题,79人答对第四题,74人答对第五题。问这100人中,至少有多少人及格?
\end{example}
\begin{proof}[提示]
  考虑如何才能在满足题目条件的情况下使不及格的人尽可能的多,那么该情况下的及格人数就是及格人数的最少值,为问题之解。

  不及格等价于答错三题或三题以上。
\end{proof}

\section{数理逻辑}
\label{sec:mathematical-logic }

\section{悖论}
\label{sec:paradox}

悖论(Paradox),也叫佯谬或诡局,是指一种导致矛盾的命题。称之为悖论的命题,如果承认它是真的,经过一系列正确的推理,却又得出它是假的;如果承认它是假的,经过一系列正确的推理,却又得出它是真的。

\begin{example}[白马非马]
  相传,春秋战国时赵国的马匹流行烈性传染病,秦国严防瘟疫传入国内,就在函谷关口贴出告示,禁止赵国马匹入关。这天,正巧公孙龙骑着白马来到函谷关。

关吏说,“你人可入关,但马不能”。

公孙龙辩道:“白马非马,怎么不可以过关?”

关吏说:“白马是马”。

公孙龙说:“我公孙龙是龙吗?”

关吏一愣,但仍坚持说:“按照规定只要是赵国的马就不能入关,管你是白马还是黑马。”

公孙龙微微一笑,道:“‘马’是指名称而言,‘白’是指颜色而说,名称和颜色不是一个概念。‘白马’这个概念,分开来就是‘白’和‘马’或‘马’和‘白’,这是两个不同的概念。比如说你要马,给黄马、黑马可以,但是如果要白马,给黑马、给黄马就不可以,由此证明‘白马’和‘马’不是一回事!所以说白马非马。”

关吏越听越迷糊,被公孙龙这套高谈阔论搞得晕头转向,被侃晕了,不知该如何对答,无奈只好让公孙龙骑白马过关。
\end{example}

这就是历史上著名的白马非马。不过,这应该叫诡辩,而不是悖论。一般人在理解“白马非马”这句话时是理解为白马不属于马,然而公孙龙在论述的过程中偷换了概念来与关吏抬杠,将“非”的“不属于”转换成了“不等同于”,于是“白马非马”就变为了“白马不等同于马”,这就是偷换概念,是诡辩。

\begin{example}[柏拉图---苏格拉底悖论]\mbox{}\par
  柏拉图说:“苏格拉底下面要说的话是真的。”

  苏格拉底说:“柏拉图说的是假的。”
\end{example}

按照以上两句话的信息,柏拉图的话是真的还是假的呢?

如果柏拉图说的是真的,那么苏格拉底说的也是真的,从而柏拉图说的就是假的,矛盾。

如果柏拉图说的是假的,那么苏格拉底说的也是假的,从而柏拉图说的就是真的,矛盾。

这就是悖论,又不能是真,又不能是假,否则逻辑不自恰。


\section{策略}
\label{sec:logic-strategy}

\begin{example}[赛马问题]
  64匹马,8个赛道,至少需要赛多少场才能选出其中最快的4匹?假设没有任何计时设备,每场次中每个赛道最多只能有1匹马,每匹马的速度都不一样,且每匹马体力无限不管赛多少场都能保持一致的成绩。
\end{example}
\begin{example}[赛马简化版]
  可以考虑简化版本,以寻找其中的关键所在。同样的限制条件下,16匹马,4赛道,至少需要赛多少场才能选出其中最快的2匹?

  一场比赛最多只能有4匹马出赛,首先按每组4匹分组可分得4组(此时的分组可以随意分),对其编号,那么16匹马可以排成下面第一个队形:
  \begin{center}
    \begin{tikzpicture}[scale=.8]
      \begin{scope}[shift={(0,0)}]
        \draw(0,0)grid(4,4);
        \foreach \x/\y/\v in{%
          0/3/A1,1/3/A2,2/3/A3,3/3/A4,
          0/2/B1,1/2/B2,2/2/B3,3/2/B4,
          0/1/C1,1/1/C2,2/1/C3,3/1/C4,
          0/0/D1,1/0/D2,2/0/D3,3/0/D4}{
          \node at(\x+.5,\y+.5){\small $\v$};
          \node at(2,-1){初始分组};
        }
      \end{scope}
      \begin{scope}[shift={(6,0)}]
        \draw(0,0)grid(4,4);
        \foreach \x/\y/\v in{%
          0/3/A1',1/3/A2',2/3/A3',3/3/A4',
          0/2/B1',1/2/B2',2/2/B3',3/2/B4',
          0/1/C1',1/1/C2',2/1/C3',3/1/C4',
          0/0/D1',1/0/D2',2/0/D3',3/0/D4'}{
          \node at(\x+.5,\y+.5){\small $\v$};
          \node at(2,-1){小组赛结束后};
        }
      \end{scope}
      \foreach \u/\v in{1/1,2/3,3/4,4/2}{
        \draw[->](10.2,4.5-\u)--(11.8,4.5-\v);
      }
      \begin{scope}[shift={(12,0)}]
        \fill[pattern=north west lines,pattern color=red!50](2,0)rectangle(4,4);
        \fill[pattern=north east lines,pattern color=blue!50](0,0)rectangle(4,2);
        \fill[color=gray](1,2)rectangle(2,3);
        \draw(0,0)grid(4,4);
        \foreach \x/\y/\v in{%
          0/3/A1',1/3/A2',2/3/A3',3/3/A4',
          0/2/D1',1/2/D2',2/2/D3',3/2/D4',
          0/1/B1',1/1/B2',2/1/B3',3/1/B4',
          0/0/C1',1/0/C2',2/0/C3',3/0/C4'}{
          \node at(\x+.5,\y+.5){\small $\v$};
          \node at(2,-1){小组互换};
        }
      \end{scope}

    \end{tikzpicture}
  \end{center}
  其中每一行是一个小组。然后进行小组赛,每个小组内4匹赛一场,可以得到小组内排名,根据排名,变换小组内顺序,跑得快在前,可以得到第二个队形,其中满足
  \begin{align*}
    A1' > A2' > A3' > A4', \qquad B1' > B2' > B3' > B4'\\
    C1' > C2' > C3' > C4', \qquad D1' > D2' > D3' > D4'
  \end{align*}
  其中$x>y$表示$x$比$y$快。此时共赛了4场。

  然后,将每个小组的第一名共4匹马拉出来赛一场,根据其成绩调整每一行的位置。比如,若比赛结果按快慢排序是$A1'>D1'>B1'>C1'$,那么调整后的队形如第3图所示。此时可以知道$A1'$就是所有马中跑得最快的,并且所有马中跑得最快的前2名一定不在前2列之外,也一定不在前2行之外,所以只剩余左上角4匹马有可能。另外,由于$A1'>D1'>D2'$,从而$D2'$也不可能是是所有马中前2名之一,所以$A2'$与$D1'$其中之一必有一个是所有马中的第2名。从而这两匹马再赛一场即可确定所有马中的第2名。

  总共需要赛$4+1+1=6$场。
\end{example}

\begin{proof}[64匹马问题的提示]
  按简化版本的思路,随机分8组,每组8匹马,每小组赛一场确定小组内的排序,然后各个小组的每1名出来赛一场确定小组间的排序。这样可以得到以下的队形:
  \begin{center}
    \begin{tikzpicture}[scale=.5]
      \fill[pattern=crosshatch,pattern color=gray](0,0)--(8,0)--(8,8)--(4,8)--(4,4)--(0,4)--cycle;
      \fill[color=gray](4,4)--(4,7)--(3,7)--(3,6)--(2,6)--(2,5)--(1,5)--(1,4)--cycle;
      \draw(0,0)grid(8,8);
      \foreach \y/\v in{7/A,6/B,5/C,4/D,3/E,2/F,1/G,0/H}{
        \foreach \x in{1,2,3,4,5,6,7,8}{
          \node at(\x-.5,\y+.5){\tiny $\v\x$};
        }
      }
    \end{tikzpicture}
  \end{center}
  其中队形里每一行从左到右是按从快到慢排序的,第1列从上到下也是按从快到慢排序的。排除掉不可能是前4名的之外,总共还剩余10匹,即图中左上角不带阴影的部分,其中$A1$已经可以确定是最快的第1名。此时总共赛了$8+1=9$场。

  观察此队形,$A4$和$B1$是不能同时出现在前4名里的\footnote{同样互斥的还有$B3$和$C1$,$C2$和$D1$,但这两对互斥的在不知道$A4$与$B1$的比较结果之前,无论是哪一个场景都无法只赛一场就能确定前4名的顺序。$A4$比$B1$快,相当于排除了$B1$及其右边及下边的所有马匹。$B3$比$C1$快,同样相当于排除了$C1$及其右边及下边的所有马匹,但$A2,A3,A4$无法排除。}。而$A4$出现在前4名对应于唯一的一种情况:分组运气特别好,前4名都分到同一组里了,此时前4名为$A1,A2,A3,A4$。所以,$A4$与$B1$再赛一场,如果$A4$比$B1$快,那么可以确定前4名为$A1,A2,A3,A4$,否则$A4$不在前4名里,从而表格中剩余的10匹马中除去$A1,A4$外的8匹再赛一场即可赛出剩余的第2、3、4名。

  所以运气好的时候,10场即可决出前4名的排名,运气不好时则需要11场。
\end{proof}



\section{逻辑益智游戏方格}
\label{sec:logic-puzzle}

爱因斯坦的难题也可以通过Logic Grid Puzzle\footnote{\url{https://en.wikipedia.org/wiki/Logic_puzzle}}的方法来解决。


\begin{example}
  Fill the logic puzzle grid with hints:
  \begin{enumerate}
  \item Bryan likes Spiderman.
  \item Tony doesn't like Superman.
  \item The youngest kid likes Spiderman.
  \item The kid who likes Superman is 8.
  \end{enumerate}
  \begin{center}
    \begin{tikzpicture}[scale=0.5]
      \foreach \xc/\yc/\n/\th/\tw in{2/2/3/4.5/2.5}{
        \fill[color=gray!10](0,0)rectangle(\n,-\n);
        \draw(6, 0) grid (0, -3) grid (3, -6);
        \foreach \y/\xa/\xb in {0/0/6, -3/0/6, -6/0/3}{
          \draw[thick](\xa,\y)--(\xb,\y);
        }
        \foreach \x/\ya/\yb in {0/0/-6, 3/0/-6, 6/0/-3}{
          \draw[thick](\x,\ya)--(\x,\yb);
        }
        \foreach \x in{0,1,2,3,4,5,6}{
          \draw(\x,0)--(\x,\th);
        }
        % \draw(0,3)--(6,3);
        \foreach \x/\t/\ta/\tb/\tc/\c in {0/Heroes/Batman/Spiderman/Superman/red!20,
          3/Age/6/8/10/blue!50}{
          \draw[fill=\c](\x, \th) rectangle (\x + \n, \th + 1) node[midway]{\t};
          \node[rotate=90, anchor=west]at(0.5 + \x, 0){\ta};
          \node[rotate=90, anchor=west]at(1.5 + \x, 0){\tb};
          \node[rotate=90, anchor=west]at(2.5 + \x, 0){\tc};
        }
        \foreach \y in{0,1,2,3,4,5,6}{
          \draw(0,-\y)--(-\tw,-\y);
        }
        \draw(-\tw,0)--(-\tw, -6);
        \foreach \y/\t/\ta/\tb/\tc/\c in {0/Name/Brian/Sean/Tony/red!20,
          3/Age/6/8/10/blue!50}{
          \draw[fill=\c](-\tw, -\y) rectangle (-\tw-1, -\y - \n) node[midway, rotate=90, anchor=center]{\t};
          % \node[rotate=90, anchor=west]at(-4.5, -\y-4){\t};
          \node[left]at(0, -0.5 - \y){\ta};
          \node[left]at(0, -1.5 - \y){\tb};
          \node[left]at(0, -2.5 - \y){\tc};
        }
      }        
    \end{tikzpicture}
  \end{center}

  Fill in each grid a True(e.g. \tikz{\draw[fill=green]circle(0.2);}) or False(e.g.\redcross). Similar to sodoku, there must contain one and only one True mark for every rows and columns in every $4\times 4$ sub-grid.

  From (1), the grids can be filled as the following left
  figure. Considering the (2), (3) and (4) conditions, the right
  figure can be filled.
  \begin{center}
    \begin{tikzpicture}[scale=0.5]
      \foreach \xc/\yc/\n/\th/\tw in{2/2/3/4.5/2.5}{
        \fill[color=gray!10](0,0)rectangle(\n,-\n);
        \draw(6, 0) grid (0, -3) grid (3, -6);
        \foreach \y/\xa/\xb in {0/0/6, -3/0/6, -6/0/3}{
          \draw[thick](\xa,\y)--(\xb,\y);
        }
        \foreach \x/\ya/\yb in {0/0/-6, 3/0/-6, 6/0/-3}{
          \draw[thick](\x,\ya)--(\x,\yb);
        }
        \foreach \x in{0,1,2,3,4,5,6}{
          \draw(\x,0)--(\x,\th);
        }
        % \draw(0,3)--(6,3);
        \foreach \x/\t/\ta/\tb/\tc/\c in {0/Heroes/Batman/Spiderman/Superman/red!20,
          3/Age/6/8/10/blue!50}{
          \draw[fill=\c](\x, \th) rectangle (\x + \n, \th + 1) node[midway]{\t};
          \node[rotate=90, anchor=west]at(0.5 + \x, 0){\ta};
          \node[rotate=90, anchor=west]at(1.5 + \x, 0){\tb};
          \node[rotate=90, anchor=west]at(2.5 + \x, 0){\tc};
        }
        \foreach \y in{0,1,2,3,4,5,6}{
          \draw(0,-\y)--(-\tw,-\y);
        }
        \draw(-\tw,0)--(-\tw, -6);
        \foreach \y/\t/\ta/\tb/\tc/\c in {0/Name/Brian/Sean/Tony/red!20,
          3/Age/6/8/10/blue!50}{
          \draw[fill=\c](-\tw, -\y) rectangle (-\tw-1, -\y - \n) node[midway, rotate=90, anchor=center]{\t};
          % \node[rotate=90, anchor=west]at(-4.5, -\y-4){\t};
          \node[left]at(0, -0.5 - \y){\ta};
          \node[left]at(0, -1.5 - \y){\tb};
          \node[left]at(0, -2.5 - \y){\tc};
        }
      }
      \foreach \x/\y/\t in {1/1/0, 2/1/1, 3/1/0, 2/2/0, 2/3/0}{
        \ifnum\t=1
        \draw[fill=green](\x-0.5,-\y+0.5)circle(0.3);
        \else
        \node at(\x-0.5, -\y+0.5){\redcross};
        \fi
      }

      \begin{scope}[shift={(14,0)}]
        \foreach \xc/\yc/\n/\th/\tw in{2/2/3/4.5/2.5}{
          \fill[color=gray!10](0,0)rectangle(\n,-\n);
          \draw(6, 0) grid (0, -3) grid (3, -6);
          \foreach \y/\xa/\xb in {0/0/6, -3/0/6, -6/0/3}{
            \draw[thick](\xa,\y)--(\xb,\y);
          }
          \foreach \x/\ya/\yb in {0/0/-6, 3/0/-6, 6/0/-3}{
            \draw[thick](\x,\ya)--(\x,\yb);
          }
          \foreach \x in{0,1,2,3,4,5,6}{
            \draw(\x,0)--(\x,\th);
          }
          % \draw(0,3)--(6,3);
          \foreach \x/\t/\ta/\tb/\tc/\c in {0/Heroes/Batman/Spiderman/Superman/red!20,
            3/Age/6/8/10/blue!50}{
            \draw[fill=\c](\x, \th) rectangle (\x + \n, \th + 1) node[midway]{\t};
            \node[rotate=90, anchor=west]at(0.5 + \x, 0){\ta};
            \node[rotate=90, anchor=west]at(1.5 + \x, 0){\tb};
            \node[rotate=90, anchor=west]at(2.5 + \x, 0){\tc};
          }
          \foreach \y in{0,1,2,3,4,5,6}{
            \draw(0,-\y)--(-\tw,-\y);
          }
          \draw(-\tw,0)--(-\tw, -6);
          \foreach \y/\t/\ta/\tb/\tc/\c in {0/Name/Brian/Sean/Tony/red!20,
            3/Age/6/8/10/blue!50}{
            \draw[fill=\c](-\tw, -\y) rectangle (-\tw-1, -\y - \n) node[midway, rotate=90, anchor=center]{\t};
            % \node[rotate=90, anchor=west]at(-4.5, -\y-4){\t};
            \node[left]at(0, -0.5 - \y){\ta};
            \node[left]at(0, -1.5 - \y){\tb};
            \node[left]at(0, -2.5 - \y){\tc};
          }
        }
        \foreach \x/\y/\t in {1/1/0, 2/1/1, 3/1/0, 2/2/0, 2/3/0, 3/3/0,2/4/1, 3/5/1}{
          \ifnum\t=1
          \draw[fill=green](\x-0.5,-\y+0.5)circle(0.3);
          \else
          \node at(\x-0.5, -\y+0.5){\redcross};
          \fi
        }
      \end{scope}
    \end{tikzpicture}
  \end{center}

\end{example}

\begin{example}
Base on the hints:
\begin{enumerate}
\item Exactly one girl has the same initial in her name and in the color of her basket.
\item Trisha is 4 or 6 years old, and had the green or red basket.
\item The girl who collected 28 apples is 1 year older than Trisha.
\item The green basket had 4 apples less than the blue one. The other two baskets belonged to Berenice and the 6-year-old girl.
\item Gina is 5 years old, or she had the white basket. She collected more apples than Berenice.
\item If the 4 years girl had the blue basket, then Gina got 24 apples.
\item The 7-year-old girl didn't collect 20 apples.
\end{enumerate}

Fill the Logic Puzzle Grid, the 4 \redcross of which is filled by the hint (2).

\begin{center}
  \begin{tikzpicture}[scale=0.5]
    % \foreach \rows/\cols/\individuals in{3/3/5}{%4 rows, 4 cols, everty grid contains 5x5 sub-grids
    \fill[color=gray!10](8,0)rectangle(4,-4)rectangle(0,-8);
    \draw(12, 0) grid (0, -4) grid (8, -8) (4, -8) grid (0, -12);
    \foreach \y/\xa/\xb in {0/0/12, -4/0/12, -8/0/8, -12/0/4}{
      \draw[thick](\xa,\y)--(\xb,\y);
    }
    \foreach \x/\ya/\yb in {0/0/-12, 4/0/-12, 8/0/-8, 12/0/-4}{
      \draw[thick](\x,\ya)--(\x,\yb);
    }
    \foreach \x in{0,1,2,3,4,5,6,7,8,9,10,11,12}{
      \draw(\x,0)--(\x,3);
    }
    \draw(0,3)--(12,3);
    \foreach \x/\t/\ta/\tb/\tc/\td/\c in {0/Age/4/5/6/7/red!20,
      4/Basket/blue/green/red/white/blue!50,
      8/Quantity/20/24/25/28/green}{
      \draw[fill=\c](\x, 3) rectangle (\x + 4, 4) node[midway]{\t};
      \node[rotate=90, anchor=south west]at(1 + \x, 0){\ta};
      \node[rotate=90, anchor=south west]at(2 + \x, 0){\tb};
      \node[rotate=90, anchor=south west]at(3 + \x, 0){\tc};
      \node[rotate=90, anchor=south west]at(4 + \x, 0){\td};
    }
    \foreach \y in{0,1,2,3,4,5,6,7,8,9,10,11,12}{
      \draw(0,-\y)--(-4,-\y);
    }
    \draw(-4,0)--(-4, -12);
    \foreach \y/\t/\ta/\tb/\tc/\td/\c in {0/Name/Berenice/Gina/Rachel/Trisha/red!20,
      4/Basket/blue/green/red/white/blue!50,
      8/Quantity/20/24/25/28/green}{
      \draw[fill=\c](-4, -\y) rectangle (-5, -\y - 4) node[midway, rotate=90, anchor=center]{\t};
      % \node[rotate=90, anchor=west]at(-4.5, -\y-4){\t};
      \node[left]at(0, -0.5 - \y){\ta};
      \node[left]at(0, -1.5 - \y){\tb};
      \node[left]at(0, -2.5 - \y){\tc};
      \node[left]at(0, -3.5 - \y){\td};
    }

    \foreach \x/\y/\t in {2/4/0, 4/4/0, 5/4/0, 8/4/0%Trisha
    }{%
      \ifnum\t=1
      \draw[fill=green](\x-0.5,-\y+0.5)circle(0.3);
      \else
      \node at(\x-0.5, -\y+0.5){\redcross};
      \fi
    }

  \end{tikzpicture}
\end{center}

\end{example}


\begin{example}
  10 people line up, s.t. the taller one is behind. Everyone has a hat either black or blue. Everyone can see the color of the hats before him. Do they have a strategy that they can make sure at least 9 people can tell the color of his own hat?

  \begin{center}
    \begin{tikzpicture}[scale=.8]
      \foreach \s/\x/\h in{1cm/15.4/black, 1.2cm/14.2/yellow, 1.3cm/12.9/yellow, 1.4cm/11.5/black, 1.5cm/10/yellow, 1.6cm/8.4/black, 1.7cm/6.7/black, 1.8cm/4.9/yellow, 1.9cm/3/black, 2.0cm/1/yellow}{
        \node[conductor, minimum size=\s, hat=\h] at(\x,0){};
      }      
    \end{tikzpicture}
  \end{center}

  Who to sacrify to help others? The shortest one? No, since he has no information on the color of all the hats. He sacrifies in vain.

  Sacrifice the tallest one to tell the even-odd property of the other nine hats. Then the second tallest can tell the color of his own hat. After that, so could the third tallest. And so on.
\end{example}


\section{广泛流传的经典问题}
\label{sec:classic-puzzles}

\begin{example}[海盗分金\footnote{\url{https://www.mysmth.net/nForum/\#!elite/path?v=\%2Fgroups\%2Frec.faq\%2FIQDoor\%2Ftoold\%2Fgold}, \url{https://omohundro.files.wordpress.com/2009/03/stewart99_a_puzzle_for_pirates.pdf, 英国数学家伊恩·斯图尔特(Ian Stewart)在1999年5月的《科学美国人》上,发表了A Puzzle for Pirate的文章分析了这个问题。所以,你看,数学家们也会考虑这些看上去不正经的问题,比如诺贝尔经济奖得主纳什研究的博弈理论。}}]
  10名海盗抢得了窖藏的100块金子,并打算 瓜分这些战利 品。这是一些讲民主
  的海盗(当然是他们自己特有的民主), 他们的习惯是按下面的方式进行分配:
  最厉害的一名海盗提出分配方案,然 后所有的海盗(包括提出方案者本人)就
  此方案进行表决。如果50\%或更多的 海盗赞同此方案,此方案就获得通过并据
  此分配战利品。否则提出方案的海 盗将被扔到海里,然后下一名最厉害的海盗
  又重复上述过程。
\end{example}

所有的海盗都乐于看到他们的一位同伙被扔进海里,不过,如果让他们 选择的话,
他们还是宁可得一笔现金。他们当然也不愿意自己被扔到海里。  所有的海盗都
是有理性的,而且知道其他的海盗也是有理性的。此外,没有 两名海盗是同等厉
害的——这些海盗按照完全由上到下的等级排好了座次, 并且每个人都清楚自己和
其他所有人的等级。这些金块不能再分,也不允许 几名海盗共有金块,因为任何
海盗都不相信他的同伙会遵守关于共享金块的 安排。这是一伙每人都只为自己打
算的海盗。

最凶的一名海盗应当提出什么样的分配方案才能使他获得最多的金子呢?
    
为方便起见,我们按照这些海盗 那优 程度来给他们编号。最怯懦的海 盗为1号
海盗,次怯懦的海盗为2号海盗,如此类推。这样最厉害的海盗就应 当得到最大
的编号,而方案的提出就将倒过来从上至下地进行。
    
分析所有这类策略游戏的奥妙就在于应当从结尾出发倒推回去。游戏结 束时,你
容易知道何种决策有利而何种决策不利。确定了这一点后,你就可 以把它用到倒
数第2次决策上,如此类推。如果从游戏的开头出发进行分析, 那是走不了多远
的。其原因在于,所有的战略决策都是要确定:“如果我这 样做,那么下一个人
会怎样做?” 因此在你以下海盗所做的决定对你来说是 重要的,而在你之前的海
盗所做的决定并不重要,因为你反正对这些决定也 无能为力了。
    
记住了这一点,就可以知道我们的出发点应当是游戏进行到只剩两名
海 盗——即1号和2号——的时候。这时最厉害的海盗是2号,而他的最佳分配方 案是
一目了然的:100块金子全归他一人所有,1号海盗什么也得不到。由于 他自己肯
定为这个方案投赞成票,这样就占了总数的50\%,因此方案获得通过。  现在加
上3号海盗。1号海盗知道,如果3号的方案被否决,那么最后将只剩2 个海盗,
而1号将肯定一无所获——此外,3号也明白1号了解这一形势。因此, 只要3号的分
配方案,给1号一点甜头使他不至于空手而归,那么不论3号提出 什么样的分配方
案,1号都将投赞成票。因此3号需要分出尽可能少的一点金子 来贿赂1号海盗,
这样就有了下面的分配方案: 3号海盗分得99块金子,2号海 盗一无所获,1号海
盗得1块金子。4号海盗的策略也差不多。他需要有50\%的支 持票,因此同3号一
样也需再找一人做同党。他可以给同党的最低贿赂是1块金 子,而他可以用这块
金子来收买2号海盗。因为如果4号被否决而3号得以通过, 则2号将一文不名。因
此,4号的分配方案应是:99块金子归自己,3号一块也 得不到,2号得1块金
子,1号也是一块也得不到。5号海盗的策略稍有不同。他 需要收买另两名海盗,
因此至少得用2块金子来贿赂,才能使自己的方案得到 采纳。他的分配方案应该
是:98块金子归自己,1块金子给3号,1块金子给1号。
  
    
这一分析过程可以照着上述思路继续进行下去。每个分配方案都是唯一确定的,
它可以使提出该方案的海盗获得尽可能多的金子,同时又保证该方案肯 定能通过。
照这一模式进行下去,10号海盗提出的方案将是96块金子归他所有,其他编号为
偶数的海盗各得1 块金子,而编号为奇数的海盗则什么也得不到。这就解决
了10名海盗的分配难题。
    
Omohundro的贡献是他把这一问题扩大到有500名海盗的情形,即500名海 盗瓜
分100块金子。显然,类似的规律依然成立——至少是在一定范围内成立。  事实上,
前面所述的规律直到第200号海盗都成立。 200号海盗的方案将是: 从1到199号
的所有奇数号的海盗都将一无所获,而从2到198号的所有偶数号 海盗将各得1块
金子,剩下的1块金子归200号海盗自己所有。
    
乍看起来,这一论证方法到200号之后将不再适用了,因为201号拿不出更 多的金
子来收买其他海盗。但是即使分不到金子,201号至少还希望自己不会 被扔进海
里,因此他可以这样分配:给1到199号的所有奇数号海盗每人1块金 子,自己一
块也不要。202号海盗同样别无选择,只能一块金子都不要了—— 他必须把这100块
金子全部用来收买100名海盗,而且这100名海盗还必须是那 些按照201号方案将
一无所获的人。

由于这样的海盗有101名,因此202号的方案将不再是唯一的——贿赂方
案 有101种。203号海盗必须获得102张赞成票,但他显然没有足够的金子去收
买 101名同伙。因此,无论提出什么样的分配方案,他都注定会被扔到海里去
喂 鱼。不过,尽管203号命中注定死路一条,但并不是说他在游戏进程中不起
任 何作用。相反,204号现在知道,203号为了能保住性命,就必须避免由他
自 己来提出分配方案这么一种局面,所以无论204号海盗提出什么样的方
案,203 号都一定会投赞成票。这样204号海盗总算 男 拣到一条命:他可以得到
他自 己的1票、203号的1票、以及另外100名收买的海盗的赞成票,刚好达到保命
所 需的50\%。获得金子的海盗,必属于根据202号方案肯定将一无所获的
那101名 海盗之列。

205号海盗的命运又如何呢?他可没有这样走运了。他不能指望203号和 204号支
持他的方案,因为如果他们投票反对205号方案,就可以 以掷 祸地 看到205号被
扔到海里去喂鱼,而他们自己的性命却仍然能够保全。这样,无 论205号海盗提
出什么方案都必死无疑。

206号海盗也是如此——他肯定可以得到205号的支持,但这不足以救他一命。类似
地,207号海盗需要104张赞成票——除了他收买的100张赞成票以及他自己的1张赞
成票之外,他还需3张赞成票才能免于一死。他可以获得205 号和206号的支持,
但还差一张票却是无论如何也弄不到了,因此207号海盗的命运也是下海喂鱼。

208号又时来运转了。他需要104张赞成票,而205、206、207号都会支持 他,加
上他自己一票及收买的100票,他得以过关保命。获得他贿赂的必属于 那些根
据204号方案肯定将一无所获的人(候选人包括2到200号中所有偶数号 的海盗、
以及201、203、204 号)。
    
现在可以看出一条新的、此后将一直有效的规律:那些方案能过关的海盗(他们
的分配方案全都是把金子用来收买100名同伙而自己一点都得不到)相 隔的距离
越来越远,而在他们之间的海盗则无论提什么样的方案都会被扔进海 里——因此为
了保命,他们必会投票支持比他们厉害的海盗提出的任何分配方 案。得以避免葬
身鱼腹的海盗包括201、202、204、208、216、232、264、328、 456号,即其号
码等于200加2的某一方幂的海盗。
    
现在我们来看看哪些海盗是获得贿赂的幸运儿。分配贿赂的方法是不唯一 的,其
中一种方法是让201号海盗把贿赂分给1到199号的所有奇数编号的海盗,
  
让202号分给2到200号的所有偶数编号的海盗,然后是让204号贿赂奇数编号的 海
盗,208号贿赂偶数编号的海盗,如此类推,也就是轮流贿赂奇数编号和偶 数编
号的海盗。
    
结论是:当500名海盗运用最优策略来瓜分金子时,头44名海盗必死无疑,
而456号海盗则给从1到199号中所有奇数编号的海盗每人分1块金子,问题就解决
了。由于这些海盗所实行的那种民主制度,他们的事情就搞成了最厉害的一 批海
盗多半都是下海喂鱼,不过有时他们也会觉得自己很幸运——虽然分不到 抢来的金
子,但总可以免于一死。只有最怯懦的200名海盗有可能分得一份脏物,而他们
之中又只有一半的人能真正得到一块金子,的确是怯懦者继承财富。

\begin{example}[Impossible Puzzle, Sum-Product Puzzle\footnote{https://math.stackexchange.com/questions/165311/the-impossible-puzzle-now-i-know-your-product}]
  1969年,荷兰数学家汉斯·弗莱登塔尔(Hans Freudenthal)发表了一个当时被称为“弗莱登塔尔问题”(Freudenthal Problem)的谜题:

  有两个不相等的整数 $x$,$y$ ,它们都大于 1 且和小于 100 ,数学家“和先
  生”知道这两个数的和,数学家“积先生”知道这两个数的积,他们进行了如下对
  话:

  积先生:我不知道 $x$ 和 $y$ 分别是啥。

  和先生:我知道你不知道。

  积先生:我现在知道了。

  和先生:如果你知道了,那我也知道了。

  那么,$x$ 和 $y$ 各是多少?
\end{example}
\include{game}
\include{nim-game}
\include{music}

\include{magic-square}

\include{fractal}

\include{algorithm}
\include{dp}


\chapter{颜色}
\label{chap:color}

\newcommand{\RGB}{{\color{red}R}{\color{green}G}{\color{blue}B}}
\newcommand{\RYB}{{\color{red}R}{\color{yellow}Y}{\color{blue}B}}
\newcommand{\CMY}{{\color{cyan}C}{\color{magenta}M}{\color{yellow}Y}}
\newcommand{\CMYK}{\CMY{}{\color{black}K}}

\definecolor{chartreuse}{RGB}{127,255,0}
\definecolor{springgreen}{RGB}{0,255,127}
\definecolor{azure}{RGB}{0,127,255}
\definecolor{rose}{RGB}{255,0,127}
\definecolor{yellowgreen}{RGB}{64,255,0}
\definecolor{yellowblue}{RGB}{0,64,255}
\definecolor{redorange}{RGB}{255,64,0}
\definecolor{yelloworange}{RGB}{255,192,0}

% RYB color wheel
% https://bahamas10.github.io/ryb/about.html
\definecolor{rybRed}{RGB}{255,0,0}
\definecolor{rybYellow}{RGB}{255,255,0}
\definecolor{rybBlue}{RGB}{42,96,153}
% Secondary
% violet in ryb [1,0,1]->[255,0,255]
\definecolor{rybViolet}{RGB}{128, 0, 128}
% green in ryb  [0,1,1]->[0,255,255]
\definecolor{rybGreen}{RGB}{0, 169, 51}
% orange in ryb [1,1,0]->[255,255,0]
\definecolor{rybOrange}{RGB}{255, 128, 0}
% Tertiary
% blue green [0,1/2,1]->[0,128,255]
\definecolor{rybBlueGreen}{RGB}{21, 132, 102}
% yellow green [0,1,1/2]->[0,255,128]
\definecolor{rybYellowGreen}{RGB}{127, 212, 26}
% yellow orange [1/2,1,0]->[128,255,0]
\definecolor{rybYellowOrange}{RGB}{255, 191, 0}
% red orange [1,1/2,0]->[255,128,0]
\definecolor{rybRedOrange}{RGB}{255, 65, 0}
% red violet [1,0,1/2]->[255,0,128]
\definecolor{rybRedViolet}{RGB}{191, 0, 65}
% blue violet [1/2,0,1]->[128,0,255]
\definecolor{rybBlueViolet}{RGB}{85, 48, 141}


\section{色彩空间}
\label{sec:color-space}

\begin{definition}[色彩模型,Color Model]
  色彩模型是指通过一组数字描述颜色的数学模型,通常有3个或4个分量。
\end{definition}

\begin{example}[RGB色彩模型,RGB]
  \RGB{}表示由{\color{red}Red}--{\color{green}Green}--{\color{blue}Blue}三原色(Primary colors)生成颜色的色彩模型,通常用于电子屏幕,是一种加法模型,即通过叠加不同亮度的三原色产生所需的颜色。
\end{example}

\begin{example}[CMYK色彩模型]
  \CMYK{}是由{\color{cyan}Cyan}--{\color{magenta}Magenta}--{\color{yellow}Yellow}--{\color{black}blacK}四原色生成颜色的色彩模型,通常用于印刷行业,是一种减法模型。即通过印刷不同的四原色组合,将照射到载体(书籍、杂志等)上的光线吸收一部分(这个被吸收的过程就是减法过程)后反射,使得反射光被人眼感知到的色彩是期望的色彩。

  在实际应用中一方面\CMY{}三色很难叠加成真正的黑色,一般只能达到褐色,另一方面黑色需求比其它颜色要大的多且黑色原料的成本比起\CMY{}三原色的成本低的多,因此在\CMY{}的基础上又引入了K,即黑色,一方面强化暗调,加深暗部色彩,一方面降低印刷成本。

  这种色彩不光与印刷有关,还与照射到印刷载体上的光线有关,这就是为什么同一本书在不同的环境下可能会看到不同的颜色。
\end{example}

\begin{example}[HSV色彩模型]
  
\end{example}

\begin{definition}[YUV]
  
\end{definition}

\begin{definition}[色域]
  
\end{definition}

\begin{definition}[色彩空间,Color Space]
  色彩模型
\end{definition}

\begin{example}[sRGB色彩空间]
  
\end{example}

\begin{example}[Adobe RGB色彩空间]
  
\end{example}

\begin{definition}[绝对色彩空间]
  
\end{definition}


\begin{example}\RGB{}色彩模型在三维空间中的表示,如下图,分别以三原色为三维直角坐标系的三个正交基,原点为黑色,坐标$(1,1,1)$处为白色。

  \begin{center}
    \begin{tikzpicture}[scale=2.0]
      % \begin{scope}
      %   \shade[upper left=green, upper right=yellow, lower left=black, lower right=red](0,0)rectangle(1,1);
      % \end{scope}
      \begin{scope}[shift={(-2.5,0)}]
        % \draw(0,0,0)--(0,1,0)--(0,1,1)--(0,0,1)--cycle;
        % \draw(0,0,0)--(1,0,0)--(1,1,0)--(0,1,0)--cycle;
        % \draw(0,0,0)--(0,0,1)--(1,0,1)--(1,0,0)--cycle;
        % \draw(1,0,1)--(1,1,1)--(0,1,1) (1,1,1)--(1,1,0);
        \draw(0,1,0)--(1,1,0)--(1,1,1)--(0,1,1)--cycle;
        \draw(0,0,1)--(1,0,1)--(1,1,1)--(0,1,1)--cycle;
        \draw(1,0,0)--(1,1,0)--(1,1,1)--(1,0,1)--cycle;
        \draw[help lines,dashed](0,0,0)--(1,0,0) (0,0,0)--(0,1,0) (0,0,0)--(0,0,1);

        \draw[->](1,0,0)--(1.5,0,0)node[right]{\color{red}$R$};
        \draw[->](0,1,0)--(0,1.5,0)node[above]{\color{green}$G$};
        \draw[->](0,0,1)--(0,0,1.5)node[below left]{\color{blue}$B$};

        \fill[ball color=black](0,0,0)circle(.1);
        \fill[ball color=red](1,0,0)circle(.1);
        \fill[ball color=green](0,1,0)circle(.1);
        \fill[ball color=blue](0,0,1)circle(.1);

        \fill[ball color=cyan](0,1,1)circle(.1);
        \fill[ball color=magenta](1,0,1)circle(.1);
        \fill[ball color=yellow](1,1,0)circle(.1);

        \filldraw[draw=black,ball color=white](1,1,1)circle(.1);
      \end{scope}
      \begin{scope}
        \shade[lower left=black,lower right=red,upper left=green,upper right=yellow](0,0,0)--(1,0,0)--(1,1,0)--(0,1,0)--cycle;
        \shade[lower left=blue,lower right=black,upper left=cyan,upper right=green](0,0,0)--(0,1,0)--(0,1,1)--(0,0,1)--cycle;
        \shade[lower left=blue, lower right=magenta,upper left=black,upper right=red](0,0,0)--(0,0,1)--(1,0,1)--(1,0,0)--cycle;
      \end{scope}

      \begin{scope}[shift={(2,0,0)}]
        % \shade[lower left=white,lower right=cyan,upper left=magenta,upper right=green](0,0,0)--(1,0,0)--(1,1,0)--(0,1,0)--cycle;
        % \shade[lower left=yellow,lower right=white,upper left=red,upper right=magenta](0,0,0)--(0,1,0)--(0,1,1)--(0,0,1)--cycle;
        % \shade[lower left=yellow, lower right=blue,upper left=white,upper right=cyan](0,0,0)--(0,0,1)--(1,0,1)--(1,0,0)--cycle;

        \shade[lower left=blue, lower right=magenta,
               upper left=cyan,  upper right=white]
               (0,0,1)--(1,0,1)--(1,1,1)--(0,1,1)--cycle;
        \shade[lower left=magenta, lower right=red,
               upper left=white,  upper right=yellow]
               (1,0,0)--(1,1,0)--(1,1,1)--(1,0,1)--cycle;
        \shade[lower left=cyan, lower right=white,
               upper left=green,  upper right=yellow]
               (0,1,0)--(1,1,0)--(1,1,1)--(0,1,1)--cycle;               
      \end{scope}
    \end{tikzpicture}
  \end{center}
\end{example}

\begin{example}[Color Wheel]\mbox{}\par
  \begin{center}
    \begin{tikzpicture}[scale=.75]
      \begin{scope}
        \foreach \r/\R/\n/\dt/\rot in{1/2/3/120/30}{
          \foreach \i/\c in{0/red,1/blue,2/yellow}{
            \fill[color=\c](\i*\dt+\rot:\r)--(\i*\dt+\rot:\R)arc(\i*\dt+\rot:\i*\dt+\dt+\rot:\R)--(\i*\dt+\dt+\rot:\r)
            arc(\i*\dt+\dt+\rot:\i*\dt+\rot:\r);
            \path%[decorate,decoration={text along path,text align=center,text={\c}]
            (\i*\dt+\rot:\R+.25)arc(\i*\dt+\rot:\i*\dt+\dt+\rot:\R+.25)node[midway,sloped]{\color{\c}\c};
          }
        }
      \end{scope}
      \begin{scope}[shift={(5.5,0)}]
        \foreach \r/\R/\n/\dt/\rot in{1/2/6/60/60}{
          \foreach \i/\c in{0/red,1/violet,2/blue,3/green,4/yellow,5/orange}{
            \fill[color=\c](\i*\dt+\rot:\r)--(\i*\dt+\rot:\R)arc(\i*\dt+\rot:\i*\dt+\dt+\rot:\R)--(\i*\dt+\dt+\rot:\r)
            arc(\i*\dt+\dt+\rot:\i*\dt+\rot:\r);
            \path%[decorate,decoration={text along path,text align=center,text={\c}]
            (\i*\dt+\rot:\R+.25)arc(\i*\dt+\rot:\i*\dt+\dt+\rot:\R+.25)node[midway,sloped]{\color{\c}\c};
          }
        }
      \end{scope}
      % \begin{scope}[shift={(11,0)}]
      %   \foreach \r/\R/\n/\dt/\rot in{1/2/12/30/75}{
      %     \foreach \i/\c in{0/red,1/violet, 2/magenta,3/violet,4/blue,5/black,6/cyan,7/black,8/green,9/black,10/yellow,11/orange}{
      %       \fill[color=\c](\i*\dt+\rot:\r)--(\i*\dt+\rot:\R)arc(\i*\dt+\rot:\i*\dt+\dt+\rot:\R)--(\i*\dt+\dt+\rot:\r)
      %       arc(\i*\dt+\dt+\rot:\i*\dt+\rot:\r);
      %       \path%[decorate,decoration={text along path,text align=center,text={\c}]
      %       (\i*\dt+.5*\dt+\rot:\R+.25)--(\i*\dt+.5*\dt+\rot:\R+1.5)node[midway,sloped]{\color{\c}\c};
      %     }
      %   }
      %   \end{scope}
    \end{tikzpicture}
  \end{center}
\end{example}


\begin{example}[\RYB{} Color Wheel]\mbox{}\par
  \begin{center}
    \begin{tikzpicture}[scale=1]
      \begin{scope}[shift={(-6,-6)}]
          \foreach \i/\h/\a/\c/\coord/\t in{%
            0/1.5/right/rybRed/{(1,0,0)}/{Red},
            1/0.5/left/rybRedViolet/{(1,.5,0)}/{Red Violet},
            2/1.0/left/rybViolet/{(1,0,1)}/Violet,
            3/0.5/left/rybBlueViolet/{(0,.5,1)}/{Blue Violet},
            4/1.5/left/Blue/{(0,0,1)}/Blue,
            5/0.5/left/rybBlueGreen/{(0,1,.5)}/{Blue Green},
            6/1.0/right/rybGreen/{(0,1,1)}/Green,
            7/0.5/right/rybYellowGreen/{(0,1,.5)}/{Yellow Green},
            8/1.5/right/rybYellow/{(0,1,0)}/Yellow,
            9/0.5/right/rybYellowOrange/{(.5,1,0)}/{Yellow Orange},
           10/1.0/right/rybOrange/{(1,1,0)}/Orange,
           11/0.5/right/rybRedOrange/{(1,.5,0)}/{Red Orange}%
          }{
            \fill[color=\c](\i,0)rectangle(\i+1,\h);
            % \node[anchor=west,rotate=30]at(\i,\h+.1){\tiny \coord};
            \node[above]at(\i+.5,\h){\tiny \coord};
            \node[anchor=west,rotate=-45]at(\i+.4,-.1){\tiny \t};
          }        
      \end{scope}
      \begin{scope}[shift={(0,0)}]
        \foreach \r/\R/\n/\dt/\rot in{1.8/3/12/30/75}{
          \foreach \i/\a/\c/\t in{%
            0/right/rybRed/{(1,0,0) Red},
            1/left/rybRedViolet/{(1,1/2,0) Red Violet},
            2/left/rybViolet/{(1,0,1) Violet},
            3/left/rybBlueViolet/{(0,1/2,1) Blue Violet},
            4/left/Blue/{(0,0,1) Blue},
            5/left/rybBlueGreen/{(0,1,1/2) Blue Green},
            6/right/rybGreen/{(0,1,1) Green},
            7/right/rybYellowGreen/{(0,1,1/2) Yellow Green},
            8/right/rybYellow/{(0,1,0) Yellow},
            9/right/rybYellowOrange/{(1/2,1,0) Yellow Orange},
            10/right/rybOrange/{(1,1,0) Orange},
            11/right/rybRedOrange/{(1,1/2,0) Red Orange}%
          }{
            \fill[color=\c](\i*\dt+\rot:\r)--(\i*\dt+\rot:\R)arc(\i*\dt+\rot:\i*\dt+\dt+\rot:\R)--(\i*\dt+\dt+\rot:\r)
            arc(\i*\dt+\dt+\rot:\i*\dt+\rot:\r);
            % \path%[decorate,decoration={text along path,text align=center,text={\c}}]
            % (\i*\dt+.5*\dt+\rot:\R+.25)--(\i*\dt+.5*\dt+\rot:\R+.1)node[pos=1,sloped,\a]{\color{\c}\t};
          }

          \coordinate(R)at(\rot+.5*\dt:\r);
          \coordinate(V)at(\rot+2.5*\dt:\r);
          \coordinate(B)at(\rot+4.5*\dt:\r);
          \coordinate(G)at(\rot+6.5*\dt:\r);
          \coordinate(Y)at(\rot+8.5*\dt:\r);
          \coordinate(O)at(\rot+10.5*\dt:\r);
          \coordinate(C)at(0,0);
          \coordinate(RY)at($.5*(R)+.5*(Y)$);
          \coordinate(YB)at($.5*(Y)+.5*(B)$);
          \coordinate(BR)at($.5*(B)+.5*(R)$);
          \fill[color=rybRed](R)--(RY)--(C)--(BR)--cycle;
          \fill[color=rybYellow](Y)--(YB)--(C)--(RY)--cycle;
          \fill[color=rybBlue](B)--(BR)--(C)--(YB)--cycle;
          \foreach \x/\y/\z/\c in {R/Y/O/rybOrange,Y/B/G/rybGreen,B/R/V/rybViolet}{%
            \fill[color=\c](\x)--(\y)--(\z)--cycle;
          }
        }
      \end{scope}
    \end{tikzpicture}
  \end{center}
\end{example}

\section{RGB颜色的运算}
\label{sec:RGB-color-operations}

\RGB{}模型是光线的加法模型,等强度的三原色光线混合在一起就是白色,即
\begin{center}
  \begin{tikzpicture}[scale=.5]
    \fill[color=red,draw=black](0,0)circle(1);
    \node at(1.5,0){$+$};
    \fill[color=green,draw=black](3,0)circle(1);
    \node at(4.5,0){$+$};
    \fill[color=blue,draw=black](6,0)circle(1);
    \node at(7.5,0){$=$};
    \fill[color=white,draw=black](9,0)circle(1);
  \end{tikzpicture}
\end{center}

\begin{definition}[互补色]
  若等强度的两个RGB光源混合后是白光,则这两个光源的对应的颜色称为互补色。
\end{definition}

由等强度的三原色中的两个混合成的颜色称为二次色(Secondary Colors)。图~\ref{fig:complementary-colors-of-light}中三个大圆中的颜色为三原色,三个小圆中的颜色即为与其相邻的两个原色等强度混合而成的二次色。

由等强度的RGB三原色混合后为白色,可知在图~\ref{fig:complementary-colors-of-light}中处于对角线上的两个颜色为互补色。

\begin{figure}[htbp]
  \centering
  \begin{tikzpicture}[scale=0.5]
    \foreach \r/\R/\L in{.8/1/3.5}{
      \foreach \x in{30,90,150}{
        \draw[dashed](\x:\L)--(\x+180:\L);
      }
      \foreach \x in{30,150,270}{
        \draw[very thick,->](\x-60:\L)--($(\x:\L)+(\x-120:\r)$);
        \draw[very thick,->](\x+60:\L)--($(\x:\L)+(\x+120:\r)$);
      }
      \foreach \a/\rc/\c/\t in{90/\R/red/红,150/\r/magenta/洋红,210/\R/blue/蓝,270/\r/cyan/青,330/\R/green/绿,30/\r/yellow/黄}{%
        \fill[color=\c,draw=black](\a:\L)circle(\rc);
        \node at(\a:\L){\small \t};
      }
    }
  \end{tikzpicture}
  \caption{光的互补色}
  \label{fig:complementary-colors-of-light}
\end{figure}

\section{颜料的色彩运算}
\label{sec:operation-of-pigments}

与自主发光的色彩光源不同,颜料呈现的颜色是通过吸收某种颜色的光线达成的,是一种减法模型,比如印刷中常见的\CMYK{}模型。

\CMYK{}模型中的原色是{\color{cyan}Cyan}(青色)、{\color{magenta}Magenta}(洋红)、{\color{yellow}Yellow}(黄色),因为这些颜色都只吸收单一频率的光线\footnote{光线也是一种波,如果光线中波的频率都一样,就是纯色波。如果由不同频率的波混合出来的光线就是杂色波。},如表~\ref{tab:absorb-of-primary-colors-of-pigments}所示。

\begin{table}[htbp]
  \centering
  \caption{CMY颜料吸收表}
  \label{tab:absorb-of-primary-colors-of-pigments}
  \newcommand{\cc}[1]{\tikz[baseline=-.5ex]{\fill[color=#1](0,0)circle(.2)}}
  % % The point of \arraybackslash is to return \\ to its original
  % % meaning because the \centering command alters this and could
  % % possibly give you a noalign error during compilation.
  % \newcolumntype{C}{>{\centering\arraybackslash}m{.6cm}} 
  % % \newcolumntype{D}{>{\centering\arraybackslash}m{9cm}}
  \begin{tabular}{cccl}
    \hline
    颜料 & 吸收 & 反射 & 说明\\\hline
    % \cc{yellow} & \cc[red] & \cc[green] \cc[blue]\\
    \cc{yellow} & \cc{blue} & \multicolumn{1}{>{\centering\arraybackslash}m{.6cm}}{\cc{red} \cc{green}} & \multicolumn{1}{m{9cm}}{白光遇到\cc{yellow}颜料时\cc{blue}被吸收,反射剩余的\cc{red}和\cc{green}光线,相当于人眼看到的是$\cc{red}+\cc{green}=\cc{yellow}$。} \\\hline
    \cc{cyan}   & \cc{red}  & \multicolumn{1}{>{\centering\arraybackslash}m{.6cm}}{\cc{green} \cc{blue}}\\\hline
    \cc{magenta}& \cc{green}& \multicolumn{1}{>{\centering\arraybackslash}m{.6cm}}{\cc{blue} \cc{red}}\\\hline
  \end{tabular}
\end{table}

颜料的叠加,实际上就是做色彩的减法,减去入射的光线中更多的成分,如
\begin{align*}
  % \newcommand{\cc}[1]{\tikz[baseline=(current bounding box.center)]{\fill[color=#1](0,0)circle(.3)}}
  \newcommand{\cc}[1]{\tikz[baseline=-.5ex]{\fill[color=#1](0,0)circle(.3)}}
  \cc{magenta} + \cc{yellow} = \cc{red}   \quad \quad \quad
  \cc{yellow}  + \cc{cyan}   = \cc{green} \quad \quad \quad
  \cc{cyan}    + \cc{magenta}= \cc{blue}
\end{align*}

对于加法的RGB光源,三原色的和是白色。对于减法的CMY颜料,三原色的和(实际是做减法,吸收了对应颜色的光线)是黑色。即
\begin{align*}
  \newcommand{\cc}[1]{\tikz[baseline=-.5ex]{\fill[color=#1](0,0)circle(.3)}}
  \text{光源:} \cc{red} + \cc{green} + \cc{blue} = \tikz[baseline=-.5ex]{\draw(0,0)circle(.3)}\\
  \newcommand{\cc}[1]{\tikz[baseline=-.5ex]{\fill[color=#1](0,0)circle(.3)}}
  \text{颜料:} \cc{cyan} + \cc{magenta} + \cc{yellow} = \cc{black}
\end{align*}



\section{三原色原理}
\label{sec:three-primary-color-theory}

在数学上而言,三维线性空间的正交基一般都不是唯一的。那么RGB颜色既然可以由R、G、B三原色生成,自然就带来以下问题:
\begin{quotation}
  RGB是否是线性无关的?{\color{red}R}、{\color{green}G}、{\color{blue}B}三原色是正交的吗?
\end{quotation}

\begin{theorem}[格拉斯曼定律,Grassmann's Law]
  若两单色色光组合成一测试色光,则观测者感知到的三原色数值为两单色光分别被观测者单独观测到的三原色数值之和。

  格拉斯曼定律是一个根据实验数据得出的经验法则,说明了人对于色彩的感知是线性的。
\end{theorem}


\section{Color Systems}
\label{sec:color-systems}

\subsection{YUV}
\label{sec:yuv}



\include{intuition}

\include{genius-solution}
\include{genius-solution-from-the-book}

\include{traps}
\include{logic-print}

\url{https://en.wikipedia.org/wiki/International_Mathematical_Olympiad}
\url{brilliant.org/wiki/muirhead-inequality/}

\chapter{Calculation Tricks}
\label{chap:calculation-tricks}


\chapter{Division and Remainder}
\label{chap:division-and-remainder}

\chapter{Pigeonhole Principle}
\label{chap:pigeonhole-principle}

For infinite sets, the Hilbert's Grand Hotel violates the Pigeonhole Principle.
Location: \url{https://en.wikipedia.org/wiki/Hilbert's_paradox_of_the_Grand_Hotel}

\section{Worse Case Scenario}
\label{sec:worst-case-scenario}



\chapter{Even Odd}
\label{chap:even-odd}

\url{https://en.wikipedia.org/wiki/Mutilated_chessboard_problem}

\chapter{Sequence}
\label{chap:sequence}

\section{Arithmetic Progression}
\label{sec:arithmetic-progression}
\url{https://en.wikipedia.org/wiki/Arithmetic_progression}

\section{Geometric Progression}
\label{sec:geometric-progression}
\url{https://en.wikipedia.org/wiki/Geometric_progression}

\chapter{Geometry}
\label{chap:geometry}

\section{Angle}
\label{sec:angle}

\subsection{Parallel}
\label{sec:parallel}

\subsection{Perpendicular}
\label{sec:perpendicular}


\section{Polygon}
\label{sec:polygon}



\chapter{Measurement: Perimeter, Area, Volume}
\label{chap:perimeter-area-volume}

\section{Volume}
\label{sec:volume}

\subsection{Volume of Cones}
\label{sec:volume-of-cones}
\url{https://math.stackexchange.com/questions/623/why-is-the-volume-of-a-cone-one-third-of-the-volume-of-a-cylinder}

\section{Time, Velocity, Range}
\label{sec:time-velocity-range}

\section{Weight, Density, Volume}
\label{sec:weight-density-volume}


\chapter{Logic}
\label{chap:logic}



\chapter{Calender}
\label{chap:calendar}

\chapter{Combination and Permutation}
\label{chap:combination-and-permutation}

\chapter{Graph Theory}
\label{chap:graph-theory}

\section{Bipartite}
\label{sec:bipartite}




\chapter{Set Theory}
\label{chap:set-theory}

\section{Venn's Diagram}
\label{sec:venn's-diagram}



\chapter{Probability}
\label{chap:probability}



\chapter{深圳鹏程杯}
\label{chap:pengchengbei}

\begin{example}
  在下图中的钉子板上,用橡皮筋最多可以围出几个正方形?

  \begin{center}
    \begin{tikzpicture}[scale=.5, every node/.style={draw,shape=circle,fill=blue,scale=0.1}]
      \foreach \x in{1,2,3,4}{%
        \foreach \y in{1,2,3,4}{%
          \node(N\x\y) at(\x,\y){};
        }
      }
    \end{tikzpicture}
  \end{center}
\end{example}
\begin{proof}[分类讨论] 算法里的分治思想(Divide and Conquer),先分解成小问题,然后对每个小问题分别求解。
  
  可以围出的正方形共有5种形状,考虑每一种形状的个数,然后求和。

  \vspace{0.5cm}

  \begin{center}
    \begin{tikzpicture}[scale=0.5, every node/.style={draw,shape=circle,fill=blue,scale=0.1}]
      \begin{scope}[shift={(0,0)}]
        \foreach \x in{1,2,3,4}{%
          \foreach \y in{1,2,3,4}{%
            \node(N\x\y) at(\x,\y){};
          }
        }
        \draw (N11)--(N12)--(N22)--(N21)--(N11);
      \end{scope}

      \begin{scope}[shift={(5,0)}]
        \foreach \x in{1,2,3,4}{%
          \foreach \y in{1,2,3,4}{%
            \node(N\x\y) at(\x,\y){};
          }
        }
        \draw (N11)--(N13)--(N33)--(N31)--(N11);
      \end{scope}

      \begin{scope}[shift={(10,0)}]
        \foreach \x in{1,2,3,4}{%
          \foreach \y in{1,2,3,4}{%
            \node(N\x\y) at(\x,\y){};
          }
        }
        \draw (N11)--(N14)--(N44)--(N41)--(N11);
      \end{scope}

      \begin{scope}[shift={(15,0)}]
        \foreach \x in{1,2,3,4}{%
          \foreach \y in{1,2,3,4}{%
            \node(N\x\y) at(\x,\y){};
          }
        }
        \draw (N21)--(N32)--(N23)--(N12)--(N21);
      \end{scope}

      \begin{scope}[shift={(20,0)}]
        \foreach \x in{1,2,3,4}{%
          \foreach \y in{1,2,3,4}{%
            \node(N\x\y) at(\x,\y){};
          }
        }
        \draw (N21)--(N42)--(N34)--(N13)--(N21);
      \end{scope}

    \end{tikzpicture}
  \end{center}
\end{proof}


\begin{example}
  一个展厅有120盏灯,分别由编号为1~120的开关控制,这些开关每按一下,对应灯的亮灭状态就会变化一次。开始的时候120盏灯全是熄灭的。老师请大毛把所有编号是2的倍数的开关都按一下,然后又请二毛把所有编号是3的倍数的开关都按一下,然后又请三毛把所有编号是5的倍数的开关都按一下。现在有几盏灯是亮的?
\end{example}

这种题用文氏图表达就很清晰。用三个圆分别表示2的倍数的编号集合,3的倍数的编号集合,5的倍数的编号集合。那么被一个圆覆盖的表示被按了1次,被2个圆覆盖的表示被按了2次,被3个圆覆盖的表示被按了3次。因为开始全是灭的,因此被按了奇数次的开关对应的灯最后是改变了状态变成了亮的,即最终亮着的灯的个数对应于图中阴影部分元素个数。

\begin{center}
  \begin{tikzpicture}[scale=1.0]
    \draw(0,0)--(8,0)--(8,6)--(0,6)--cycle;
    % \draw[fill=blue!30, even odd rule]
    \coordinate[label=2的倍数] (A) at (3,2);
    \coordinate[label=3的倍数] (B) at (5,2);
    \coordinate[label=5的倍数] (C) at (4,4);
    \coordinate[label=$S_{23}$] (D23) at ($0.5*(A)+0.5*(B)$);
    \coordinate[label=$S_{35}$] (D35) at ($0.5*(B)+0.5*(C)$);
    \coordinate[label=$S_{52}$] (D52) at ($0.5*(C)+0.5*(A)$);
    \coordinate[label=$\Delta$] (D235) at ($1/3*(A)+1/3*(B)+1/3*(C)$);

    \fill[blue!30] (A) circle(1.5);
    \fill[blue!30] (B) circle(1.5);
    \fill[blue!30] (C) circle(1.5);
    
    \begin{scope}
      \clip (A) circle(1.5);
      \clip (B) circle(1.5);
      \fill[white] (A) circle(1.5);
    \end{scope}
    \begin{scope}
      \clip (B) circle(1.5);
      \clip (C) circle(1.5);
      \fill[white] (B) circle(1.5);
    \end{scope}
    \begin{scope}
      \clip (C) circle(1.5);
      \clip (A) circle(1.5);
      \fill[white] (C) circle(1.5);
    \end{scope}
    
    \begin{scope}
      \clip (A) circle(1.5);
      \clip (B) circle(1.5);
      \clip (C) circle(1.5);
      \fill[blue!30] (A) circle(1.5);
    \end{scope}

    \draw (A) circle(1.5);
    \draw (B) circle(1.5);
    \draw (C) circle(1.5);

  \end{tikzpicture}
\end{center}


\section{2014}
\label{sec:pengcheng2014}

\begin{example}
  The area of the shadow is 48. What the total area of the 10 circles?
  \begin{center}
    \begin{tikzpicture}[scale=1.0]
      \fill[color=blue!20](0,0)rectangle(8,2);
      \foreach \x/\y in{%
        2/0, 4/0, 6/0,
        2/2, 4/2, 6/2
      }{
        \draw[fill=white](\x,\y) circle(1);
      }        
      \foreach \x/\y in{%
        0/0, 0/2, 8/0, 8/2
      }{
        \draw[fill=blue!20](\x,\y) circle(1);
      }
    \end{tikzpicture}
  \end{center}
\end{example}

The area of the shadow is 4 circles \tikz{\draw[fill=blue!20](0,0)circle(1);} and 4 \tikz{\draw[fill=blue!20](1,0)arc(0:90:1)arc(270:360:1)arc(180:270:1)arc(90:180:1)--cycle;}
% \foreach \x/\y/\s/\t in{1/0/0/90,0/1/270/360,1/2/180/270,2/1/180/270}{\draw[fill=white](\x,\y)arc(\s:\t:1);}}

One circle and one star shape could reshape into a square:
\begin{center}
  \begin{tikzpicture}
    \draw[fill=blue!20](0,0)rectangle(2,2);
    \draw(1,1)circle(1);
  \end{tikzpicture}
\end{center}

So the area of the square is $48\div4=12$, and the length of each side of the square is $12\div 4=3$, the radius of the circle is $3\div2=\frac32$, the total area of the 10 circles is $10\times \pi\times\left(\frac32\right)^2=10\times\pi\times\frac94=\frac{45\pi}{2}$.


\chapter{Graph}
\label{ch:graph}

\begin{example}
  60 biscuits, 5 kids, some guests. Each kid gives every guest he knows 1 biscuit. And each guest gives every kid he doesn't know 1 biscuit. Exactly 60 biscuits used. How many guests?  
\end{example}

Assumption: KNOW is bi-direction, which means if A knows B, then B must knows A.

偶图(Bipartite graph)
\begin{tikzpicture}[scale=1.0]
  \foreach \y/\n in {5/1,4/2,3/3,2/4,1/5}{%
    \node(K\n)[draw, circle] at(0, \y) {$K_{\n}$};
  }
  \foreach \y/\n in {7/1,6/2,0/3,-1/4}{%
    \node(G\n)[draw, circle] at(3,\y) {$G_{\n}$};
  }
  \foreach \k in {1,2,3,4,5}{%
    \foreach \g in {1,2,3,4}{%
      \draw(K\k)--(G\g);
    }
  }
  % override with some edges where guest hands biscuit to kids
  \draw[color=red](G1)--(K3) (G2)--(K1) (G3)--(K4) (G3)--(K5) (G4)--(K2);
\end{tikzpicture}


\chapter{六年级}
\label{chap:pengcheng-grade6}

\section{2019年}
\label{sec:pengcheng-grade6-2019}

\begin{question}
  $\overline{\text{少年}} + \overline{\text{科技}} + \overline{\text{创新}} + \overline{\text{能力}} = 314$,其中不同的汉字表示不同的非0数字,则分数
  \begin{align*}
    \frac{\text{少} + \text{科} + \text{创} + \text{能}}{\text{年} + \text{技} + \text{新} + \text{力}}
  \end{align*}
  的值是$\blankline$。
\end{question}


\begin{question}
  把一笔奖金分给甲乙两个组,平均每人可得到600元;如果只分给甲级,平均每人可得到1000元;如果只分给乙组,平均每人可得$\blankline$元。
\end{question}


\begin{question}
  把如图所示的6个单位正方形组成的$2\times 3$矩形中,标示出两个角$\alpha$和$\beta$,则$\alpha + \beta$的角度是$\blankline$。

  \begin{center}
    \begin{tikzpicture}[scale=1]
      \coordinate(A1)at(0,2);
      \coordinate(A2)at(0,0);
      \coordinate(O)at(1,0);
      \coordinate(B2)at(2,3);
      \coordinate(B1)at(2,0);
      \draw(0,0)rectangle(2,3) (0,1)--(2,1) (0,2)--(2,2) (1,0)--(1,3) (0,2)--(1,0)--(2,3);
      \draw pic["$\alpha$",<->,angle eccentricity=1.6,draw=orange]{angle=A1--O--A2};
      \draw pic["$\beta$",<->,angle eccentricity=1.6,draw=orange]{angle=B1--O--B2};
    \end{tikzpicture}
  \end{center}
\end{question}


\begin{question}
  从十个数$1,2,3,4,5,6,7,8,9,10$中去掉一个数,使得剩下的九个数可分为两组,且这两组数的乘积相等,则去掉的数是$\blankline$。
\end{question}


\begin{question}
  五个不同的自然数,两两之和依次等于$3,4,5,6,7,8,11,12,13,15$这10个值,则这五个自然数的平均数是$\blankline$。
\end{question}


\begin{question}
  梯形$ABCD$中,$AD\parallel BC$,$\angle ABC=90^\circ$。对角线$AC$和$BD$相交于点$O$,且$AB=6$厘米,$BO=3DO$,三角形$AOD$的面积为3平方厘米。则梯形$ABCD$的周长为$\blankline$厘米。

  \begin{center}
    \begin{tikzpicture}[scale=0.4]
      \coordinate[label=below left:$B$](B)at(0,0);
      \coordinate[label=above left:$A$](A)at(0,6);
      \coordinate[label=above right:$D$](D)at(4,6);
      \coordinate[label=below right:$C$](C)at(12,0);
      \draw(A)--node[midway,left]{6}(B)--(C)--(D)--(A)--(C) (B)--(D);
      \tkzInterLL(A,C)(B,D)\tkzGetPoint{O}\tkzLabelPoints[below](O)
      \fill[color=blue!20](A)--(O)--(D)--cycle;
      \node at($1/3*(A)+1/3*(O)+1/3*(D)$){3};
      \tkzMarkRightAngle[color=blue](A,B,C)
    \end{tikzpicture}
  \end{center}
\end{question}


\begin{question}
  从28个自然数$1,2,\cdots,28$中任取$n$个数,使得其中必有2个数的差是7,则$n$的最小值是$\blankline$。
\end{question}



\begin{question}
  核研所每天按时出车沿规定路线定时到达$A$站,接上同时到达$A$站的专家准时到达核研所。有一天,该专家提前55分钟到达$A$站,因接他的车还没来,他就步行向核研所走去。在途中遇到接他的汽车,立即乘上车,这样比通常提前10分钟到达核研所。则汽车速度是专家步行速度的$\blankline$倍。
\end{question}


\begin{question}
  一个长方体的棱长都是质数,其中相邻的两个表面长方形的面积之和是209平方厘米,则这个长方体的体积是$\blankline$立方厘米。
\end{question}


\begin{question}
  设$a,b,c,d$是1--9之间的四个不同的数字,用这四个数字(不能重复)可以组成很多不同的四位数,小明把所有可能组成的四位数加起来,但他不小心把其中一个四位数多加了一遍,结果为128313,那么正确的结果应该是$\blankline$。
\end{question}


\begin{question}
  计算
  \begin{align*}
    \left( 4\frac79 - 0.8 + 3\frac29 \right) \times
    \left[ \left( 7\frac25 + 2.6 \right) \times 1.25 \right]
  \end{align*}
\end{question}


\begin{question}
  正方形$ABCD$的面积等于8平方厘米,它的对角线交点为$O$,分别以$A,B,C,D$为圆心画过$O$点的四条圆弧,如图所示,图中四个花瓣形(阴影部分)的总面积是多少平方厘米?(圆周率$=3.14$)

  \begin{center}
    \begin{tikzpicture}[scale=1.5]
      \coordinate[label=above left:$A$](A)at(0,2);
      \coordinate[label=below left:$B$](B)at(0,0);
      \coordinate[label=below right:$C$](C)at(2,0);
      \coordinate[label=above right:$D$](D)at(2,2);
      \coordinate(O)at(1,1);
      \coordinate(A1)at($(A)!1!45:(O)$);
      \coordinate(A2)at($(A)!1!-45:(O)$);
      \coordinate(B1)at($(B)!1!45:(O)$);
      \coordinate(B2)at($(B)!1!-45:(O)$);
      \coordinate(C1)at($(C)!1!45:(O)$);
      \coordinate(C2)at($(C)!1!-45:(O)$);
      \coordinate(D1)at($(D)!1!45:(O)$);
      \coordinate(D2)at($(D)!1!-45:(O)$);

      \calcLength(O,A){r}
      %\centerarc(A)(270:360:\r pt);
      % \draw[fill=red!20, non zero rule] (A1)arc(0:-90:\r pt) --(B1) arc(90:0:\r pt)
      %         (C1) arc(180:90:\r pt) -- (D1) arc(270:180:\r pt);
      \begin{scope}
        \clip (B1)arc(90:0:\r pt) -- (B) -- cycle;
        \fill[color=red!20] (A1)arc(0:-90:\r pt) -- (A)-- cycle;
      \end{scope}
      \begin{scope}
        \clip (D1)arc(270:180:\r pt) -- (D) -- cycle;
        \fill[color=red!20] (A1)arc(0:-90:\r pt) -- (A)-- cycle;
      \end{scope}
      \begin{scope}
        \clip (B1)arc(90:0:\r pt) -- (B) -- cycle;
        \fill[color=red!20] (C1)arc(180:90:\r pt) -- (C)-- cycle;
      \end{scope}
      \begin{scope}
        \clip (D1)arc(270:180:\r pt) -- (D) -- cycle;
        \fill[color=red!20] (C1)arc(180:90:\r pt) -- (C)-- cycle;
      \end{scope}

      \draw(A)--(B)--(C)--(D)--(A)--(C) (B)--(D)
          (A1)arc(0:-90:\r pt)
          (B1)arc(90:0:\r pt)
          (C1)arc(180:90:\r pt)
          (D1)arc(270:180:\r pt);
      \tkzLabelPoints[right](O)
    \end{tikzpicture}
  \end{center}
\end{question}
\begin{proof}[提示]
  可以用方程组。

  也可以用几何变换。其中每个花瓣都可以变换成一个弓形,四个花瓣加起来可以变换成一个圆挖去一个跟$ABCD$一样大的正方形。
  \begin{center}
    \begin{tikzpicture}[scale=1.5]
      \coordinate[label=above left:$A$](A)at(0,2);
      \coordinate[label=below left:$B$](B)at(0,0);
      \coordinate[label=below right:$C$](C)at(2,0);
      \coordinate[label=above right:$D$](D)at(2,2);
      \coordinate(O)at(1,1);
      \coordinate(A1)at($(A)!1!45:(O)$);
      \coordinate(A2)at($(A)!1!-45:(O)$);
      \coordinate(B1)at($(B)!1!45:(O)$);
      \coordinate(B2)at($(B)!1!-45:(O)$);
      \coordinate(C1)at($(C)!1!45:(O)$);
      \coordinate(C2)at($(C)!1!-45:(O)$);
      \coordinate(D1)at($(D)!1!45:(O)$);
      \coordinate(D2)at($(D)!1!-45:(O)$);

      \calcLength(O,A){r}

      \begin{scope}
        \clip (B1)arc(90:0:\r pt) -- (B) -- cycle;
        \fill[color=red!20] (A1)arc(0:-90:\r pt) -- (A)-- cycle;
      \end{scope}

      \filldraw[pattern=north west lines](O)arc(45:135:\r pt)--(O);

      \coordinate(O1)at($(B)!1!-90:(O)$);
      \coordinate(O2)at($(B)!1!180:(O)$);
      \coordinate(O3)at($(B)!1!90:(O)$);

      
      \draw(A)--(B)--(C)--(D)--(A)--(C) (B)--(D)
          (O)arc(-45:-90:\r pt);
          %(O)arc(45:135:\r pt);
      \tkzLabelPoints[right](O)

      \begin{scope}[shift={(5,0)}]
        \coordinate[label=above left:$A$](A)at(0,2);
        \coordinate(B)at(0,0);
        \coordinate[label=below right:$C$](C)at(2,0);
        \coordinate[label=above right:$D$](D)at(2,2);
        \coordinate(O)at(1,1);
        \coordinate(A1)at($(A)!1!45:(O)$);
        \coordinate(A2)at($(A)!1!-45:(O)$);
        \coordinate(B1)at($(B)!1!45:(O)$);
        \coordinate(B2)at($(B)!1!-45:(O)$);
        \coordinate(C1)at($(C)!1!45:(O)$);
        \coordinate(C2)at($(C)!1!-45:(O)$);
        \coordinate(D1)at($(D)!1!45:(O)$);
        \coordinate(D2)at($(D)!1!-45:(O)$);
        \coordinate(O1)at($(B)!1!-90:(O)$);
        \coordinate(O2)at($(B)!1!180:(O)$);
        \coordinate(O3)at($(B)!1!90:(O)$);

        \filldraw[pattern=north west lines](B) circle(\r pt);
        \draw($(B)!1!135:(O)$)--(B)--($(B)!1!-135:(O)$);
        \draw[fill=white](O)--(O1)--(O2)--(O3)--cycle;
        \draw(A)--(B)node[below right]{$B$}--(C)--(D)--(A)--(C) (B)--(D);
      \end{scope}

    \end{tikzpicture}
  \end{center}
\end{proof}


\begin{question}
  如图是一个边长为100米的正方形跑道$ABCE$,甲、乙两人同时分别从$A,C$两点出发,沿着跑道顺时针方向出发,甲的速度为每秒7米,乙的速度为每秒5米,他们每到转弯处都要停留5秒钟,请问,当甲第一次追上乙时,要用多少时间?

  \begin{center}
    \begin{tikzpicture}
      \draw(0,0)node[below left]{$B$}--(2,0)node[below right]{$C$}
         --(2,2)node[above right]{$D$}--(0,2)node[above left]{$A$}--cycle;
      \draw[->](0.5,2.3)--(1.5,2.3);
      \draw[->](1.5,-0.3)--(0.5,-0.3);
    \end{tikzpicture}
  \end{center}
\end{question}


\begin{question}
  四只容量相同且有刻度的玻璃杯,其中三只分别装满三种不同的果汁,另外一只为空杯。你可以利用这只空杯,怎样操作得到三杯成分相同的混合果汁?如果增加一个同容量,而且装满与以上三种不相同的饮料的玻璃杯,你又怎样操作得到四杯成分相同的混合果汁?
\end{question}



\begin{question}
  阅读以下材料:如图所示,长方形$ABCD$中,$AB=a$,$AD=b$,分割成四个小正方形,其中$AG=a-1$,$AE=b-1$。由于$S_{ABCD}=S_{AEPG}+S_{EFCD}+S_{BCHG}-S_{PFCH}$,即
  \begin{align*}
    ab=(a-1)(b-1)+a+b-1,\quad \therefore\ (a-1)(b-1)=ab-a-b+1
  \end{align*}

  \begin{center}
    \begin{tikzpicture}[scale=0.5]
      \coordinate[label=below left:$D$](D)at(0,0);
      \coordinate[label=below:$H$](H)at(5,0);
      \coordinate[label=below right:$C$](C)at(6,0);
      \coordinate[label=left:$E$](E)at(0,1);
      \coordinate[label=above left:$P$](P)at(5,1);
      \coordinate[label=right:$F$](F)at(6,1);
      \coordinate[label=above left:$A$](A)at(0,4);
      \coordinate[label=above:$G$](G)at(5,4);
      \coordinate[label=above right:$B$](B)at(6,4);
      \draw(A)--(B)--(C)--(D)--(A) (E)--(F) (G)--(H);
    \end{tikzpicture}
  \end{center}
  
  运用上述公式,解决以下问题:

  一个数,其所有位数上的非零数字之积恰好等于这些数字之和,这样的数称为“鹏程数”,例如,8000,11125都是五位数的“鹏程数”。

  特别地,我们把各个数字均不为零的“鹏程数”叫做“真鹏程数”。
  \begin{enumerate}
  \item 求出所有三位“鹏程数”之和。
  \item 求出四位“真鹏程数”的四个数字。
  \item 写出一个2019位的“鹏程数”,其中包含数字2,0,1,9。
  \end{enumerate}
\end{question}



\section{2020年}
\label{chap:pengcheng-2020}

\begin{question}
  若三个质数$x$,$y$,$z$使得等式$x\times y\times z + 7 = 2020$成立,则$x+y+z=\blankline$.
\end{question}
\begin{proof}[提示]
对$x\times y\times z=2013$做质因式分解。
\end{proof}


\begin{question}
  如图所示,由$O$点引出的6条射线形成的角满足:
  \begin{align*}
    \angle AOB=\angle BOC=\angle COD=\angle DOE=\angle EOF=18^\circ
  \end{align*}
  直线$l$分别交这条射线依次于点$M,G,H,K,L,N$,则图中至少有锐角$\blankline$个。

  \begin{center}
    \begin{tikzpicture}[scale=1.0]
      \coordinate[label=below:$O$](O) at (0,0);
      \coordinate[label=above:$A$](A) at (-0.5,3);
      \coordinate[label=above:$B$](B) at ($(O)!1!-18:(A)$); % rotate A around O by 18 degrees (negative for clockwise)
      \coordinate[label=above:$C$](C) at ($(O)!1!-18:(B)$);
      \coordinate[label=above:$D$](D) at ($(O)!1!-18:(C)$);
      \coordinate[label=above:$E$](E) at ($(O)!1!-18:(D)$);
      \coordinate[label=above:$F$](F) at ($(O)!1!-18:(E)$);

      \coordinate(L1) at (-1,2.8);
      \coordinate(L2) at (3,-0.1);

      \draw(O)--(A) (O)--(B) (O)--(C) (O)--(D) (O)--(E) (O)--(F);

      \draw(L1)--(L2)node[below right]{$l$};

      \tkzInterLL(O,A)(L1,L2)\tkzGetPoint{M}
      \tkzInterLL(O,B)(L1,L2)\tkzGetPoint{G}
      \tkzInterLL(O,C)(L1,L2)\tkzGetPoint{H}
      \tkzInterLL(O,D)(L1,L2)\tkzGetPoint{K}
      \tkzInterLL(O,E)(L1,L2)\tkzGetPoint{L}
      \tkzInterLL(O,F)(L1,L2)\tkzGetPoint{N}
      \tkzLabelPoints[below left](M,G)
      \tkzLabelPoints[below](K,N)
      \tkzLabelPoints[above](H,L)

      \tkzDrawPoints(O,M,G,H,K,L,N)
    \end{tikzpicture}
  \end{center}
\end{question}
\begin{proof}[提示]
  分类讨论。
  \begin{itemize}
  \item 点$O$的锐角有4种:$18^\circ$的,$36^\circ$的,$54^\circ$的,$72^\circ$的。
  \item 直线$l$上的角。每条射线与$l$的夹角,当垂直时没有锐角,当不垂直时有2个锐角。$l$最多只能与其中一条射线垂直,从而最少有5条射线与$l$不垂直,即这些射线与$l$的夹角最少有$5\times2=10$个锐角的夹角。
  \end{itemize}
\end{proof}

\begin{question}
  四个两位数和乘积$\overline{\text{众志}}\times\overline{\text{成城}}\times\overline{\text{防控}}\times\overline{\text{疫情}}$中,相同的汉字表示相同的数字,不同的汉字表示不同的数字,这个乘积数值的结尾最多可连续有$\blankline$个零?
\end{question}

\begin{question}
  边长分别为 8cm 跟 6cm 的两个正方形$ABCD$和$BEFG$如图并排放在了一起,连接$AF$交$BD$于$P$,则四边形$BPEF$的面积是$\blankline$平方厘米。

  \begin{center}
    \begin{tikzpicture}[scale=0.5]
      \coordinate[label=below left :$A$](A) at(0,0);
      \coordinate[label=below      :$B$](B) at(8,0);
      \coordinate[label=above      :$C$](C) at(8,8);
      \coordinate[label=above left :$D$](D) at(0,8);
      \coordinate[label=below right:$E$](E) at(14,0);
      \coordinate[label=above right:$F$](F) at(14,6);
      \coordinate[label=above right:$G$](G) at(8,6);
      \tkzInterLL(A,F)(B,D)\tkzGetPoint{P}\tkzLabelPoints[above](P)
      \fill[color=blue!20](P)--(B)--(E)--(F)--cycle;
      \draw(A)--node[below]{8cm}(B)--(C)--(D)--(A)--(F)--(G) (D)--(B)--node[below]{6cm}(E)--(F);
    \end{tikzpicture}
  \end{center}
\end{question}


\begin{question}
  由$0,1,2,3,4,5,6,7,8,9$这十个数字,每个数字只用一次排出可能的十位数,将这十位数从小到大自左向右排成一行,则从左向右的第6个数是$\blankline$。
\end{question}


\begin{question}
  一杯盐水,第一次加入一定量的水后,盐水的含盐百分比变为15\%;第二次又加入同样多的水,含盐百分比变为12\%;第三次再加入同样多的水,盐水的含盐百分比将变为$\blankline$\%。
\end{question}


\begin{question}
  1--1000000的自然数中,所有的7的倍数的数之和等于$\blankline$?
\end{question}
\begin{proof}[提示]
  本质是 $7,14,21,\cdots$ 这样一个等差数列求和。
\end{proof}


\begin{question}
  今年,祖父的的年龄是小学生明明年龄的6倍,几年过去了,祖父的年龄将是明明年龄的5倍,又过去了几年以后,祖父的年龄是明明年龄的4倍,祖父今年是$\blankline$岁?
\end{question}
\begin{proof}[提示]
  寻找不变量。随着时间的变化,祖父和明明的年龄差是永远不变的。
  \begin{itemize}
  \item 6倍的时候,年龄差是明明年龄的5倍。
  \item 5倍的时候,年龄差是明明年龄的4倍。
  \item 4倍的时候,年龄差是明明年龄的3倍。
  \end{itemize}
  也就是年龄差是5,4,3的倍数,年龄差是$60,120,180,\cdots$。一般来说,在这个数列里,60是较为合理的爷孙年龄差。故今年,年龄差60是明明年龄的5倍,明明今年是$60\div5=12$岁,祖父今年是$12\times6=72$岁。
\end{proof}


\begin{question}
  如图是一个容积为30立方厘米的正方体铁皮盒被剪去一个“角”后的平面展开图(图中相同字母表示长度相等的线段)各侧面剪掉的三个阴影三角形的面积分别是2平方厘米,3平方厘米,3平方厘米,则该铁皮盒最多可以装$\blankline$立方厘米的水。

  \begin{center}
    \begin{tikzpicture}[scale=2.0]
      \draw(0,0)--++(1,0)--++(0,1)--++(2,0)--++(0,1)--++(-2,0)--++(0,1)--++(-1,0)--++(0,-1)--++(-1,0)--++(0,-1)--++(1,0)--cycle;
      \draw[dashed](0,1)rectangle(1,2) (2,1)--++(0,1);
      \coordinate(A0)at(-1,1);
      \coordinate(A1)at(-1,1.4);
      \coordinate(A2)at(-0.4,1);
      \coordinate(B0)at(0,0);
      \coordinate(B1)at(0,0.6);
      \coordinate(B2)at(0.4,0);
      \coordinate(C0)at(3,1);
      \coordinate(C1)at(2.6,1);
      \coordinate(C2)at(3,1.4);
      \foreach \a/\b/\c in{A0/A1/A2,B0/B1/B2,C0/C1/C2}{
        \draw(\b)--(\c);
        \fill[color=blue!20](\a)--(\b)--(\c)--cycle;
      }
      % Av --> A vertical, Ah --> A horizontal
      \coordinate[label=left:$a$](FAv)at($0.5*(A0)+0.5*(A1)$);
      \coordinate[label=below:$b$](FAh)at($0.5*(A0)+0.5*(A2)$);
      \coordinate[label=left:$b$](FBv)at($0.5*(B0)+0.5*(B1)$);
      \coordinate[label=below:$c$](FBh)at($0.5*(B0)+0.5*(B2)$);
      \coordinate[label=below:$c$](FCv)at($0.5*(C0)+0.5*(C1)$);
      \coordinate[label=right:$a$](FCh)at($0.5*(C0)+0.5*(C2)$);
    \end{tikzpicture}
  \end{center}
\end{question}
\begin{proof}[提示]
  用方程组很方便。三个阴影部分面积分别是$2,3,3$,不妨设
  \begin{align*}
    \begin{cases}
      \frac12\times ab=2\\
      \frac12\times bc=3\\
      \frac12\times ca=3
    \end{cases}
  \end{align*}
  从而$a\times b\times c=12$且$a,b,c$是$2,2,3$的组合,对应的被剪掉部分的体积是
  \begin{align*}
    \frac13 \times \frac12 \times abc=2
  \end{align*}
\end{proof}


\begin{question}
  在2名六年级选手与至少10名五年级选手一起进行比赛象棋,每两个人彼此都恰好比赛一场,每场比赛胜者得2分,负者得0分,若和局则各得到1分。比赛结束后,已知2名六年级的选手得分之和是20分,且每名五年级选手都得到了$N$分,则$N=\blankline$。
\end{question}


\begin{question}
  \begin{align*}
    \left( 5\frac59 - 0.8 + 2\frac49 \right) \times
    \left( 7.6\div\frac45 + 2\frac25 \times 1.25 \right)
  \end{align*}
\end{question}


\begin{question}
  一个六位数$\overline{abcdef}$满足$\overline{defabc}=6\times\overline{abcdef}$,试确定这个六位数$\overline{abcdef}$的值,并且写出求解过程。
\end{question}
\begin{proof}[提示]
  如果想到了$1/7$,那就容易联想。

  想不到,就只能中规中矩。考察$\overline{abcdef}$和$\overline{defabc}$,其实本质只有两部分$\overline{abc}$和$\overline{def}$。从而
  \begin{align*}
    &6\times\overline{abcdef}=\overline{defabc}\\
    \iff & 6\times\left(1000\times\overline{abc} + \overline{def}\right)
           = 1000\times\overline{def} + \overline{abc}\\
    \iff & 857\times\overline{abc}= 142\times\overline{def}
  \end{align*}
  由857与142互质可知三位数$\overline{abc}$和$\overline{def}$分别是142与857。怎么判断两个整数互质?可以用欧拉辗转相除法求其公约数。
\end{proof}


\begin{theorem}[欧拉辗转相除法]
  记$\gcd(a,b)$表示两个整数$a,b$的最大公约数,若$r=a\pmod b$,则有$\gcd(a,b)=\gcd(r,b)$。
\end{theorem}
\begin{example}
求857与142的最大公约数。把大的记为$a$,小的记为$b$,这样余数$r$与$b$的组合才会比$a,b$的组合小,迭代下去总有结束的时候。
\begin{align*}
  847 \pmod{142} = 137 & \implies \,\, \gcd(847,142) = \gcd(137,142) = \gcd(142,137) \\
  142 \pmod{137}= 5   & \implies \,\, \gcd(142,137) = \gcd(5,137)
\end{align*}
由这里已经容易知道5和137是互质的,从而$\gcd(5,137)=1$。若还猜不出来,可以继续下去
\begin{align*}
  137 \pmod 5 = 2 & \implies \,\, \gcd(5,137) = \gcd(2,5) = \gcd(5,2) \\
  5 \pmod 2 = 1   & \implies \,\, \gcd(5,2) = \gcd(1,2)
\end{align*}
这下总是能看出来$\gcd(1,2)=1$了。
\end{example}


\begin{question}
  如图所示,$ABCD$是正方形,$PCD$是面积为1的正三角形,线段$AP$交$BD, CD$分别于$E$跟$L$,$BP$交$CD$于点$K$,取$AB$中点$M$,连接$MK,ML$。
  \begin{enumerate}
  \item 证明:$BE=PE$。
  \item 求四边形$PKML$的面积。
  \end{enumerate}

  \begin{center}
    \begin{tikzpicture}[scale=4.0]
      \coordinate[label=below left :$A$](A)at(0,0);
      \coordinate[label=above left :$B$](B)at(0,1);
      \coordinate[label=above right:$C$](C)at(1,1);
      \coordinate[label=below right:$D$](D)at(1,0);
      \coordinate[label=      left :$M$](M)at($0.5*(A)+0.5*(B)$);
      \coordinate[label=right      :$P$](P)at($(C)!1!60:(D)$);
      % K, L 不是一半
      %\coordinate[label=above right:$K$](K)at($0.75*(C)+0.25*(D)$);
      %\coordinate[label=below right:$L$](L)at($0.25*(C)+0.75*(D)$);
      \tkzInterLL(B,P)(C,D)\tkzGetPoint{K}\tkzLabelPoints[above right](K)
      \tkzInterLL(A,P)(C,D)\tkzGetPoint{L}\tkzLabelPoints[below right](L)
      \tkzInterLL(B,D)(A,P)\tkzGetPoint{E}\tkzLabelPoints[below](E)
      \draw(A)--(B)--(C)--(D)--(A)--(P)--(B) (C)--(P)--(D) (B)--(D) (K)--(M)--(L);
    \end{tikzpicture}
  \end{center}
\end{question}


\begin{question}
  如果一个三角形的三条边长是彼此不等的三个质数,这样的三角形叫做“鹏程三角形”。
  \begin{enumerate}
  \item 试举一个鹏程三角形的实例。
  \item 证明:不存在周长为2020的鹏程三角形。
  \item 证明:鹏程三角形一定不是直角三角形。
  \end{enumerate}
\end{question}


\begin{question}
  在一个平地上站着$n$个人,对每个人来说,他到其他的人的距离均不相同,当有火灾信号发出的时候,每人都用水枪击中距离他最近的人。
  \begin{enumerate}
  \item 当$n=2020$时,请你举例说明,可能每个人都是湿的。
  \item 当$n=2021$时,证明至少有一个人身上是干的。
  \end{enumerate}
\end{question}



\section{2021六年级预赛}
\label{sec:pengcheng-grade6-2021-pre}

\begin{question}
  计算:
  \begin{align*}
    \frac{ \frac7{18} \times 4\frac12 + \frac16 }
    { 13\frac12 - \frac34 \div \frac5{16}}
    \times 18 = (\quad \quad)
  \end{align*}

  \testxxfive{21}{22}{23}{$2\frac59$}{以上都不对}
\end{question}


\begin{question}
  下面是由六个相同的正方形重叠起来构成的图形,连接点正好是正方形边的中点,正方形边长是$a$,则该图形的周长是$(\quad \quad)$。

  \begin{center}
    \begin{tikzpicture}[scale=0.5]
      \draw(0,0)rectangle(2,2);
      \begin{scope}[shift={(-1,-1)}]
        \draw[fill=white](0,0)rectangle(2,2);
      \end{scope}
      \begin{scope}[shift={(-2,-2)}]
        \draw[fill=white](0,0)rectangle(2,2);
      \end{scope}
      \begin{scope}[shift={(-3,-3)}]
        \draw[fill=white](0,0)rectangle(2,2);
      \end{scope}
      \begin{scope}[shift={(-4,-4)}]
        \draw[fill=white](0,0)rectangle(2,2);
      \end{scope}
      \begin{scope}[shift={(-5,-5)}]
        \draw[fill=white](0,0)rectangle(2,2);
        \coordinate[label=below:$a$](A)at(1,0);
      \end{scope}
    \end{tikzpicture}
  \end{center}

  \testxxfive{$24a$}{$14a$}{$12a$}{$18a$}{以上都不对}
\end{question}


\begin{question}
  由3个不同的自然数组成一等式:
  \begin{align*}
    \Box + \bigtriangleup + \bigcirc = \Box \times \bigtriangleup - \bigcirc
  \end{align*}
  这三个数中最多有$(\quad\quad)$ 个奇数。

  \testxxfive{1}{2}{3}{0}{无法确定}
\end{question}


\section{2022年}
\label{sec:pengcheng-2022-grade-6}

\begin{question}
  算式 $\dfrac{0.25\times2+\frac14}{3.2-2.95} + \dfrac{4\times0.9}{2.3-1\frac25}=(\quad\quad).$
  \testxxfive{0}{1}{2}{3}{7}
\end{question}

\begin{question}
  $A,B,C,D$四位小朋友分成两组做游戏,每组两个人,问$A,B$分在同一组的可能性是
  \testxxfive{$\dfrac16$}{$\dfrac14$}{$\dfrac12$}{$\dfrac13$}{以上都不对}
\end{question}

\begin{question}
  图中的正方形$ABCD$中,$E$为$AB$边的中点,$DE$把正方形分成了两部分,已知这两部分的周长相差4厘米,则正方形的面积为$(\quad\quad)$平方厘米。

  \begin{center}
    \begin{tikzpicture}
      \coordinate[label=above left:$A$](A) at(0,2);
      \coordinate[label=below left:$B$](B) at(0,0);
      \coordinate[label=below right:$C$](C) at(2,0);
      \coordinate[label=above right:$D$](D) at(2,2);
      \coordinate[label=left:$E$](E) at(0,1);
      \draw(D)--(A)--(B)--(C)--(D)--(E);
    \end{tikzpicture}
  \end{center}

  \testxxfive{9}{4}{1}{25}{16}  
\end{question}

\begin{question}
  6位中国象棋选手进行比赛,每两人之间比赛一局。如果平局,参赛选手各得1分;否则赢者得3分,输者得0分。最后六位选手的得分之和为39分,则平了$(\quad\quad)$局。
  \testxxfive{2}{3}{4}{5}{6}
\end{question}

\begin{question}
  若三角形的三个内角$\angle A$,$\angle B$,$\angle C$满足条件:$\angle A+2\angle B=\angle C$,则这个三角形可能是$(\quad\quad)$三角形。
  \testxxfive{锐角}{直角}{钝角}{等腰}{以上都不对}
\end{question}

\begin{question}
  两支粗细、长短都不同的蜡烛,长的可以点4小时,短的可以点6小时。将它们同时点燃,两小时后,两支蜡烛所余下的部分长度正好相等。那么,原来短蜡烛的长度是长蜡烛的$(\quad\quad)$。
  \testxxfive{$\dfrac45$}{$\dfrac35$}{$\dfrac34$}{$\dfrac12$}{以上都不对}
\end{question}


\begin{question}
  如图,$AEBO$是四分之一圆。$CEDO$是正方形,面积是16平方厘米,则阴影部分面积是$(\quad\quad)$平方厘米。(取$\pi=3.14$)

  \begin{center}
    \begin{tikzpicture}[scale=2.0]
      \coordinate[label=below left:$O$](O)at(0,0);
      \coordinate[label=below right:$B$](B)at(1,0);
      \coordinate[label=above left:$A$](A)at(0,1);
      \coordinate[label=above right:$E$](E)at($(O)!1!45:(B)$);
      \coordinate[label=left:$C$](C)at($(O)!(E)!(A)$); % project of E to OA
      \coordinate[label=below:$D$](D)at($(O)!(E)!(B)$); % project of E to OB
      \filldraw[fill=blue!20](O)--(B)arc(0:90:1)--(O);
      \filldraw[fill=white](O)--(D)--(E)--(C)--(O);
    \end{tikzpicture}
  \end{center}

  \testxxfive{4.12}{9.12}{10.12}{5.12}{11.12}
\end{question}

\section{Using mind maps to find the way out}
\label{sec:mind-map}

\subsection{Forward mind map}
\label{sec:mind-map-forward}

\subsection{Backward mind map}
\label{sec:mind-map-backward}





\end{document}

