
\chapter{单调函数}
\label{chap:monotonic-functions}

\begin{example}
  任意非负数$a,b,c$满足$a\le b+c$,则
  \begin{align*}
    \frac{a}{1+a}\le \frac{b}{1+b} + \frac{c}{1+c}
  \end{align*}
  并说明等号成立的充要条件。
\end{example}
\begin{proof}
  容易知道$f(x)=\dfrac{x}{1+x}$在$x\ge0$时是严格单调递增函数,从而
  \begin{align*}
    \frac{a}{1+a}&\le\frac{b+c}{1+b+c}&\text{等号成立}\iff a=b+c\\
    &=\frac{b}{1+b+c} + \frac{c}{1+b+c}\\
    &\le\frac{b}{1+b} + \frac{c}{1+c}&\text{等号成立}\iff b=c=0
  \end{align*}
  从而原不等式得证,且当且仅当$a=b=c=0$时等号成立。
\end{proof}

\begin{example}
  比较两数$\dfrac{2018}{2019}$和$\dfrac{2019}{2020}$的大小。
\end{example}
\begin{proof}[解]
  这个简单的题目有很多方法,此处利用单调函数的方法。考虑函数
  \begin{align*}
    f(x)\equiv \frac{x}{x+1}
  \end{align*}
  当$x\ne0$时,将$f(x)$变换为如下形式
  \begin{align*}
    f(x) = \frac1{1+\frac1x}
  \end{align*}
  由这种形式可以很容易看出$f(x)$在$x>0$时是严格单调递增的,从而$f(2018)<f(2019)$。
\end{proof}


\begin{example}[清华自主招生题]
  已知$f(x)=x^3-6$,求方程$f(f(x))=x$的解。
\end{example}
\begin{proof}[提示]
  利用$f(x)$的严格单调性,可以证明$f(f(x))=x$的解等价于$f(x)=x$的解。

  事实上,若$x=x_0$是$f(x)=x$的解,则
  \begin{align*}
    f(x_0) = x_0
  \end{align*}
  从而$f(f(x_0))=f(x_0)=x_0$,即$x=x_0$是$f(f(x))=x$的解。

  反之,若$x=x_0$不是$f(x)=x$的解,即
  \begin{align*}
    f(x_0)\ne x_0
  \end{align*}
  不妨记$f(x_0) < x_0$($f(x_0)>x_0$的情况类似),则由$f(x)$的严格递增,有
  \begin{align*}
    f(f(x_0)) < f(x_0) < x_0
  \end{align*}
  从而$x=x_0$不是$f(f(x))=x$的解。

  综合可知
  \begin{align*}
        & f(f(x))=x\\
    \iff& f(x) = x\\
    \iff& x^3 - 6 = x\\
    \iff& (x-2)(x^2+2x+3) = 0
  \end{align*}
  由最后一个方程可得原问题的解。实数域只有1个解,复数域有3个解。
\end{proof}
