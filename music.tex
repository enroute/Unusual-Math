
\chapter{数学与音乐}
\label{chap:math-and-music}

\section{调和数列}
\label{sec:harmonic-series}

按振动力学的理论,弦乐器的声音声调高低与弦的长度有关。古希腊的毕达哥拉斯学派发现,绷得一样紧的几根同样材质的弦,如果弦的长度比是整数比,那么发出的声音就比较和谐。更进一步,如果一根弦是另一弦的长度的2倍,那么两弦发出的音正好相差八度;如果三根弦的长度比是$15:12:10$,它们将分别发出十分调和的乐声do、mi、so。

后来数学家们研究了比例$15:12:10$,发现这三个数的倒数成等差数列:
\begin{align*}
  \frac1{15}-\frac1{12} = \frac1{12}-\frac1{10} = -\frac1{60}
\end{align*}
又由于当弦的长度具有这样的比例时,其发出的声音很调和,所以数学上就把倒数成等差数列的数列称为调和数列。比如
\begin{align*}
  1,\,\frac12,\, \frac13,\, \frac14,\, \frac15,\, \frac16,\, \frac17,\, \cdots
\end{align*}

