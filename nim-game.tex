
\chapter{尼姆游戏}
\label{chap:nim-game}

\section{一元尼姆游戏}
\label{sec:one-pile-nim-game}

\begin{example}
  两人用一堆石头玩游戏,规则是轮流拿,每次只能拿1或2或3块石头,谁拿走最后一块石头的人将是胜利者。你的最佳策略是什么?
\end{example}
\begin{proof}[提示]
  也可以用硬币或纸牌等替换石头,尼姆游戏特别适合旅途中玩。

  最佳策略的关键点是找到必赢和必输点。考虑剩下$n$块石头时,下一次要走的人是必赢还是必输,结果如表~\ref{tab:win-lose-position-of-one-pile-nim-game}所示。
  \begin{table}[htbp]
    \begin{minipage}{\textwidth}\centering
    \begin{tabular}{c|ccccccccccc}
      \toprule
      $n$ & 1 & 2 & 3 & \textcircled{4} & 5 & 6 & 7 & \textcircled{8} & 9 & 10 & $\cdots$ \\\hline
          & W & W & W & \textcircled{L} & W & W & W & \textcircled{L} & W & W  & $\cdots$ \\
      \bottomrule
    \end{tabular}
    \end{minipage}
    \caption{输赢点,W表示必赢点(Win),L表示必输点(Lose)}
    \label{tab:win-lose-position-of-one-pile-nim-game}
  \end{table}

  其中$1,2,3$是必赢点,是因为玩家可以直接拿完全部剩余的石头获取胜利。$4$是必输点是因为不管如何走,走完后都会让对手处于必赢点$1,2,3$中的某一个。$5,6,7$是必赢点,是因为玩家可以拿走适当的石头,让对手面临必输点$4$。依次类推,可知$4n$是必输点,其余为必赢点,可用数学归纳法证明。

  最佳策略就是设法让对手面临必输点,即设法让走完后剩余$4n$块石头。
\end{proof}

\begin{example}
  上题中,如果胜利规则改为谁拿到最后一块石头谁就输,那么最佳策略又该是如何呢?
\end{example}
\begin{proof}[提示]
  找输赢点,如下表所示。
  \begin{center}
    \begin{tabular}{c|ccccccccccc}
      \toprule
      $n$ & \textcircled{1} & 2 & 3 & 4 & \textcircled{5} & 6 & 7 & 8 & \textcircled{9} & 10 & $\cdots$ \\\hline
          & \textcircled{L} & W & W & W & \textcircled{L} & W & W & W & \textcircled{L} & W  & $\cdots$ \\
      \bottomrule
    \end{tabular}
  \end{center}
  最佳策略就是设法剩余$4n+1$块石头留给对手。
\end{proof}

\begin{question}
  有100个乒乓球,两个人轮流拿球,能拿到最后一个乒乓球的人为胜利者。条件是:每次拿球者至少要拿1个,但最多不能超过5个,那么先拿的人或者后拿的人是否能保证其必赢?
\end{question}

\begin{example}[拿平方数]
  两人玩游戏,轮流从一堆硬币里拿若干个,规定每次只能拿$1$、$4$、$9$等平方数个,谁拿走最后一个谁赢。记开始时共有$n$个硬币,若两人都足够聪明,那么什么情况下先拿的人会赢?
\end{example}
\begin{proof}[提示]
  对于先手,同样考虑W点和L点。显然,当$n$是完全平方数时,这是一个W点,因为先手可以一次性全拿完。
  \begin{center}
    \begin{tabular}{c|cccccccccccccc}
      \toprule
      $n$ & 1 & 2 & 3 & 4 & 5 & 6 & 7 & 8 & 9 & 10 & 11 & 12 & 13 & $\cdots$ \\\hline
          & W &   &   & W &   &   &   &   & W &    &    &    &    & $\cdots$ \\
      \bottomrule
    \end{tabular}
  \end{center}
  考虑先手在剩余$n$个硬币时一次操作后可剩余多少个硬币给对手,若剩余的硬币数$k$都是W点,则$n$就是L点,因为无论怎么走,都只能让对手处于W点。而若有某种操作使剩余的硬币数$k$是L点,则$n$就是W点,因为在此种操作下对手要面对L点的必输局面。
  \begin{center}
    \begin{tabular}{cc|cc||cc|cc}
      \toprule
      $n$ & $k$ & $k$的W/L & $n$的W/L & $n$ & $k$ & $k$的W/L & $n$的W/L\\\hline
      2   & 1   & 全是W    & L        & 8   & 4,7 & 7是L     & W\\
      3   & 2   & 全是L    & W        & 10  & 1,6,9  & 全是W    & L\\
      5   & 1,4 & 全是W    & L        & 11  & 2,7,10 & 全是L    & W\\
      6   & 2,5 & 全是L    & W        & 12  & 3,8,11 & 全是W    & L\\
      7   & 3,6 & 全是W    & L        & 13  & 4,9,12 & 12是L    & W\\
      \bottomrule
    \end{tabular}
  \end{center}
  按上面的分析方法,可以填充下面的表格:
  \begin{center}
    \begin{tabular}{c|cccccccccccccc}
      \toprule
      $n$ & 1 & 2 & 3 & 4 & 5 & 6 & 7 & 8 & 9 & 10 & 11 & 12 & 13 & $\cdots$ \\\hline
          & W & L & W & W & L & W & L & W & W & L  & W  & L  & W  & $\cdots$ \\
      \bottomrule
    \end{tabular}
  \end{center}
  容易发现,若$k$是L点,则任意正整数$t$,有$n=t^2+k$是W点。但除了通过前序点的W/L情况判断之外,不容易找到一个通用的模式来直接根据$n$来判断是W点还是L点,所以与前面的尼姆游戏相比,这种游戏的可玩性更强些。
\end{proof}

\section{二元尼姆游戏}
\label{sec:two-piles-nim-game}

\begin{example}
  有两堆石头,规则变为每人可以从任意一堆中取走任意数量的石头,但不可不取。谁拿到最后一块石头谁获得胜利。此时的最佳策略又该如何?
\end{example}
\begin{proof}[提示]
  还是找输赢点。
  \begin{enumerate}
  \item 首先,如果两堆中有一堆已经被拿光,那么直接拿剩余一堆的所有石头可以获取胜利;
  \item 考虑下一个位置(1,1),即两堆都只剩一块石头,则不管如何,只能走到(1,0)或者(0,1)这两个W点,从而让对手获胜,即(1,1)是L点;
  \item 考虑$(n,1)$和$(1,n)$,其中$n>1$,这些位置走一步都可以到达(1,1)这个L点让对手必输,所以这些都是W点;
  \item 再考虑(2,2)这个点,它的上方和左方都是W点,从而不管如何走都会让对手处于W点,即这个位置是L点;
  \item $\cdots$
  \end{enumerate}
  \begin{center}
    \begin{tabular}{|c|c|c|c|c|c|c|}
      \hline
        & 0 & 1 & 2 & 3 & 4 & 5\\\hline
      0 &   & W & W & W & W & W\\\hline
      1 & W &   &   &   &   &  \\\hline
      2 & W &   &   &   &   &  \\\hline
      3 & W &   &   &   &   &  \\\hline
      4 & W &   &   &   &   &  \\\hline
      5 & W &   &   &   &   &  \\\hline
    \end{tabular}
    $\quad\to\quad$
    \begin{tabular}{|c|c|c|c|c|c|c|}
      \hline
        & 0 & 1 & 2 & 3 & 4 & 5\\\hline
      0 &   & W & W & W & W & W\\\hline
      1 & W & \cellcolor{blue!20}L &   &   &   &  \\\hline
      2 & W &   &   &   &   &  \\\hline
      3 & W &   &   &   &   &  \\\hline
      4 & W &   &   &   &   &  \\\hline
      5 & W &   &   &   &   &  \\\hline
    \end{tabular}
  \end{center}
  \begin{center}
    \begin{tabular}{|c|c|c|c|c|c|c|}
      \hline
        & 0 & 1 & 2 & 3 & 4 & 5\\\hline
      0 &   & W & W & W & W & W\\\hline
      1 & W & L & \cellcolor{blue!20}W & \cellcolor{blue!20}W & \cellcolor{blue!20}W & \cellcolor{blue!20}W\\\hline
      2 & W & \cellcolor{blue!20}W &   &   &   &  \\\hline
      3 & W & \cellcolor{blue!20}W &   &   &   &  \\\hline
      4 & W & \cellcolor{blue!20}W &   &   &   &  \\\hline
      5 & W & \cellcolor{blue!20}W &   &   &   &  \\\hline
    \end{tabular}
    $\quad\to\quad$
    \begin{tabular}{|c|c|c|c|c|c|c|}
      \hline
        & 0 & 1 & 2 & 3 & 4 & 5\\\hline
      0 &   & W & W & W & W & W\\\hline
      1 & W & L & W & W & W & W\\\hline
      2 & W & W & \cellcolor{blue!20}L  &   &   &  \\\hline
      3 & W & W &   &   &   &  \\\hline
      4 & W & W &   &   &   &  \\\hline
      5 & W & W &   &   &   &  \\\hline
    \end{tabular}
  \end{center}
依次填下去,就会发现对角线上的点$(n,n)$都是L点,而且除了这些点外,其余的都是W点。
\end{proof}

