
\chapter{数值方法}
\label{chap:numerical-method}

数值方法通常是指应用计算机程序求解数值问题的近似解的各种方法与工具。

\section{牛顿法}
\label{sec:Newton-method}

牛顿法是一种迭代方法。迭代方法的思想是通过一个初始点,不断地迭代得到一系列的点,而这些点序列最终会收敛到目标点上。具体来说,牛顿法是通过曲线的切线来构造下一个点,因而牛顿法通常也叫做切线法。

\begin{enumerate}
\item 从一个初始点$(x_0, 0)$开始,找到曲线上的点$(x_0, f(x_0))$;
\item 作切线,找到切线与$x$轴的交点$(x_1,0)$;
\item 从$(x_1,0)$开始重复上述步骤
\end{enumerate}

按上述步骤,可得到一系列的值$x_0, x_1, x_2, \cdots, $。在某些条件下,$x_i$会收敛到曲线与$x$轴的交点。其递推关系为
\begin{align*}
  x_{n+1}=x_0 - \frac{f(x_0)}{f'(x_0)}
\end{align*}
上述迭代的不动点(即迭代完了之后还是它本身),在$f'$不为零的情况下等价于$f(x)$的零点。这是因为
\begin{align*}
  x=x-\frac{f(x)}{f'(x)} \quad \overset{f'\ne0}{\iff}\quad f(x)=0
\end{align*}

\begin{theorem}
  若函数$f(x)$在$x=\alpha$处为零,即$f(\alpha)=0$,且$f(x)$有连续的导数,且$f'(\alpha)\ne0$。则存在$x=\alpha$的一个邻域,在此邻域内任意一点$x_0$出发按牛顿切线法得出的点序列$\{x_i\}$收敛于$\alpha$。
\end{theorem}
由此定理可知,有连续导数的函数的零点如果其导数不会零,则该零点有一个吸引域,在此吸引域内的点按牛顿切线法得到的序列总会收敛于该零点。

\begin{figure}[htbp]
  \centering
  \begin{tikzpicture}[scale=1.0]
    \begin{scope}[shift={(0,0)},decoration={
        markings,
        mark=at position 0.5 with {\arrow{>}}
      }]
      \draw[->](-1,0)--(11,0)node[right]{$x$};
      \draw[->](0,-2)--(0,9)node[above]{$y$};
      \draw[very thick,domain=-.75:9.5,smooth,variable=\x]plot({\x},{.1*(\x-1)*(\x-1)-1});
      \coordinate[label=below:$x_0$] (x0) at (9.3,0);
      \coordinate[label=above left:{$\left( x_0,f(x_0)\right)$}] (f0) at (9.3,5.889);
      \coordinate[label=below:$x_1$] (x1) at (5.7524,0);
      \coordinate (f1) at (5.7524,1.2585);
      \coordinate[label=below:$x_2$] (x2) at (4.4283,0);
      \coordinate (f2) at (4.4283,0.1753);
      \draw[help lines,postaction={decorate}](x0)--(f0);
      \draw[help lines,postaction={decorate}](f0)--(x1);
      \draw[help lines,postaction={decorate}](x1)--(f1);
      \draw[help lines,postaction={decorate}](f1)--(x2);%--(f2);
      \tkzDrawPoints(x0,f0,x1,f1,x2)
      \draw[help lines,|<->|](9.3,-1)--(5.7524,-1)node[midway,fill=white,black]{$\dfrac{f(x_0)}{f'(x_0)}$};
      \draw[help lines, <->|](10,0)--(10,5.889)node[midway,fill=white,black]{$f(x_0)$};
      \draw pic["$\theta$",<->,draw=orange,angle eccentricity=1.6,angle radius=.6cm]{angle=x0--x1--f0};
      \coordinate (f) at ($.7*(x1)+.3*(f0)$);
      \coordinate[label=above:{切线斜率$\tan\theta=f'(x_0)$}] (t) at (5,3);
      \draw[dashed,->,help lines](f)--(t);
    \end{scope}
    \begin{scope}[shift={(0,0)}]
      
    \end{scope}
  \end{tikzpicture}
  \caption{牛顿切线法}
  \label{fig:numeric-newton-method}
\end{figure}