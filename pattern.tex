
\chapter{找规律}
\label{chap:pattern}

\section{序列}
\label{sec:series-pattern}

\begin{example}\mbox{}\par
  \begin{center}
    \begin{tikzpicture}[scale=1.0]
      % \begin{scope}[shift={(0,0)}]
      %   \coordinate(A) at (0,0);
      %   \coordinate(B) at (2,0);
      %   \coordinate(C) at (60:2);
      %   \coordinate(O) at ($1/3*(A)+1/3*(B)+1/3*(C)$);
      %   \draw(A)--(B)--(C)--cycle;
      %   \node at ($(90:.5)+(O)$) {1};
      %   \node at ($(210:.5)+(O)$) {2};
      %   \node at ($(330:.5)+(O)$) {3};
      % \end{scope}

      \begin{scope}[shift={(3,0)}]
        \coordinate(A) at (0,0);
        \coordinate(B) at (2,0);
        \coordinate(C) at (60:2);
        \coordinate(O) at ($1/3*(A)+1/3*(B)+1/3*(C)$);
        \draw(A)--(B)--(C)--cycle;
        \node at ($(90:.5)+(O)$) {1};
        \node at ($(210:.5)+(O)$) {2};
        \node at ($(330:.5)+(O)$) {3};
      \end{scope}

      \begin{scope}[shift={(6,0)}]
        \coordinate(A) at (0,0);
        \coordinate(B) at (2,0);
        \coordinate(C) at (60:2);
        \coordinate(O) at ($1/3*(A)+1/3*(B)+1/3*(C)$);
        \draw(A)--(B)--(C)--cycle;
        \node at ($(90:.5)+(O)$) {4};
        \node at ($(210:.5)+(O)$) {5};
        \node at ($(330:.5)+(O)$) {6};
      \end{scope}

      \begin{scope}[shift={(9,0)}]
        \coordinate(A) at (0,0);
        \coordinate(B) at (2,0);
        \coordinate(C) at (60:2);
        \coordinate(O) at ($1/3*(A)+1/3*(B)+1/3*(C)$);
        \draw(A)--(B)--(C)--cycle;
        \node at ($(90:.5)+(O)$) {7};
        \node at ($(210:.5)+(O)$) {8};
        \node at ($(330:.5)+(O)$) {9};
      \end{scope}
      
      \begin{scope}[shift={(12,0)}]
        \coordinate(A) at (0,0);
        \coordinate(B) at (2,0);
        \coordinate(C) at (60:2);
        \coordinate(O) at ($1/3*(A)+1/3*(B)+1/3*(C)$);
        \draw(A)--(B)--(C)--cycle;
        \node at ($(90:.5)+(O)$) {10};
        \node at ($(210:.5)+(O)$) {18};
        \node at ($(330:.5)+(O)$) {?};
      \end{scope}
    \end{tikzpicture}
  \end{center}
  \begin{align*}
    (\mathrm{A})\ 200 \quad\quad (\mathrm{B})\ 600 \quad\quad (\mathrm{C})\ 800\quad\quad (\mathrm{D})\ 1200
  \end{align*}
\end{example}
\begin{proof}[提示]平方和的数字和,直到最后是一位数。
  \begin{align*}
    &1^2+2^2+3^2 = 14, && \implies 1 + 4 = 5\\
    &4^2+5^2+6^2 = 74, && \implies 7 + 4 = 14, && \implies 1+4=5\\
    &7^2+8^2+9^2 = 194,&& \implies 1+9+4=14, && \implies 1+4=5
  \end{align*}
  按这个规律,?号处填200,最后会得到5。
\end{proof}