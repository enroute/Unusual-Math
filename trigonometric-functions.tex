
\chapter{三角形与三角函数}
\label{chap:triangle-and-trigonometric-functions}

\section{余弦定理}
\label{sec:law-of-cosines}

\begin{theorem}[余弦定理,Law of Cosine]
  任意三角形,记三边边长分别是$a,b,c$,且$a$和$b$的夹角为$C$,则有$c^2=a^2+b^2-2ab\cos C$。
  \begin{center}
    \begin{tikzpicture}[scale=1.0]
      \coordinate (A) at (0,0);
      \coordinate (B) at (3,0);
      \coordinate (C) at (2, 1.5);
      \draw(A)--(B) node[midway, below]{$c$};
      \draw(B)--(C) node[midway,right]{$a$};
      \draw(C)--(A) node[midway,left]{$b$};
      \draw pic["$C$",<->,draw=orange,angle eccentricity=1.6,angle radius=.6cm]{angle=A--C--B};
    \end{tikzpicture}
  \end{center}
\end{theorem}
\begin{proof}
  余弦定理有证明方法有很多,这里有一种直观的构造方法。分$\angle C$是锐角及钝角两种情况(直角时可直接用勾股定理)。
  \begin{center}
    \begin{tikzpicture}[scale=1.0]
      \begin{scope}
        \coordinate(A) at (0,0);
        \coordinate(B) at (3,3);
        \coordinate(G) at (3,0);
        \coordinate(H) at (0,3);
        \coordinate(C) at (2.3,3.8);
        \coordinate(D) at ($(C)!1!90:(B)$);
        \coordinate(E) at ($(B)!1!-90:(C)$);
        \coordinate(F) at ($(G) + (E) - (B)$);
        % \fill[color=red!20](A)rectangle(B);
        % \fill[color=red!20](B)--(C)--(D)--(E)--cycle;
        \fill[color=red!20](B)--(C)--(H);
        \fill[pattern color=red!20,pattern=north east lines](B)--(E)--(F);
        \fill[pattern color=blue!20,pattern=north west lines](B)--(G)--(F);
        \draw(A)rectangle(B)--(E)--(D)--(C)--(B)--(F)--(E) (C)--(H) (G)--(F) ;
        \draw pic["$C$",<->,draw=orange,angle eccentricity=1.6,angle radius=.4cm]{angle=C--B--H};
        \node[below] at ($.5*(H)+.5*(B)$) {$a$};
        \node[above] at ($.5*(B)+.5*(C)$) {$b$};
        \node[above left] at ($.5*(H)+.5*(C)$) {$c$};
        \draw[dashed](B)--($(B)+(0,1)$) node(BB){};
        \draw[dashed](E)--($(BB)!(E)!(B)$);
      \end{scope}
      \begin{scope}[shift={(6,0)}]
        \coordinate(A) at (0,0);
        \coordinate(B) at (3,3);
        \coordinate(G) at (3,0);
        \coordinate(H) at (0,3);
        \coordinate(C) at (2.3,3.8);
        \coordinate(D) at ($(C)!1!90:(B)$);
        \coordinate(E) at ($(B)!1!-90:(C)$);
        \coordinate(F) at ($(G) + (E) - (B)$);

        \coordinate(U) at ($(H)!1!-90:(C)$);
        \coordinate(V) at ($(C)!1!90:(H)$);

        \fill[color=red!20](U)--(V)--(F)--cycle;
        \fill[pattern color=red!20,pattern=north east lines](C)--(D)--(V)--cycle;
        \fill[pattern color=blue!20,pattern=north west lines](H)--(A)--(U)--cycle;

        % \fill[color=red!20](A)rectangle(B);
        % \fill[color=red!20](B)--(C)--(D)--(E)--cycle;
        % \fill[pattern color=red!20,pattern=north west lines](B)--(C)--(H);
        % \fill[pattern color=blue!20,pattern=north east lines](B)--(G)--(F);
        % \fill[color=red!20](B)--(E)--(F);
        \draw(A)rectangle(B)--(E)--(D)--(C)--(B)--(F)--(E) (C)--(H) (G)--(F) ;
        % \draw pic["$C$",<->,draw=orange,angle eccentricity=1.6,angle radius=.4cm]{angle=C--B--H};
        \node[below] at ($.5*(H)+.5*(B)$) {$a$};
        \node[above] at ($.7*(B)+.3*(C)$) {$b$};
        \node[above left] at ($.5*(H)+.5*(C)$) {$c$};


        \draw[line width=1pt](H)--(U)--(V)--(C)--cycle;
        \draw[line width=1pt](A)--(H)--(U)--(A) (V)--(D)--(C)--(V);
        \draw[line width=1pt](A)--(G)--(F)--(U) (V)--(F)--(E)--(D);

        \draw pic["",draw=orange,angle eccentricity=1.6,angle radius=.5cm]{angle=A--H--U};
        \draw pic["",draw=orange,angle eccentricity=1.6,angle radius=.5cm]{angle=B--H--C};
        \draw pic["",draw=orange,angle eccentricity=1.6,angle radius=.55cm]{angle=A--H--U};
        \draw pic["",draw=orange,angle eccentricity=1.6,angle radius=.55cm]{angle=B--H--C};

        \draw pic["",draw=black,angle eccentricity=1.6,angle radius=.3cm]{angle=H--C--V};
        \draw pic["",draw=black,angle eccentricity=1.6,angle radius=.2cm]{angle=H--C--V};
        \draw pic["",draw=black,angle eccentricity=1.6,angle radius=.3cm]{angle=B--C--D};
        \draw pic["",draw=black,angle eccentricity=1.6,angle radius=.2cm]{angle=B--C--D};
        
        \draw pic["",<->, draw=black,angle eccentricity=1.6,angle radius=.4cm]{angle=G--A--U};
        \node[above right] at (.5,0) {$90^\circ - C$};
      \end{scope}

    \end{tikzpicture}
  \end{center}
  如左图,以三角形为基础,$a,b$两边向外做正方形,再以这两个正方形的两边作平行四边形。则带阴影的三个角形都是$a$为底,$b\cos C$为高,故其面积都为$\frac12ab\cos C$。

  对左图换一种如右图的切割方式,以$c$为边长向内作正方形,然后连接对应的顶点。则由“边角边”可知\tikz{\fill[draw,pattern=north west lines, pattern color=blue!20](0,0)--(1,0)--(.6,.4)--cycle}与\tikz{\fill[draw,pattern=north east lines, pattern color=red!20](0,0)--(1,0)--(.6,.4)--cycle}的两个三角形都与原三角形全等。从而白色的两个是全等的平行四边形,\tikz{\filldraw[draw=black,fill=red!20](0,0)--(1,0)--(.6,.4)--cycle}也与原三角形全等。且白色平行四边形的一个内角为$90^\circ - C$,从而白色平行四边形的面积都是$ab\sin(90^\circ-C)=ab\cos C$。

  左右两图白色部分面积相等,从而有
  \begin{align*}
    c^2 + 2ab\cos C = a^2 + b^2
  \end{align*}

  钝角的情形则可按下图方式构造。
  \begin{center}
    \begin{tikzpicture}[scale=.9]
      \begin{scope}
        \coordinate(A) at (0,0);
        \coordinate(B) at (3,0);
        \coordinate(C) at (-2,1.8);
        \coordinate(D) at ($(B)!1!90:(A)$);
        \coordinate(E) at ($(A)!1!-90:(B)$);
        \coordinate(F) at ($(E) + (C) - (A)$);
        \coordinate(G) at ($(F)!1!-90:(E)$);
        \coordinate(H) at ($(E)!1!90:(F)$);
        \coordinate(I) at ($(H) + (B) - (A)$);
        \fill[color=blue!20](A)--(B)--(C)--cycle (C)--(F)--(G)--cycle;
        \draw(B)--(C)node[midway, above]{$c$};
        \draw(A)--(B)node[midway, below]{$b$};
        \draw(A)--(C)node[pos=.4,below left]{$a$};
        \draw pic["$C$",<->, draw=black,angle eccentricity=1.9,angle radius=.2cm]{angle=B--A--C};
        \draw(G)--(C)--(F)--(G)--(H)--(E)--(F) (A)--(E)--(D)--(B) (H)--(I)--(D);
        \node at ($.25*(A)+.25*(B)+.25*(D)+.25*(E)$) {$b^2$};
        \node at ($.25*(E)+.25*(F)+.25*(G)+.25*(H)$) {$a^2$};
        \node at ($.25*(E)+.25*(F)+.25*(A)+.25*(C)$) {$-ab\cos C$};
        \node at ($.25*(E)+.25*(D)+.25*(I)+.25*(H)$) {$-ab\cos C$};
      \end{scope}
      \begin{scope}[shift={(7.5,0)}]
        \coordinate(A) at (0,0);
        \coordinate(B) at (3,0);
        \coordinate(C) at (-2,1.8);
        \coordinate(D) at ($(B)!1!90:(A)$);
        \coordinate(E) at ($(A)!1!-90:(B)$);
        \coordinate(F) at ($(E) + (C) - (A)$);
        \coordinate(G) at ($(F)!1!-90:(E)$);
        \coordinate(H) at ($(E)!1!90:(F)$);
        \coordinate(I) at ($(H) + (B) - (A)$);
        \fill[color=blue!20](G)--(I)--(H)--cycle (I)--(D)--(B)--cycle;
        \draw[help lines](B)--(A)--(C) (G)--(F)--(C);
        \draw[help lines](H)--(E)--(F) (A)--(E)--(D)--(B);
        \draw(B)--(C)node[midway,above]{$c$} (C)--(G)--(H)--(I)--(D)--(B) (G)--(I)--(B);
        \node at ($.25*(B)+.25*(C)+.25*(G)+.25*(I)$) {$c^2$};
      \end{scope}
    \end{tikzpicture}
  \end{center}
  左图是先分别沿一边$a,b$向外作一正方形和平行四边形,然后再作另一正方形和平行四边形。右图是连接对应的顶点将图重新分割成边长为$c$的正方形与两个带阴影的三角形。左右两图空白面积相等,从而有
  \begin{align*}
    a^2+b^2-2ab\cos C=c^2&\qedhere
  \end{align*}
\end{proof}

\section{角平分线}

\subsection{角平分线的长度}
\label{sec:length-of-angular-bisector}

三角形的角平分线的长度公式有好几种形式,其中常用的有以下几种。

\begin{theorem}
  $AD$是$\triangle ABC$中$\angle A$的角平分线,则角平分线$AD$的长度$d$可由下面公式给出:
  \begin{align*}
    d=\frac{2 bc\cdot\cos\frac{A}{2}}{b+c}
  \end{align*}
\end{theorem}
\begin{proof}如图,考虑$\triangle ABC$的面积$S_{\triangle ABC}=S_{\triangle ABD}+S_{\triangle ACD}$,有
  \begin{center}
    \begin{tikzpicture}[scale=1.0]
      \coordinate[label=below left:$A$](A) at (0,0);
      \coordinate[label=below right:$B$](B) at (3,0);
      \coordinate[label=right:$D$](D) at (30:2.5);
      \coordinate(C') at (60:5);
      \tkzInterLL(A,C')(B,D)\tkzGetPoint{C}
      \tkzLabelPoint[above](C){$C$}
      \draw[line width=2pt](A)--(B) node[midway, below]{$c$};
      \draw[line width=2pt](A)--(C) node[midway, left]{$b$};
      \draw[line width=2pt](A)--(D) node[midway, above]{$d$};
      \draw(B)--(C);
      \draw pic["",<->,draw=orange,angle eccentricity=1.6,angle radius=.4cm]{angle=B--A--C};
    \end{tikzpicture}
  \end{center}
  \begin{align*}
    S_{\triangle ABC} = \frac12 bc\cdot \sin A = \frac12 bd\cdot \sin\frac{A}2 + \frac12 cd\cdot\sin\frac{A}2
  \end{align*}
  将$\sin\alpha=2\sin\frac{\alpha}2\cos\frac{\alpha}2$代入可得。
\end{proof}


\begin{example}
  任意正数$a,b,c$,有
  \begin{align*}
    \sqrt{a^2+ac+c^2}\le \sqrt{a^2-ab+b^2}+\sqrt{b^2-bc+c^2}
  \end{align*}
\end{example}
\begin{proof}[提示]利用余弦定理构造线段,如图:
  \begin{center}
    \begin{tikzpicture}[scale=1.0]
      \coordinate[label=below left:$D$] (D) at (0,0);
      \coordinate[label=below right:$C$] (C) at (3,0);
      \coordinate[label=above left:$A$] (A) at (120:2);
      \coordinate[label=above:$B$] (B) at (60:4);
      \draw[line width=2pt](D)--(A)node[midway,below left]{$a$};
      \draw[line width=2pt](D)--(C)node[midway,below left]{$c$};
      \draw[line width=2pt](D)--(B)node[midway,above left]{$b$};
      \draw(A)--(B)--(C)--cycle;
      \draw pic["$60^\circ$",<->,draw=orange,angle eccentricity=1.8,angle radius=.4cm]{angle=B--D--A};
      \draw pic["$60^\circ$",<->,draw=orange,angle eccentricity=1.6,angle radius=.6cm]{angle=C--D--B};
    \end{tikzpicture}
  \end{center}
  如上图,以点$D$为起点,用长度为$a,b,c$的线段及两个$60^\circ$的夹角构造图形,其中$AD=a$,$BD=b$,$CD=c$,$\angle ADB=BDC=60^\circ$。则由余弦定理有
  \begin{align*}
    AC=&\,\sqrt{a^2+ac+c^2}\\
    AB=&\,\sqrt{a^2-ab+b^2}\\
    BC=&\,\sqrt{b^2-bc+c^2}
  \end{align*}
  再由三角形$ABC$的两边和大于等于(三角形退化为三顶点共线时等号成立)第三边,可知原不等式成立。当且仅当$B$落在线段$AC$上时等号成立,此时应用角平分线长度公式,有
  \begin{align*}
    b=\frac{2ac\cos60^\circ}{a+c}=\frac{ac}{a+c}&\qedhere
  \end{align*}
\end{proof}

\begin{example}\label{ex:triangle-sides-inequality}
  三角形的三条边边长$a,b,c$满足
  \begin{align*}
    (a+b-c)(b+c-a)(c+a-b)\le abc
  \end{align*}
\end{example}
\begin{proof}
  令$s=\frac12(a+b+c)$是三角形的半周长,且
  \begin{align*}
    x\equiv s-a, \quad y\equiv s-b, \quad z\equiv s-c
  \end{align*}
  由三角形的性质,可知$x,y,z$都是正的,且
  \begin{gather*}
    a=y+z,\quad b=z+x,\quad c=x+y\\
    a+b-c = 2z, \quad b+c-a=2x, \quad c+a-b = 2y
  \end{gather*}
  代入并重新各项顺序排列,原不等式等价于$8xyz\le (x+y)(y+z)(z+x)$。对右边每项应用AM--GM不等式可得。
\end{proof}

\begin{example}\label{ex:triangle-sides-inequality-simplified}
  $a,b,c$是三角形的三边边长,则
  \begin{align*}
    (a-b)(b-c)(c-a)<abc
  \end{align*}
\end{example}
\begin{proof}
  由对称性,不妨设$a\le b\le c$,从而存在非负数$u,v$,使得
  \begin{align*}
    b=a+u,\quad c=a+u+v
  \end{align*}
  由$c<a+b$可知$a+u+v<a + (a+u)\implies a>v$。从而
  \begin{align*}
    (a-b)(b-c)(c-a)=(-u)(-v)(u+v) = uv(u+v)\\
    abc = a(a+u)(a+u+v) > v(v+u)(u+v) \ge uv(u+v)
  \end{align*}
  比较两式可得。
\end{proof}

\begin{example}
  若$a,b,c$是一个三角形的三边边长,则
  \begin{align*}
    \left| \frac{a-b}{a+b} + \frac{b-c}{b+c} + \frac{c-a}{c+a}
    \right|
    < \frac18
  \end{align*}
\end{example}
\begin{proof}[提示]
  应用例~\ref{ex:sum-is-negative-to-product},将和化为积的形式:
  \begin{align*}
    \left| \frac{a-b}{a+b} + \frac{b-c}{b+c} + \frac{c-a}{c+a} \right|
    =&\, \left| \frac{a-b}{a+b} \cdot \frac{b-c}{b+c} \cdot \frac{c-a}{c+a} \right|
    = \frac{|a-b|\cdot|b-c|\cdot|c-a|}{(a+b)(b+c)(c+a)}\\
    % <&\, \frac{2\sqrt{ab}\cdot 2\sqrt{bc}\cdot 2\sqrt{ca}}{(a+b)(b+c)(c+a)}\\
    % \le&\,\frac{}{(a+b)(b+c)(c+a)}\\
    \intertext{再由例~\ref{ex:triangle-sides-inequality-simplified}的结论,有}
    <&\, \frac{abc}{(a+b)(b+c)(c+a)}
    \le \frac{abc}{2\sqrt{ab} \cdot 2\sqrt{bc}\cdot 2\sqrt{ca}} = \frac18&&\qedhere
  \end{align*}
\end{proof}